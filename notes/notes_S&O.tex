%\documentclass[UTF8]{ctexart} % use larger type; default would be 10pt
\documentclass[a4paper]{article}
\usepackage{xeCJK}
%\usepackage{ctex}
%\usepackage{luatexja-fontspec}
%\setmainjfont{FandolSong}
%\usepackage[utf8]{inputenc} % set input encoding (not needed with XeLaTeX)

%%% Examples of Article customizations
% These packages are optional, depending whether you want the features they provide.
% See the LaTeX Companion or other references for full information.

%%% PAGE DIMENSIONS
\usepackage{geometry} % to change the page dimensions
\geometry{a4paper} % or letterpaper (US) or a5paper or....
% \geometry{margin=2in} % for example, change the margins to 2 inches all round
% \geometry{landscape} % set up the page for landscape
%   read geometry.pdf for detailed page layout information

\usepackage{graphicx} % support the \includegraphics command and options

% \usepackage[parfill]{parskip} % Activate to begin paragraphs with an empty line rather than an indent

%%% PACKAGES
\usepackage{booktabs} % for much better looking tables
\usepackage{array} % for better arrays (eg matrices) in maths
\usepackage{paralist} % very flexible & customisable lists (eg. enumerate/itemize, etc.)
\usepackage{verbatim} % adds environment for commenting out blocks of text & for better verbatim
\usepackage{subfig} % make it possible to include more than one captioned figure/table in a single float
% These packages are all incorporated in the memoir class to one degree or another...

%%% HEADERS & FOOTERS
\usepackage{fancyhdr} % This should be set AFTER setting up the page geometry
\pagestyle{fancy} % options: empty , plain , fancy
\renewcommand{\headrulewidth}{0pt} % customise the layout...
\lhead{}\chead{}\rhead{}
\lfoot{}\cfoot{\thepage}\rfoot{}

%%% SECTION TITLE APPEARANCE
\usepackage{sectsty}
%\allsectionsfont{\sffamily\mdseries\upshape} % (See the fntguide.pdf for font help)
% (This matches ConTeXt defaults)

%%% ToC (table of contents) APPEARANCE
\usepackage[nottoc,notlof,notlot]{tocbibind} % Put the bibliography in the ToC
\usepackage[titles,subfigure]{tocloft} % Alter the style of the Table of Contents
%\renewcommand{\cftsecfont}{\rmfamily\mdseries\upshape}
%\renewcommand{\cftsecpagefont}{\rmfamily\mdseries\upshape} % No bold!

%%% END Article customizations

%%% The "real" document content comes below...

\setlength{\parindent}{0pt}
\usepackage{physics}
\usepackage{amsmath}
%\usepackage{symbols}
\usepackage{AMSFonts}
\usepackage{bm}
%\usepackage{eucal}
\usepackage{mathrsfs}
\usepackage{amssymb}
\usepackage{float}
\usepackage{multicol}
\usepackage{abstract}
\usepackage{empheq}
\usepackage{extarrows}
\usepackage{textcomp}
\usepackage{mhchem}
\usepackage{braket}
\usepackage{siunitx}
\usepackage[utf8]{inputenc}
\usepackage{tikz-feynman}
\usepackage{feynmp}
\usepackage{fontspec}
%\sffamily
\setmainfont{CMU Serif}
\setsansfont{CMU Sans Serif}
\setmonofont{CMU Typewriter Text}




\DeclareMathOperator{\p}{\prime}
\DeclareMathOperator{\ti}{\times}

\DeclareMathOperator{\e}{\mathrm{e}}
\DeclareMathOperator{\I}{\mathrm{i}}
\DeclareMathOperator{\Arg}{\mathrm{Arg}}
\newcommand{\NA}{N_\mathrm{A}}
\newcommand{\kB}{k_\mathrm{B}}

\DeclareMathOperator{\ra}{\rightarrow}
\DeclareMathOperator{\llra}{\longleftrightarrow}
\DeclareMathOperator{\lra}{\longrightarrow}
\DeclareMathOperator{\dlra}{\;\Leftrightarrow\;}
\DeclareMathOperator{\dra}{\;\Rightarrow\;}

%%%%%%%%%%%% QUANTUM MECHANICS %%%%%%%%%%%%%%%%%%%%%%%%
\newcommand{\bkk}[1]{\Braket{#1|#1}}
\newcommand{\bk}[2]{\Braket{#1|#2}}
\newcommand{\bkev}[2]{\Braket{#2|#1|#2}}

\DeclareMathOperator{\na}{\bm{\nabla}}
\DeclareMathOperator{\nna}{\nabla^2}
\DeclareMathOperator{\drrr}{\dd[3]\vb{r}}

\DeclareMathOperator{\psis}{\psi^\ast}
\DeclareMathOperator{\Psis}{\Psi^\ast}
\DeclareMathOperator{\hi}{\hat{\vb{i}}}
\DeclareMathOperator{\hj}{\hat{\vb{j}}}
\DeclareMathOperator{\hk}{\hat{\vb{k}}}
\DeclareMathOperator{\hr}{\hat{\vb{r}}}
\DeclareMathOperator{\hT}{\hat{\vb{T}}}
\DeclareMathOperator{\hH}{\hat{H}}

\DeclareMathOperator{\hL}{\hat{\vb{L}}}
\DeclareMathOperator{\hp}{\hat{\vb{p}}}
\DeclareMathOperator{\hx}{\hat{\vb{x}}}
\DeclareMathOperator{\ha}{\hat{\vb{a}}}
\DeclareMathOperator{\hs}{\hat{\vb{s}}}
\DeclareMathOperator{\hS}{\hat{\vb{S}}}
\DeclareMathOperator{\hSigma}{\hat{\bm\Sigma}}
\DeclareMathOperator{\hJ}{\hat{\vb{J}}}

\DeclareMathOperator{\Tdv}{-\dfrac{\hbar^2}{2m}\dv[2]{x}}
\DeclareMathOperator{\Tna}{-\dfrac{\hbar^2}{2m}\nabla^2}

%\DeclareMathOperator{\s}{\sum_{n=1}^{\infty}}
\DeclareMathOperator{\intinf}{\int_0^\infty}
\DeclareMathOperator{\intdinf}{\int_{-\infty}^\infty}
%\DeclareMathOperator{\suminf}{\sum_{n=0}^\infty}
\DeclareMathOperator{\sumnzinf}{\sum_{n=0}^\infty}
\DeclareMathOperator{\sumnoinf}{\sum_{n=1}^\infty}
\DeclareMathOperator{\sumndinf}{\sum_{n=-\infty}^\infty}
\DeclareMathOperator{\sumizinf}{\sum_{i=0}^\infty}

%%%%%%%%%%%%%%%%% PARTICLE PHYSICS %%%%%%%%%%%%%%%%
\DeclareMathOperator{\hh}{\hat{h}}               % helicity
\DeclareMathOperator{\hP}{\hat{\vb{P}}}          % Parity
\DeclareMathOperator{\hU}{\hat{U}}
\DeclareMathOperator{\hG}{\hat{G}}

\DeclareMathOperator{\GeV}{\si{GeV}}
\DeclareMathOperator{\LI}{\mathscr{L}.I.}
%\DeclareMathOperator{\g5}{\gamma^5}
\DeclareMathOperator{\gmuu}{\gamma^\mu}
\DeclareMathOperator{\gmud}{\gamma_\mu}
\DeclareMathOperator{\gnuu}{\gamma^\nu}
\DeclareMathOperator{\gnud}{\gamma_\nu}

\renewcommand{\u}{\mathrm{u}}
\renewcommand{\d}{\mathrm{d}}
\DeclareMathOperator{\s}{\mathrm{s}}

\DeclareMathOperator{\q}{\mathrm{q}}
\DeclareMathOperator{\bq}{\bar{\mathrm{q}}}

\DeclareMathOperator{\g}{\mathrm{g}}
\DeclareMathOperator{\W}{\mathrm{W}}
\DeclareMathOperator{\Z}{\mathrm{Z}}

%%% Feynman Diagram
\newcommand{\pa}{particle}
\newcommand{\mo}{momentum}
\newcommand{\el}{edge label}

%%% MQC
\newcommand{\iden}{{\large \bm{1}}}
\newcommand{\qed}{$ \Square $}
\newcommand{\tPhi}{\tilde{\Phi} }
\newcommand{\hsP}{\hat{\mathscr{P}}}
\newcommand{\hsS}{\hat{\mathscr{S}}}
\DeclareMathOperator{\core}{\mathrm{core}}

\newcommand{\dis}{\displaystyle}
\numberwithin{equation}{section}

\title{Notes of \textbf{Modern Quantum Chemistry, Szabo \& Ostlund}}
\author{hebrewsnabla}
%\date{} % Activate to display a given date or no date (if empty),
         % otherwise the current date is printed 

\begin{document}
% \boldmath
\maketitle

\tableofcontents

\newpage

\setcounter{section}{-1}
\section{}
spatial mol orb -- $ \psi $ -- $ i,j,k,... $\\
spatial basis fxn -- $ \phi $ -- $ \mu,\nu,\lambda,... $\\
spin orb -- $ \chi $\\
occ mol orb -- $ a,b,c,... $\\
vir mol orb -- $ r,s,t,... $\\
exact many-elec wfn -- $ \Phi $\\
approx many-elec wfn -- $ \Psi $\\
exact energy -- $ \mathscr{E} $\\
approx energy -- $ E $

\section{}
\subsection{}
\subsection{}
\subsection{}
\subsection{N-D Complex Vector Spaces}
Suppose
\begin{equation}\label{key}
\mathcal{O}\ket{a} = \ket{b}
\end{equation}
\begin{equation}\label{key}
\Braket{i | \mathcal{O} | j} = O_{ij}
\end{equation}
def the \textbf{adjoint} of $ \mathcal{O} $ as $ \mathcal{O}^\dagger $
\begin{equation}\label{key}
\bra{a}\mathcal{O}^\dagger = \bra{b}
\end{equation}
\begin{equation}\label{key}
\Braket{i | \mathcal{O}^\dagger | j} = O_{ji}^*
\end{equation}

\subsubsection{Change of Basis}
\begin{equation}\label{key}
\ket{\alpha} = \sum_i\ket{i}\braket{i | \alpha} = \sum_i\ket{i} U_{i\alpha}
\end{equation}
\begin{equation}\label{key}
\ket{i} = \sum_\alpha \ket{\alpha}\braket{i | \alpha} = \sum_\alpha \ket{\alpha} U_{i\alpha}^*
\end{equation}
If $ i, \alpha $ are all orthonormal, $ \vb{U} $ must be unitary.

\begin{equation}\label{key}
\Omega_{\alpha\beta} = \Braket{\alpha | \mathcal{O} | \beta} = ... \sum_{ij} U_{\alpha i}^* O_{ij} U_{j\beta}
\end{equation}
or
\begin{equation}\label{key}
\bm\Omega = \vb{U}^\dagger \vb{O} \vb{U}
\end{equation}


\section{}
\subsection{The Electronic Problem}
\subsubsection{Atomic Units}
\subsubsection{The B-O Approximation}
\subsubsection{The Antisymmetry or Pauli Exclusion Principle}
\subsection{Orbitals, Slater Determinants, and Basis Functions}
\subsubsection{Spin Orbitals and Spatial Orbitals}
\subsubsection{Hartree Products}
\subsubsection{Slater Determinants}
def
\begin{equation}\label{key}
\ket{\chi_i(\vb{x}_1)\chi_j(\vb{x}_2)\cdots\chi_k(\vb{x}_N)} 
\equiv \dfrac{1}{\sqrt{N!}}
\mqty| \chi_i(\vb{x}_1) & \chi_j(\vb{x}_1) & \cdots & \chi_k(\vb{x}_1)\\
        \chi_i(\vb{x}_2) & \chi_j(\vb{x}_2) & \cdots & \chi_k(\vb{x}_2)\\
        \vdots & \vdots & & \vdots \\
        \chi_i(\vb{x}_N) & \chi_j(\vb{x}_N) & \cdots & \chi_k(\vb{x}_N)|
\end{equation}
It can be further shortened to
\begin{equation}\label{key}
\ket{\chi_i\chi_j\cdots\chi_k}
\end{equation}
\subsubsection{The Hartree-Fock Approximation}
\subsubsection{The Minimal Basis $ \ce{H_2} $ Model}
gerade, ungerade
\subsubsection{Excited Determinants}
Suppose the ground state det
\begin{equation}\label{key}
\ket{\psi_0} = \ket{\chi_1\cdots\chi_a\cdots\chi_b\cdots\chi_N}
\end{equation}
thus, singly excited det
\begin{equation}\label{key}
\ket{\psi_a^r} = \ket{\chi_1\cdots\chi_r\cdots\chi_b\cdots\chi_N}
\end{equation}
\begin{equation}\label{key}
\ket{\psi_{ab}^{rs}} = \ket{\chi_1\cdots\chi_r\cdots\chi_s\cdots\chi_N}
\end{equation}

How does program determine what dets can exist? by gerade/ungerade?

\subsection{Operators and Matrix Elements}
\subsubsection{Minimal Basis $ \ce{H_2} $ Matrix Elements}

\subsubsection{Notations for 1- and 2-Electron Integrals}
For spin orb,\\
physicists'
\begin{equation}\label{key}
\Braket{ij|kl} = \Braket{\chi_i(1)\chi_j(2) | \dfrac{1}{r_{12}} | \chi_k(1)\chi_l(2)}
\end{equation}
\begin{equation}\label{key}
\Braket{ij||kl} = \Braket{ij|kl} - \Braket{ij|lk}
\end{equation}
chemists'
\begin{equation}\label{key}
[ij|kl] = \Braket{\chi_i(1)\chi_j(1) | \dfrac{1}{r_{12}} | \chi_k(2)\chi_l(2)}
\end{equation}
For spatial orb
\begin{equation}\label{key}
(ij|kl) = \Braket{\psi_i(1)\psi_j(1) | \dfrac{1}{r_{12}} | \psi_k(2)\psi_l(2)}
\end{equation}

\subsubsection{General Rules for Matrix Elements}
\begin{enumerate}
	\item $ \ket{K} = \ket{\cdots m n\cdots} $
	\begin{equation}\label{key}
	\Braket{K | \mathscr{H} | K} = \sum_m^N [m|h|m] + \dfrac{1}{2}\sum_m^N\sum_n^N \qty([mm|nn] - [mn|nm])
	\end{equation}
	or (Since $ [mm|mm] - [mm|mm] = 0 $)
	\begin{equation}\label{key}
	\Braket{K | \mathscr{H} | K} = \sum_m^N [m|h|m] + \sum_m^N\sum_{n>m}^N \qty([mm|nn] - [mn|nm])
	\end{equation}
	\item $ \ket{K} = \ket{\cdots m n\cdots} $, $ \ket{L} = \ket{\cdots p n\cdots} $
	\begin{equation}\label{key}
	\Braket{K | \mathscr{H} | L} = [m|h|p] + \sum_n^N \qty([mp|nn] - [mn|np])
	\end{equation}
	\item $ \ket{K} = \ket{\cdots m n\cdots} $, $ \ket{L} = \ket{\cdots p q\cdots} $
	\begin{equation}\label{key}
	\Braket{K | \mathscr{H} | L} = [mp|nq] - [mq|np]
	\end{equation}
\end{enumerate}

\subsection{Second Quantization}
\subsubsection{Creation and Annihilation Operators and Their Anticommutation Relations}
\begin{equation}\label{key}
a_i^\dagger a_j^\dagger + a_j^\dagger a_i^\dagger = 0 \quad a_i a_j + a_j a_i^ = 0
\end{equation}
\begin{equation}\label{key}
\{a_i, a_j^\dagger\} \equiv a_i a_j^\dagger + a_j^\dagger a_i = \delta_{ij}
\end{equation}

\subsection{Spin-Adapted Configurations}
\subsubsection{Spin Operators}
total spin
\begin{align}
\hat{\mathscr{P}} &= \sum_i^N \hs(i)\\
\hsS_x &= \sum_i^N \hs_x(i)\\
\hsS_+ &= \sum_i^N \hs_+(i)\\
\hsS^2 &= \hsS_+\hsS_- - \hsS_z + \hsS_z^2
\end{align}
\begin{align}
\hsS^2\ket{\Phi} &= S(S+1)\ket{\Phi}\\
\hsS_z\ket{\Phi} &= M_S\ket{\Phi}
\end{align}
\begin{equation}\label{key}
\hsS_z\ket{ij\cdots k} = \dfrac{1}{2}(N^\alpha - N^\beta)\ket{ij\cdots k}
\end{equation}

\subsubsection{Restricted Determinants and Spin-Adapted Configurations}
\begin{equation}\label{key}
\ket{^1\Psi_1^2} = \dfrac{1}{\sqrt{2}}(\ket{1\bar{2}} + \ket{\bar{1}2})
= \dfrac{1}{\sqrt{2}}(\psi_1(1)\psi_2(2)\alpha(1)\beta(2) - \psi_2(1)...)
\end{equation}

\section{The Hartree-Fock Approximation}
\subsection{The HF Equations}
\subsubsection{The Coulomb and Exchange Operators}
\begin{equation}\label{key}
\mathscr{J}_b(1) = \int\dd x_2 \chi_b^*(2)r_{12}^{-1} \chi_b(2)
\end{equation}
\begin{align}\label{key}
\mathscr{K}_b(1) &= \int\dd x_2 \chi_b^*(2) r_{12}^{-1} \hsP_{12} \chi_b(2) \\
&= \int\dd x_2 \chi_b^*(2) r_{12}^{-1} \chi_b(1) \hsP_{12} 
\end{align}
\begin{align}
\Braket{\chi_a(1) | \mathscr{J}_b(1) | \chi_a(1)} &= J_{ab} \\
\Braket{\chi_a(1) | \mathscr{K}_b(1) | \chi_a(1)} &= K_{ab} 
\end{align}
\subsubsection{The Fock Operator}
\subsection{Derivation of the HF Equations}
\subsubsection{Functional Variation}
\subsubsection{Minimization of the Energy of a Single Determinant}
\subsubsection{The Canonical HF Equations}

\subsection{Interpretation of Solutions to the HF Equations}
\subsubsection{Orbital Energies and Koopmans' Theorem}
\begin{align}\label{key}
\varepsilon_i &= \Braket{i|h|i} + \sum_b \Braket{ib||ib} \\
&= \Braket{i|h|i} + \sum_b \qty(\Braket{ib|ib} - \Braket{ib|bi})
\end{align}
\paragraph{Koopmans' Theorem}
\begin{equation}\label{key}
\text{IP} = -\varepsilon_a \quad 
\text{EA} = -\varepsilon_r
\end{equation}
Koopmans' EA is often bad.
\subsubsection{Brillouin's Theorem}
\begin{equation}\label{key}
\Braket{\Psi_0 | \mathscr{H} | \Psi_a^r} = 0
\end{equation}

\subsubsection{The HF Hamiltonian}
\begin{equation}\label{key}
\mathscr{H}_0 = \sum_i^N f(i) 
\end{equation}

\subsection{Restricted Closed-shell HF: The Roothaan Equations}
\subsubsection{Closed-shell HF: Restricted Spin Orbitals}
\begin{equation}\label{key}
E_0 = 2\sum_a h_{aa} + \sum_a\sum_b(2J_{ab} - K_{ab})
\end{equation}

\subsubsection{Introduction of a Basis: The Roothaan Equations}

\subsubsection{The Charge Density}

\subsubsection{Expression for the Fock Matrix}

\subsubsection{Orthogonalization of the Basis}
\begin{equation}\label{key}
\vb{X}^\dagger \vb{S} \vb{X} = \iden
\end{equation}
$ \vb{S} $ can be diagonalized by unitary matrix $ \vb{U} $:
\begin{equation}\label{key}
\vb{U}^\dagger \vb{S} \vb{U} = \vb{s}
\end{equation}
\paragraph{Symmetric Orthogonalization}
\begin{equation}\label{key}
\vb{X} = \vb{S}^{-1/2} = \vb{U}\vb{s}^{-1/2}\vb{U}^\dagger
\end{equation}
(linear dependence must be removed)
\paragraph{Canonical Orthogonalization}
\begin{equation}\label{key}
\vb{X} = \vb{U}\vb{S}^{-1/2} = \vb{U}\vb{s}^{-1/2}\vb{U}^\dagger
\end{equation}
Suppose $ \vb{s} $ has $ m $ small values, we make a truncated $ K\cross (K-m) $matrix
\begin{equation}\label{key}
\tilde{\vb{X}} = ...
\end{equation}
thus
\begin{equation}\label{key}
\phi_\mu' = \sum_\nu^K \phi_\nu \tilde{X}_{\nu\mu} \qquad \mu = 1,2,\cdots,K-m
\end{equation}
However, calculate 2e integrals in transformed matrix is very time-consuming.\\
Since
\begin{equation}\label{key}
\bm\phi' = \bm\phi \vb{X}
\end{equation}
\begin{equation}\label{key}
\bm\psi = \bm\phi \vb{C}
\end{equation}
we have
\begin{equation}\label{key}
\bm\psi = \bm\phi' \vb{X}^{-1} \vb{C}
\end{equation}
Let
\begin{equation}\label{key}
\vb{C}' = \vb{X}^{-1} \vb{C} \qquad or \; \vb{C} = \vb{X}\vb{C}'
\end{equation}
thus
\begin{equation}\label{key}
\vb{F} \vb{X}\vb{C}' = \vb{S} \vb{X}\vb{C}' \bm\varepsilon
\end{equation}
\begin{equation}\label{key}
(\vb{X}^\dagger \vb{F}\vb{X}) \vb{C}' = (\vb{X}^\dagger \vb{S}\vb{X}) \vb{C}' \bm\varepsilon = \vb{C}'\bm\varepsilon 
\end{equation}
def:
\begin{equation}\label{key}
\vb{F}' = \vb{X}^\dagger \vb{F}\vb{X}
\end{equation}

\subsubsection{The SCF Procedure}
P. 148\\
I think $ \vb{C}' $ should be $ (K-m)\cross K $.

\subsubsection{Expectation Values and Population Analysis}
\begin{equation}\label{key}
N = \sum_\mu\sum_\nu P_{\mu\nu} S_{\mu\nu} = \tr \vb{PS}
\end{equation}
Mulliken:
\begin{equation}\label{key}
q_A = Z_A - \sum_{\mu\in A}(\vb{PS})_{\mu\mu}
\end{equation}
L\"owin:
\begin{equation}\label{key}
q_A = Z_A - \sum_{\mu\in A}(\vb{S}^{1/2} \vb{P} \vb{S}^{1/2})_{\mu\mu}
\end{equation}

\subsection{Model Calculations on \ce{H_2} and \ce{HeH^+}}
\subsubsection{The $ 1s $ Minimal STO-3G Basis Set}
\begin{equation}\label{key}
\alpha = \alpha_{(\zeta=1.0)}\cross \zeta^2
\end{equation}
\subsubsection{STO-3G \ce{H_2}}

\subsubsection{An SCF Calculation on STO-3G \ce{HeH^+}}

\subsection{Polyatomic Basis Sets}
\subsubsection{Contracted Gaussian Functions}
Notation: (pGTO)/[cGTO] (ignore $ p_x,p_y,... $)
\begin{table}[H]
	\begin{tabular}{cc}
		STO-3G & (6s3p/3s)/[2s1p/1s]\\
		4-31G & (8s4p/4s)/[3s2p/2s] \\
		T. Dunning (JCP 1970) & (9s5p/4s)/[3s2p/2s] \\ 
		6-31G* (sph) & (10s4p1d)/[3s2p1d/2s] \\
		6-31G** (sph) & (10s4p1d/4s1p)/[3s2p1d/2s1p]
	\end{tabular}
\end{table}
\subsubsection{Minimal Basis Sets: STO-3G}

\subsubsection{Double Zeta Basis Sets: 4-31G}

\subsubsection{Polarized Basis Sets: 6-31G* and 6-31G**}
What's D polarized basis?\\
Cartesian: $ xx, yy, zz, xy, yz, xz  $\\
Spherical: $ 3z^2-r^2, x^2-y^2, xy, yz, zx $, ($ r^2 $ is removed)

\subsection{Some Illustrative Closed-shell Calculations}
\subsubsection{Total Energies}

\subsubsection{Ionization Potentials}

\subsubsection{Equilibrium Geometries}

\subsubsection{Population Analysis and Dipole Moments}

\subsection{Unrestricted Open-shell HF: The Pople-Nesbet Equations}
\subsubsection{Open-shell HF: Unrestricted Spin Orbitals}
\begin{equation}\label{key}
f^\alpha(1) = h(1) + \sum_a^{N_\alpha}\qty[J_a^\alpha(1) - K_a^{\alpha}(1) ] + \sum_a^{N_\beta} J_a^\beta(1) 
\end{equation}
\begin{equation}\label{key}
E_0 = \sum_a h_{aa} + \dfrac{1}{2}\sum_a^{N_\alpha}\sum_b^{N_\alpha} (J_{ab}^{\alpha\alpha} - K_{ab}^{\alpha\alpha}) + \dfrac{1}{2}\sum_a^{N_\beta}\sum_b^{N_\beta} (J_{ab}^{\beta\beta} - K_{ab}^{\beta\beta}) 
+ \sum_a^{N_\alpha}\sum_b^{N_\beta} J_{ab}^{\alpha\beta}
\end{equation}

\subsubsection{Introduction of a Basis: The Pople-Nesbet Equations}
\begin{align}
\vb{F}^\alpha \vb{C}^\alpha = \vb{S} \vb{C}^\alpha \bm\varepsilon^\alpha \\
\vb{F}^\beta \vb{C}^\beta = \vb{S} \vb{C}^\beta \bm\varepsilon^\beta
\end{align}

\subsubsection{Unrestricted Density Matrices}
spin density
\begin{equation}\label{key}
\rho^S(\vb{r}) = \rho^\alpha(\vb{r}) - \rho^\beta(\vb{r})
\end{equation}
\begin{equation}\label{key}
\vb{P}^S = \vb{P}^\alpha - \vb{P}^\beta
\end{equation}

\subsubsection{Expression for the Fock Matrices}
\begin{align}
F^\alpha_{\mu\nu} &= H^{\core}_{\mu\nu} + \sum_\lambda\sum_\sigma \qty[ P^T_{\lambda\sigma}(\mu\nu|\sigma\lambda) - P^\alpha_{\lambda\sigma}(\mu\lambda|\sigma\nu)] \\
F^\beta_{\mu\nu} &= H^{\core}_{\mu\nu} + \sum_\lambda\sum_\sigma \qty[ P^T_{\lambda\sigma}(\mu\nu|\sigma\lambda) - P^\beta_{\lambda\sigma}(\mu\lambda|\sigma\nu)]
\end{align}

\subsubsection{Solution of the Unrestricted SCF Equations}



\end{document}
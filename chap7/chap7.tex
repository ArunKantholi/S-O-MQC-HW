%\documentclass[UTF8]{ctexart} % use larger type; default would be 10pt
\documentclass[a4paper]{article}
\usepackage{xeCJK}
%\usepackage[utf8]{inputenc} % set input encoding (not needed with XeLaTeX)

%%% Examples of Article customizations
% These packages are optional, depending whether you want the features they provide.
% See the LaTeX Companion or other references for full information.

%%% PAGE DIMENSIONS
\usepackage{geometry} % to change the page dimensions
\geometry{a4paper} % or letterpaper (US) or a5paper or....
\geometry{margin=1in} % for example, change the margins to 2 inches all round
% \geometry{landscape} % set up the page for landscape
%   read geometry.pdf for detailed page layout information

\usepackage{graphicx} % support the \includegraphics command and options

% \usepackage[parfill]{parskip} % Activate to begin paragraphs with an empty line rather than an indent

%%% PACKAGES
\usepackage{booktabs} % for much better looking tables
\usepackage{array} % for better arrays (eg matrices) in maths
\usepackage{paralist} % very flexible & customisable lists (eg. enumerate/itemize, etc.)
\usepackage{verbatim} % adds environment for commenting out blocks of text & for better verbatim
\usepackage{subfig} % make it possible to include more than one captioned figure/table in a single float
% These packages are all incorporated in the memoir class to one degree or another...

%%% HEADERS & FOOTERS
\usepackage{fancyhdr} % This should be set AFTER setting up the page geometry
\pagestyle{fancy} % options: empty , plain , fancy
\renewcommand{\headrulewidth}{0pt} % customise the layout...
\lhead{}\chead{}\rhead{}
\lfoot{}\cfoot{\thepage}\rfoot{}

%%% SECTION TITLE APPEARANCE
\usepackage{sectsty}
%\allsectionsfont{\sffamily\mdseries\upshape} % (See the fntguide.pdf for font help)
% (This matches ConTeXt defaults)

%%% ToC (table of contents) APPEARANCE
\usepackage[nottoc,notlof,notlot]{tocbibind} % Put the bibliography in the ToC
\usepackage[titles,subfigure]{tocloft} % Alter the style of the Table of Contents
%\renewcommand{\cftsecfont}{\rmfamily\mdseries\upshape}
%\renewcommand{\cftsecpagefont}{\rmfamily\mdseries\upshape} % No bold!

%%% END Article customizations

%%% The "real" document content comes below...

\setlength{\parindent}{0pt}
\usepackage{physics}
\usepackage{amsmath}
%\usepackage{symbols}
\usepackage{AMSFonts}
\usepackage{bm}
%\usepackage{eucal}
\usepackage{mathrsfs}
\usepackage{amssymb}
\usepackage{float}
\usepackage{multicol}
\usepackage{abstract}
\usepackage{empheq}
\usepackage{extarrows}
\usepackage{textcomp}
\usepackage{fontspec}
%\usepackage{physymb}

\setmainfont{CMU Serif}
\setsansfont{CMU Sans Serif}
\setmonofont{CMU Typewriter Text}

\usepackage{braket}
\usepackage{siunitx}
\sisetup{
	separate-uncertainty = true,
	inter-unit-product = \ensuremath{{}\cdot{}}
}
\usepackage{mhchem}
\usepackage{multirow}
\usepackage{booktabs}

\DeclareMathOperator{\p}{\prime}
\DeclareMathOperator{\ti}{\times}
\DeclareMathOperator{\intinf}{\int_0^\infty}
\DeclareMathOperator{\intdinf}{\int_{-\infty}^\infty}
\DeclareMathOperator{\intzpi}{\int_0^\pi}
\DeclareMathOperator{\intztpi}{\int_0^{2\pi}}
\DeclareMathOperator{\sumninf}{\sum_{n=1}^{\infty}}
\DeclareMathOperator{\sumninfz}{\sum_{n=0}^\infty}
\DeclareMathOperator{\sumiinf}{\sum_{i=1}^{\infty}}
\DeclareMathOperator{\sumiinfz}{\sum_{i=0}^\infty}
\DeclareMathOperator{\sumkinf}{\sum_{k=1}^{\infty}}
\DeclareMathOperator{\sumkinfz}{\sum_{k=0}^\infty}
\DeclareMathOperator{\e}{\mathrm{e}}
\DeclareMathOperator{\I}{\mathrm{i}}
\DeclareMathOperator{\Arg}{\mathrm{Arg}}
\DeclareMathOperator{\ra}{\rightarrow}
\DeclareMathOperator{\llra}{\longleftrightarrow}
\DeclareMathOperator{\lra}{\longrightarrow}
\DeclareMathOperator{\dlra}{\Leftrightarrow}
\DeclareMathOperator{\dra}{\Rightarrow}
\newcommand{\bkk}[1]{\Braket{#1|#1}}
\newcommand{\bk}[2]{\Braket{#1|#2}}
\newcommand{\bkev}[2]{\Braket{#2|#1|#2}}



\DeclareMathOperator{\hV}{\hat{\vb{V}}}

\DeclareMathOperator{\hx}{\hat{\vb{x}}}
\DeclareMathOperator{\hy}{\hat{\vb{y}}}
\DeclareMathOperator{\hz}{\hat{\vb{z}}}

\DeclareMathOperator{\hA}{\hat{\vb{A}}}

\DeclareMathOperator{\hQ}{\hat{\vb{Q}}}
\DeclareMathOperator{\hI}{\hat{\vb{I}}}
\DeclareMathOperator{\psis}{\psi^\ast}
\DeclareMathOperator{\Psis}{\Psi^\ast}
\DeclareMathOperator{\hi}{\hat{\vb{i}}}
\DeclareMathOperator{\hj}{\hat{\vb{j}}}
\DeclareMathOperator{\hk}{\hat{\vb{k}}}
\DeclareMathOperator{\hr}{\hat{\vb{r}}}
\DeclareMathOperator{\hT}{\hat{\vb{T}}}
\DeclareMathOperator{\hH}{\hat{H}}
\DeclareMathOperator{\hh}{\hat{h}}               % helicity
\DeclareMathOperator{\hL}{\hat{\vb{L}}}
\DeclareMathOperator{\hp}{\hat{\vb{p}}}

\DeclareMathOperator{\ha}{\hat{\vb{a}}}
\DeclareMathOperator{\hs}{\hat{\vb{s}}}
\DeclareMathOperator{\hS}{\hat{\vb{S}}}
\DeclareMathOperator{\hSigma}{\hat{\bm\Sigma}}
\DeclareMathOperator{\hJ}{\hat{\vb{J}}}
\DeclareMathOperator{\hP}{\hat{\vb{P}}}          % Parity
\DeclareMathOperator{\hC}{\hat{\vb{C}}} 
\DeclareMathOperator{\Tdv}{-\dfrac{\hbar^2}{2m}\dv[2]{x}}
\DeclareMathOperator{\Tna}{-\dfrac{\hbar^2}{2m}\nabla^2}
\DeclareMathOperator{\vna}{\vnabla}
\DeclareMathOperator{\nna}{\nabla^2}
\newcommand{\naCarExpd}[1]{\pdv[2]{#1}{x} + \pdv[2]{#1}{y} + \pdv[2]{#1}{z}}
\newcommand{\naCyl}{\qty[\dfrac{1}{\rho}\pdv{\rho}\qty(\rho\pdv{\rho}) + \dfrac{1}{\rho^2}\pdv[2]{\phi} + \pdv[2]{z}]}

%\DeclareMathOperator{\g#0}{\gamma^0}
%\DeclareMathOperator{\g1}{\gamma^1}
%\DeclareMathOperator{\g2}{\gamma^2}
%\DeclareMathOperator{\g3}{\gamma^3}
%\DeclareMathOperator{\g5}{\gamma^5}
\newcommand{\g}[1]{\gamma^{#1}}
\DeclareMathOperator{\gmuu}{\gamma^\mu}
\DeclareMathOperator{\gmud}{\gamma_\mu}
\newcommand{\G}[2]{g^{#1#2}}


%% MQC
\DeclareMathOperator{\sH}{\mathscr{H}}
\DeclareMathOperator{\sA}{\mathscr{A}}
\newcommand{\iden}{{\large \bm{1}}}
\newcommand{\qed}{$ \Square $}
\newcommand{\tPhi}{\tilde{\Phi} }
\newcommand{\hsP}{\hat{\mathscr{P}}}
\newcommand{\hsS}{\hat{\mathscr{S}}}
\DeclareMathOperator{\core}{\mathrm{core}}
\DeclareMathOperator{\GF}{\mathrm{GF}}
\DeclareMathOperator{\SF}{\mathrm{SF}}
\DeclareMathOperator{\corr}{\mathrm{corr}}
\DeclareMathOperator{\gvb}{\mathrm{GVB}}


\newcommand{\subsbul}{\subsection*{$ \bullet $}}
\newcommand{\ex}[1]{\paragraph{Ex #1}}
\newcommand{\subex}[1]{\subparagraph{#1}}
\newcommand{\dis}{\displaystyle}


\numberwithin{equation}{subsection}
%\setcounter{secnumdepth}{4}
\setcounter{tocdepth}{4}
\allowdisplaybreaks[1]

\usepackage{xcolor}
\definecolor{codegray}{gray}{0.9}
\newfontfamily\Consolas{Consolas}
\newcommand{\code}[1]{\colorbox{codegray}{{\Consolas#1}}}

%\usepackage{tikz}
%\usepackage[european]{circuitikz}
\usepackage{tikz-feynman}
%\usetikzlibrary{arrows.meta, bending, positioning}

\title{\textbf{Modern Quantum Chemistry, Szabo \& Ostlund}\\HW}
\author{wsr
\vspace{5pt}\\
}
\date{\today} % Activate to display a given date or no date (if empty),
         % otherwise the current date is printed 

\begin{document}
% \boldmath

\maketitle

\tableofcontents

\newpage

\setcounter{section}{6}

\section{The 1-Particle Many-body Green's Function}
\subsection{Green's Function in Single-Particle Systems}
\ex{7.1}
\begin{equation}\label{key}
\vb{V} = \vb{G}_0(E)^{-1} - \vb{G}(E)^{-1}
\end{equation}
thus
\begin{align}
\vb{G}_0(E)\vb{V}\vb{G}(E) 
&= \vb{G}_0(E) [\vb{G}_0(E)^{-1} - \vb{G}(E)^{-1}] \vb{G}(E) \notag\\
&= \vb{G}(E) - \vb{G}_0(E)
\end{align}
i.e.
\begin{equation}\label{key}
\vb{G}(E) = \vb{G}_0(E) + \vb{G}_0(E)\vb{V}\vb{G}(E) 
\end{equation}

\ex{7.2}
\subex{a.}
When $ x=0 $,
\begin{align}
\dv[2]{x} \abs{x} \Bigg|_{x=0} &= \lim_{\epsilon\ra 0} \dfrac{\dv{\abs{x}}{x}\Big|_{x=\epsilon} - \dv{\abs{x}}{x}\Big|_{x=-\epsilon}}{2\epsilon} \qquad (\epsilon>0)\notag\\
&= \lim_{\epsilon\ra 0} \dfrac{1 - (-1)}{2\epsilon} \notag\\
&= \infty
\end{align}
otherwise,
\begin{align}
\dv[2]{x} \abs{x}  &= \dv[2]{x} [x \;\mathrm{sgn}(x)] \notag\\
&= \dv{x} [1\times \mathrm{sgn}(x) + x\times 0] \notag\\
&= 0
\end{align}
\subex{b.}
\begin{align}\label{key}
\intdinf \dv[2]{x} \abs{x} \dd x &= \intdinf \dd(\dv{x} \abs{x}) \notag\\
&= \dv{x} \abs{x} \Bigg|_{-\infty}^{\infty} \notag\\
&= 1 - (-1) \notag\\
&= 2
\end{align}
thus
\begin{equation}\label{key}
\dv[2]{x} \abs{x} = 2\delta(x)
\end{equation}
\subex{c.}
\begin{align}
\dv[2]{x} a(x) &= \dv[2]{x} \dfrac{1}{2}\int_\alpha^\beta \dd x' 
\abs{x - x'} b(x') \notag\\
&= \dv[2]{x} \dfrac{1}{2}\int_\alpha^x \dd x' 
(x - x') b(x') + \dv[2]{x} \dfrac{1}{2}\int_x^\beta \dd x' 
[-(x - x')] b(x') \notag\\
&= \dv{x} \dfrac{1}{2}\int_\alpha^x \dd x' 
 b(x') - \dv{x} \dfrac{1}{2}\int_x^\beta \dd x' b(x') \notag\\
&= \dfrac{1}{2}b(x) - \dfrac{1}{2} [-b(x)] \notag\\
&= b(x)
\end{align}

\ex{7.3}

\begin{equation}\label{key}
\hskip-1cm
\begin{aligned}
\qty(E + \dfrac{1}{2}\dv[2]{x}) G_0(x,x',E) 
&= \qty(E + \dfrac{1}{2}\dv[2]{x}) \dfrac{1}{\I(2E)^{1/2}} \e^{\I(2E)^{1/2}\abs{x-x'}} \\
&=  \dfrac{E}{\I(2E)^{1/2}} \e^{\I(2E)^{1/2}\abs{x-x'}} +  \dfrac{1}{2}\dfrac{1}{\I(2E)^{1/2}} \dv[2]{x} \e^{\I(2E)^{1/2}\abs{x-x'}} \\
&= \dfrac{E}{\I(2E)^{1/2}} \e^{\I(2E)^{1/2}\abs{x-x'}} +  \dfrac{1}{2}\dfrac{1}{\I(2E)^{1/2}} \dv{x} \qty [\e^{\I(2E)^{1/2}\abs{x-x'}} \I(2E)^{1/2}\dv{x}\abs{x-x'} ] \\
&= \dfrac{E}{\I(2E)^{1/2}} \e^{\I(2E)^{1/2}\abs{x-x'}} 
+ \dfrac{1}{2} 
\qty [\e^{\I(2E)^{1/2}\abs{x-x'}}\I(2E)^{1/2} \qty(\dv{x}\abs{x-x'} )^2
+ \e^{\I(2E)^{1/2}\abs{x-x'}} \dv[2]{x}\abs{x-x'}] \\
&= \dfrac{E}{\I(2E)^{1/2}} \e^{\I(2E)^{1/2}\abs{x-x'}} 
+ \dfrac{1}{2} \e^{\I(2E)^{1/2}\abs{x-x'}}
\qty [\I(2E)^{1/2} \times 1
+ 2\delta(x-x')] \\
&= \e^{\I(2E)^{1/2}\abs{x-x'}}
\qty [\dfrac{E}{\I(2E)^{1/2}} + \dfrac{-E}{\I(2E)^{1/2}} + \delta(x-x')] \\
&= \e^{\I(2E)^{1/2}\abs{x-x'}} \delta(x-x') \\
&= \delta(x-x') 
\end{aligned}
%\qquad
\end{equation}

\ex{7.4}
\begin{align}
\phi_n(x)\phi_n^*(x') &= \lim_{E\ra E_n} (E - E_n) \dfrac{1}{\I(2E)^{1/2}} \qty[\e^{\I(2E)^{1/2}\abs{x-x'}} 
- \dfrac{\e^{\I(2E)^{1/2}(\abs{x} + \abs{x'})}}{1 + \I(2E)^{1/2}}] \notag\\
&= \lim_{E\ra -1/2} (E + 1/2) \dfrac{1}{-1} \qty[\e^{-\abs{x-x'}} 
- \dfrac{\e^{-(\abs{x} + \abs{x'})}}{1 + \I(2E)^{1/2}}] \notag\\
&= -\lim_{E\ra -1/2} (E + 1/2) \e^{-\abs{x-x'}} 
+ \lim_{E\ra -1/2}(E + 1/2)\dfrac{\e^{-(\abs{x} + \abs{x'})}}{1 + \I(2E)^{1/2}} \notag\\
&= 0 + \lim_{E\ra -1/2}(E + 1/2)\dfrac{\e^{-(\abs{x} + \abs{x'})}(1 - \I(2E)^{1/2})}{(1 + \I(2E)^{1/2})(1 - \I(2E)^{1/2})} \notag\\
&= \lim_{E\ra -1/2}(E + 1/2)\dfrac{\e^{-(\abs{x} + \abs{x'})}(1 - \I(2E)^{1/2})}{1 + 2E} \notag\\
&= \dfrac{1}{2}\e^{-(\abs{x} + \abs{x'})}(1 - (-1)) \notag\\
&= \e^{-(\abs{x} + \abs{x'})}
\end{align}
Let $ x=x' $,
\begin{equation}\label{key}
\phi_n^2(x) = \e^{-2\abs{x}}
\end{equation}
thus
\begin{equation}\label{key}
\phi_n(x) = \e^{-\abs{x}}
\end{equation}

\ex{7.5}
\begin{align}
\sH \phi &= \qty[-\dfrac{1}{2}\dv[2]{x} - \delta(x)]\e^{-\abs{x}} \notag\\
&= -\dfrac{1}{2}\dv{x}\qty[\e^{-\abs{x}}\qty(-\dv{x}\abs{x})] - \delta(x)\e^{-\abs{x}} \notag\\
&= \dfrac{1}{2}\qty[-\e^{-\abs{x}}(\dv{x}\abs{x})^2 + \e^{-\abs{x}}\dv[2]{x}\abs{x}] - \delta(x)\e^{-\abs{x}} \notag\\
&= \dfrac{1}{2}\qty[-\e^{-\abs{x}} + \e^{-\abs{x}}\times 2\delta(x)] - \delta(x)\e^{-\abs{x}} \notag\\
&= -\dfrac{1}{2}\e^{-\abs{x}} 
\end{align}
thus the eigenvalue is $ -\dfrac{1}{2} $.

\ex{7.6}
\subex{a.}
\begin{align}
\I\pdv{t}\phi(x,t) &= \I\int\dd x' \pdv{G(x,x',t)}{t} \psi(x') \notag\\
&= \int\dd x' \sH G(x,x',t) \psi(x') \notag\\
&= \sH \phi(x,t)
\end{align}
\subex{b.}
From
\begin{equation}\label{key}
\I\pdv{G(x,x',t)}{t} = \sH G(x,x',t)
\end{equation}
we get
\begin{equation}\label{key}
\lim_{\varepsilon\ra 0} 
\intinf\dd t \I\pdv{G(x,x',t)}{t} [-\I\e^{(\I E - \varepsilon)t}] 
= \lim_{\varepsilon\ra 0} 
 \intinf\dd t \sH G(x,x',t)[-\I\e^{(\I E - \varepsilon)t}]
\end{equation}
\begin{align}\label{key}
\lim_{\varepsilon\ra 0}  \intinf\dd t \pdv{G(x,x',t)}{t} \e^{(\I E - \varepsilon)t}
&= \intinf\dd t \sH G(x,x',t) [-\I\e^{\I E t}] \notag\\
&= \sH G(x,x',E)
\end{align}
thus
\begin{align}\label{key}
\lim_{\varepsilon\ra 0} 
\qty[G(x,x',t)\e^{(\I E - \varepsilon)t} \Bigg|_{t=0}^\infty 
- \intinf\dd t G(x,x',t) \e^{(\I E - \varepsilon)t} (\I E - \varepsilon) ] 
= \sH G(x,x',E)
\end{align}
\begin{align}
\sH G(x,x',E) &= -G(x,x',0) 
- \I E \intinf\dd t G(x,x',t) \e^{\I E t}   \notag\\
&= -G(x,x',0) - \I E G(x,x',E) / (-\I)  \notag\\
&= -\delta(x-x') + E G(x,x',E) 
\end{align}
$ \therefore $
\begin{equation}\label{key}
(E - \sH)G(x,x',E) = \delta(x-x')
\end{equation}

\subex{c.}
\begin{align}
\I \pdv{t}\mathscr{G}(t) &= \I \pdv{t}\e^{-\I\sH t} \notag\\
&= \I \e^{-\I\sH t}(-\I\sH) \notag\\
&= \sH \mathscr{G}(t)
\end{align}
\begin{align}
\lim_{\varepsilon\ra 0} \intinf\dd t \e^{(\I E - \varepsilon)t} \I \pdv{t}\mathscr{G}(t) = \lim_{\varepsilon\ra 0} \intinf\dd t \e^{(\I E - \varepsilon)t} \sH \mathscr{G}(t)
\end{align}
\begin{align}
\lim_{\varepsilon\ra 0} 
\qty[\e^{(\I E - \varepsilon)t}\mathscr{G}(t) \Bigg|_0^\infty 
- (\I E - \varepsilon)\intinf\dd t \e^{(\I E - \varepsilon)t} \mathscr{G}(t) ] 
= \sH\mathscr{G}(E)
\end{align}
$ \therefore $
\begin{align}
\sH\mathscr{G}(E) &=\lim_{\varepsilon\ra 0} 
\qty[-\mathscr{G}(0)
- (\I E - \varepsilon)\intinf\dd t \e^{(\I E - \varepsilon)t} \mathscr{G}(t) ] \notag\\
&= -\mathscr{G}(0) + E \mathscr{G}(E) \notag\\
&= -1 + E \mathscr{G}(E) 
\end{align}
thus
\begin{equation}\label{key}
\mathscr{G}(E) = \dfrac{1}{E - \sH}
\end{equation}

\subsection{The 1-Particle Many-body Green's Function}
\subsubsection{The Self-Energy}
\ex{7.7}
\begin{align}
\Sigma_{ij}^{(2)}(E) &= \dfrac{1}{2} \sum_{ars}\dfrac{\Braket{rs||ia}\Braket{ja||rs}}{E+\varepsilon_a -\varepsilon_r-\varepsilon_s} 
+ \dfrac{1}{2} \sum_{abr}\dfrac{\Braket{ab||ir}\Braket{jr||ab}}{E+\varepsilon_r -\varepsilon_a-\varepsilon_b} \notag\\
&= \dfrac{1}{2} \sum_{ars} 
\dfrac{(\Braket{rs|ia} - \Braket{rs|ai}) (\Braket{ja|rs} - \Braket{ja|sr})
}{E+\varepsilon_a -\varepsilon_r-\varepsilon_s} 
+ \dfrac{1}{2} \sum_{abr} 
\dfrac{(\Braket{ab|ir} - \Braket{ab|ri}) (\Braket{jr|ab} - \Braket{jr|ba}) }{E+\varepsilon_r -\varepsilon_a-\varepsilon_b} %\notag\\
%&= 
\end{align}
%For convenience, suppose $ i $'s spin is $ \alpha $
In the 1st summation: \\
To make the terms non-zero, the spin of $ r $ is fixed in the first and last term, and $ r,s,a $ are all fixed in the second and third term, thus
\begin{align}
\text{the 1st term} &= \dfrac{1}{2} \sum_{ars}^{N/2}
\dfrac{1}{E+\varepsilon_a -\varepsilon_r-\varepsilon_s} \qty[2\Braket{rs|ia}\Braket{ja|rs} - \Braket{rs|ai}\Braket{ja|rs} 
- \Braket{rs|ia}\Braket{ja|sr} + 2\Braket{rs|ai}\Braket{ja|sr}] \notag\\
&= \sum_{ars}^{N/2} 
\dfrac{1}{E+\varepsilon_a -\varepsilon_r-\varepsilon_s} \qty[2\Braket{rs|ia}\Braket{ja|rs} - \Braket{rs|ia}\Braket{ja|sr} ] \notag\\
&= \sum_{ars}^{N/2} 
\dfrac{\Braket{rs|ia} \qty[2\Braket{ja|rs} - \Braket{aj|rs}] }{E+\varepsilon_a -\varepsilon_r-\varepsilon_s} 
\end{align}
Similarly,
\begin{align}
\Sigma_{ij}^{(2)}(E) = \sum_{ars}^{N/2} 
\dfrac{\Braket{rs|ia} \qty[2\Braket{ja|rs} - \Braket{aj|rs}] }{E+\varepsilon_a -\varepsilon_r-\varepsilon_s} 
+ \sum_{abr}^{N/2} 
\dfrac{\Braket{ab|ir} \qty[2\Braket{jr|ab} - \Braket{rj|ab}]
}{E+\varepsilon_r -\varepsilon_a-\varepsilon_b} 
\end{align}

\ex{7.8}
\begin{align}
[\vb{G}_0(E)]_{ij} &= \sum_m 
\dfrac{\Braket{^N\Psi_0 | a_i^\dagger a_m | ^N\Psi_0} 
	\Braket{a_m {^N\Psi}_0| a_j | ^N\Psi_0}
}{E - \qty(\Braket{^N\Psi_0|\sH |^N\Psi_0} 
- \Braket{a_m {^N\Psi}_0|\sH|a_m {^N\Psi}_0} )}
+ \sum_p
\dfrac{\Braket{^N\Psi_0 | a_j a_p^\dagger | ^N\Psi_0} 
	\Braket{a_p^\dagger {^N\Psi}_0| a_i^\dagger | ^N\Psi_0}
}{E + \qty(\Braket{^N\Psi_0|\sH |^N\Psi_0} 
	- \Braket{a_p^\dagger {^N\Psi}_0|\sH|a_p^\dagger {^N\Psi}_0} )} \notag\\
%&= \sum_m 
%\dfrac{\Braket{^N\Psi_0 | a_i^\dagger a_m | ^N\Psi_0} 
%	\Braket{ {^N\Psi}_0|a_m^\dagger a_j | ^N\Psi_0}
%}{E - \qty(\Braket{^N\Psi_0|\sH |^N\Psi_0} 
%	- \Braket{ {^N\Psi}_0|a_m^\dagger\sH a_m| {^N\Psi}_0} )}
%+ \sum_p
%\dfrac{\Braket{^N\Psi_0 | a_j a_p^\dagger | ^N\Psi_0} 
%	\Braket{ {^N\Psi}_0| a_p a_i^\dagger | ^N\Psi_0}
%}{E + \qty(\Braket{^N\Psi_0|\sH |^N\Psi_0} 
%	- \Braket{{^N\Psi}_0|a_p \sH a_p^\dagger| {^N\Psi}_0} )} \notag\\
&= \sum_m \dfrac{\delta_{im}\delta_{mj}}{E - \varepsilon_m} + 0 \notag\\
&= \sum_m \dfrac{\delta_{ij}}{E - \varepsilon_m}
\end{align}

\subsubsection{The Solution of the Dyson Equation}

\subsection{Application of the Formalism to \ce{H_2} and \ce{HeH^+}}
\ex{7.9}
\subex{a.}



\ex{7.10}



\ex{7.11}



\ex{7.12}



\subsection{Perturbation Theory and the Green's Function Method}
\ex{7.13}



\ex{7.14}



\ex{7.15}


\subsection{Some Illustrative Calculations}
\ex{7.16}




\end{document}
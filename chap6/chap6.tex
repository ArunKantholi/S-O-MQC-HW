%\documentclass[UTF8]{ctexart} % use larger type; default would be 10pt
\documentclass[a4paper]{article}
\usepackage{../mqc}

% TikZFeynman
\newcommand{\qr}{quarter right}
\newcommand{\ql}{quarter left}
\newcommand{\hfr}{half right}
\newcommand{\hfl}{half left}
\newcommand{\el}{edge label}

%\usepackage{tikz}
%\usepackage[european]{circuitikz}
\usepackage{tikz-feynman}
%\usetikzlibrary{arrows.meta, bending, positioning}

\title{\textbf{Modern Quantum Chemistry, Szabo \& Ostlund}\\HW}
\author{wsr
\vspace{5pt}\\
}
\date{\today} % Activate to display a given date or no date (if empty),
         % otherwise the current date is printed 

\begin{document}
% \boldmath

\maketitle

\tableofcontents

\newpage

\setcounter{section}{5}

\section{Many-body Perturbation Theory}
\subsection{RS Perturbation Theory}

\subsection{Diagrammatic Representation of RS Perturbation Theory}
\subsubsection{Diagrammatic Perturbation Theory for Two States}
\ex{6.1}
~\\
\tikzfeynmanset{
every vertex={black,dot},
}


\begin{minipage}{0.3\linewidth}
	\hfill
	\begin{tikzpicture}
	\begin{feynman}[small]
	\vertex (a);
	\vertex [below=of a] (b);
	\vertex [below=of b] (c);
	\vertex [below=of c] (d);
	\vertex [below=of d] (e);
	\diagram[inline=(c)]{
		(a) -- [fermion, quarter right, edge label'=1] (b)
		-- [fermion, quarter right, edge label'=1] (c)
		-- [fermion, quarter right, edge label'=1] (d)
		-- [fermion, quarter right, edge label'=1] (e)
		-- [fermion, quarter right, edge label'=2] (a),
	};
	\end{feynman}
	\end{tikzpicture}
\end{minipage}
\begin{minipage}{0.7\linewidth}
	$ = (-1)^5 \dfrac{V_{12}V_{21}V_{11}^3}{(E_1^{(0)} - E_2^{(0)})^4} = -\dfrac{V_{12}V_{21}V_{11}^3}{(E_1^{(0)} - E_2^{(0)})^4} $\\
\end{minipage}
~\vspace{15pt}\\
\begin{minipage}{0.3\linewidth}
	\hfill
	\begin{tikzpicture}
	\begin{feynman}[small]
	\vertex (a);
	\vertex [below=of a] (b);
	\vertex [below=of b] (c);
	\vertex [below=of c] (d);
	\vertex [below=of d] (e);
	\diagram[inline=(c)]{
		(e) -- [fermion, quarter left, edge label=2] (d)
		-- [fermion, quarter left, edge label=2] (c)
		-- [fermion, quarter left, edge label=2] (b)
		-- [fermion, quarter left, edge label=2] (a)
		-- [fermion, quarter left, edge label=1] (e),
	};
	\end{feynman}
	\end{tikzpicture}
\end{minipage}
\begin{minipage}{0.7\linewidth}
	$ = (-1)^2 \dfrac{V_{12}V_{21}V_{22}^3}{(E_1^{(0)} - E_2^{(0)})^4} = \dfrac{V_{12}V_{21}V_{22}^3}{(E_1^{(0)} - E_2^{(0)})^4} $\\
\end{minipage}

~\vspace{15pt}\\
\begin{minipage}{0.3\linewidth}
	\hfill
	\begin{tikzpicture}
	\begin{feynman}[small]
	\vertex (a);
	\vertex [below=of a] (b);
	\vertex [below=of b] (c);
	\vertex [below=of c] (d);
	\vertex [below=of d] (e);
	\diagram[inline=(c)]{
		(a) -- [fermion, quarter right, edge label'=1] (b)
		-- [fermion, quarter right, edge label'=1] (d)
		-- [fermion, quarter right, edge label'=1] (e)
		-- [fermion, quarter right, edge label'=2] (c)
		-- [fermion, quarter right, edge label'=2] (a),
	};
	\end{feynman}
	\end{tikzpicture}
\end{minipage}
\begin{minipage}{0.7\linewidth}
	$ = (-1)^4 \dfrac{V_{12}V_{21}V_{11}^2 V_{22}}{(E_1^{(0)} - E_2^{(0)})^4} = \dfrac{V_{12}V_{21}V_{11}^2 V_{22}}{(E_1^{(0)} - E_2^{(0)})^4} $\\
\end{minipage}

~\vspace{15pt}\\
\begin{minipage}{0.3\linewidth}
	\hfill
	\begin{tikzpicture}
	\begin{feynman}[small]
	\vertex (a);
	\vertex [below=of a] (b);
	\vertex [below=of b] (c);
	\vertex [below=of c] (d);
	\vertex [below=of d] (e);
	\diagram[inline=(c)]{
		(a) -- [fermion, quarter left, edge label=1] (c)
		-- [fermion, quarter left, edge label=1] (e)
		-- [fermion, quarter left, edge label=2] (d)
		-- [fermion, quarter left, edge label=2] (b)
		-- [fermion, quarter left, edge label=2] (a),
	};
	\end{feynman}
	\end{tikzpicture}
\end{minipage}
\begin{minipage}{0.7\linewidth}
	$ = (-1)^3 \dfrac{V_{12}V_{21}V_{11} V_{22}^2}{(E_1^{(0)} - E_2^{(0)})^4} = -\dfrac{V_{12}V_{21}V_{11} V_{22}^2}{(E_1^{(0)} - E_2^{(0)})^4} $\\
\end{minipage}

\newpage
Similarly,\\
%~\vspace{15pt}\\
\begin{minipage}{0.2\linewidth}
	\hfill
	\begin{tikzpicture}
	\begin{feynman}[small]
	\vertex (a);
	\vertex [below=of a] (b);
	\vertex [below=of b] (c);
	\vertex [below=of c] (d);
	\vertex [below=of d] (e);
	\diagram[inline=(c)]{
		(a) -- [fermion, quarter right, edge label'=1] (c)
		-- [fermion, quarter right, edge label'=1] (d)
		-- [fermion, quarter right, edge label'=1] (e)
		-- [fermion, quarter right, edge label'=2] (b)
		-- [fermion, quarter right, edge label'=2] (a),
	};
	\end{feynman}
	\end{tikzpicture}
\end{minipage}
,
\begin{minipage}{0.15\linewidth}
	%\hfill
	\begin{tikzpicture}
	\begin{feynman}[small]
	\vertex (a);
	\vertex [below=of a] (b);
	\vertex [below=of b] (c);
	\vertex [below=of c] (d);
	\vertex [below=of d] (e);
	\diagram[inline=(c)]{
		(a) -- [fermion, quarter right, edge label'=1] (b)
		-- [fermion, quarter right, edge label'=1] (c)
		-- [fermion, quarter right, edge label'=1] (e)
		-- [fermion, quarter right, edge label'=2] (d)
		-- [fermion, quarter right, edge label'=2] (a),
	};
	\end{feynman}
	\end{tikzpicture}
\end{minipage}
\begin{minipage}{0.6\linewidth}
	$ = \dfrac{V_{12}V_{21}V_{11}^2 V_{22}}{(E_1^{(0)} - E_2^{(0)})^4} $\\
\end{minipage}

~\vspace{15pt}\\
\begin{minipage}{0.2\linewidth}
	\hfill
	\begin{tikzpicture}
	\begin{feynman}[small]
	\vertex (a);
	\vertex [below=of a] (b);
	\vertex [below=of b] (c);
	\vertex [below=of c] (d);
	\vertex [below=of d] (e);
	\diagram[inline=(c)]{
		(a) -- [fermion, quarter left, edge label=1] (b)
		-- [fermion, quarter left, edge label=1] (e)
		-- [fermion, quarter left, edge label=2] (d)
		-- [fermion, quarter left, edge label=2] (c)
		-- [fermion, quarter left, edge label=2] (a),
	};
	\end{feynman}
	\end{tikzpicture}
\end{minipage}
,
\begin{minipage}{0.15\linewidth}
	\begin{tikzpicture}
	\begin{feynman}[small]
	\vertex (a);
	\vertex [below=of a] (b);
	\vertex [below=of b] (c);
	\vertex [below=of c] (d);
	\vertex [below=of d] (e);
	\diagram[inline=(c)]{
		(a) -- [fermion, quarter left, edge label=1] (d)
		-- [fermion, quarter left, edge label=1] (e)
		-- [fermion, quarter left, edge label=2] (c)
		-- [fermion, quarter left, edge label=2] (b)
		-- [fermion, quarter left, edge label=2] (a),
	};
	\end{feynman}
	\end{tikzpicture}
\end{minipage}
\begin{minipage}{0.6\linewidth}
	$ = -\dfrac{V_{12}V_{21}V_{11} V_{22}^2}{(E_1^{(0)} - E_2^{(0)})^4} $\\
\end{minipage}
~\vspace{10pt}\\
thus, the sum of above terms is
\begin{align}
\dfrac{V_{12}V_{21}(V_{22}^3 - V_{11}^3)}{(E_1^{(0)} - E_2^{(0)})^4} + 3\times\dfrac{V_{12}V_{21}(V_{11}^2 V_{22} - V_{11}V_{22}^2)}{(E_1^{(0)} - E_2^{(0)})^4} 
= \dfrac{V_{12}V_{21}(V_{22} - V_{11})^3}{(E_1^{(0)} - E_2^{(0)})^4} 
\end{align}

\subsubsection{Diagrammatic Perturbation Theory for $ N $ States}
\ex{6.2}
The 4th-order perturbation energy of state $ i $ can be expressed as
\begin{align}
&\sum_{k,n,m\neq i} \dfrac{V_{ki}V_{nk}V_{mn}V_{im}}{(E_i^{(0)}-E_k^{(0)})(E_i^{(0)}-E_n^{(0)})(E_i^{(0)}-E_m^{(0)})} + \sum_{n\neq i} \dfrac{V_{ii}^2 V_{ni}V_{in}}{(E_i^{(0)}-E_n^{(0)})^3} - \sum_{m,n\neq i} \dfrac{V_{ii}V_{mi}V_{in}V_{nm}}{(E_i^{(0)}-E_m^{(0)})^2(E_i^{(0)}-E_n^{(0)})} \notag\\
&- \sum_{m,n\neq i} \dfrac{V_{ii}V_{ni}V_{im}V_{mn}}{(E_i^{(0)}-E_m^{(0)})^2(E_i^{(0)}-E_n^{(0)})} - \sum_{m,n\neq i} \dfrac{V_{mi}V_{im}V_{in}V_{ni}}{(E_i^{(0)}-E_m^{(0)})(E_i^{(0)}-E_n^{(0)})(2E_i^{(0)}-E_n^{(0)}-E_m^{(0)})} \notag\\
&- \sum_{m,n\neq i} \dfrac{V_{mi}V_{im}V_{in}V_{ni}}{(E_i^{(0)}-E_n^{(0)})^2(2E_i^{(0)}-E_n^{(0)}-E_m^{(0)})} \notag\\
= &\sum_{k,n,m\neq i} \dfrac{V_{ki}V_{nk}V_{mn}V_{im}}{(E_i^{(0)}-E_k^{(0)})(E_i^{(0)}-E_n^{(0)})(E_i^{(0)}-E_m^{(0)})} + \sum_{n\neq i} \dfrac{V_{ii}^2 V_{ni}V_{in}}{(E_i^{(0)}-E_n^{(0)})^3} - 2\sum_{m,n\neq i} \dfrac{V_{ii}V_{mi}V_{in}V_{nm}}{(E_i^{(0)}-E_m^{(0)})^2(E_i^{(0)}-E_n^{(0)})} \notag\\
&%- \sum_{m,n\neq i} \dfrac{V_{ii}V_{ni}V_{im}V_{mn}}{(E_i^{(0)}-E_m^{(0)})^2(E_i^{(0)}-E_n^{(0)})} 
- \sum_{m,n\neq i} \dfrac{V_{mi}V_{im}V_{in}V_{ni}}{(E_i^{(0)}-E_m^{(0)})(E_i^{(0)}-E_n^{(0)})^2} 
\end{align}
while
\begin{equation}\label{key}
\Braket{n | \sH | \Psi_i^{(3)}} + \Braket{n | \mathscr{V} | \Psi_i^{(2)}} = E_i^{(0)} \Braket{n|\Psi_i^{(3)}} + E_i^{(1)} \Braket{n|\Psi_i^{(2)}} + E_i^{(2)} \Braket{n|\Psi_i^{(1)}}
\end{equation}
\begin{align}\label{key}
(E_i^{(0)} - E_n^{(0)})\Braket{n | \Psi_i^{(3)}} 
&= \Braket{n | \mathscr{V} | \Psi_i^{(2)}} - E_i^{(1)} \Braket{n|\Psi_i^{(2)}} - E_i^{(2)} \Braket{n|\Psi_i^{(1)}} \notag\\
&= \Braket{n | \mathscr{V} | \Psi_i^{(2)}} 
- E_i^{(1)} \dfrac{\Braket{n | \mathscr{V} | \Psi_i^{(1)}} - E_i^{(1)}\Braket{n|\Psi_i^{(1)}}}{E_i^{(0)} - E_n^{(0)}} 
- E_i^{(2)} \Braket{n|\Psi_i^{(1)}} \notag\\
&= \Braket{n | \mathscr{V} | \Psi_i^{(2)}} 
- E_i^{(1)} \dfrac{\Braket{n | \mathscr{V} | \Psi_i^{(1)}}}{E_i^{(0)} - E_n^{(0)}} 
+ \qty[E_i^{(1)}]^2 \dfrac{\Braket{n|\mathscr{V}|i}}{\qty[E_i^{(0)} - E_n^{(0)}]^2} 
- E_i^{(2)} \dfrac{\Braket{n|\mathscr{V}|i}}{E_i^{(0)} - E_n^{(0)}} 
\end{align}
\begin{align}
E_i^{(4)} &= \Braket{i | \mathscr{V} | \Psi_i^{(3)}} \notag\\
&= \sum_{n\neq i} \dfrac{\Braket{i | \mathscr{V} | n}}{E_i^{(0)} - E_n^{(0)}} \qty{\Braket{n | \mathscr{V} | \Psi_i^{(2)}} 
- E_i^{(1)} \dfrac{\Braket{n | \mathscr{V} | \Psi_i^{(1)}}}{E_i^{(0)} - E_n^{(0)}} 
+ \qty[E_i^{(1)}]^2 \dfrac{\Braket{n|\mathscr{V}|i}}{\qty[E_i^{(0)} - E_n^{(0)}]^2} 
- E_i^{(2)} \dfrac{\Braket{n|\mathscr{V}|i}}{E_i^{(0)} - E_n^{(0)}} }\notag\\
&= \sum_{n\neq i} \dfrac{\Braket{i | \mathscr{V} | n}}{E_i^{(0)} - E_n^{(0)}} \Braket{n | \mathscr{V} | \Psi_i^{(2)}} 
- E_i^{(1)} \sum_{n\neq i} \dfrac{\Braket{i | \mathscr{V} | n}}{\qty[E_i^{(0)} - E_n^{(0)}]^2} \Braket{n | \mathscr{V} | \Psi_i^{(1)}} \notag\\
&{}\quad +  \qty[E_i^{(1)}]^2 \sum_{n\neq i} \dfrac{V_{in}V_{ni}}{\qty[E_i^{(0)} - E_n^{(0)}]^3} 
-  E_i^{(2)} \sum_{n\neq i} \dfrac{V_{in}V_{ni}}{\qty[E_i^{(0)} - E_n^{(0)}]^2} \notag\\
&= \sum_{n,m\neq i} \dfrac{\Braket{i | \mathscr{V} | n}}{E_i^{(0)} - E_n^{(0)}} \Braket{n | \mathscr{V} | m} \Braket{m |\Psi_i^{(2)}} 
- E_i^{(1)} \sum_{n,m\neq i} \dfrac{\Braket{i | \mathscr{V} | n}\Braket{n | \mathscr{V} | m}\Braket{m | \mathscr{V} | i} }{\qty[E_i^{(0)} - E_n^{(0)}]^2\qty[E_i^{(0)} - E_m^{(0)}]}  \notag\\
&{}\quad +  \qty[E_i^{(1)}]^2 \sum_{n\neq i} \dfrac{V_{in}V_{ni}}{\qty[E_i^{(0)} - E_n^{(0)}]^3} 
-  E_i^{(2)} \sum_{n\neq i} \dfrac{V_{in}V_{ni}}{\qty[E_i^{(0)} - E_n^{(0)}]^2} \notag\\
&= \sum_{n,m\neq i} \dfrac{V_{in} V_{nm}}{E_i^{(0)} - E_n^{(0)}} \dfrac{\Braket{m | \mathscr{V} | \Psi_i^{(1)}} - E_i^{(1)}\Braket{m|\Psi_i^{(1)}}}{E_i^{(0)} - E_m^{(0)}} 
- E_i^{(1)} \sum_{n,m\neq i} \dfrac{V_{in}V_{nm}V_{mi} }{\qty[E_i^{(0)} - E_n^{(0)}]^2\qty[E_i^{(0)} - E_m^{(0)}]}  \notag\\
&{}\quad +  \qty[E_i^{(1)}]^2 \sum_{n\neq i} \dfrac{V_{in}V_{ni}}{\qty[E_i^{(0)} - E_n^{(0)}]^3} 
-  E_i^{(2)} \sum_{n\neq i} \dfrac{V_{in}V_{ni}}{\qty[E_i^{(0)} - E_n^{(0)}]^2} \notag\\
&= \sum_{n,m,k\neq i} \dfrac{V_{in} V_{nm}}{E_i^{(0)} - E_n^{(0)}} \dfrac{\Braket{m | \mathscr{V} | k} \Braket{k| \mathscr{V} | i} }{\qty[E_i^{(0)} - E_m^{(0)}]\qty[E_i^{(0)} - E_k^{(0)}]} 
- E_i^{(1)}\sum_{n,m\neq i} \dfrac{V_{in} V_{nm}}{E_i^{(0)} - E_n^{(0)}} \dfrac{ \Braket{m| \mathscr{V} |i}}{\qty[E_i^{(0)} - E_m^{(0)}]^2} \notag\\
&{}\quad- E_i^{(1)} \sum_{n,m\neq i} \dfrac{V_{in}V_{nm}V_{mi} }{\qty[E_i^{(0)} - E_n^{(0)}]^2\qty[E_i^{(0)} - E_m^{(0)}]} 
 +  \qty[E_i^{(1)}]^2 \sum_{n\neq i} \dfrac{V_{in}V_{ni}}{\qty[E_i^{(0)} - E_n^{(0)}]^3} 
-  E_i^{(2)} \sum_{n\neq i} \dfrac{V_{in}V_{ni}}{\qty[E_i^{(0)} - E_n^{(0)}]^2} \notag\\
&= \sum_{n,m,k\neq i} \dfrac{V_{in} V_{nm}V_{mk}V_{ki}}{\qty[E_i^{(0)} - E_n^{(0)}]\qty[E_i^{(0)} - E_m^{(0)}]\qty[E_i^{(0)} - E_k^{(0)}]} 
- 2V_{ii}\sum_{n,m\neq i} \dfrac{V_{in} V_{nm} V_{mi}}{\qty[E_i^{(0)} - E_n^{(0)}]\qty[E_i^{(0)} - E_m^{(0)}]^2}  \notag\\
&{}\quad %- V_{ii} \sum_{n,m\neq i} \dfrac{V_{in}V_{nm}V_{mi} }{\qty[E_i^{(0)} - E_n^{(0)}]^2\qty[E_i^{(0)} - E_m^{(0)}]} 
+  V_{ii}^2 \sum_{n\neq i} \dfrac{V_{in}V_{ni}}{\qty[E_i^{(0)} - E_n^{(0)}]^3} 
-  \sum_{m\neq i} \dfrac{V_{mi}V_{im}}{\qty[E_i^{(0)} - E_m^{(0)}]} \sum_{n\neq i} \dfrac{V_{in}V_{ni}}{\qty[E_i^{(0)} - E_n^{(0)}]^2} 
\end{align}
which agrees with diagrammatic results above.

\subsubsection{Summation of Diagrams}

\newpage
\subsection{Orbital Perturbation Theory: One-Particle Perturbations}
\ex{6.3}
Since $ n\neq 0 $ and $ v(i) $ is one-particle operator, $ n $ must be single-excited, i.e. $ \ket{\Psi_a^r} $. Thus,
\begin{align}
E_0^{(2)} &= \sum_{a,r} \dfrac{\abs{\Braket{\Psi_0 | \sum_i v(i) | \Psi_a^r}}^2}{\Braket{\Psi_0 | \sH | \Psi_0} - \Braket{\Psi_a^r| \sH | \Psi_a^r}} \notag\\
&= \sum_{a,r} \dfrac{v_{ar} v_{ra}}{\sum_b \varepsilon_b^{(0)} - \qty(\sum_{b\neq a}\varepsilon_b^{(0)} + \varepsilon_r^{(0)})} \notag\\
&= \sum_{a,r} \dfrac{v_{ar} v_{ra}}{\varepsilon_a^{(0)} - \varepsilon_r^{(0)}}
\end{align}

\ex{6.4}
Eq 6.15 in textbook gives
\begin{align}
E_i^{(3)} &= \sum_{n,m\neq i} \dfrac{\Braket{i|\mathscr{V}|n} \Braket{n|\mathscr{V}|m} \Braket{m|\mathscr{V}|i}}{(E_i^{(0)}-E_n^{(0)}) (E_i^{(0)}-E_m^{(0)})} 
- E_i^{(1)}\sum_{n\neq i} \dfrac{\abs{\Braket{i|\mathscr{V}|n}}^2 }{(E_i^{(0)}-E_n^{(0)})^2} \notag\\
&= A_i^{(3)} + B_i^{(3)}
\end{align}
\subex{a.}
\begin{align}
B_0^{(3)} &= - E_0^{(1)}\sum_{n\neq 0} \dfrac{\abs{\Braket{\Psi_0|\mathscr{V}|n}}^2 }{(E_0^{(0)}-E_n^{(0)})^2} \notag\\
&= -\sum_b v_{bb} \sum_{a,r}\dfrac{v_{ar}v_{ra}}{(\varepsilon_a^{(0)} - \varepsilon_r^{(0)})^2} \notag\\
&= - \sum_{a,b,r}\dfrac{v_{aa} v_{br}v_{rb}}{(\varepsilon_b^{(0)} - \varepsilon_r^{(0)})^2} 
\end{align}
\subex{b.}
\begin{align}
A_0^{(3)} &= \sum_{n,m\neq 0} \dfrac{\Braket{\Psi_0|\mathscr{V}|n} \Braket{n|\mathscr{V}|m} \Braket{m|\mathscr{V}|\Psi_0}}{(E_0^{(0)}-E_n^{(0)}) (E_0^{(0)}-E_m^{(0)})} \notag\\
&= \sum_{a,r,b,s} \dfrac{\Braket{\Psi_0|\mathscr{V}|\Psi_a^r} \Braket{\Psi_a^r|\mathscr{V}|\Psi_b^s} \Braket{\Psi_b^s|\mathscr{V}|\Psi_0}}{(\varepsilon_a^{(0)} - \varepsilon_r^{(0)}) (\varepsilon_b^{(0)} - \varepsilon_s^{(0)})} \notag\\
&= \sum_{a,r,b,s} \dfrac{v_{ar}v_{sb} \Braket{\Psi_a^r|\mathscr{V}|\Psi_b^s} }{(\varepsilon_a^{(0)} - \varepsilon_r^{(0)}) (\varepsilon_b^{(0)} - \varepsilon_s^{(0)})} 
\end{align}
\subex{c.}
Clearly, if $ a\neq b, r\neq s $
\begin{align}
\Braket{\Psi_a^r | \mathscr{V} | \Psi_b^s} = 0
\end{align}
If $ a=b, r\neq s $,
\begin{align}
\Braket{\Psi_a^r | \mathscr{V} | \Psi_b^s} 
&= \Braket{r| v |s} \notag\\
&= v_{rs}
\end{align}
If $ a\neq b, r= s $,
\begin{align}
\Braket{\Psi_a^r | \mathscr{V} | \Psi_b^s} 
&= \Braket{\Psi_a^r | \mathscr{V} | \Psi_b^r} \notag\\
&= \Braket{\Psi_a^r | \mathscr{V} | -\Psi_{ab}^{ra}} \notag\\
&= -\Braket{b| v |a} \notag\\
&= -v_{ba}
\end{align}
If $ a= b, r= s $,
\begin{align}
\Braket{\Psi_a^r | \mathscr{V} | \Psi_b^s} 
&= \Braket{\Psi_a^r | \mathscr{V} | \Psi_a^r} \notag\\
&= \sum_c v_{cc} - v_{aa} + v_{rr}
\end{align}

\subex{d.}
\begin{align}
E_0^{(3)} &= A_0^{(3)} + B_0^{(3)} \notag\\
&= \sum_{a,r,b,s} 
\dfrac{v_{ar}v_{sb} \Braket{\Psi_a^r|\mathscr{V}|\Psi_b^s} }{(\varepsilon_a^{(0)} - \varepsilon_r^{(0)}) (\varepsilon_b^{(0)} - \varepsilon_s^{(0)})} 
- \sum_{a,b,r}\dfrac{v_{aa} v_{br}v_{rb}}{(\varepsilon_b - \varepsilon_r)^2}
\notag\\
&=  \sum_{a,r\neq s} 
\dfrac{v_{ar}v_{sa} v_{rs} }{(\varepsilon_a^{(0)} - \varepsilon_r^{(0)}) (\varepsilon_a^{(0)} - \varepsilon_s^{(0)})} 
+ \sum_{a\neq b,r} 
\dfrac{v_{ar}v_{rb}(-v_{ba}) }{(\varepsilon_a^{(0)} - \varepsilon_r^{(0)}) (\varepsilon_b^{(0)} - \varepsilon_r^{(0)})} 
\notag\\
&\quad {} + \sum_{a,r} 
\dfrac{v_{ar}v_{ra} (\sum_c v_{cc} - v_{aa} + v_{rr}) }{(\varepsilon_a^{(0)} - \varepsilon_r^{(0)})^2} 
- \sum_{a,b,r}\dfrac{v_{aa} v_{br}v_{rb}}{(\varepsilon_b^{(0)} - \varepsilon_r^{(0)})^2}
\notag\\
&=  \sum_{a,r\neq s} 
\dfrac{v_{ar}v_{sa} v_{rs} }{(\varepsilon_a^{(0)} - \varepsilon_r^{(0)}) (\varepsilon_a^{(0)} - \varepsilon_s^{(0)})} 
+ \sum_{a\neq b,r} 
\dfrac{v_{ar}v_{rb}(-v_{ba}) }{(\varepsilon_a^{(0)} - \varepsilon_r^{(0)}) (\varepsilon_b^{(0)} - \varepsilon_r^{(0)})} 
\notag\\
&\quad {} + \sum_{a,r} 
\dfrac{v_{ar}v_{ra} (\sum_c v_{cc} - v_{aa} + v_{rr}) }{(\varepsilon_a^{(0)} - \varepsilon_r^{(0)})^2} 
- \sum_{a,r}\dfrac{\sum_c v_{cc} v_{ar}v_{ra}}{(\varepsilon_a^{(0)} - \varepsilon_r^{(0)})^2}
\notag\\
&=  \sum_{a,r\neq s} 
\dfrac{v_{ar}v_{sa} v_{rs} }{(\varepsilon_a^{(0)} - \varepsilon_r^{(0)}) (\varepsilon_a^{(0)} - \varepsilon_s^{(0)})} 
+ \sum_{a\neq b,r} 
\dfrac{v_{ar}v_{rb}(-v_{ba}) }{(\varepsilon_a^{(0)} - \varepsilon_r^{(0)}) (\varepsilon_b^{(0)} - \varepsilon_r^{(0)})} 
+ \sum_{a,r} 
\dfrac{v_{ar}v_{ra} ( - v_{aa} + v_{rr}) }{(\varepsilon_a^{(0)} - \varepsilon_r^{(0)})^2} 
\notag\\
&=  \sum_{a,r,s} 
\dfrac{v_{ar}v_{sa} v_{rs} }{(\varepsilon_a^{(0)} - \varepsilon_r^{(0)}) (\varepsilon_a^{(0)} - \varepsilon_s^{(0)})} 
- \sum_{a,b,r} 
\dfrac{v_{ar}v_{rb}v_{ba} }{(\varepsilon_a^{(0)} - \varepsilon_r^{(0)}) (\varepsilon_b^{(0)} - \varepsilon_r^{(0)})} 
\end{align}

\subex{e.}
That's obvious.

\ex{6.5}
Since $ a,b $ run over all $ n $ occupied orbitals $ i,j $ and $ r $ runs over all $ n $ unoccupied orbitals $ k^* $, we have
\begin{align}
-2\sum_{a,b,r}^{N/2} \dfrac{v_{ra}v_{ab}v_{br}}{(\varepsilon_a^{(0)} - \varepsilon_r^{(0)}) (\varepsilon_b^{(0)} - \varepsilon_r^{(0)})} 
&= -\dfrac{2}{(2\beta)^2} \sum_i^n \sum_j^n \sum_k^n \Braket{i|v|j} \Braket{j|v|k^*} \Braket{k^*|v|i} \notag\\
&= -\dfrac{2}{(2\beta)^2} \sum_i^3
\mqty[\Braket{i|v|i+1} \Braket{i+1|v|(i+2)^*} \Braket{(i+2)^*|v|i} \\
+ \Braket{i|v|i+2} \Braket{i+2|v|(i+1)^*} \Braket{(i+1)^*|v|i} ] \notag\\
&= -\dfrac{2}{(2\beta)^2} \sum_i^3 \qty[(\beta/2)(\beta/2)(-\beta/2) + (\beta/2)(-\beta/2)(\beta/2)] \notag\\
&= -\dfrac{2}{(2\beta)^2} \times 3\times (-\beta^3/4) \notag\\
&= 3\beta/8
\end{align}

\ex{6.6}
\subex{a.}
Using the general expression, we get
\begin{align}
\mathscr{E}_0 &= 6\alpha - 2\sum_{j=-1}^1 (\beta_1^2 + \beta_2^2 + 2\beta_1\beta_2\cos\dfrac{2j\pi}{3})^{1/2} \notag\\
&= 6\alpha - 2(\beta_1^2 + \beta_2^2 + 2\beta_1\beta_2\cos\dfrac{-2\pi}{3})^{1/2}
- 2(\beta_1^2 + \beta_2^2 + 2\beta_1\beta_2\cos 0)^{1/2}
- 2(\beta_1^2 + \beta_2^2 + 2\beta_1\beta_2\cos\dfrac{2\pi}{3})^{1/2} 
\notag\\
&= 6\alpha - 2(\beta_1^2 + \beta_2^2 - \beta_1\beta_2)^{1/2}
- 2(\beta_1^2 + \beta_2^2 + 2\beta_1\beta_2)^{1/2}
- 2(\beta_1^2 + \beta_2^2 - \beta_1\beta_2)^{1/2} \notag\\
&= 6\alpha 
- 2\abs{\beta_1 + \beta_2}
- 4(\beta_1^2 + \beta_2^2 - \beta_1\beta_2)^{1/2} \notag\\
&= 6\alpha + 2(\beta_1 + \beta_2)
- 4(\beta_1^2 + \beta_2^2 - \beta_1\beta_2)^{1/2}
\end{align}

Using H\"uckel matrix:
\begin{align}
\vb{H} = \mqty( 
\alpha  & \beta_1 & 0       & 0       & 0       & \beta_2\\
\beta_1 & \alpha  & \beta_2 & 0       & 0       & 0      \\
0       & \beta_2 & \alpha  & \beta_1 & 0       & 0      \\
0       & 0       & \beta_1 & \alpha  & \beta_2 & 0      \\
0       & 0       & 0       & \beta_2 & \alpha  & \beta_1\\
\beta_2 & 0       & 0       & 0       & \beta_1 & \alpha  
)
\end{align}
%Diagonalize $ \vb{H} - \bm\varepsilon $, we get
Eigenvalues of $ \vb{H} $ are
\begin{align}\label{key}
&\alpha + (\beta_1 + \beta_2), \notag\\
&\alpha - \sqrt{\beta_1^2 + \beta_2^2 - \beta_1\beta_2} \;\text{ (2-fold)}, \notag\\
&\alpha + \sqrt{\beta_1^2 + \beta_2^2 - \beta_1\beta_2} \;\text{ (2-fold)}, \notag\\
&\alpha - (\beta_1 + \beta_2),
\end{align}
thus
\begin{align}
\mathscr{E}_0 &= 2\qty[\alpha + (\beta_1 + \beta_2)] + 4\qty[\alpha - \sqrt{\beta_1^2 + \beta_2^2 - \beta_1\beta_2}] \notag\\
&= 6\alpha + 2(\beta_1 + \beta_2) - 4\sqrt{\beta_1^2 + \beta_2^2 - \beta_1\beta_2}
\end{align}

\subex{b.}
\begin{align}
E_R &= \mathscr{E}_0 - (N\alpha + N\beta) \notag\\
&= 6\alpha + 2(\beta_1 + \beta_2) - 4\sqrt{\beta_1^2 + \beta_2^2 - \beta_1\beta_2} - (6\alpha + 6\beta) \notag\\
&= -4\beta_1 + 2\beta_2 - 4\sqrt{\beta_1^2 + \beta_2^2 - \beta_1\beta_2} \notag\\
&= 4\beta\qty(-1 + \dfrac{1}{2}x + \sqrt{1 + x^2 - x})
\end{align}

\subex{c.}
\begin{align}
E_R &= 4\beta\qty(-1 + \dfrac{1}{2}x + \sqrt{1 + x^2 - x}) \notag\\
&= 4\beta\qty[ -1 + \dfrac{1}{2}x 
+ 1 + \dfrac{1}{2}(x^2-x) - \dfrac{1}{8}(x^2-x)^2 + \dfrac{1}{16}(x^2-x)^3 - \dfrac{5}{128}(x^2-x)^4 ] \notag\\
&= 4\beta\qty[ 
\dfrac{1}{2}x^2 - \dfrac{1}{8}(x^4 + x^2 - 2x^3) 
+ \dfrac{1}{16}(-x^3 + 3x^4) - \dfrac{5}{128}x^4 + \cdots ] \notag\\
&= 4\beta\qty[ 
\dfrac{3}{8}x^2 + \dfrac{3}{16}x^3 + \dfrac{3}{128}x^4 + \cdots ] \notag\\
&= \beta\qty[ 
\dfrac{3}{2}x^2 + \dfrac{3}{4}x^3 + \dfrac{3}{32}x^4 + \cdots ]
\end{align}

\newpage
\subsection{Diagrammatic Representation of Orbital Perturbation Theory}
\ex{6.7}
\subex{a.}
~\\
\begin{minipage}{0.2\linewidth}
		\hfill
	\begin{tikzpicture}
	\begin{feynman}[small]
	\vertex (a);
	\vertex [below=of a] (b);
	\vertex [below=of b] (c);
	\vertex [below=of c] (d);
	%\vertex [below=of d] (e);
	\diagram[inline=(c)]{
		(a) -- [fermion, quarter right, edge label'=a] (b)
		-- [fermion, quarter right, edge label'=b] (d)
		-- [fermion, quarter right, edge label'=s] (c)
		-- [fermion, quarter right, edge label'=r] (a)
		%-- [fermion, quarter right, edge label'=2] (a),
	};
	\end{feynman}
	\end{tikzpicture}
	\\
	\vspace{25pt}\\
\end{minipage}
\begin{minipage}{0.8\linewidth}
	\vspace{25pt}~\\
	\begin{align}
	&= - \sum_{a,b,r,s} \dfrac{v_{ab}v_{bs}v_{sr}v_{ra}}{(\varepsilon_a^{(0)} - \varepsilon_r^{(0)})(\varepsilon_b^{(0)} - \varepsilon_r^{(0)})(\varepsilon_b^{(0)} - \varepsilon_b^{(0)})}  &\notag\\
	&= -\dfrac{1}{(2\beta)^3} \sum_{i,j,k,l} \Braket{i|v|j}\Braket{j|v|k^*}\Braket{k^*|v|l^*}\Braket{l^*|v|i} & \notag\\
	&= -\dfrac{2}{(2\beta)^3} \sum_i^{N/2} [-1 + 1 -1 -1 +1 -1]\times (\beta/2)^4  & \notag\\
	&= \dfrac{N\beta}{64}  & \notag\\
	\end{align}
\end{minipage}
%\tikzfeynmanset{
%}
The pictorial representation of the summation are as follows\\
\begin{center}
	\begin{tikzpicture}
	\begin{feynman}[nodes=circle, large]
	\vertex (a) {$ (i) $};
	\vertex [below right=of a] (b1) {$ (i-1) $};
	\vertex [above right=of a] (b2) {$ (i+1) $};
	\vertex [below right=of b1] (c1) {$ (i-2)^* $};
	\vertex [above right=of b1] (c2) {$ (i)^* $};
	\vertex [above right=of b2] (c3) {$ (i+2)^* $};
	\vertex [above right=of c1] (d1) {$ (i-1)^* $};
	\vertex [above right=of c2] (d2) {$ (i+1)^* $};
	\vertex [above right=of d1] (e) {$ (i) $};
	\diagram[inline=(c)]{
		(a) -- [plain, edge label'=+] (b1);
		(a) -- [plain, edge label=+] (b2);
		(b1) -- [plain, edge label'=--] (c1);
		(b1) -- [plain, edge label'=+] (c2);
		(b2) -- [plain, edge label=--] (c2);
		(b2) -- [plain, edge label=+] (c3);
		(c1) -- [plain, edge label'=--] (d1);
		(c2) -- [plain, edge label'=--] (d1);
		(c2) -- [plain, edge label=--] (d2);
		(c3) -- [plain, edge label=--] (d2);
		(d1) -- [plain, edge label'=--] (e);
		(d2) -- [plain, edge label=+] (e);
		%-- [fermion, quarter right, edge label'=2] (a),
	};
	\end{feynman}
	\end{tikzpicture}
\end{center}

\begin{minipage}{0.2\linewidth}
	\hfill
	\begin{tikzpicture}
	\begin{feynman}[small]
	\vertex (a);
	\vertex [below=of a] (b);
	\vertex [below=of b] (c);
	\vertex [below=of c] (d);
	%\vertex [below=of d] (e);
	\diagram[inline=(c)]{
		(a) -- [fermion, out=200, in=150, edge label'=a] (d)
		-- [fermion, out=45, in=330, edge label'=r] (b)
		-- [fermion, quarter left, edge label'=b] (c)
		-- [fermion, quarter left, edge label=s] (a)
		%-- [fermion, quarter right, edge label'=2] (a),
	};
	\end{feynman}
	\end{tikzpicture}
	\\
	\vspace{25pt}\\
\end{minipage}
\begin{minipage}{0.8\linewidth}
	\vspace{25pt}~\\
	\begin{align}
	&= - \sum_{a,r,b,s} \dfrac{v_{ar}v_{rb}v_{bs}v_{sa}}{(\varepsilon_a^{(0)} - \varepsilon_r^{(0)})(\varepsilon_a^{(0)} - \varepsilon_s^{(0)})(\varepsilon_a^{(0)} + \varepsilon_b^{(0)} - \varepsilon_r^{(0)} - \varepsilon_s^{(0)})}  &\notag\\
	&= -\dfrac{1}{(2\beta)^2\times 4\beta} \sum_{i,j,k,l} \Braket{i|v|j^*}\Braket{j^*|v|k}\Braket{k|v|l^*}\Braket{l^*|v|i} & \notag\\
	&= -\dfrac{2}{(2\beta)^2\times 4\beta} \sum_i^{N/2} 6\times (\beta/2)^4  & \notag\\
	&= -\dfrac{3N\beta}{128}  & \notag\\
	\end{align}
\end{minipage}
%\tikzfeynmanset{
%}
The pictorial representation of the summation are as follows\\
\begin{center}
	\begin{tikzpicture}
	\begin{feynman}[nodes=circle, large]
	\vertex (a) {$ (i) $};
	\vertex [below right=of a] (b1) {$ (i-1)^* $};
	\vertex [above right=of a] (b2) {$ (i+1)^* $};
	\vertex [below right=of b1] (c1) {$ (i-2) $};
	\vertex [above right=of b1] (c2) {$ (i) $};
	\vertex [above right=of b2] (c3) {$ (i+2) $};
	\vertex [above right=of c1] (d1) {$ (i-1)^* $};
	\vertex [above right=of c2] (d2) {$ (i+1)^* $};
	\vertex [above right=of d1] (e) {$ (i) $};
	\diagram[inline=(c)]{
		(a) -- [plain, edge label'=--] (b1);
		(a) -- [plain, edge label=+] (b2);
		(b1) -- [plain, edge label'=+] (c1);
		(b1) -- [plain, edge label'=--] (c2);
		(b2) -- [plain, edge label=+] (c2);
		(b2) -- [plain, edge label=--] (c3);
		(c1) -- [plain, edge label'=+] (d1);
		(c2) -- [plain, edge label'=--] (d1);
		(c2) -- [plain, edge label=+] (d2);
		(c3) -- [plain, edge label=--] (d2);
		(d1) -- [plain, edge label'=--] (e);
		(d2) -- [plain, edge label=+] (e);
		%-- [fermion, quarter right, edge label'=2] (a),
	};
	\end{feynman}
	\end{tikzpicture}
\end{center}
thus
\begin{equation}\label{key}
E_0^{(4)} = 4\times \dfrac{N\beta}{64} + 3\times\qty(-\dfrac{3N\beta}{128}) =  \dfrac{N\beta}{64}
\end{equation}

\subex{b.}
Let $ N=6 $, we get
\begin{equation}\label{key}
E_0^{(4)} = \dfrac{3\beta}{32}
\end{equation}
which agrees with the result in Ex 6.6.

\subsection{Perturbation Expansion of the Correlation Energy}
\ex{6.8}
\begin{align}
E_0^{(2)} &= 
\dfrac{1}{4}\sum_{a,b,r,s} \dfrac{\abs{\Braket{ab||rs}}^2}{\varepsilon_a + \varepsilon_b - \varepsilon_r - \varepsilon_s} \notag\\
&= \dfrac{1}{4}\sum_{a,b,r,s} 
\dfrac{\qty(\Braket{ab|rs} - \Braket{ab|sr}) \qty(\Braket{rs|ab} - \Braket{sr|ab})}{\varepsilon_a + \varepsilon_b - \varepsilon_r - \varepsilon_s} \notag\\
&= \dfrac{1}{4}\sum_{a,b,r,s} 
\dfrac{\Braket{ab|rs}\Braket{rs|ab} - \Braket{ab|sr}\Braket{rs|ab}  - \Braket{ab|rs}\Braket{sr|ab} + \Braket{ab|sr}\Braket{sr|ab}}{\varepsilon_a + \varepsilon_b - \varepsilon_r - \varepsilon_s} \notag\\
&= \dfrac{1}{4}\qty[\sum_{a,b,r,s} 
\dfrac{\Braket{ab|rs}\Braket{rs|ab}}{\varepsilon_a + \varepsilon_b - \varepsilon_r - \varepsilon_s}
- \sum_{a,b,r,s} 
\dfrac{\Braket{ab|sr}\Braket{rs|ab}}{\varepsilon_a + \varepsilon_b - \varepsilon_r - \varepsilon_s}
- \sum_{a,b,r,s} 
\dfrac{ \Braket{ab|rs}\Braket{sr|ab}}{\varepsilon_a + \varepsilon_b - \varepsilon_r - \varepsilon_s}
+ \sum_{a,b,r,s} 
\dfrac{\Braket{ab|sr}\Braket{sr|ab}}{\varepsilon_a + \varepsilon_b - \varepsilon_r - \varepsilon_s}
] \notag\\
&= \dfrac{1}{4}\qty[2\sum_{a,b,r,s} 
\dfrac{\Braket{ab|rs}\Braket{rs|ab}}{\varepsilon_a + \varepsilon_b - \varepsilon_r - \varepsilon_s}
- 2\sum_{a,b,r,s} 
\dfrac{\Braket{ab|rs}\Braket{sr|ab}}{\varepsilon_a + \varepsilon_b - \varepsilon_r - \varepsilon_s}
] \notag\\
&= \dfrac{1}{2}\sum_{a,b,r,s} 
\dfrac{\Braket{ab|rs}\Braket{rs|ab}}{\varepsilon_a + \varepsilon_b - \varepsilon_r - \varepsilon_s}
- \dfrac{1}{2}\sum_{a,b,r,s} 
\dfrac{\Braket{ab|rs}\Braket{rs|ba}}{\varepsilon_a + \varepsilon_b - \varepsilon_r - \varepsilon_s}
\end{align}
For a closed-shell system, the possible spin part of $ a,b,r,s $ of the non-zero terms are\\
first term: $ \alpha,\alpha,\alpha,\alpha $;\quad $ \alpha,\beta,\alpha,\beta $;\quad $ \beta,\alpha,\beta,\alpha $;\quad $ \beta,\beta,\beta,\beta $\\
second term: $ \alpha,\alpha,\alpha,\alpha $;\quad $ \beta,\beta,\beta,\beta $\\
thus
\begin{align}
E_0^{(2)} &= 2\sum_{a,b,r,s}^{N/2}
\dfrac{\Braket{ab|rs}\Braket{rs|ab}}{\varepsilon_a + \varepsilon_b - \varepsilon_r - \varepsilon_s}
- \sum_{a,b,r,s}^{N/2}
\dfrac{\Braket{ab|rs}\Braket{rs|ba}}{\varepsilon_a + \varepsilon_b - \varepsilon_r - \varepsilon_s}
\end{align}

\ex{6.9}
\begin{align}
E_{\corr} &= \Delta - (\Delta^2 + K_{12}^2)^{1/2} \notag\\
&= \Delta - \qty[\Delta + \dfrac{K_{12}^2}{2\Delta}] \notag\\
&= - \dfrac{K_{12}^2}{2\Delta} \notag\\
&= - \dfrac{K_{12}^2}{2(\varepsilon_2-\varepsilon_1) + J_{11} + J_{22} - 4J_{12} + 2K_{12}} \notag\\
&= - K_{12}^2 \qty(\dfrac{1}{2(\varepsilon_2-\varepsilon_1)} - \dfrac{ J_{11} + J_{22} - 4J_{12} + 2K_{12}}{4(\varepsilon_2-\varepsilon_1)^2}) \notag\\
&= \dfrac{K_{12}^2}{2(\varepsilon_1-\varepsilon_2)} + \dfrac{K_{12}^2 (J_{11} + J_{22} - 4J_{12} + 2K_{12})}{4(\varepsilon_1-\varepsilon_2)^2})
\end{align}

\subsection{The $ N $-dependence of the RS Perturbation Expansion}
\ex{6.10}
From Eq 6.68, we get
\begin{align}
E_0^{(1)} &= \Braket{\Psi_0 | \mathscr{V} | \Psi_0} = -\dfrac{1}{2}\sum_{ab}\Braket{ab||ab} \notag\\
&= -\dfrac{1}{2}\sum_{i=1}^N \qty[\Braket{1_i \bar{1}_i|| 1_i \bar{1}_i} + \Braket{\bar{1}_i 1_i || \bar{1}_i 1_i}] \notag\\
&= -\dfrac{1}{2}\sum_{i=1}^N 
\qty[\Braket{1_i \bar{1}_i| 1_i \bar{1}_i} 
- \Braket{1_i \bar{1}_i| \bar{1}_i 1_i} 
+ \Braket{\bar{1}_i 1_i | \bar{1}_i 1_i}
- \Braket{\bar{1}_i 1_i | 1_i \bar{1}_i} ] \notag\\
&= -\dfrac{1}{2}\times 2N [1_i 1_i| 1_i 1_i] \notag\\
&= -N J_{11}
\end{align}
\begin{align}
\Braket{\Psi_{1_i\bar{1}_i}^{2_i\bar{2}_i} | \mathscr{V} | \Psi_{1_i\bar{1}_i}^{2_i\bar{2}_i}} 
&= \Braket{\Psi_{1_i\bar{1}_i}^{2_i\bar{2}_i} | \sH | \Psi_{1_i\bar{1}_i}^{2_i\bar{2}_i}} 
- \Braket{\Psi_{1_i\bar{1}_i}^{2_i\bar{2}_i} | \sH_0 | \Psi_{1_i\bar{1}_i}^{2_i\bar{2}_i}} \notag\\
&= (2N-2) h_{11} + 2h_{22} + (N-1)J_{11} + J_{22} - (2N-2)\varepsilon_1 - 2\varepsilon_2 \notag\\
&= (2N-2) h_{11} + 2h_{22} + (N-1)J_{11} + J_{22} - (2N-2)(h_{11} + J_{11}) - 2(h_{22} + 2J_{12} - K_{12}) \notag\\
&= -(N-1)J_{11} + J_{22} - 4J_{12} + 2K_{12}
\end{align}

\subsection{Diagrammatic Representation of the Perturbation Expansion of the Correlation Energy}

\subsubsection{Hugenholtz Diagrams}
\ex{6.11}
The numerator and denominator are obvious.\\
$ h=5 $, and $ l=2 $ since closed loops are $ r\ra a\ra d\ra t\ra e\ra r; \; s\ra c\ra b\ra s $. The number of quivalent line pairs is one ($ r,s $). Thus the pre-factor is $ -\dfrac{1}{2} $.

\subsubsection{Goldstone Diagrams}
\ex{6.12}
~\\
1
\begin{minipage}{0.4\linewidth}
	\centering
\begin{tikzpicture}
\begin{feynman}[large]
\vertex (a); \vertex [below=of a] (b);
\vertex [right=of a] (c); \vertex [right=of b] (d);

\diagram[%inline=(e.base)
]{
	(a) -- [fermion,  quarter left, edge label=r] (b)
	-- [fermion, quarter left, edge label=a] (a),
	(c)-- [fermion, quarter left, edge label=s] (d)
	-- [fermion, quarter left, edge label=b] (c),
	(a) -- [scalar] (c),
	(b) -- [scalar] (d),
	%-- [fermion, quarter right, edge label'=2] (a),
};
\end{feynman}
\end{tikzpicture}
\end{minipage}
$ \lra  $
\begin{minipage}{0.4\linewidth}
	\centering
	\begin{tikzpicture}
	\begin{feynman}[large]
	\vertex (a); \vertex [below=of a] (b);
	%\vertex [right=of a] (c); %\vertex [right=of b] (d);
	\diagram[%inline=(e.base)
	]{
		(a) -- [fermion,  half left, edge label=r] (b)
		-- [fermion, half left, edge label=a] (a),
		(a)-- [fermion, quarter left, edge label=s] (b)
		-- [fermion, quarter left, edge label=b] (a),
		%(a) -- [scalar] (c),
		%(b) -- [scalar] (d),
		%-- [fermion, quarter right, edge label'=2] (a),
	};
	\end{feynman}
	\end{tikzpicture}
\end{minipage}
~\\
~\vspace{10pt}\\
2
\begin{minipage}{0.4\linewidth}
	\centering
	\begin{tikzpicture}
	\begin{feynman}[large]
	\vertex (a); \vertex [below=of a] (b);
	\vertex [right=of a] (c); \vertex [right=of b] (d);
	
	\diagram[%inline=(e.base)
	]{
		(a) -- [fermion, out=-60, in=150, edge label'=b] (d)
		-- [fermion, quarter right, edge label=s] (c),
		(c)-- [fermion, out=240, in=30, edge label=a] (b)
		-- [fermion, quarter left, edge label=r] (a),
		(a) -- [scalar] (c),
		(b) -- [scalar] (d),
		%-- [fermion, quarter right, edge label'=2] (a),
	};
	\end{feynman}
	\end{tikzpicture}
\end{minipage}
$ \lra  $
\begin{minipage}{0.4\linewidth}
	\centering
	\begin{tikzpicture}
	\begin{feynman}[large]
	\vertex (a); \vertex [below=of a] (b);
	%\vertex [right=of a] (c); %\vertex [right=of b] (d);
	\diagram[%inline=(e.base)
	]{
		(b) -- [fermion,  half right, edge label'=r] (a),
		(b) -- [fermion, half left, edge label'=s] (a),
		(a)-- [fermion, quarter left, edge label=a] (b),
		(a) -- [fermion, quarter right, edge label=b] (b),
		%(a) -- [scalar] (c),
		%(b) -- [scalar] (d),
		%-- [fermion, quarter right, edge label'=2] (a),
	};
	\end{feynman}
	\end{tikzpicture}
\end{minipage}
~\\
~\vspace{10pt}\\
1
\begin{minipage}{0.4\linewidth}
	\centering
	\begin{tikzpicture}
	\begin{feynman}[large]
	\vertex (a); \vertex [below=2.4cm of a] (b);
	\vertex [right=2.5cm of a] (c); \vertex [right=2.5cm of b] (d);
	\vertex at ($(a)!0.5!(b) + (0.7cm, 0)$) (e); \vertex at ($(c)!0.5!(d) + (-0.7cm, 0)$) (f);
	
	\diagram[%inline=(e.base)
	]{
		(a) -- [anti fermion,  quarter left, edge label=t] (e) 
		-- [anti fermion,  quarter left, edge label=r] (b)
		-- [anti fermion, quarter left, edge label=a] (a),
		(c) -- [fermion, quarter left, edge label=b] (d)
		-- [fermion, quarter left, edge label=u] (f)
		-- [fermion, quarter left, edge label=s] (c),
		(a) -- [scalar] (c),
		(b) -- [scalar] (d),
		(e) -- [scalar] (f),
		%-- [fermion, quarter right, edge label'=2] (a),
	};
	\end{feynman}
	\end{tikzpicture}
\end{minipage}
$ \lra  $
\begin{minipage}{0.4\linewidth}
	\centering
	\begin{tikzpicture}
	\begin{feynman}[large]
	\vertex (a); \vertex [below=of a] (b);
	%\vertex [right=of a] (c); %\vertex [right=of b] (d);
	\vertex at ($(a)!0.5!(b) $) (e);
	\diagram[%inline=(e.base)
	]{
		(a) -- [anti fermion,  quarter left, edge label=t] (e) 
		-- [anti fermion,  quarter left, edge label=r] (b)
		-- [anti fermion, half left, edge label=a] (a),
		(a) -- [fermion, half left, edge label=b] (b)
		-- [fermion, quarter left, edge label=u] (e)
		-- [fermion, quarter left, edge label=s] (a),
	};
	\end{feynman}
	\end{tikzpicture}
\end{minipage}
~\\
~\vspace{10pt}\\
2
\begin{minipage}{0.4\linewidth}
	\centering
	\begin{tikzpicture}
	\begin{feynman}[large]
	\vertex (a); \vertex [below=2.4cm of a] (b);
	\vertex [right=2.5cm of a] (c); \vertex [right=2.5cm of b] (d);
	\vertex at ($(a)!0.5!(b) + (0.7cm, 0)$) (e); \vertex at ($(c)!0.5!(d) + (-0.7cm, 0)$) (f);
	
	\diagram[%inline=(e.base)
	]{
		(a) -- [fermion,  quarter left, edge label=c] (e) 
		-- [ fermion,  quarter left, edge label=a] (b)
		-- [ fermion, quarter left, edge label=r] (a),
		(c) -- [anti fermion, quarter left, edge label=s] (d)
		-- [anti fermion, quarter left, edge label=d] (f)
		-- [anti fermion, quarter left, edge label=b] (c),
		(a) -- [scalar] (c),
		(b) -- [scalar] (d),
		(e) -- [scalar] (f),
		%-- [fermion, quarter right, edge label'=2] (a),
	};
	\end{feynman}
	\end{tikzpicture}
\end{minipage}
$ \lra  $
\begin{minipage}{0.4\linewidth}
	\centering
	\begin{tikzpicture}
	\begin{feynman}[large]
	\vertex (a); \vertex [below=2.4cm of a] (b);
	%\vertex [right=2.5cm of a] (c); \vertex [right=2.5cm of b] (d);
	\vertex at ($(a)!0.5!(b) $) (e); 
	%\vertex at ($(c)!0.5!(d) + (-0.7cm, 0)$) (f);
	
	\diagram[%inline=(e.base)
	]{
		(a) -- [fermion,  quarter left, edge label=c] (e) 
		-- [ fermion,  quarter left, edge label=a] (b)
		-- [ fermion, half left, edge label=r] (a),
		(a) -- [anti fermion, half left, edge label=s] (b)
		-- [anti fermion, quarter left, edge label=d] (e)
		-- [anti fermion, quarter left, edge label=b] (a),
		%(a) -- [scalar] (c),
		%(b) -- [scalar] (d),
		%(e) -- [scalar] (f),
		%-- [fermion, quarter right, edge label'=2] (a),
	};
	\end{feynman}
	\end{tikzpicture}
\end{minipage}
~\\
~\vspace{10pt}\\
3
\begin{minipage}{0.4\linewidth}
	\centering
	\begin{tikzpicture}
	\begin{feynman}[large]
	\vertex (a); \vertex [below=2.4cm of a] (b);
	\vertex [right=2.5cm of a] (c); \vertex [right=2.5cm of b] (d);
	\vertex at ($(a)!0.5!(b) + (0.7cm, 0)$) (e); \vertex at ($(c)!0.5!(d) + (-0.7cm, 0)$) (f);
	
	\diagram[%inline=(e.base)
	]{
		(a) -- [anti fermion,  quarter left, edge label=s] (e) 
		-- [anti fermion,  quarter left, edge label=r] (b)
		-- [anti fermion, quarter left, edge label=a] (a),
		(c) -- [anti fermion, quarter left, edge label=t] (d)
		-- [anti fermion, quarter left, edge label=c] (f)
		-- [anti fermion, quarter left, edge label=b] (c),
		(a) -- [scalar] (c),
		(b) -- [scalar] (d),
		(e) -- [scalar] (f),
		%-- [fermion, quarter right, edge label'=2] (a),
	};
	\end{feynman}
	\end{tikzpicture}
\end{minipage}
$ \lra $
\begin{minipage}{0.4\linewidth}
	\centering
	\begin{tikzpicture}
	\begin{feynman}[large]
	\vertex (a); \vertex [below=2.4cm of a] (b);
	%\vertex [right=2.5cm of a] (c); \vertex [right=2.5cm of b] (d);
	\vertex at ($(a)!0.5!(b) $) (e); 
	%\vertex at ($(c)!0.5!(d) + (-0.7cm, 0)$) (f);
	
	\diagram[%inline=(e.base)
	]{
		(a) -- [anti fermion,  quarter left, edge label=s] (e) 
		-- [anti fermion,  quarter left, edge label=r] (b)
		-- [anti fermion, half left, edge label=a] (a),
		(a) -- [anti fermion, half left, edge label=t] (b)
		-- [anti fermion, quarter left, edge label=c] (e)
		-- [anti fermion, quarter left, edge label=b] (a),
		%(a) -- [scalar] (c),
		%(b) -- [scalar] (d),
		%(e) -- [scalar] (f),
		%-- [fermion, quarter right, edge label'=2] (a),
	};
	\end{feynman}
	\end{tikzpicture}
\end{minipage}
~\\
~\vspace{10pt}\\
4
\begin{minipage}{0.4\linewidth}
	\centering
	\begin{tikzpicture}
	\begin{feynman}[large]
	\vertex (a); \vertex [below=2.4cm of a] (b);
	\vertex [right=2.5cm of a] (c); \vertex [right=2.5cm of b] (d);
	\vertex at ($(a)!0.5!(b) + (-0.7cm, 0)$) (e); \vertex at ($(a)!0.5!(b) + (0.7cm, 0)$) (f);
	
	\diagram[%inline=(e.base)
	]{
		(a) -- [fermion,  quarter left, edge label=a] (f) 
		-- [ fermion,  quarter left, edge label=b] (b)
		-- [ fermion, quarter left, edge label=r] (e)
		-- [ fermion, quarter left, edge label=s] (a),
		(c) -- [fermion, quarter left, edge label=c] (d)
		%-- [anti fermion, quarter left, edge label=d] (f)
		-- [fermion, quarter left, edge label=t] (c),
		(a) -- [scalar] (c),
		(b) -- [scalar] (d),
		(e) -- [scalar] (f),
		%-- [fermion, quarter right, edge label'=2] (a),
	};
	\end{feynman}
	\end{tikzpicture}
\end{minipage}
$ \lra  $
\begin{minipage}{0.4\linewidth}
	\centering
	\begin{tikzpicture}
	\begin{feynman}[large]
	\vertex (a); \vertex [below=2.4cm of a] (b);
	%\vertex [right=2.5cm of a] (c); \vertex [right=2.5cm of b] (d);
	\vertex at ($(a)!0.5!(b) $) (e); 
	%\vertex at ($(a)!0.5!(b) + (0.7cm, 0)$) (f);
	
	\diagram[%inline=(e.base)
	]{
		(a) -- [fermion,  quarter left, edge label=a] (e) 
		-- [ fermion,  quarter left, edge label=b] (b)
		-- [ fermion, quarter left, edge label=r] (e)
		-- [ fermion, quarter left, edge label=s] (a),
		(a) -- [fermion, half left, edge label=c] (b)
		%-- [anti fermion, quarter left, edge label=d] (f)
		-- [fermion, half left, edge label=t] (a),
		%(a) -- [scalar] (c),
		%(b) -- [scalar] (d),
		%(e) -- [scalar] (f),
		%-- [fermion, quarter right, edge label'=2] (a),
	};
	\end{feynman}
	\end{tikzpicture}
\end{minipage}
~\\
~\vspace{10pt}\\
5
\begin{minipage}{0.4\linewidth}
	\centering
	\begin{tikzpicture}
	\begin{feynman}[large]
	\vertex (a); \vertex [below=2.4cm of a] (b);
	\vertex [right= of a] (c); \vertex [right=4cm of b] (d);
	\vertex [below=1.2cm of c] (e); \vertex [right= of e] (f);
	
	\diagram[%inline=(e.base)
	]{
		(a) %-- [fermion,  quarter left, edge label=a] (f) 
		-- [ fermion,  quarter left, edge label=a] (b)
		%-- [ fermion, quarter left, edge label=r] (e)
		-- [ fermion, quarter left, edge label=r] (a),
		(c) -- [fermion, quarter left, edge label=b] (e)
		%-- [anti fermion, quarter left, edge label=d] (f)
		-- [fermion, quarter left, edge label=s] (c),
		(f) -- [fermion, quarter left, edge label=c] (d)
		%-- [anti fermion, quarter left, edge label=d] (f)
		-- [fermion, quarter left, edge label=t] (f),
		(a) -- [scalar] (c),
		(b) -- [scalar] (d),
		(e) -- [scalar] (f),
		%-- [fermion, quarter right, edge label'=2] (a),
	};
	\end{feynman}
	\end{tikzpicture}
\end{minipage}
$ \lra  $
\begin{minipage}{0.4\linewidth}
	\centering
	\begin{tikzpicture}
	\begin{feynman}[large]
	\vertex (a); \vertex [below=2.4cm of a] (b);
	%\vertex [right= of a] (c); \vertex [right=4cm of b] (d);
	\vertex [below=1.2cm of a] (e); %\vertex [right= of e] (f);
	
	\diagram[%inline=(e.base)
	]{
		(a) %-- [fermion,  quarter left, edge label=a] (f) 
		-- [ fermion,  half left, edge label=a] (b)
		%-- [ fermion, quarter left, edge label=r] (e)
		-- [ fermion, half left, edge label=r] (a),
		(a) -- [fermion, quarter left, edge label=b] (e)
		%-- [anti fermion, quarter left, edge label=d] (f)
		-- [fermion, quarter left, edge label=s] (a),
		(e) -- [fermion, quarter left, edge label=c] (b)
		%-- [anti fermion, quarter left, edge label=d] (f)
		-- [fermion, quarter left, edge label=t] (e),
		%(a) -- [scalar] (c),
		%(b) -- [scalar] (d),
		%(e) -- [scalar] (f),
		%-- [fermion, quarter right, edge label'=2] (a),
	};
	\end{feynman}
	\end{tikzpicture}
\end{minipage}

~\\
~\vspace{10pt}\\
6
\begin{minipage}{0.4\linewidth}
	\centering
	\begin{tikzpicture}
	\begin{feynman}[large]
	\vertex (a); \vertex [below=2.4cm of a] (b);
	\vertex [right=2.5cm of a] (c); \vertex [right=2.5cm of b] (d);
	\vertex at ($(a)!0.5!(b) + (0.6cm, 0)$) (e); \vertex at ($(c)!0.5!(d) + (-0.6cm, 0)$) (f);
	
	\diagram[%inline=(e.base)
	]{
		(a) -- [fermion,  quarter left, edge label=a] (e) 
		-- [ fermion,  edge label=s] (c)
		-- [ fermion, quarter left, edge label=b] (d)
		-- [ fermion, quarter left, edge label=t] (f)
		-- [fermion, edge label=c] (b)
		%-- [anti fermion, quarter left, edge label=d] (f)
		-- [fermion, quarter left, edge label=r] (a),
		(a) -- [scalar] (c),
		(b) -- [scalar] (d),
		(e) -- [scalar] (f),
		%-- [fermion, quarter right, edge label'=2] (a),
	};
	\end{feynman}
	\end{tikzpicture}
\end{minipage}
$ \lra  $
\begin{minipage}{0.4\linewidth}
	\centering
	\begin{tikzpicture}
	\begin{feynman}[large]
	\vertex (a); \vertex [below=2.4cm of a] (b);
	%\vertex [right=2.5cm of a] (c); \vertex [right=2.5cm of b] (d);
	\vertex at ($(a)!0.5!(b)$) (e); %\vertex at ($(c)!0.5!(d) + (-0.6cm, 0)$) (f);
	
	\diagram[%inline=(e.base)
	]{
		(a) -- [fermion,  quarter left, edge label=a] (e) 
		-- [ fermion,  quarter left, edge label=s] (a)
		-- [ fermion, half left, edge label=b] (b)
		-- [ fermion, quarter left, edge label=t] (e)
		-- [fermion, quarter left, edge label=c] (b)
		%-- [anti fermion, quarter left, edge label=d] (f)
		-- [fermion, half left, edge label=r] (a),
		%(a) -- [scalar] (c),
		%(b) -- [scalar] (d),
		%(e) -- [scalar] (f),
		%-- [fermion, quarter right, edge label'=2] (a),
	};
	\end{feynman}
	\end{tikzpicture}
\end{minipage}
~\\
~\vspace{10pt}\\
7
\begin{minipage}{0.4\linewidth}
	\centering
	\begin{tikzpicture}
	\begin{feynman}[large]
	\vertex (a); \vertex [below=2.4cm of a] (b);
	\vertex [right=2.5cm of a] (c); \vertex [right=2.5cm of b] (d);
	\vertex at ($(a)!0.5!(b) + (-0.6cm, 0)$) (e); \vertex at ($(c)!0.5!(d) + (0.6cm, 0)$) (f);
	
	\diagram[]{
		(a) -- [fermion,  quarter right, edge label'=d] (e) 
		-- [ fermion,  quarter right,  edge label'=a] (b)
		-- [ fermion,out=20, in=-120,  edge label'=r] (c)
		-- [ fermion, quarter left, edge label=b] (f)
		-- [fermion, quarter left, edge label=c] (d)
		%-- [anti fermion, quarter left, edge label=d] (f)
		-- [fermion,out=160, in=-60, edge label=s] (a),
		(a) -- [scalar] (c),
		(b) -- [scalar] (d),
		(e) -- [scalar] (f),
	};
	\end{feynman}
	\end{tikzpicture}
\end{minipage}
$ \lra  $
\begin{minipage}{0.4\linewidth}
	\centering
	\begin{tikzpicture}
	\begin{feynman}[large]
	\vertex (a); \vertex [below=2.4cm of a] (b);
	%\vertex [right=2.5cm of a] (c); \vertex [right=2.5cm of b] (d);
	\vertex at ($(a)!0.5!(b) $) (e); %\vertex at ($(c)!0.5!(d) + (0.6cm, 0)$) (f);
	
	\diagram[]{
		(a) -- [fermion,  \qr, \el'=d] (e) -- [ fermion,  \qr,  \el'=a] (b)
		-- [ fermion, \hfr,  \el'=r] (a) -- [ fermion, \ql, \el=b] (e)
		-- [fermion, \ql, \el=c] (b)
		%-- [anti fermion, quarter left, edge label=d] (f)
		-- [fermion, \hfl, \el=s] (a),
		%(a) -- [scalar] (c),
		%(b) -- [scalar] (d),
		%(e) -- [scalar] (f),
	};
	\end{feynman}
	\end{tikzpicture}
\end{minipage}
~\\
~\vspace{10pt}\\
8
\begin{minipage}{0.4\linewidth}
	\centering
	\begin{tikzpicture}
	\begin{feynman}[large]
	\vertex (a); \vertex [below=2.4cm of a] (b);
	\vertex [right=2.5cm of a] (c); \vertex [right=2.5cm of b] (d);
	\vertex at ($(a)!0.5!(b) + (0.7cm, 0)$) (e); \vertex at ($(c)!0.5!(d) + (-0.7cm, 0)$) (f);	
	\diagram[]{
		(a) -- [fermion,  \qr, \el'=a] (b) -- [ fermion, \el=r] (f)
		-- [ fermion,  \el'=s] (c) -- [ fermion, \ql, \el=b] (d)
		-- [fermion,  \el'=t] (e) -- [fermion, \el=u] (a),
		(a) -- [scalar] (c), 	(b) -- [scalar] (d), (e) -- [scalar] (f),
	};
	\end{feynman}
	\end{tikzpicture}
\end{minipage}
$ \lra  $
\begin{minipage}{0.4\linewidth}
	\centering
	\begin{tikzpicture}
	\begin{feynman}[large]
	\vertex (a); \vertex [below=2.4cm of a] (b);
	%\vertex [right=2.5cm of a] (c); \vertex [right=2.5cm of b] (d);
	\vertex at ($(a)!0.5!(b) $) (e); %\vertex at ($(c)!0.5!(d) + (-0.7cm, 0)$) (f);	
	\diagram[]{
		(a) -- [fermion,  \hfr, \el'=a] (b) -- [ fermion,\qr, \el'=r] (e)
		-- [ fermion, \qr, \el'=s] (a) -- [ fermion, \hfl, \el=b] (b)
		-- [fermion, \ql, \el=t] (e) -- [fermion, \ql,\el=u] (a),
		%(a) -- [scalar] (c), 	(b) -- [scalar] (d), (e) -- [scalar] (f),
	};
	\end{feynman}
	\end{tikzpicture}
\end{minipage}
~\\
~\vspace{10pt}\\
9
\begin{minipage}{0.4\linewidth}
	\centering
	\begin{tikzpicture}
	\begin{feynman}[large]
	\vertex (a); \vertex [below=2.4cm of a] (b);
	\vertex [right=2.5cm of a] (c); \vertex [right=2.5cm of b] (d);
	\vertex at ($(a)!0.5!(b) + (-0.7cm, 0)$) (e); \vertex at ($(a)!0.5!(b) + (0.7cm, 0)$) (f);	
	\diagram[]{
		(a) -- [fermion,  \qr, \el'=a] (e) -- [ fermion, \qr, \el'=b] (b)
		-- [ fermion,  \el'=r] (f) -- [ fermion, \el=s] (c)
		-- [fermion, \ql, \el'=c] (d) -- [fermion, \el'=t] (a),
		(a) -- [scalar] (c), 	(b) -- [scalar] (d), (e) -- [scalar] (f),
	};
	\end{feynman}
	\end{tikzpicture}
\end{minipage}
$ \lra  $
\begin{minipage}{0.4\linewidth}
	\centering
	\begin{tikzpicture}
	\begin{feynman}[large]
	\vertex (a); \vertex [below=2.4cm of a] (b);
	%\vertex [right=2.5cm of a] (c); \vertex [right=2.5cm of b] (d);
	\vertex at ($(a)!0.5!(b) $) (e); %\vertex at ($(a)!0.5!(b) + (0.7cm, 0)$) (f);	
	\diagram[]{
		(a) -- [fermion,  \qr, \el'=a] (e) -- [ fermion, \qr, \el'=b] (b)
		-- [ fermion, \qr, \el'=r] (e) -- [ fermion, \qr, \el'=s] (a)
		-- [fermion, \hfl, \el=c] (b) -- [fermion, \hfl, \el=t] (a),
		%(a) -- [scalar] (c), 	(b) -- [scalar] (d), (e) -- [scalar] (f),
	};
	\end{feynman}
	\end{tikzpicture}
\end{minipage}
~\\
~\vspace{10pt}\\
10
\begin{minipage}{0.4\linewidth}
	\centering
	\begin{tikzpicture}
	\begin{feynman}[large]
	\vertex (a); \vertex [below=2.4cm of a] (b);
	\vertex [right=2.5cm of a] (c); \vertex [right=2.5cm of b] (d);
	\vertex at ($(a)!0.5!(b) + (-0.7cm, 0)$) (e); \vertex at ($(a)!0.5!(b) + (0.7cm, 0)$) (f);	
	\diagram[]{
		(a) -- [fermion,  \el'=s] (f) -- [ fermion, \el'=r] (d)
		-- [ fermion, \qr,  \el'=c] (c) -- [ fermion, \el'=t] (b)
		-- [fermion, \ql, \el=b] (e) -- [fermion,\ql, \el=a] (a),
		(a) -- [scalar] (c), 	(b) -- [scalar] (d), (e) -- [scalar] (f),
	};
	\end{feynman}
	\end{tikzpicture}
\end{minipage}
$ \lra  $
\begin{minipage}{0.4\linewidth}
	\centering
	\begin{tikzpicture}
	\begin{feynman}[large]
	\vertex (a); \vertex [below=2.4cm of a] (b);
	%\vertex [right=2.5cm of a] (c); \vertex [right=2.5cm of b] (d);
	\vertex at ($(a)!0.5!(b) $) (e); %\vertex at ($(a)!0.5!(b) + (0.7cm, 0)$) (f);	
	\diagram[]{
		(a) -- [fermion, \ql, \el=s] (e) -- [ fermion,\ql, \el=r] (b)
		-- [ fermion, \hfr,  \el'=c] (a) -- [ fermion, \hfr, \el'=t] (b)
		-- [fermion, \ql, \el=b] (e) -- [fermion,\ql, \el=a] (a),
		%(a) -- [scalar] (c), 	(b) -- [scalar] (d), (e) -- [scalar] (f),
	};
	\end{feynman}
	\end{tikzpicture}
\end{minipage}
~\\
~\vspace{10pt}\\
11
\begin{minipage}{0.4\linewidth}
	\centering
	\begin{tikzpicture}
	\begin{feynman}[large]
	\vertex (a); \vertex [below=2.4cm of a] (b);
	\vertex [right=3.5cm of a] (c); \vertex [right= of b] (d);
	\vertex [above=1.2cm of d] (e); \vertex [below=1.2cm of c] (f);	
	\diagram[]{
		(a) -- [fermion,  \el'=b] (d) -- [ fermion, \qr, \el'=s] (e)
		-- [ fermion,  \el=a] (b) -- [ fermion, \ql, \el=r] (a),
		(c) -- [fermion, \ql, \el=c] (f) -- [fermion,\ql, \el=t] (c),
		(a) -- [scalar] (c), 	(b) -- [scalar] (d), (e) -- [scalar] (f),
	};
	\end{feynman}
	\end{tikzpicture}
\end{minipage}
$ \lra  $
\begin{minipage}{0.4\linewidth}
	\centering
	\begin{tikzpicture}
	\begin{feynman}[large]
	\vertex (a); \vertex [below=2.4cm of a] (b);
	%\vertex [right=3.5cm of a] (c); \vertex [right= of b] (d);
	\vertex at ($(a)!0.5!(b)$) (e); %\vertex [below=1.2cm of c] (f);	
	\diagram[]{
		(a) -- [fermion, \hfl, \el=b] (b) -- [ fermion, \qr, \el'=s] (e)
		-- [ fermion, \qr, \el'=a] (b) -- [ fermion, \hfl, \el=r] (a),
		(a) -- [fermion, \ql, \el=c] (e) -- [fermion,\ql, \el=t] (a),
		%(a) -- [scalar] (c), 	(b) -- [scalar] (d), (e) -- [scalar] (f),
	};
	\end{feynman}
	\end{tikzpicture}
\end{minipage}
~\\~\vspace{10pt}\\
12
\begin{minipage}{0.4\linewidth}
	\centering
	\begin{tikzpicture}
	\begin{feynman}[large]
	\vertex (a); \vertex [below=2.4cm of a] (b);
	\vertex [right=of a] (c); \vertex [right=3.5cm of b] (d);
	\vertex [above=1.2cm of d] (f); \vertex [below=1.2cm of c] (e);	
	\diagram[]{
		(a) -- [fermion,  \el'=a] (e) -- [ fermion, \qr, \el'=s] (c)
		-- [ fermion,  \el=b] (b) -- [ fermion, \ql, \el=r] (a),
		(f) -- [fermion, \ql, \el=c] (d) -- [fermion,\ql, \el=t] (f),
		(a) -- [scalar] (c), 	(b) -- [scalar] (d), (e) -- [scalar] (f),
	};
	\end{feynman}
	\end{tikzpicture}
\end{minipage}
$ \lra  $
\begin{minipage}{0.4\linewidth}
	\centering
	\begin{tikzpicture}
	\begin{feynman}[large]
	\vertex (a); \vertex [below=2.4cm of a] (b);
	%\vertex [right=of a] (c); \vertex [right=3.5cm of b] (d);
	%\vertex [above=1.2cm of d] (f); 
	\vertex [below=1.2cm of a] (e);	
	\diagram[]{
		(a) -- [fermion, \qr, \el'=a] (e) -- [ fermion, \qr, \el'=s] (a)
		-- [ fermion, \hfl, \el=b] (b) -- [ fermion, \hfl, \el=r] (a),
		(e) -- [fermion, \ql, \el=c] (b) -- [fermion,\ql, \el=t] (e),
		%(a) -- [scalar] (c), 	(b) -- [scalar] (d), (e) -- [scalar] (f),
	};
	\end{feynman}
	\end{tikzpicture}
\end{minipage}
~\\
~\\~\\
For the Hugenholtz diagram provided, its value is\\
\begin{minipage}{0.3\linewidth}
	\centering
	\begin{tikzpicture}
	\begin{feynman}[large]
	\vertex (a); \vertex [below=of a] (b);
	%\vertex [right=of a] (c); %\vertex [right=of b] (d);
	\vertex at ($(a)!0.5!(b) $) (e);
	\diagram[%inline=(e.base)
	]{
		(a) -- [anti fermion,  quarter left, edge label=t] (e) 
		-- [anti fermion,  quarter left, edge label=r] (b)
		-- [anti fermion, half left, edge label=a] (a),
		(a) -- [fermion, half left, edge label=b] (b)
		-- [fermion, quarter left, edge label=u] (e)
		-- [fermion, quarter left, edge label=s] (a),
	};
	\end{feynman}
	\end{tikzpicture}
\end{minipage}
\begin{minipage}{0.6\linewidth}
	\vskip0.6cm
\begin{align}
&= \qty(\dfrac{1}{2})^3 (-1)^{2+2} \sum_{a,b,r,s,u,t} \dfrac{\Braket{ab||ru} \Braket{ru||ts}\Braket{ts||ab}}{(\varepsilon_a + \varepsilon_b - \varepsilon_u - \varepsilon_r)(\varepsilon_a + \varepsilon_b - \varepsilon_s - \varepsilon_t)} &&& \notag\\
& =\dfrac{1}{8} \sum_{a,b,r,s,u,t} \dfrac{\Braket{ab||ru} \Braket{ru||ts}\Braket{ts||ab}}{(\varepsilon_a + \varepsilon_b - \varepsilon_u - \varepsilon_r)(\varepsilon_a + \varepsilon_b - \varepsilon_s - \varepsilon_t)}  &&&
\notag
\end{align}
\end{minipage}
\begin{align}
&=\dfrac{1}{8} \sum_{a,b,r,s,u,t} \dfrac{(\Braket{ab|ru}-\Braket{ab|ur}) (\Braket{ru|ts}-\Braket{ru|st}) (\Braket{ts|ab}-\Braket{ts|ba})}{(\varepsilon_a + \varepsilon_b - \varepsilon_u - \varepsilon_r)(\varepsilon_a + \varepsilon_b - \varepsilon_s - \varepsilon_t)}  \notag\\
&= \dfrac{1}{8} \sum_{a,b,r,s,u,t} \dfrac{\Braket{ab|ru} \Braket{ru|ts} \Braket{ts|ab}}{(\varepsilon_a + \varepsilon_b - \varepsilon_u - \varepsilon_r)(\varepsilon_a + \varepsilon_b - \varepsilon_s - \varepsilon_t)} 
- \dfrac{1}{8} \sum_{a,b,r,s,u,t} \dfrac{\Braket{ab|ur} \Braket{ru|ts} \Braket{ts|ab}}{(\varepsilon_a + \varepsilon_b - \varepsilon_u - \varepsilon_r)(\varepsilon_a + \varepsilon_b - \varepsilon_s - \varepsilon_t)} \notag\\
&\quad{} - \dfrac{1}{8} \sum_{a,b,r,s,u,t} \dfrac{\Braket{ab|ru} \Braket{ru|st} \Braket{ts|ab}}{(\varepsilon_a + \varepsilon_b - \varepsilon_u - \varepsilon_r)(\varepsilon_a + \varepsilon_b - \varepsilon_s - \varepsilon_t)} 
+ \dfrac{1}{8} \sum_{a,b,r,s,u,t} \dfrac{\Braket{ab|ur} \Braket{ru|st} \Braket{ts|ab}}{(\varepsilon_a + \varepsilon_b - \varepsilon_u - \varepsilon_r)(\varepsilon_a + \varepsilon_b - \varepsilon_s - \varepsilon_t)} 
\notag\\
&\quad{} - \dfrac{1}{8} \sum_{a,b,r,s,u,t} \dfrac{\Braket{ab|ru} \Braket{ru|ts} \Braket{ts|ba}}{(\varepsilon_a + \varepsilon_b - \varepsilon_u - \varepsilon_r)(\varepsilon_a + \varepsilon_b - \varepsilon_s - \varepsilon_t)} 
+ \dfrac{1}{8} \sum_{a,b,r,s,u,t} + \dfrac{\Braket{ab|ur} \Braket{ru|ts} \Braket{ts|ba}}{(\varepsilon_a + \varepsilon_b - \varepsilon_u - \varepsilon_r)(\varepsilon_a + \varepsilon_b - \varepsilon_s - \varepsilon_t)} \notag\\
&\quad{} + \dfrac{1}{8} \sum_{a,b,r,s,u,t} \dfrac{\Braket{ab|ru} \Braket{ru|st} \Braket{ts|ba}}{(\varepsilon_a + \varepsilon_b - \varepsilon_u - \varepsilon_r)(\varepsilon_a + \varepsilon_b - \varepsilon_s - \varepsilon_t)} 
- \dfrac{1}{8} \sum_{a,b,r,s,u,t} \dfrac{\Braket{ab|ur} \Braket{ru|st} \Braket{ts|ba}}{(\varepsilon_a + \varepsilon_b - \varepsilon_u - \varepsilon_r)(\varepsilon_a + \varepsilon_b - \varepsilon_s - \varepsilon_t)}  \notag\\
&= \dfrac{1}{8} \sum_{a,b,r,s,u,t} \dfrac{\Braket{ab|ru} \Braket{ru|ts} \Braket{ts|ab}}{(\varepsilon_a + \varepsilon_b - \varepsilon_u - \varepsilon_r)(\varepsilon_a + \varepsilon_b - \varepsilon_s - \varepsilon_t)} 
- \dfrac{1}{8} \sum_{a,b,r,s,u,t} \dfrac{\Braket{ab|ur} \Braket{ru|ts} \Braket{ts|ab}}{(\varepsilon_a + \varepsilon_b - \varepsilon_u - \varepsilon_r)(\varepsilon_a + \varepsilon_b - \varepsilon_s - \varepsilon_t)} \notag\\
&\quad{} - \dfrac{1}{8} \sum_{a,b,r,s,u,t} \dfrac{\Braket{ab|ur} \Braket{ur|st} \Braket{ts|ab}}{(\varepsilon_a + \varepsilon_b - \varepsilon_r - \varepsilon_u)(\varepsilon_a + \varepsilon_b - \varepsilon_s - \varepsilon_t)} 
+ \dfrac{1}{8} \sum_{a,b,r,s,u,t} \dfrac{\Braket{ab|ru} \Braket{ur|st} \Braket{ts|ab}}{(\varepsilon_a + \varepsilon_b - \varepsilon_r - \varepsilon_u)(\varepsilon_a + \varepsilon_b - \varepsilon_s - \varepsilon_t)} 
\notag\\
&\quad{} - \dfrac{1}{8} \sum_{a,b,r,s,u,t} \dfrac{\Braket{ab|ru} \Braket{ru|ts} \Braket{ts|ba}}{(\varepsilon_a + \varepsilon_b - \varepsilon_u - \varepsilon_r)(\varepsilon_a + \varepsilon_b - \varepsilon_s - \varepsilon_t)} 
+ \dfrac{1}{8} \sum_{a,b,r,s,u,t} + \dfrac{\Braket{ab|ur} \Braket{ru|ts} \Braket{ts|ba}}{(\varepsilon_a + \varepsilon_b - \varepsilon_u - \varepsilon_r)(\varepsilon_a + \varepsilon_b - \varepsilon_s - \varepsilon_t)} \notag\\
&\quad{} + \dfrac{1}{8} \sum_{a,b,r,s,u,t} \dfrac{\Braket{ab|ur} \Braket{ur|st} \Braket{ts|ba}}{(\varepsilon_a + \varepsilon_b - \varepsilon_r - \varepsilon_u)(\varepsilon_a + \varepsilon_b - \varepsilon_s - \varepsilon_t)} 
- \dfrac{1}{8} \sum_{a,b,r,s,u,t} \dfrac{\Braket{ab|ru} \Braket{ur|st} \Braket{ts|ba}}{(\varepsilon_a + \varepsilon_b - \varepsilon_r - \varepsilon_u)(\varepsilon_a + \varepsilon_b - \varepsilon_s - \varepsilon_t)}  \notag\\
&= \dfrac{1}{4} \sum_{a,b,r,s,u,t} \dfrac{\Braket{ab|ru} \Braket{ru|ts} \Braket{ts|ab}}{(\varepsilon_a + \varepsilon_b - \varepsilon_u - \varepsilon_r)(\varepsilon_a + \varepsilon_b - \varepsilon_s - \varepsilon_t)} 
- \dfrac{1}{4} \sum_{a,b,r,s,u,t} \dfrac{\Braket{ab|ur} \Braket{ru|ts} \Braket{ts|ab}}{(\varepsilon_a + \varepsilon_b - \varepsilon_u - \varepsilon_r)(\varepsilon_a + \varepsilon_b - \varepsilon_s - \varepsilon_t)} \notag\\
&\quad{} - \dfrac{1}{4} \sum_{a,b,r,s,u,t} \dfrac{\Braket{ab|ru} \Braket{ru|ts} \Braket{ts|ba}}{(\varepsilon_a + \varepsilon_b - \varepsilon_u - \varepsilon_r)(\varepsilon_a + \varepsilon_b - \varepsilon_s - \varepsilon_t)} 
+ \dfrac{1}{4} \sum_{a,b,r,s,u,t} + \dfrac{\Braket{ab|ur} \Braket{ru|ts} \Braket{ts|ba}}{(\varepsilon_a + \varepsilon_b - \varepsilon_u - \varepsilon_r)(\varepsilon_a + \varepsilon_b - \varepsilon_s - \varepsilon_t)} \notag\\
&= \dfrac{1}{4} \sum_{a,b,r,s,u,t} \dfrac{\Braket{ab|ru} \Braket{ru|ts} \Braket{ts|ab}}{(\varepsilon_a + \varepsilon_b - \varepsilon_u - \varepsilon_r)(\varepsilon_a + \varepsilon_b - \varepsilon_s - \varepsilon_t)} 
- \dfrac{1}{4} \sum_{a,b,r,s,u,t} \dfrac{\Braket{ab|ur} \Braket{ru|ts} \Braket{ts|ab}}{(\varepsilon_a + \varepsilon_b - \varepsilon_u - \varepsilon_r)(\varepsilon_a + \varepsilon_b - \varepsilon_s - \varepsilon_t)} \notag\\
&\quad{} - \dfrac{1}{4} \sum_{a,b,r,s,u,t} \dfrac{\Braket{ba|ru} \Braket{ru|ts} \Braket{ts|ab}}{(\varepsilon_a + \varepsilon_b - \varepsilon_u - \varepsilon_r)(\varepsilon_a + \varepsilon_b - \varepsilon_s - \varepsilon_t)} 
+ \dfrac{1}{4} \sum_{a,b,r,s,u,t} + \dfrac{\Braket{ba|ur} \Braket{ru|ts} \Braket{ts|ab}}{(\varepsilon_a + \varepsilon_b - \varepsilon_u - \varepsilon_r)(\varepsilon_a + \varepsilon_b - \varepsilon_s - \varepsilon_t)} \notag\\
&= \dfrac{1}{2} \sum_{a,b,r,s,u,t} \dfrac{\Braket{ab|ru} \Braket{ru|ts} \Braket{ts|ab}}{(\varepsilon_a + \varepsilon_b - \varepsilon_u - \varepsilon_r)(\varepsilon_a + \varepsilon_b - \varepsilon_s - \varepsilon_t)} 
- \dfrac{1}{2} \sum_{a,b,r,s,u,t} \dfrac{\Braket{ab|ur} \Braket{ru|ts} \Braket{ts|ab}}{(\varepsilon_a + \varepsilon_b - \varepsilon_u - \varepsilon_r)(\varepsilon_a + \varepsilon_b - \varepsilon_s - \varepsilon_t)} 
\end{align}
=
\begin{minipage}{0.4\linewidth}
	\centering
	\begin{tikzpicture}
	\begin{feynman}[large]
	\vertex (a); \vertex [below=2.4cm of a] (b);
	\vertex [right=2.5cm of a] (c); \vertex [right=2.5cm of b] (d);
	\vertex at ($(a)!0.5!(b) + (0.7cm, 0)$) (e); \vertex at ($(c)!0.5!(d) + (-0.7cm, 0)$) (f);
	
	\diagram[%inline=(e.base)
	]{
		(a) -- [anti fermion,  quarter left, edge label=t] (e) 
		-- [anti fermion,  quarter left, edge label=r] (b)
		-- [anti fermion, quarter left, edge label=a] (a),
		(c) -- [fermion, quarter left, edge label=b] (d)
		-- [fermion, quarter left, edge label=u] (f)
		-- [fermion, quarter left, edge label=s] (c),
		(a) -- [scalar] (c),
		(b) -- [scalar] (d),
		(e) -- [scalar] (f),
		%-- [fermion, quarter right, edge label'=2] (a),
	};
	\end{feynman}
	\end{tikzpicture}
\end{minipage}
+
\begin{minipage}{0.4\linewidth}
	\centering
	\begin{tikzpicture}
	\begin{feynman}[large]
	\vertex (a); \vertex [below=2.4cm of a] (b);
	\vertex [right=2.5cm of a] (c); \vertex [right=2.5cm of b] (d);
	\vertex at ($(a)!0.5!(b) + (0.7cm, 0)$) (e); \vertex at ($(c)!0.5!(d) + (-0.7cm, 0)$) (f);	
	\diagram[]{
		(a) -- [fermion,  \qr, \el'=a] (b) -- [ fermion, \el=u] (f)
		-- [ fermion,  \el'=s] (c) -- [ fermion, \ql, \el=b] (d)
		-- [fermion,  \el'=r] (e) -- [fermion, \el=t] (a),
		(a) -- [scalar] (c), 	(b) -- [scalar] (d), (e) -- [scalar] (f),
	};
	\end{feynman}
	\end{tikzpicture}
\end{minipage}

\subsubsection{Summation of Diagrams}

\subsubsection{What Is the Linked-Cluster Theorem?}
\ex{6.13}
For the 3rd-order Goldstone diagrams in Table 6.2,
\begin{equation}\label{key}
\text{diagram1} = (-1)^4 \qty(\dfrac{1}{2}) \sum_{ab}\sum_{rsut} \dfrac{\Braket{ab|ru}\Braket{ru|ts}\Braket{ts|ab} }{(\varepsilon_a + \varepsilon_b - \varepsilon_r - \varepsilon_u)(\varepsilon_a + \varepsilon_b - \varepsilon_t - \varepsilon_t)}
\end{equation}
$ a,b,r,s,u,t $ must come from 1 or 2 molecules. If they come from 2 molecules, $ \Braket{ru|ts} $ must be zero. Thus they only come from 1 molecule, i.e. the value of each Goldstone diagram is $ N $ times the result for a single molecule.





\subsection{Some Illustrative Calculations}






\end{document}
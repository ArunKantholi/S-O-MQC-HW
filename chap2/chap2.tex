%\documentclass[UTF8]{ctexart} % use larger type; default would be 10pt
\documentclass[a4paper]{article}
\usepackage{xeCJK}
%\usepackage[utf8]{inputenc} % set input encoding (not needed with XeLaTeX)

%%% Examples of Article customizations
% These packages are optional, depending whether you want the features they provide.
% See the LaTeX Companion or other references for full information.

%%% PAGE DIMENSIONS
\usepackage{geometry} % to change the page dimensions
\geometry{a4paper} % or letterpaper (US) or a5paper or....
\geometry{margin=1in} % for example, change the margins to 2 inches all round
% \geometry{landscape} % set up the page for landscape
%   read geometry.pdf for detailed page layout information

\usepackage{graphicx} % support the \includegraphics command and options

% \usepackage[parfill]{parskip} % Activate to begin paragraphs with an empty line rather than an indent

%%% PACKAGES
\usepackage{booktabs} % for much better looking tables
\usepackage{array} % for better arrays (eg matrices) in maths
\usepackage{paralist} % very flexible & customisable lists (eg. enumerate/itemize, etc.)
\usepackage{verbatim} % adds environment for commenting out blocks of text & for better verbatim
\usepackage{subfig} % make it possible to include more than one captioned figure/table in a single float
% These packages are all incorporated in the memoir class to one degree or another...

%%% HEADERS & FOOTERS
\usepackage{fancyhdr} % This should be set AFTER setting up the page geometry
\pagestyle{fancy} % options: empty , plain , fancy
\renewcommand{\headrulewidth}{0pt} % customise the layout...
\lhead{}\chead{}\rhead{}
\lfoot{}\cfoot{\thepage}\rfoot{}

%%% SECTION TITLE APPEARANCE
\usepackage{sectsty}
\allsectionsfont{\sffamily\mdseries\upshape} % (See the fntguide.pdf for font help)
% (This matches ConTeXt defaults)

%%% ToC (table of contents) APPEARANCE
\usepackage[nottoc,notlof,notlot]{tocbibind} % Put the bibliography in the ToC
\usepackage[titles,subfigure]{tocloft} % Alter the style of the Table of Contents
\renewcommand{\cftsecfont}{\rmfamily\mdseries\upshape}
\renewcommand{\cftsecpagefont}{\rmfamily\mdseries\upshape} % No bold!

%%% END Article customizations

%%% The "real" document content comes below...

\setlength{\parindent}{0pt}
\usepackage{physics}
\usepackage{amsmath}
%\usepackage{symbols}
\usepackage{AMSFonts}
\usepackage{bm}
%\usepackage{eucal}
\usepackage{mathrsfs}
\usepackage{amssymb}
\usepackage{float}
\usepackage{multicol}
\usepackage{abstract}
\usepackage{empheq}
\usepackage{extarrows}
\usepackage{textcomp}
\usepackage{fontspec}
\usepackage{braket}
\usepackage{siunitx}
\sisetup{
	separate-uncertainty = true,
	inter-unit-product = \ensuremath{{}\cdot{}}
}
\usepackage{mhchem}

\DeclareMathOperator{\p}{\prime}
\DeclareMathOperator{\ti}{\times}
\DeclareMathOperator{\intinf}{\int_0^\infty}
\DeclareMathOperator{\intdinf}{\int_{-\infty}^\infty}
\DeclareMathOperator{\intzpi}{\int_0^\pi}
\DeclareMathOperator{\intztpi}{\int_0^{2\pi}}
\DeclareMathOperator{\sumninf}{\sum_{n=1}^{\infty}}
\DeclareMathOperator{\sumninfz}{\sum_{n=0}^\infty}
\DeclareMathOperator{\sumiinf}{\sum_{i=1}^{\infty}}
\DeclareMathOperator{\sumiinfz}{\sum_{i=0}^\infty}
\DeclareMathOperator{\sumkinf}{\sum_{k=1}^{\infty}}
\DeclareMathOperator{\sumkinfz}{\sum_{k=0}^\infty}
\DeclareMathOperator{\e}{\mathrm{e}}
\DeclareMathOperator{\I}{\mathrm{i}}
\DeclareMathOperator{\Arg}{\mathrm{Arg}}
\DeclareMathOperator{\ra}{\rightarrow}
\DeclareMathOperator{\llra}{\longleftrightarrow}
\DeclareMathOperator{\lra}{\longrightarrow}
\DeclareMathOperator{\dlra}{\Leftrightarrow}
\DeclareMathOperator{\dra}{\Rightarrow}
\newcommand{\bkk}[1]{\Braket{#1|#1}}
\newcommand{\bk}[2]{\Braket{#1|#2}}
\newcommand{\bkev}[2]{\Braket{#2|#1|#2}}



\DeclareMathOperator{\hV}{\hat{\vb{V}}}

\DeclareMathOperator{\hx}{\hat{\vb{x}}}
\DeclareMathOperator{\hy}{\hat{\vb{y}}}
\DeclareMathOperator{\hz}{\hat{\vb{z}}}

\DeclareMathOperator{\hA}{\hat{\vb{A}}}

\DeclareMathOperator{\hQ}{\hat{\vb{Q}}}
\DeclareMathOperator{\hI}{\hat{\vb{I}}}
\DeclareMathOperator{\psis}{\psi^\ast}
\DeclareMathOperator{\Psis}{\Psi^\ast}
\DeclareMathOperator{\hi}{\hat{\vb{i}}}
\DeclareMathOperator{\hj}{\hat{\vb{j}}}
\DeclareMathOperator{\hk}{\hat{\vb{k}}}
\DeclareMathOperator{\hr}{\hat{\vb{r}}}
\DeclareMathOperator{\hT}{\hat{\vb{T}}}
\DeclareMathOperator{\hH}{\hat{H}}
\DeclareMathOperator{\hh}{\hat{h}}               % helicity
\DeclareMathOperator{\hL}{\hat{\vb{L}}}
\DeclareMathOperator{\hp}{\hat{\vb{p}}}

\DeclareMathOperator{\ha}{\hat{\vb{a}}}
\DeclareMathOperator{\hs}{\hat{\vb{s}}}
\DeclareMathOperator{\hS}{\hat{\vb{S}}}
\DeclareMathOperator{\hSigma}{\hat{\bm\Sigma}}
\DeclareMathOperator{\hJ}{\hat{\vb{J}}}
\DeclareMathOperator{\hP}{\hat{\vb{P}}}          % Parity
\DeclareMathOperator{\hC}{\hat{\vb{C}}} 
\DeclareMathOperator{\Tdv}{-\dfrac{\hbar^2}{2m}\dv[2]{x}}
\DeclareMathOperator{\Tna}{-\dfrac{\hbar^2}{2m}\nabla^2}
\DeclareMathOperator{\vna}{\vnabla}
\DeclareMathOperator{\nna}{\nabla^2}
\newcommand{\naCarExpd}[1]{\pdv[2]{#1}{x} + \pdv[2]{#1}{y} + \pdv[2]{#1}{z}}
\newcommand{\naCyl}{\qty[\dfrac{1}{\rho}\pdv{\rho}\qty(\rho\pdv{\rho}) + \dfrac{1}{\rho^2}\pdv[2]{\phi} + \pdv[2]{z}]}

%% MQC
\DeclareMathOperator{\sH}{\mathscr{H}}
\DeclareMathOperator{\sA}{\mathscr{A}}
\newcommand{\iden}{{\large \bm{1}}}
\newcommand{\qed}{$ \Square $}
\newcommand{\tPhi}{\tilde{\Phi} }
\newcommand{\hsP}{\hat{\mathscr{P}}}
\newcommand{\hsS}{\hat{\mathscr{S}}}

%\DeclareMathOperator{\g#0}{\gamma^0}
%\DeclareMathOperator{\g1}{\gamma^1}
%\DeclareMathOperator{\g2}{\gamma^2}
%\DeclareMathOperator{\g3}{\gamma^3}
%\DeclareMathOperator{\g5}{\gamma^5}
\newcommand{\g}[1]{\gamma^{#1}}
\DeclareMathOperator{\gmuu}{\gamma^\mu}
\DeclareMathOperator{\gmud}{\gamma_\mu}
\newcommand{\G}[2]{g^{#1#2}}

\newcommand{\subsbul}{\subsection*{$ \bullet $}}
\newcommand{\ex}[1]{\paragraph{Ex #1}}
\newcommand{\subex}[1]{\subparagraph{#1}}
\newcommand{\dis}{\displaystyle}


\numberwithin{equation}{subsection}
%\setcounter{secnumdepth}{4}
\setcounter{tocdepth}{4}
\allowdisplaybreaks[1]

\usepackage{xcolor}
\definecolor{codegray}{gray}{0.9}
\newfontfamily\Consolas{Consolas}
\newcommand{\code}[1]{\colorbox{codegray}{{\Consolas#1}}}

\title{\textbf{Modern Quantum Chemistry, Szabo \& Ostlund}\\HW}
\author{wsr
\vspace{5pt}\\
}
\date{\today} % Activate to display a given date or no date (if empty),
         % otherwise the current date is printed 

\begin{document}
% \boldmath

\maketitle

\tableofcontents

\newpage

\setcounter{section}{1}
\section{Many-electron Wave Functions and Operators}
\subsection{The Electronic Problem}
\subsubsection{Atomic Units}
\subsubsection{The B-O Approximation}
\subsubsection{The Antisymmetry or Pauli Exclusion Principle}
\subsection{Orbitals, Slater Determinants, and Basis Functions}
\subsubsection{Spin Orbitals and Spatial Orbitals}
\ex{2.1}
Consider $ \Braket{\chi_k | \chi_m} $. If $ k=m $,
\begin{equation}\label{key}
\Braket{\chi_{2i-1} | \chi_{2i-1}} = \Braket{\psi_i^\alpha | \psi_i^\alpha} \Braket{\alpha|\alpha} = 1
\end{equation}
\begin{equation}\label{key}
\Braket{\chi_{2i} | \chi_{2i}} = \Braket{\psi_i^\beta | \psi_i^\beta} \Braket{\alpha|\alpha} = 1
\end{equation}
thus
\begin{equation}\label{key}
\Braket{\chi_k | \chi_k} = 1
\end{equation}
If $ k\neq m $, three cases may occur as below
\begin{equation}\label{key}
\Braket{\chi_{2i-1} | \chi_{2j-1}} = \Braket{\psi_i^\alpha | \psi_j^\alpha}\Braket{\alpha|\alpha} = 0\cdot 1 = 0 \qquad (i\neq j)
\end{equation}
\begin{equation}
\Braket{\chi_{2i-1} | \chi_{2j}} = \Braket{\psi_i^\alpha | \psi_j^\beta}\Braket{\alpha|\beta} = S_{ij}\cdot 0 = 0
\end{equation}
\begin{equation}\label{key}
\Braket{\chi_{2i} | \chi_{2j}} = \Braket{\psi_i^\beta | \psi_j^\beta}\Braket{\beta|\beta} = 0\cdot 1 = 0 \qquad (i\neq j)
\end{equation}
thus
\begin{equation}\label{key}
\Braket{\chi_k | \chi_m} = 0 \qquad (k\neq m)
\end{equation}
Overall,
\begin{equation}\label{key}
\Braket{\chi_k | \chi_m} = \delta_{km}
\end{equation}

\subsubsection{Hartree Products}
%.
\ex{2.2}
\begin{equation}\label{key}
\begin{aligned}
\mathscr{H}\Psi^{HP} &= \sum_{i=1}^N h(i)\chi_i(\vb{x}_1)\chi_j(\vb{x}_2)\cdots\chi_k(\vb{x}_N)\\
&= \varepsilon_i\chi_i(\vb{x}_1)\chi_j(\vb{x}_2)\cdots\chi_k(\vb{x}_N) + \chi_i(\vb{x}_1)[\varepsilon_j\chi_j(\vb{x}_2)]\cdots\chi_k(\vb{x}_N) + \cdots + \chi_i(\vb{x}_1)\chi_j(\vb{x}_2)\cdots[\varepsilon_k\chi_k(\vb{x}_N)]\\
&= (\varepsilon_i +\varepsilon_j + \cdots + \varepsilon_k)\Psi^{HP}
\end{aligned}
\end{equation}

\subsubsection{Slater Determinants}
%.
\ex{2.3}
\begin{equation}\label{key}
\begin{aligned}
\Braket{\Psi | \Psi} 
&= \dfrac{1}{2}\qty(\Braket{\chi_i|\chi_i}\Braket{\chi_j|\chi_j} - \Braket{\chi_i|\chi_j}\Braket{\chi_j|\chi_i} - \Braket{\chi_j|\chi_i}\Braket{\chi_i|\chi_j} + \Braket{\chi_j|\chi_j}\Braket{\chi_i|\chi_i})\\
&= \dfrac{1}{2}(1+0+0+1) = 1
\end{aligned}
\end{equation}
\ex{2.4}
According to Ex. 2.2, we know that $ \chi_i(\vb{x}_1)\chi_j(\vb{x}_2) $ are an eigenfunction of $ \mathscr{H} $ and has the eigenvalue $ \varepsilon_i \varepsilon_j $. Similarly, we have the same conclusion for $ \chi_i(\vb{x}_2)\chi_j(\vb{x}_1) $.\\
For the antisymmetrized wave function,
\begin{equation}\label{key}
\begin{aligned}
\Braket{\Psi | \mathscr{H} | \Psi} &= \dfrac{1}{2}\left(\Braket{\chi_i(\vb{x}_1)\chi_j(\vb{x}_2) | \mathscr{H} | \chi_i(\vb{x}_1)\chi_j(\vb{x}_2)} - \Braket{\chi_i(\vb{x}_1)\chi_j(\vb{x}_2) | \mathscr{H} | \chi_j(\vb{x}_1)\chi_i(\vb{x}_2)}\right.\\
&\left. \quad - \Braket{\chi_j(\vb{x}_1)\chi_i(\vb{x}_2) | \mathscr{H} | \chi_i(\vb{x}_1)\chi_j(\vb{x}_2)} + \Braket{\chi_j(\vb{x}_1)\chi_i(\vb{x}_2) | \mathscr{H} | \chi_j(\vb{x}_1)\chi_i(\vb{x}_2)}\right)\\
&= \dfrac{1}{2}\qty(\varepsilon_i + \varepsilon_j - 0 - 0 + \varepsilon_i + \varepsilon_j)\\
&= \varepsilon_i + \varepsilon_j
\end{aligned}
\end{equation}

\ex{2.5}
\begin{equation}\label{key}
\begin{aligned}
\Braket{K | L} &= \dfrac{1}{2}\Braket{\chi_i(\vb{x}_1)\chi_j(\vb{x}_2) - \chi_j(\vb{x}_1)\chi_i(\vb{x}_2) | \chi_k(\vb{x}_1)\chi_l(\vb{x}_2) - \chi_l(\vb{x}_1)\chi_k(\vb{x}_2)}\\
&= \dfrac{1}{2}\qty(\Braket{\chi_i|\chi_k}\Braket{\chi_j|\chi_l} - \Braket{\chi_i|\chi_l}\Braket{\chi_j|\chi_k} - \Braket{\chi_j|\chi_k}\Braket{\chi_i|\chi_l} + \Braket{\chi_j|\chi_l}\Braket{\chi_i|\chi_k})\\
&= \dfrac{1}{2}\qty(\delta_{ik}\delta_{jl} - \delta_{il}\delta_{jk} - \delta_{jk}\delta_{il} + \delta_{jl}\delta_{ik})\\
&= \delta_{ik}\delta_{jl} - \delta_{il}\delta_{jk}
\end{aligned}
\end{equation}

\subsubsection{The Hartree-Fock Approximation}
\subsubsection{The Minimal Basis $ \ce{H_2} $ Model}
\ex{2.6}
\begin{equation}\label{key}
\Braket{\psi_1 | \psi_1} = \dfrac{1}{2(1+S_{12})}\qty(\Braket{\phi_1|\phi_1} + 2\Braket{\phi_1|\phi_2} + \Braket{\phi_2|\phi_2}) = \dfrac{2 + 2S_{12}}{2(1+S_{12})} = 1
\end{equation}
\begin{equation}\label{key}
\Braket{\psi_2 | \psi_2} = \dfrac{1}{2(1-S_{12})}\qty(\Braket{\phi_1|\phi_1} - 2\Braket{\phi_1|\phi_2} + \Braket{\phi_2|\phi_2}) = \dfrac{2 - 2S_{12}}{2(1-S_{12})} = 1
\end{equation}
\begin{equation}\label{key}
\Braket{\psi_1 | \psi_2} = \dfrac{1}{2\sqrt{1+S_{12}}\sqrt{1-S_{12}}}\qty(\Braket{\phi_1|\phi_1} - \Braket{\phi_2|\phi_2}) = 0
\end{equation}
\subsubsection{Excited Determinants}
\subsubsection{Form of the Exact Wfn and CI}
\ex{2.7}
Size of full CI matrix
\begin{equation}\label{key}
C_{72}^{42} = 164307576757973059488 \approx \num{1.64e20}
\end{equation}
The number of singly excited determinants
\begin{equation}\label{key}
42\cross 30 = 1260
\end{equation}
The number of doubly excited determinants
\begin{equation}\label{key}
C_{42}^2 C_{30}^2 = 374535
\end{equation}

\subsection{Operators and Matrix Elements}
\subsubsection{Minimal Basis $ \ce{H_2} $ Matrix Elements}
\ex{2.8}\label{2.8}
\begin{equation}\label{key}
\begin{aligned}
\Braket{\Psi_{12}^{34} | h(1) | \Psi_{12}^{34}} &= \dfrac{1}{2}\Braket{\chi_3(\vb{x}_1)\chi_4(\vb{x}_2) - \chi_3(\vb{x}_2)\chi_4(\vb{x}_1) | h(1) | \chi_3(\vb{x}_1)\chi_4(\vb{x}_2) - \chi_3(\vb{x}_2)\chi_4(\vb{x}_1)}\\
&= \dfrac{1}{2}\qty(\Braket{\chi_3|h(1)|\chi_3} - 0 - 0 + \Braket{\chi_4|h(1)|\chi_4})\\
&= \dfrac{1}{2}\qty(\Braket{\chi_3|h(1)|\chi_3} + \Braket{\chi_4|h(1)|\chi_4})
\end{aligned}
\end{equation}
thus
\begin{equation}\label{key}
\Braket{\Psi_{12}^{34} | \mathcal{O}_1 | \Psi_{12}^{34}} = \Braket{3|h|3} + \Braket{4|h|4}
\end{equation}
\begin{equation}\label{key}
\begin{aligned}
\Braket{\Psi_0 | h(1) | \Psi_{12}^{34}} &= \dfrac{1}{2}\Braket{\chi_1(\vb{x}_1)\chi_2(\vb{x}_2) - \chi_2(\vb{x}_2)\chi_1(\vb{x}_1) | h(1) | \chi_3(\vb{x}_1)\chi_4(\vb{x}_2) - \chi_3(\vb{x}_2)\chi_4(\vb{x}_1)}\\
&= \dfrac{1}{2}\qty(0 - 0 - 0 + 0)\\
&= 0
\end{aligned}
\end{equation}
thus
\begin{equation}\label{key}
\Braket{\Psi_0 | \mathcal{O}_1 | \Psi_{12}^{34}} = 0
\end{equation}
Similarly, we get
\begin{equation}\label{key}
\Braket{\Psi_{12}^{34} | \mathcal{O}_1 | \Psi_0} = 0
\end{equation}

\ex{2.9}
From Eq. (2.92) in textbook, we get
\begin{equation}\label{key}
\Braket{\Psi_0 | \mathscr{H} | \Psi_0} = \Braket{1|h|1} + \Braket{2|h|2} + \Braket{12|12} - \Braket{12|21}
\end{equation}
From Ex 2.8, we get
\begin{equation}\label{key}
\Braket{\Psi_0 | \mathcal{O}_1 | \Psi_{12}^{34}} = \Braket{\Psi_{12}^{34} | \mathcal{O}_1 | \Psi_0} = 0
\end{equation}
thus
\begin{equation}\label{key}
\begin{aligned}
\Braket{\Psi_0 | \mathscr{H} | \Psi_{12}^{34}} &= \Braket{\Psi_0 | \mathcal{O}_2 | \Psi_{12}^{34}}\\
&= \dfrac{1}{2}\Braket{\chi_1(\vb{x}_1)\chi_2(\vb{x}_2) - \chi_1(\vb{x}_2)\chi_2(\vb{x}_1) | \dfrac{1}{r_{12}} | \chi_3(\vb{x}_1)\chi_4(\vb{x}_2) - \chi_3(\vb{x}_2)\chi_4(\vb{x}_1)}\\
&= \Braket{12|34} - \Braket{12|43}
\end{aligned}
\end{equation}
\begin{equation}\label{key}
\begin{aligned}
\Braket{\Psi_{12}^{34} | \mathscr{H} | \Psi_0} &= \Braket{\Psi_{12}^{34} | \mathcal{O}_2 | \Psi_0}\\
&= \dfrac{1}{2}\Braket{\chi_3(\vb{x}_1)\chi_4(\vb{x}_2) - \chi_3(\vb{x}_2)\chi_4(\vb{x}_1) | \dfrac{1}{r_{12}} | \chi_1(\vb{x}_1)\chi_2(\vb{x}_2) - \chi_2(\vb{x}_2)\chi_1(\vb{x}_1)}\\
&= \Braket{34|12} - \Braket{34|21}
\end{aligned}
\end{equation}
\begin{equation}\label{key}
\begin{aligned}
\Braket{\Psi_{12}^{34} | \mathscr{H} | \Psi_{12}^{34}} &= \Braket{\Psi_{12}^{34} | h(1) +h(2) + \dfrac{1}{r_{12}} | \Psi_{12}^{34}}\\
&= 2\cross\dfrac{1}{2}\Braket{\chi_3(\vb{x}_1)\chi_4(\vb{x}_2) - \chi_3(\vb{x}_2)\chi_4(\vb{x}_1) | h(1) | \chi_3(\vb{x}_1)\chi_4(\vb{x}_2) - \chi_3(\vb{x}_2)\chi_4(\vb{x}_1)}\\
&+ \dfrac{1}{2}\Braket{\chi_3(\vb{x}_1)\chi_4(\vb{x}_2) - \chi_3(\vb{x}_2)\chi_4(\vb{x}_1) | \dfrac{1}{r_{12}} | \chi_3(\vb{x}_1)\chi_4(\vb{x}_2) - \chi_3(\vb{x}_2)\chi_4(\vb{x}_1)}\\
&= \Braket{3|h|3} + \Braket{4|h|4} + \Braket{34|34} - \Braket{34|43}
\end{aligned}
\end{equation}

\subsubsection{Notations for 1- and 2-Electron Integrals}
\subsubsection{General Rules for Matrix Elements}
%.
\ex{2.10}
\begin{equation}\label{key}
\Braket{K | \mathscr{H} | K} = \sum_m^N [m|h|m] + \dfrac{1}{2}\sum_m^N\sum_n^N \Braket{mn||mn} = \sum_m^N [m|h|m] + \dfrac{1}{2}\sum_m^N\sum_n^N \qty([mm|nn] - [mn|nm])
\end{equation}
When $ m=n $,
\begin{equation}\label{key}
[mm|mm] - [mm|mm] = 0
\end{equation}
thus
\begin{equation}\label{key}
\Braket{K | \mathscr{H} | K} = \sum_m^N [m|h|m] + \dfrac{1}{2}\sum_m^N\sum_{n\neq m}^N \qty([mm|nn] - [mn|nm]) = \sum_m^N [m|h|m] + \sum_m^N\sum_{n> m}^N \qty([mm|nn] - [mn|nm])
\end{equation}
%.
\ex{2.11}
\begin{equation}\label{key}
\begin{aligned}
\Braket{K | \mathscr{H} | K} &= \Braket{K | \mathcal{O}_1 + \mathcal{O}_2 | K} = \sum_m^N [m|h|m] + \sum_m^N\sum_{n>m}^N \Braket{mn||mn}\\
&= \Braket{1|h|1} + \Braket{2|h|2} + \Braket{3|h|3} + \Braket{12||12} + \Braket{13||13} + \Braket{23||23} 
\end{aligned}
\end{equation}

%.
\ex{2.12}
\begin{equation}\label{key}
\begin{aligned}
\Braket{\Psi_0 | \mathscr{H} | \Psi_0} &= \Braket{1|h|1} + \Braket{2|h|2} + \Braket{12||12} \\
&= \Braket{1|h|1} + \Braket{2|h|2} + \Braket{12|12} - \Braket{12|21}
\end{aligned}
\end{equation}
\begin{equation}\label{key}
\begin{aligned}
\Braket{\Psi_0 | \mathscr{H} | \Psi_{12}^{34}} = \Braket{12||34}
= \Braket{12|34} - \Braket{12|43}
\end{aligned}
\end{equation}
\begin{equation}\label{key}
\begin{aligned}
\Braket{\Psi_{12}^{34} | \mathscr{H} | \Psi_0} = \Braket{34||12}
= \Braket{34|12} - \Braket{34|21}
\end{aligned}
\end{equation}
\begin{equation}\label{key}
\begin{aligned}
\Braket{\Psi_{12}^{34} | \mathscr{H} | \Psi_{12}^{34}} &= \Braket{3|h|3} + \Braket{4|h|4} + \Braket{34||34}\\
&= \Braket{3|h|3} + \Braket{4|h|4} + \Braket{34|34} - \Braket{34|43}
\end{aligned}
\end{equation}
Which are exactly the same with Ex 2.9.

\ex{2.13}
if $ a=b $, $ r=s $
\begin{equation}\label{key}
\Braket{\Psi_a^r | \mathcal{O} | \Psi_b^s} = \Braket{\Psi_a^r | \mathcal{O}_1 | \Psi_a^r} = \sum_c^N\Braket{c|h|c} - \Braket{a|h|a} + \Braket{r|h|r}
\end{equation}
if $ a=b $, $ r\neq s $
\begin{equation}\label{key}
\Braket{\Psi_a^r | \mathcal{O} | \Psi_b^s} = \Braket{\Psi_a^r | \mathcal{O}_1 | \Psi_a^s} = \Braket{r|h|s}
\end{equation}
if $ a\neq b $, $ r=s $
\begin{equation}\label{key}
\Braket{\Psi_a^r | \mathcal{O} | \Psi_b^s} = \Braket{\Psi_a^r | \mathcal{O}_1 | \Psi_b^r} = \Braket{\Psi_a^r | \mathcal{O}_1 | -(\Psi_a^r)_b^a} = -\Braket{b|h|a}
\end{equation}
if $ a\neq b $, $ r\neq s $
\begin{equation}\label{key}
\Braket{\Psi_a^r | \mathcal{O} | \Psi_b^s} = \Braket{\Psi_a^r | \mathcal{O}_1 | (\Psi_a^r)_{rb}^{as}} = 0
\end{equation}

\ex{2.14}
\begin{equation}\label{key}
^N E_0 = \sum_m^N \Braket{m|h|m} + \sum_m^M\sum_{n>m}^M \Braket{mn||mn}
\end{equation}
\begin{equation}\label{key}
^{N-1}E_0 = \sum_{m\neq a}^N \Braket{m|h|m} + \sum_{m\neq a}^M\sum_{n>m, n\neq a}^M \Braket{mn||mn}
\end{equation}
\begin{equation}\label{key}
^N E_0 - ^{N-1}E_0 = \Braket{a|h|a} + \sum_{b\neq a}^N \Braket{ab||ab}
\end{equation}

\subsubsection{Derivation of the Rules for Matrix Elements}
\ex{2.15}
\begin{equation}\label{key}
\begin{aligned}
\Braket{\Psi | \mathscr{H} | \Psi} &= \dfrac{1}{N!} \Braket{\sum_{n=1}^{N!} (-1)^{p_n}\mathscr{P}_n\{\chi_i(1)\chi_j(2)\cdots\chi_k(N)\} | \sum_{c=1}^N h(c) | \sum_{m=1}^{N!} (-1)^{p_m}\mathscr{P}_m\{\chi_i(1)\chi_j(2)\cdots\chi_k(N)\}}\\
&= \dfrac{1}{N!} \sum_{n=1}^{N!} \sum_{m=1}^{N!} (-1)^{p_n + p_m} \sum_{c=1}^N \Braket{ \mathscr{P}_n \{\chi_i(1)\chi_j(2)\cdots\chi_k(N)\} |  h(c) | \mathscr{P}_m \{ \chi_i(1)\chi_j(2)\cdots\chi_k(N)\}}\\
\end{aligned}
\end{equation}
Since the integral inside equals $ 0 $ when $ \mathscr{P}_n \neq \mathscr{P}_m $,
\begin{equation}\label{key}
\Braket{\Psi | \mathscr{H} | \Psi} = \dfrac{1}{N!} \sum_{n=1}^{N!} (-1)^{p_n + p_n} (\varepsilon_i+\varepsilon_j+\cdots+\varepsilon_k) = \varepsilon_i+\varepsilon_j+\cdots+\varepsilon_k
\end{equation}

\ex{2.16}
%\begin{equation}\label{key}
%\Braket{K^{HP} | \mathscr{H} | L} = \dfrac{1}{\sqrt{N!}}\Braket{K^{HP} | \mathscr{H} | \sum_{m=1}^{N!}(-1)^{p_m}\mathscr{P}_m L^{HP}}
%\end{equation}
%
\begin{align}\label{key}
\Braket{K | \mathscr{H} | L} &= \dfrac{1}{\sqrt{N!}} \sum_{n=1}^{N!}  \Braket{(-1)^{p_n} \mathscr{P}_n K^{HP} | \mathscr{H} | L} \notag\\
&= \dfrac{1}{\sqrt{N!}} \sum_{n=1}^{N!}  \Braket{  K^{HP} | \mathscr{H} | L} \notag\\
&= \dfrac{1}{\sqrt{N!}} \times N! \Braket{  K^{HP} | \mathscr{H} | L} \notag\\
&= \sqrt{N!} \Braket{  K^{HP} | \mathscr{H} | L}
\end{align}

\subsubsection{Transition from Spin Orbitals to Spatial Orbitals}
\ex{2.17}
\begin{equation}
\begin{aligned}
\ket{1} = \ket{\psi_1\alpha} & \quad \ket{2} = \ket{\psi_1\beta}\\
\ket{3} = \ket{\psi_2\alpha} & \quad \ket{4} = \ket{\psi_2\beta}\
\end{aligned}
\end{equation}
thus
\begin{equation}\label{key}
\begin{aligned}
\vb{H} &= \mqty(\Braket{1|h|1} + \Braket{2|h|2} + \Braket{12|12} - \Braket{12|21} & \Braket{12|34} - \Braket{12|43}\\
\Braket{34|12} - \Braket{34|21} & \Braket{3|h|3} + \Braket{4|h|4} + \Braket{34|34} - \Braket{34|43})\\
&= \mqty(2(1|h|1) + (11|11) & (12|12)\\
(21|21) & 2(2|h|2) + (22|22))
\end{aligned}
\end{equation}

\ex{2.18}
\begin{equation}\label{key}
\begin{aligned}
\abs{\Braket{ab||rs}}^2 &= (\Braket{ab|rs} - \Braket{ab|sr})^*(\Braket{ab|rs} - \Braket{ab|sr})\\
&= \Braket{rs|ab}\Braket{ab|rs} - \Braket{rs|ab}\Braket{ab|sr} - \Braket{sr|ab}\Braket{ab|rs} + \Braket{sr|ab}\Braket{ab|sr}\\
&= [ra|sb][ar|bs] - [ra|sb][as|br] - [sa|rb][ar|bs] + [sa|rb][as|br]\\
&= [ar|bs]^2 - 2[ar|bs][as|br] + [as|br]^2
\end{aligned}
\end{equation}
Let's calculate $ E_0^{(2)} $ term by term.
\begin{equation}\label{key}
\begin{aligned}
\qty(E_0^{(2)})_1 &= \dfrac{1}{4}\sum_{abrs} \dfrac{[ar|bs]^2}{\varepsilon_a+\varepsilon_b-\varepsilon_r-\varepsilon_s}\\
&= \dfrac{1}{4}\sum_{a,b}^{N/2}\sum_{r,s=N/2+1}^K \dfrac{[ar|bs]^2 +[\bar{a}\bar{r}|bs]^2 + [ar|\bar{b}\bar{s}]^2 +[\bar{a}\bar{r}|\bar{b}\bar{s}]^2}{\varepsilon_a+\varepsilon_b-\varepsilon_r-\varepsilon_s}\\
&= \sum_{a,b}^{N/2}\sum_{r,s=N/2+1}^K \dfrac{[ar|bs]^2}{\varepsilon_a+\varepsilon_b-\varepsilon_r-\varepsilon_s}\\
&= \sum_{a,b}^{N/2}\sum_{r,s=N/2+1}^K \dfrac{\Braket{ab|rs}\Braket{rs|ab}}{\varepsilon_a+\varepsilon_b-\varepsilon_r-\varepsilon_s}
\end{aligned}
\end{equation}
\begin{equation}\label{key}
\begin{aligned}
\qty(E_0^{(2)})_2 &= \dfrac{1}{4}\sum_{abrs} \dfrac{-2[ar|bs][as|br]}{\varepsilon_a+\varepsilon_b-\varepsilon_r-\varepsilon_s}\\
&= -\dfrac{1}{2}\sum_{a,b}^{N/2}\sum_{r,s=N/2+1}^K \dfrac{[ar|bs][as|br] +[\bar{a}\bar{r}|\bar{b}\bar{s}][\bar{a}\bar{s}|\bar{b}\bar{r}]}{\varepsilon_a+\varepsilon_b-\varepsilon_r-\varepsilon_s}\\
&= -\sum_{a,b}^{N/2}\sum_{r,s=N/2+1}^K \dfrac{[ar|bs][as|br]}{\varepsilon_a+\varepsilon_b-\varepsilon_r-\varepsilon_s}\\
&= -\sum_{a,b}^{N/2}\sum_{r,s=N/2+1}^K \dfrac{\Braket{ab|rs}\Braket{rs|ba}}{\varepsilon_a+\varepsilon_b-\varepsilon_r-\varepsilon_s}
\end{aligned}
\end{equation}
\begin{equation}\label{key}
\begin{aligned}
\qty(E_0^{(2)})_3 &= \dfrac{1}{4}\sum_{abrs} \dfrac{[as|br]^2}{\varepsilon_a+\varepsilon_b-\varepsilon_r-\varepsilon_s} = \dfrac{1}{4}\sum_{absr}\dfrac{[ar|bs]^2}{\varepsilon_a+\varepsilon_b-\varepsilon_s-\varepsilon_r}\\
&= \sum_{a,b}^{N/2}\sum_{r,s=N/2+1}^K \dfrac{\Braket{ab|rs}\Braket{rs|ab}}{\varepsilon_a+\varepsilon_b-\varepsilon_r-\varepsilon_s}
\end{aligned}
\end{equation}
thus,
\begin{equation}\label{key}
E_0^{(2)} = \sum_{a,b}^{N/2}\sum_{r,s=N/2+1}^K \dfrac{\Braket{ab|rs}(2\Braket{rs|ab} - \Braket{rs|ba})}{\varepsilon_a+\varepsilon_b-\varepsilon_r-\varepsilon_s}
\end{equation}

\subsubsection{Coulomb and Exchange Integrals}
\ex{2.19}
\begin{equation}\label{key}
J_{ii} = (ii|ii) = K_{ii} 
\end{equation}
\begin{equation}\label{key}
J_{ij}^* = \Braket{ij|ij}^* = \Braket{ij|ij} = J_{ij}
\end{equation}
\begin{equation}\label{key}
K_{ij}^* = \Braket{ij|ji}^* = \Braket{ji|ij} = \Braket{ij|ji} = K_{ij}
\end{equation}
\begin{equation}\label{key}
J_{ij} = (ii|jj) = (jj|ii) = J_{ji}
\end{equation}
\begin{equation}\label{key}
K_{ij} = (ij|ji) = (ji|ij) = K_{ji}
\end{equation}

\ex{2.20}
For real spatial orbitals
\begin{equation}\label{key}
K_{ij} = (ij|ji) = (ij|ij) = (ji|ji)
\end{equation}
\begin{equation}\label{key}
K_{ij} = \Braket{ij|ji} = \Braket{ii|jj} = \Braket{jj|ii}
\end{equation}

%.
\ex{2.21}
\begin{equation}\label{key}
\vb{H} = \mqty(2(1|h|1) + (11|11) & (12|12)\\
(21|21) & 2(2|h|2) + (22|22)) = 
\mqty(2h_{11} + J_{11} & K_{12}\\
      K_{12} & 2h_{22} + J_{22})
\end{equation}

\ex{2.22}
\begin{equation}\label{key}
E_{\uparrow\downarrow}^{HP} = \Braket{\Psi_{\uparrow\downarrow}^{HP} | h(1) + h(2) + \dfrac{1}{r_{12}} | \Psi_{\uparrow\downarrow}^{HP}} = (1|h|1)+(2|h|2)+(11|22) = h_{11}+h_{22}+J_{12}
\end{equation}
\begin{equation}\label{key}
E_{\downarrow\downarrow}^{HP} = \Braket{\Psi_{\downarrow\downarrow}^{HP} | h(1) + h(2) + \dfrac{1}{r_{12}} | \Psi_{\downarrow\downarrow}^{HP}} = (1|h|1)+(2|h|2)+(11|22) = h_{11}+h_{22}+J_{12}
\end{equation}

\subsubsection{Pseudo-Classical Interpretation of Determinantal Energies}
%.
\ex{2.23}
a.-g. can be obtained immediately with definition.

\subsection{Second Quantization}
\subsubsection{Creation and Annihilation Operators and Their Anticommutation Relations}
\ex{2.24}
Since $ a_i^\dagger a_j^\dagger + a_j^\dagger a_i^\dagger = 0 $, we have
\begin{equation}\label{key}
(a_1^\dagger a_2^\dagger + a_2^\dagger a_1^\dagger)\ket{K} = 0
\end{equation}
for any $ \ket{K} $.

\ex{2.25}
Since $ a_i a_j^\dagger + a_j^\dagger a_i = \delta_{ij} $, we have
\begin{equation}\label{key}
(a_1 a_2^\dagger + a_2^\dagger a_1)\ket{K} = 0
\end{equation}
\begin{equation}\label{key}
(a_1 a_1^\dagger + a_1^\dagger a_1)\ket{K} = \ket{K}
\end{equation}
for any $ \ket{K} $.

\ex{2.26}
\begin{equation}\label{key}
\Braket{\chi_i|\chi_j} = \Braket{0 | a_i a_j^\dagger | 0} = \Braket{0 | \delta_{ij} - a_j^\dagger a_i | 0} = \delta_{ij}
\end{equation}
where $ \ket{0} $ is the vacuum state.

\ex{2.27}
First, if $ i\notin \{1,2,\cdots,N\} $ or $ j\notin \{1,2,\cdots,N\} $, \begin{equation}\label{key}
\Braket{K | a_i^\dagger a_j | K} = 0
\end{equation}
because inexistent electron cannot be annihilated.\\
Thus, $ i,j\in \{1,2,\cdots,N\} $, and
\begin{equation}\label{key}
\Braket{K | a_i^\dagger a_j | K} = \delta_{ij}\Braket{K|K} - \Braket{K | a_j a_i^\dagger | K}
\end{equation}
$ \Braket{K | a_j a_i^\dagger | K} $ would be $ 0 $ because $ \chi_i $ is created twice. Thus,
\begin{equation}\label{key}
\Braket{K | a_i^\dagger a_j | K} = \delta_{ij}
\end{equation}
Overall, $ \Braket{K | a_i^\dagger a_j | K} = 1$ when $ i=j $ and $ i\in \{1,2,\cdots,N\} $, but is $ 0 $ otherwise.

\ex{2.28}
\subex{a.}
That's obvious since inexistent electron cannot be annihilated.
\subex{b.}
That's obvious since an electron cannot be created twice.
\subex{c.}
\begin{equation}\label{key}
\begin{aligned}
a_r^\dagger a_a\ket{\Psi_0} &= a_r^\dagger a_a (-\ket{\chi_a\cdots\chi_1\chi_b\cdots\chi_N})\\
&= -a_r^\dagger\ket{\cdots\chi_1\chi_b\cdots\chi_N}\\
&= -\ket{\chi_r\cdots\chi_1\chi_b\cdots\chi_N}\\
&= \ket{\chi_1\cdots\chi_r\chi_b\cdots\chi_N}\\
&= \ket{\Psi_a^r}
\end{aligned}
\end{equation}
\subex{d.}
That's similar to 2.28.c.
\subex{e.}
\begin{equation}\label{key}
\begin{aligned}
a_s^\dagger a_b a_r^\dagger a_a\ket{\Psi_0} 
&= a_s^\dagger a_b a_r^\dagger (-\ket{\chi_2\cdots\chi_1\chi_b\cdots\chi_N}) \\
&= -a_s^\dagger a_b \ket{\chi_r\chi_2\cdots\chi_1\chi_b\cdots\chi_N}\\
&= -a_s^\dagger (-\ket{\chi_2\cdots\chi_1\chi_r\cdots\chi_N}) \\
&= \ket{\chi_s\chi_2\cdots\chi_1\chi_r\cdots\chi_N}\\
&= \ket{\chi_1\cdots\chi_r\chi_s\cdots\chi_N}\\
&= \ket{\Psi_{ab}^{rs}}
\end{aligned}
\end{equation}
$ \therefore $
\begin{equation}\label{key}
\ket{\Psi_{ab}^{rs}} = a_s^\dagger a_b a_r^\dagger a_a\ket{\Psi_0} = a_s^\dagger (-a_r^\dagger a_b) a_a\ket{\Psi_0} = a_r^\dagger a_s^\dagger  a_b a_a\ket{\Psi_0} 
\end{equation}
\subex{f.}
That's similar to 2.28.e.

\subsubsection{Second-Quantized Operators and Their Matrix Elements}
\ex{2.29}
\begin{equation}\label{key}
\begin{aligned}
\Braket{\Psi_0 | \mathcal{O}_1 | \Psi_0} &= \sum_{ij}\Braket{i|h|j}\Bra{0}a_2 a_1 a_i^\dagger a_j a_1^\dagger a_2^\dagger \ket{0}\\
&= \sum_{ij}\Braket{i|h|j}\Bra{0}a_2 a_1 (\delta_{ij} -  a_j^\dagger a_i) a_1^\dagger a_2^\dagger \ket{0} \\
&= \sum_i\Braket{i|h|i} \Bra{0}a_2 a_1 a_1^\dagger a_2^\dagger \ket{0} - \sum_{ij}\Braket{i|h|j}\Bra{0}a_2 a_1 a_j a_i^\dagger a_1^\dagger  a_2^\dagger \ket{0}
\end{aligned}
\end{equation}
The second terms must be $ 0 $ since $ i\in{1,2} $.\\
Thus,
\begin{equation}\label{key}
\begin{aligned}
\Braket{\Psi_0 | \mathcal{O}_1 | \Psi_0} &= \sum_i\Braket{i|h|i} \Bra{0}a_2 a_1 a_1^\dagger a_2^\dagger \ket{0} = \Braket{1|h|1} +\Braket{2|h|2}
\end{aligned}
\end{equation}

\ex{2.30}
%\begin{equation}
\allowdisplaybreaks
\begin{align}
\Braket{\Psi_a^r | \mathcal{O}_1 | \Psi_0} &= \sum_{ij}\Braket{i|h|j}\Braket{\Psi_0 | a_a^\dagger a_r a_i^\dagger a_j | \Psi_0} = \sum_{ij}\Braket{i|h|j}\Braket{\Psi_0 | a_a^\dagger (\delta_{ri} -  a_i^\dagger a_r) a_j | \Psi_0}\notag\\
&= \sum_j\Braket{r|h|j}\Braket{\Psi_0 | a_a^\dagger a_j|\Psi_0} - \sum_{ij}\Braket{i|h|j}\Braket{\Psi_0 | a_a^\dagger a_i^\dagger a_r a_j | \Psi_0}\notag\\
%\displaybreak
&= \sum_j\Braket{r|h|j}\Braket{\Psi_0 | (\delta_{aj} - a_j a_a^\dagger)|\Psi_0}\notag\\
&= \Braket{r|h|a}\Braket{\Psi_0 | \Psi_0} - \sum_j\Braket{r|h|j}\Braket{\Psi_0 | a_j a_a^\dagger|\Psi_0}\notag\\
&= \Braket{r|h|a}
\end{align}
%\end{equation}

\ex{2.31}
\begin{align}
\Braket{\Psi_a^r | \mathcal{O}_2 | \Psi_0} &= \dfrac{1}{2} \sum_{ijkl} \Braket{ij|kl} \Braket{\Psi_0 | a_a^\dagger a_r a_i^\dagger a_j^\dagger a_l a_k | \Psi_0} 
\end{align}
while
\begin{align}
\Braket{\Psi_0 | a_a^\dagger a_r a_i^\dagger a_j^\dagger a_l a_k | \Psi_0} 
&= \Braket{\Psi_0 | a_a^\dagger \delta_{ri} a_j^\dagger a_l a_k | \Psi_0} - \Braket{\Psi_0 | a_a^\dagger a_i^\dagger a_r a_j^\dagger a_l a_k | \Psi_0} \notag\\
&= \delta_{ri} \qty(\Braket{\Psi_0 | a_j^\dagger \delta_{ak} a_l | \Psi_0}  - \Braket{\Psi_0 | a_j^\dagger a_k a_a^\dagger a_l | \Psi_0} ) \notag\\ 
  & \quad{} - \qty(\Braket{\Psi_0 | a_a^\dagger a_i^\dagger \delta_{rj} a_l a_k | \Psi_0} - \Braket{\Psi_0 | a_a^\dagger a_i^\dagger a_j^\dagger a_r a_l a_k | \Psi_0})\notag\\
&= \delta_{ri}\delta_{ak} \Braket{\Psi_0 | a_j^\dagger  a_l | \Psi_0}  - \delta_{ri}\delta_{al}\Braket{\Psi_0 | a_j^\dagger a_k | \Psi_0} \notag\\ 
  & \quad{} - \delta_{rj}\qty(\Braket{\Psi_0 | a_i^\dagger \delta_{ak} a_l| \Psi_0} - \Braket{\Psi_0 | a_i^\dagger a_k a_a^\dagger  a_l | \Psi_0} ) + 0 \notag\\
&= \delta_{ri}\delta_{ak} \Braket{\Psi_0 | a_j^\dagger  a_l | \Psi_0}  - \delta_{ri}\delta_{al}\Braket{\Psi_0 | a_j^\dagger a_k | \Psi_0} \notag\\ 
  & \quad{} - \delta_{rj}\delta_{ak} \Braket{\Psi_0 | a_i^\dagger a_l| \Psi_0} + \delta_{rj}\delta_{al} \Braket{\Psi_0 | a_i^\dagger a_k | \Psi_0}
\end{align}
According to Ex. 2.27, we have
\begin{align}
\Braket{\Psi_a^r | \mathcal{O}_2 | \Psi_0} &= \dfrac{1}{2} \left( \sum_{jl}\Braket{rj|al}\Braket{\Psi_0 | a_j^\dagger  a_l | \Psi_0} - \sum_{jk}\Braket{rj|ka}\Braket{\Psi_0 | a_j^\dagger  a_k | \Psi_0}\right. \notag\\
&  \left. \hspace{30pt} {} - \sum_{il}\Braket{ir|al}\Braket{\Psi_0 | a_i^\dagger  a_l | \Psi_0} + \sum_{ik}\Braket{ir|ka}\Braket{\Psi_0 | a_i^\dagger  a_k | \Psi_0}\right) \notag\\
&= \dfrac{1}{2}\qty(\sum_j^N\Braket{rj|aj} - \sum_j^N\Braket{rj|ja} - \sum_i^N\Braket{ir|ai} + \sum_i^N\Braket{ir|ia}) \notag\\
&= \sum_j^N\Braket{rj|aj} - \sum_j^N\Braket{rj|ja} \notag\\
&= \sum_j^N\Braket{rj||aj}
\end{align}

\subsection{Spin-Adapted Configurations}
\subsubsection{Spin Operators}
\ex{2.32}
\subex{a)}
\begin{align}
\hs_+\ket{\alpha} &= (\hs_x + \I\hs_y)\ket{\alpha} = \qty(\dfrac{1}{2} + \I\dfrac{\I}{2})\ket{\beta} = 0\\
\hs_+\ket{\beta} &= (\hs_x + \I\hs_y)\ket{\beta} = \qty(\dfrac{1}{2} - \I\dfrac{\I}{2})\ket{\alpha} = \ket{\alpha}\\
\hs_-\ket{\alpha} &= (\hs_x - \I\hs_y)\ket{\alpha} = \qty(\dfrac{1}{2} - \I\dfrac{\I}{2})\ket{\beta} = \ket{\beta}\\
\hs_-\ket{\beta} &= (\hs_x - \I\hs_y)\ket{\beta} = \qty(\dfrac{1}{2} + \I\dfrac{\I}{2})\ket{\alpha} = 0
\end{align}
\subex{b)}
\begin{align}
\hs_+\hs_- &= (\hs_x + \I\hs_y)(\hs_x - \I\hs_y) = \hs_x^2 + \hs_y^2 + \I(\hs_y\hs_x - \hs_x\hs_y) = \hs_x^2 + \hs_y^2 + \hs_z\\\
\hs_-\hs_+ &= (\hs_x - \I\hs_y)(\hs_x + \I\hs_y) = \hs_x^2 + \hs_y^2 + \I(\hs_x\hs_y - \hs_y\hs_x) = \hs_x^2 + \hs_y^2 - \hs_z
\end{align}
thus,
\begin{align}
\hs^2 &= \hs_x^2 + \hs_y^2 + \hs_z^2 = \hs_+\hs_- - \hs_z + \hs_z^2\\
&= \hs_-\hs_+ + \hs_z + \hs_z^2
\end{align}

\ex{2.33}
\begin{equation}\label{key}
\hs^2 = \mqty(\dfrac{3}{4} & 0\\ 0 & \dfrac{3}{4}) \quad 
\hs_z = \mqty(\dfrac{1}{2} & 0\\ 0 & -\dfrac{1}{2}) \quad
\hs_+ = \mqty(0 & 1\\ 0 & 0) \quad \hs_- = \mqty(0 & 0\\ 1 & 0)
\end{equation}
thus
\begin{align}
\hs_+\hs_- - \hs_z + \hs_z^2 &= \mqty(1 & 0\\ 0 & 0) - \mqty(\dfrac{1}{2} & 0\\ 0 & -\dfrac{1}{2}) + \mqty(\dfrac{1}{4} & 0\\ 0 & \dfrac{1}{4}) = \mqty(\dfrac{3}{4} & 0\\ 0 & \dfrac{3}{4}) = \hs^2\\
\hs_-\hs_+ + \hs_z + \hs_z^2 &= \mqty(0 & 0\\ 0 & 1) + \mqty(\dfrac{1}{2} & 0\\ 0 & -\dfrac{1}{2}) + \mqty(\dfrac{1}{4} & 0\\ 0 & \dfrac{1}{4}) = \mqty(\dfrac{3}{4} & 0\\ 0 & \dfrac{3}{4}) = \hs^2
\end{align}

\ex{2.34}
\begin{align}
[\hs^2, \hs_z] &= [\hs_+\hs_- - \hs_z + \hs_z^2, \hs_z] \notag\\
&= \hs_+[\hs_-, \hs_z] + [\hs_+, \hs_z]\hs_- - 0 + 0 \notag\\
&= \hs_+[\hs_x - \I\hs_y, \hs_z] + [\hs_x + \I\hs_y, \hs_z]\hs_- \notag\\
&= \hs_+(-\I\hs_y -\I\cdot\I\hs_x) + (-\I\hs_y + \I\cdot\I\hs_x)\hs_- \notag\\
&= \hs_+ \hs_- - \hs_+\hs_- \notag\\
&= 0
\end{align}

\ex{2.35}
\begin{equation}\label{key}
\mathscr{H} \mathscr{A}\ket{\Phi} =  \mathscr{A}\mathscr{H} \ket{\Phi} =  \mathscr{A} E \ket{\Phi} = E\mathscr{A} \ket{\Phi}
\end{equation}
thus $ \mathscr{A} \ket{\Phi} $ is also an eigenfunction of $ \mathscr{H} $ with eigenvalue $ E $.

\ex{2.36}
\begin{equation}\label{key}
\Braket{\Psi_1 | \mathscr{H} \mathscr{A} | \Psi_2} = a_2 \Braket{\Psi_1 | \sH | \Psi_2}
\end{equation}
Since $ [\sA, \sH]=0 $ and $ \sA $ is Hermitian,
\begin{equation}\label{key}
\Braket{\Psi_1 | \mathscr{H} \mathscr{A} | \Psi_2} = \Braket{\Psi_1 | \mathscr{A}\mathscr{H}  | \Psi_2} = \Braket{\Psi_1 | \mathscr{A}^\dagger \mathscr{H}  | \Psi_2} = a_1 \Braket{\Psi_1 | \mathscr{H} | \Psi_2}
\end{equation}
thus
\begin{equation}\label{key}
(a_1 - a_2) \Braket{\Psi_1 | \sH |\Psi_2} = 0
\end{equation}
Since $ a_1 \neq a_2 $,
\begin{equation}\label{key}
\Braket{\Psi_1 | \sH |\Psi_2} = 0
\end{equation}

\ex{2.37}
\begin{align}
\hsS_z \ket{\chi_i\chi_j\cdots\chi_k} &= \hsS_z \dfrac{1}{\sqrt{N!}}\sum_{n=1}^{N!}(-1)^{p_n} \hsP_n\{\chi_i(1)\chi_j(2)\cdots\chi_k(N)\} \notag\\
&= \dfrac{1}{\sqrt{N!}}\sum_{n=1}^{N!}(-1)^{p_n} \hsP_n\{\hsS_z \chi_i(1)\chi_j(2)\cdots\chi_k(N)\} \notag\\
&= \dfrac{1}{\sqrt{N!}}\sum_{n=1}^{N!}(-1)^{p_n} \hsP_n\qty{\sum_{i=1}^{N} \hs_z(i)\chi_i(1)\chi_j(2)\cdots\chi_k(N)} \notag\\
&= \dfrac{1}{\sqrt{N!}}\sum_{n=1}^{N!}(-1)^{p_n} \hsP_n\qty{\qty(\dfrac{1}{2}N^\alpha - \dfrac{1}{2}N^\beta)\chi_i(1)\chi_j(2)\cdots\chi_k(N)} \notag\\
&= \dfrac{1}{2}(N^\alpha - N^\beta) \ket{\chi_i\chi_j\cdots\chi_k} 
\end{align}

\subsubsection{Restricted Determinants and Spin-Adapted Configurations}
\ex{2.38}
From Ex 2.37, we have
\begin{equation}\label{key}
\hsS_z \ket{\psi_i\bar{\psi}_i\psi_j\bar{\psi}_j\cdots} = 0
\end{equation}
thus
\begin{equation}\label{key}
\hsS_z^2 \ket{\psi_i\bar{\psi}_i\psi_j\bar{\psi}_j\cdots} = 0
\end{equation}
While
\begin{align}
\hsS_+ \ket{\psi_i\bar{\psi}_i\cdots\psi_k\bar{\psi}_k\cdots} &= \hsS_+ \dfrac{1}{\sqrt{N!}}\sum_{n=1}^{N!}(-1)^{p_n} \hsP_n\{\psi_i\bar{\psi}_i\cdots\psi_k\bar{\psi}_k\cdots\} \notag\\
&= \dfrac{1}{\sqrt{N!}}\sum_{n=1}^{N!}(-1)^{p_n} \hsP_n\{\hsS_+ \psi_i\bar{\psi}_i\cdots\psi_k\bar{\psi}_k\cdots\}  \notag\\\
&= \dfrac{1}{\sqrt{N!}}\sum_{n=1}^{N!}(-1)^{p_n} \hsP_n\qty{\sum_a^N \hs_+(a) \psi_i\bar{\psi}_i\cdots\psi_k\bar{\psi}_k\cdots}  \notag\\
&= \sum_a^N \dfrac{1}{\sqrt{N!}}\sum_{n=1}^{N!}(-1)^{p_n} \hsP_n\qty{ \hs_+(a) \psi_i\bar{\psi}_i\cdots\psi_k\bar{\psi}_k\cdots}  
\end{align}
Since 
\begin{equation}\label{key}
\hs_+(a)\psi_k(a) = 0 \quad \hs_+(a)\bar{\psi}_k(a) = \psi_k(a)
\end{equation}
\begin{align}
\hsS_+ \ket{\psi_i\bar{\psi}_i\cdots\psi_k\bar{\psi}_k\cdots} 
= \sum_a^N 0 = 0
\end{align}
thus
\begin{equation}\label{key}
\hsS_-\hsS_+ \ket{\psi_i\bar{\psi}_i\psi_j\bar{\psi}_j\cdots} 
= 0
\end{equation}
Therefore,
\begin{equation}\label{key}
\hsS^2 \ket{\psi_i\bar{\psi}_i\psi_j\bar{\psi}_j\cdots}  = (\hsS_-\hsS_+ + \hsS_z + \hsS_z^2)\ket{\psi_i\bar{\psi}_i\psi_j\bar{\psi}_j\cdots} = 0
\end{equation}

\ex{2.39}
\subex{$ \bullet $}
\begin{align}
\hsS^2\ket{^1\Psi_1^2} &= (\hsS_-\hsS_+ + \hsS_z + \hsS_z^2)\dfrac{1}{2}(\psi_1(1)\psi_2(2) + \psi_2(1)\psi_1(2))(\alpha(1)\beta(2) - \beta(1)\alpha(2)) \notag\\
&= \dfrac{1}{2}(\psi_1(1)\psi_2(2) + \psi_2(1)\psi_1(2))(\hsS_-\hsS_+ + \hsS_z + \hsS_z^2)(\alpha(1)\beta(2) - \beta(1)\alpha(2))
\end{align}
$ \because $
\begin{align}
\hsS_-\hsS_+ (\alpha(1)\beta(2) - \beta(1)\alpha(2)) &= \hsS_-(\alpha(1)\alpha(2) - \alpha(1)\alpha(2)) = 0\\
\hsS_z (\alpha(1)\beta(2) - \beta(1)\alpha(2)) &= [1/2+(-1/2)]\alpha(1)\beta(2) - [-1/2+1/2]\beta(1)\alpha(2) = 0
\end{align}
$ \therefore $
\begin{equation}\label{key}
\hsS^2\ket{^1\Psi_1^2} = 0
\end{equation}
thus $ \ket{^1\Psi_1^2} $ is singlet.
\subex{$ \bullet $}
\begin{align}
\hsS^2\ket{^3\Psi_1^2} &= (\hsS_-\hsS_+ + \hsS_z + \hsS_z^2)\dfrac{1}{2}(\psi_1(1)\psi_2(2) - \psi_2(1)\psi_1(2))(\alpha(1)\beta(2) + \beta(1)\alpha(2)) \notag\\
&= \dfrac{1}{2}(\psi_1(1)\psi_2(2) + \psi_2(1)\psi_1(2))(\hsS_-\hsS_+ + \hsS_z + \hsS_z^2)(\alpha(1)\beta(2) - \beta(1)\alpha(2))
\end{align}
$ \because $
\begin{align}
\hsS_-\hsS_+ (\alpha(1)\beta(2) + \beta(1)\alpha(2)) &= \hsS_-(\alpha(1)\alpha(2) + \alpha(1)\alpha(2)) = 2(\alpha(1)\beta(2) + \beta(1)\alpha(2))\\
\hsS_z (\alpha(1)\beta(2) + \beta(1)\alpha(2)) &= [1/2+(-1/2)]\alpha(1)\beta(2) + [-1/2+1/2]\beta(1)\alpha(2) = 0
\end{align}
$ \therefore $
\begin{equation}\label{key}
\hsS^2\ket{^3\Psi_1^2} = 2\ket{^3\Psi_1^2}
\end{equation}
i.e. $ S=1 $,\\
thus $ \ket{^3\Psi_1^2} $ is triplet.

\subex{$ \bullet $}
\begin{align}
\hsS^2\ket{\Psi_1^{\bar{2}}} &= (\hsS_-\hsS_+ + \hsS_z + \hsS_z^2)\dfrac{-1}{2}(\psi_1(1)\psi_2(2) - \psi_2(1)\psi_1(2))\beta(1)\beta(2) \notag\\
&= \dfrac{-1}{2}(\psi_1(1)\psi_2(2) + \psi_2(1)\psi_1(2))(\hsS_-\hsS_+ + \hsS_z + \hsS_z^2)\beta(1)\beta(2)
\end{align}
$ \because $
\begin{align}
\hsS_-\hsS_+ \beta(1)\beta(2) &= \hsS_-(\alpha(1)\beta(2) + \beta(1)\alpha(2)) = 2\beta(1)\beta(2) \\
\hsS_z \beta(1)\beta(2) &= -\beta(1)\beta(2)\\
\hsS_z^2 \beta(1)\beta(2) &= \beta(1)\beta(2)\\
\end{align}
$ \therefore $
\begin{align}
\hsS^2\ket{\Psi_1^{\bar{2}}} &= 2\ket{\Psi_1^{\bar{2}}}
\end{align}
i.e. $ S=1 $,\\
thus $ \ket{\Psi_1^{\bar{2}}} $ is triplet.

\subex{$ \bullet $}
\begin{align}
\hsS^2\ket{\Psi_{\bar{1}}^2} &= (\hsS_-\hsS_+ + \hsS_z + \hsS_z^2)\dfrac{1}{2}(\psi_1(1)\psi_2(2) - \psi_2(1)\psi_1(2))\alpha(1)\alpha(2) \notag\\
&= \dfrac{1}{2}(\psi_1(1)\psi_2(2) + \psi_2(1)\psi_1(2))(\hsS_-\hsS_+ + \hsS_z + \hsS_z^2)\alpha(1)\alpha(2)
\end{align}
$ \because $
\begin{align}
\hsS_-\hsS_+ \alpha(1)\alpha(2) &= 0 \\
\hsS_z \alpha(1)\alpha(2) &= \alpha(1)\alpha(2)\\
\hsS_z^2 \alpha(1)\alpha(2) &= \alpha(1)\alpha(2)\\
\end{align}
$ \therefore $
\begin{align}
\hsS^2\ket{\Psi_{\bar{1}}^2} &= 2\ket{\Psi_{\bar{1}}^2}
\end{align}
i.e. $ S=1 $,\\
thus $ \ket{\Psi_{\bar{1}}^2} $ is triplet.

\ex{2.40}
\subex{$ \bullet $}
\begin{align}
\Braket{^1\Psi_1^2 | \sH | ^1\Psi_1^2} &= \dfrac{1}{4} \Braket{\psi_1(1)\psi_2(2) + \psi_1(2)\psi_2(1) | \sH | \psi_1(1)\psi_2(2) + \psi_1(2)\psi_2(1)} \notag\\
  &\quad{} \Braket{\alpha(1)\beta(2) - \beta(1)\alpha(2) | \alpha(1)\beta(2) - \beta(1)\alpha(2)}\notag\\
&= \dfrac{1}{4}((1|h|1) + (2|h|2) + (11|22) + (12|21) + (21|12) + (2|h|2) + (1|h|1) + (22|11) ) (1 - 0 - 0 + 1) \notag\\
&= h_{11} + h_{22} + J_{12} + K_{12}
\end{align}

\subex{$ \bullet $}
\begin{align}
\Braket{^3\Psi_1^2 | \sH | ^3\Psi_1^2} &= \dfrac{1}{4} \Braket{\psi_1(1)\psi_2(2) - \psi_1(2)\psi_2(1) | \sH | \psi_1(1)\psi_2(2) - \psi_1(2)\psi_2(1)} \notag\\
&\quad{} \Braket{\alpha(1)\beta(2) + \beta(1)\alpha(2) | \alpha(1)\beta(2) + \beta(1)\alpha(2)}\notag\\
&= \dfrac{1}{4}((1|h|1) + (2|h|2) + (11|22) - (12|21) - (21|12) + (2|h|2) + (1|h|1) + (22|11) ) (1 + 0 + 0 + 1) \notag\\
&= h_{11} + h_{22} + J_{12} - K_{12}
\end{align}

\subsubsection{Unrestricted Determinants}
\ex{2.41}
\subex{a.}
\begin{align}
\hsS^2\ket{K} &= \qty(\hsS_-\hsS_+ + \hsS_z + \hsS_z^2) \dfrac{1}{\sqrt{2}} \qty(\psi_1^\alpha(1)\psi_1^\beta(2)\alpha(1)\beta(2) - \psi_1^\beta(1)\psi_1^\alpha(2)\beta(1)\alpha(2))\notag \\
&= \dfrac{1}{\sqrt{2}} \psi_1^\alpha(1)\psi_1^\beta(2) \qty(\hsS_-\alpha(1)\alpha(2) + 0 + 0) - \psi_1^\beta(1)\psi_1^\alpha(2) \qty(\hsS_-\alpha(1)\alpha(2) + 0 + 0) \notag\\
&= \dfrac{1}{\sqrt{2}} \qty(\psi_1^\alpha(1)\psi_1^\beta(2) - \psi_1^\beta(1)\psi_1^\alpha(2)) \qty(\alpha(1)\beta(2) + \beta(1)\alpha(2)) \notag\\
&= \dfrac{1}{\sqrt{2}} \qty[\psi_1^\alpha(1)\psi_1^\beta(2)\alpha(1)\beta(2) 
+ \psi_1^\alpha(1)\psi_1^\beta(2)\beta(1)\alpha(2) %\notag\\
%&\quad{} 
- \psi_1^\beta(1)\psi_1^\alpha(2)\alpha(1)\beta(2) 
- \psi_1^\beta(1)\psi_1^\alpha(2)\beta(1)\alpha(2)] \notag\\
&= \ket{K} + \dfrac{1}{\sqrt{2}} \qty[\psi_1^\alpha(1)\psi_1^\beta(2)\beta(1)\alpha(2) 
- \psi_1^\beta(1)\psi_1^\alpha(2)\alpha(1)\beta(2)]
\end{align}
thus, $ \ket{K} $ being an eigenfunction of $ \hsS^2 $ requires
\begin{align}
\psi_1^\alpha(1)\psi_1^\beta(2)\beta(1)\alpha(2) 
- \psi_1^\beta(1)\psi_1^\alpha(2)\alpha(1)\beta(2) = k\ket{K}
\end{align}
which requires
\begin{equation}\label{key}
\psi_1^\alpha = \psi_1^\beta
\end{equation}

\subex{b.}
\begin{align}
\Braket{K | \hsS^2 | K} &= \dfrac{1}{2} \Braket{\psi_1^\alpha(1)\psi_1^\beta(2)\alpha(1)\beta(2) - \psi_1^\beta(1)\psi_1^\alpha(2)\beta(1)\alpha(2) | (\psi_1^\alpha(1)\psi_1^\beta(2) - \psi_1^\beta(1)\psi_1^\alpha(2)) (\alpha(1)\beta(2) + \beta(1)\alpha(2))} \notag\\
&= \dfrac{1}{2} \Braket{\psi_1^\alpha(1)\psi_1^\beta(2) | \psi_1^\alpha(1)\psi_1^\beta(2) - \psi_1^\beta(1)\psi_1^\alpha(2)} - \Braket{\psi_1^\beta(1)\psi_1^\alpha(2) | \psi_1^\alpha(1)\psi_1^\beta(2) - \psi_1^\beta(1)\psi_1^\alpha(2)} \notag\\
&= \dfrac{1}{2}\qty[\qty(1 - \abs{S_{11}^{\alpha\beta}}^2) - \qty(\abs{S_{11}^{\alpha\beta}}^2 - 1)] \notag\\
&= 1 - \abs{S_{11}^{\alpha\beta}}^2
\end{align}



\end{document}
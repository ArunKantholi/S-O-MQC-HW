%\documentclass[UTF8]{ctexart} % use larger type; default would be 10pt
\documentclass[a4paper]{article}
\usepackage{xeCJK}
%\usepackage[utf8]{inputenc} % set input encoding (not needed with XeLaTeX)

%%% Examples of Article customizations
% These packages are optional, depending whether you want the features they provide.
% See the LaTeX Companion or other references for full information.

%%% PAGE DIMENSIONS
\usepackage{geometry} % to change the page dimensions
\geometry{a4paper} % or letterpaper (US) or a5paper or....
\geometry{margin=1in} % for example, change the margins to 2 inches all round
% \geometry{landscape} % set up the page for landscape
%   read geometry.pdf for detailed page layout information

\usepackage{graphicx} % support the \includegraphics command and options

% \usepackage[parfill]{parskip} % Activate to begin paragraphs with an empty line rather than an indent

%%% PACKAGES
\usepackage{booktabs} % for much better looking tables
\usepackage{array} % for better arrays (eg matrices) in maths
\usepackage{paralist} % very flexible & customisable lists (eg. enumerate/itemize, etc.)
\usepackage{verbatim} % adds environment for commenting out blocks of text & for better verbatim
\usepackage{subfig} % make it possible to include more than one captioned figure/table in a single float
% These packages are all incorporated in the memoir class to one degree or another...

%%% HEADERS & FOOTERS
\usepackage{fancyhdr} % This should be set AFTER setting up the page geometry
\pagestyle{fancy} % options: empty , plain , fancy
\renewcommand{\headrulewidth}{0pt} % customise the layout...
\lhead{}\chead{}\rhead{}
\lfoot{}\cfoot{\thepage}\rfoot{}

%%% SECTION TITLE APPEARANCE
\usepackage{sectsty}
\allsectionsfont{\sffamily\mdseries\upshape} % (See the fntguide.pdf for font help)
% (This matches ConTeXt defaults)

%%% ToC (table of contents) APPEARANCE
\usepackage[nottoc,notlof,notlot]{tocbibind} % Put the bibliography in the ToC
\usepackage[titles,subfigure]{tocloft} % Alter the style of the Table of Contents
\renewcommand{\cftsecfont}{\rmfamily\mdseries\upshape}
\renewcommand{\cftsecpagefont}{\rmfamily\mdseries\upshape} % No bold!

%%% END Article customizations

%%% The "real" document content comes below...

\setlength{\parindent}{0pt}
\usepackage{physics}
\usepackage{amsmath}
%\usepackage{symbols}
\usepackage{AMSFonts}
\usepackage{bm}
%\usepackage{eucal}
\usepackage{mathrsfs}
\usepackage{amssymb}
\usepackage{float}
\usepackage{multicol}
\usepackage{abstract}
\usepackage{empheq}
\usepackage{extarrows}
\usepackage{textcomp}
\usepackage{fontspec}
\usepackage{braket}
\usepackage{siunitx}
\sisetup{
	separate-uncertainty = true,
	inter-unit-product = \ensuremath{{}\cdot{}}
}
\usepackage{mhchem}

\DeclareMathOperator{\p}{\prime}
\DeclareMathOperator{\ti}{\times}
\DeclareMathOperator{\intinf}{\int_0^\infty}
\DeclareMathOperator{\intdinf}{\int_{-\infty}^\infty}
\DeclareMathOperator{\intzpi}{\int_0^\pi}
\DeclareMathOperator{\intztpi}{\int_0^{2\pi}}
\DeclareMathOperator{\sumninf}{\sum_{n=1}^{\infty}}
\DeclareMathOperator{\sumninfz}{\sum_{n=0}^\infty}
\DeclareMathOperator{\sumiinf}{\sum_{i=1}^{\infty}}
\DeclareMathOperator{\sumiinfz}{\sum_{i=0}^\infty}
\DeclareMathOperator{\sumkinf}{\sum_{k=1}^{\infty}}
\DeclareMathOperator{\sumkinfz}{\sum_{k=0}^\infty}
\DeclareMathOperator{\e}{\mathrm{e}}
\DeclareMathOperator{\I}{\mathrm{i}}
\DeclareMathOperator{\Arg}{\mathrm{Arg}}
\DeclareMathOperator{\ra}{\rightarrow}
\DeclareMathOperator{\llra}{\longleftrightarrow}
\DeclareMathOperator{\lra}{\longrightarrow}
\DeclareMathOperator{\dlra}{\Leftrightarrow}
\DeclareMathOperator{\dra}{\Rightarrow}
\newcommand{\bkk}[1]{\Braket{#1|#1}}
\newcommand{\bk}[2]{\Braket{#1|#2}}
\newcommand{\bkev}[2]{\Braket{#2|#1|#2}}



\DeclareMathOperator{\hV}{\hat{\vb{V}}}

\DeclareMathOperator{\hx}{\hat{\vb{x}}}
\DeclareMathOperator{\hy}{\hat{\vb{y}}}
\DeclareMathOperator{\hz}{\hat{\vb{z}}}

\DeclareMathOperator{\hA}{\hat{\vb{A}}}

\DeclareMathOperator{\hQ}{\hat{\vb{Q}}}
\DeclareMathOperator{\hI}{\hat{\vb{I}}}
\DeclareMathOperator{\psis}{\psi^\ast}
\DeclareMathOperator{\Psis}{\Psi^\ast}
\DeclareMathOperator{\hi}{\hat{\vb{i}}}
\DeclareMathOperator{\hj}{\hat{\vb{j}}}
\DeclareMathOperator{\hk}{\hat{\vb{k}}}
\DeclareMathOperator{\hr}{\hat{\vb{r}}}
\DeclareMathOperator{\hT}{\hat{\vb{T}}}
\DeclareMathOperator{\hH}{\hat{H}}
\DeclareMathOperator{\hh}{\hat{h}}               % helicity
\DeclareMathOperator{\hL}{\hat{\vb{L}}}
\DeclareMathOperator{\hp}{\hat{\vb{p}}}

\DeclareMathOperator{\ha}{\hat{\vb{a}}}
\DeclareMathOperator{\hS}{\hat{\vb{S}}}
\DeclareMathOperator{\hSigma}{\hat{\bm\Sigma}}
\DeclareMathOperator{\hJ}{\hat{\vb{J}}}
\DeclareMathOperator{\hP}{\hat{\vb{P}}}          % Parity
\DeclareMathOperator{\hC}{\hat{\vb{C}}} 
\DeclareMathOperator{\Tdv}{-\dfrac{\hbar^2}{2m}\dv[2]{x}}
\DeclareMathOperator{\Tna}{-\dfrac{\hbar^2}{2m}\nabla^2}
\DeclareMathOperator{\vna}{\vnabla}
\DeclareMathOperator{\nna}{\nabla^2}
\newcommand{\naCarExpd}[1]{\pdv[2]{#1}{x} + \pdv[2]{#1}{y} + \pdv[2]{#1}{z}}
\newcommand{\naCyl}{\qty[\dfrac{1}{\rho}\pdv{\rho}\qty(\rho\pdv{\rho}) + \dfrac{1}{\rho^2}\pdv[2]{\phi} + \pdv[2]{z}]}

%\DeclareMathOperator{\g#0}{\gamma^0}
%\DeclareMathOperator{\g1}{\gamma^1}
%\DeclareMathOperator{\g2}{\gamma^2}
%\DeclareMathOperator{\g3}{\gamma^3}
%\DeclareMathOperator{\g5}{\gamma^5}
\newcommand{\g}[1]{\gamma^{#1}}
\DeclareMathOperator{\gmuu}{\gamma^\mu}
\DeclareMathOperator{\gmud}{\gamma_\mu}
\newcommand{\G}[2]{g^{#1#2}}

\newcommand{\subsbul}{\subsection*{$ \bullet $}}
\newcommand{\ex}[1]{\paragraph{Ex #1}}
\newcommand{\subex}[1]{\subparagraph{#1}}
\newcommand{\dis}{\displaystyle}
\newcommand{\iden}{{\large \bm{1}}}
\newcommand{\qed}{$ \Square $}
\newcommand{\tPhi}{\tilde{\Phi} }

\numberwithin{equation}{subsection}
%\setcounter{secnumdepth}{4}
\setcounter{tocdepth}{4}
%\allowdisplaybreaks[4]

\usepackage{xcolor}
\definecolor{codegray}{gray}{0.9}
\newfontfamily\Consolas{Consolas}
\newcommand{\code}[1]{\colorbox{codegray}{{\Consolas#1}}}

\title{\textbf{Modern Quantum Chemistry, Szabo \& Ostlund}\\HW}
\author{王石嵘
\vspace{5pt}\\
%161240065\\
%Email: shirong\_wang@berkeley.edu
}
\date{\today} % Activate to display a given date or no date (if empty),
         % otherwise the current date is printed 

\begin{document}
% \boldmath

\maketitle

\tableofcontents

\newpage

\setcounter{section}{1}
\section{Many-electron Wave Functions and Operators}
\subsection{The Electronic Problem}
\subsubsection{Atomic Units}
\subsubsection{The B-O Approximation}
\subsubsection{The Antisymmetry or Pauli Exclusion Principle}
\subsection{Orbitals, Slater Determinants, and Basis Functions}
\subsubsection{Spin Orbitals and Spatial Orbitals}
\ex{2.1}
Consider $ \Braket{\chi_k | \chi_m} $. If $ k=m $,
\begin{equation}\label{key}
\Braket{\chi_{2i-1} | \chi_{2i-1}} = \Braket{\psi_i^\alpha | \psi_i^\alpha} \Braket{\alpha|\alpha} = 1
\end{equation}
\begin{equation}\label{key}
\Braket{\chi_{2i} | \chi_{2i}} = \Braket{\psi_i^\beta | \psi_i^\beta} \Braket{\alpha|\alpha} = 1
\end{equation}
thus
\begin{equation}\label{key}
\Braket{\chi_k | \chi_k} = 1
\end{equation}
If $ k\neq m $, three cases may occur as below
\begin{equation}\label{key}
\Braket{\chi_{2i-1} | \chi_{2j-1}} = \Braket{\psi_i^\alpha | \psi_j^\alpha}\Braket{\alpha|\alpha} = 0\cdot 1 = 0 \qquad (i\neq j)
\end{equation}
\begin{equation}
\Braket{\chi_{2i-1} | \chi_{2j}} = \Braket{\psi_i^\alpha | \psi_j^\beta}\Braket{\alpha|\beta} = S_{ij}\cdot 0 = 0
\end{equation}
\begin{equation}\label{key}
\Braket{\chi_{2i} | \chi_{2j}} = \Braket{\psi_i^\beta | \psi_j^\beta}\Braket{\beta|\beta} = 0\cdot 1 = 0 \qquad (i\neq j)
\end{equation}
thus
\begin{equation}\label{key}
\Braket{\chi_k | \chi_m} = 0 \qquad (k\neq m)
\end{equation}
Overall,
\begin{equation}\label{key}
\Braket{\chi_k | \chi_m} = \delta_{km}
\end{equation}

\subsubsection{Hartree Products}
%.
\ex{2.2}
\begin{equation}\label{key}
\begin{aligned}
\mathscr{H}\Psi^{HP} &= \sum_{i=1}^N h(i)\chi_i(\vb{x}_1)\chi_j(\vb{x}_2)\cdots\chi_k(\vb{x}_N)\\
&= \varepsilon_i\chi_i(\vb{x}_1)\chi_j(\vb{x}_2)\cdots\chi_k(\vb{x}_N) + \chi_i(\vb{x}_1)[\varepsilon_j\chi_j(\vb{x}_2)]\cdots\chi_k(\vb{x}_N) + \cdots + \chi_i(\vb{x}_1)\chi_j(\vb{x}_2)\cdots[\varepsilon_k\chi_k(\vb{x}_N)]\\
&= (\varepsilon_i +\varepsilon_j + \cdots + \varepsilon_k)\Psi^{HP}
\end{aligned}
\end{equation}

\subsubsection{Slater Determinants}
%.
\ex{2.3}
\begin{equation}\label{key}
\begin{aligned}
\Braket{\Psi | \Psi} 
&= \dfrac{1}{2}\qty(\Braket{\chi_i|\chi_i}\Braket{\chi_j|\chi_j} - \Braket{\chi_i|\chi_j}\Braket{\chi_j|\chi_i} - \Braket{\chi_j|\chi_i}\Braket{\chi_i|\chi_j} + \Braket{\chi_j|\chi_j}\Braket{\chi_i|\chi_i})\\
&= \dfrac{1}{2}(1+0+0+1) = 1
\end{aligned}
\end{equation}
\ex{2.4}
According to Ex. 2.2, we know that $ \chi_i(\vb{x}_1)\chi_j(\vb{x}_2) $ are an eigenfunction of $ \mathscr{H} $ and has the eigenvalue $ \varepsilon_i \varepsilon_j $. Similarly, we have the same conclusion for $ \chi_i(\vb{x}_2)\chi_j(\vb{x}_1) $.\\
For the antisymmetrized wave function,
\begin{equation}\label{key}
\begin{aligned}
\Braket{\Psi | \mathscr{H} | \Psi} &= \dfrac{1}{2}\left(\Braket{\chi_i(\vb{x}_1)\chi_j(\vb{x}_2) | \mathscr{H} | \chi_i(\vb{x}_1)\chi_j(\vb{x}_2)} - \Braket{\chi_i(\vb{x}_1)\chi_j(\vb{x}_2) | \mathscr{H} | \chi_j(\vb{x}_1)\chi_i(\vb{x}_2)}\right.\\
&\left. \quad - \Braket{\chi_j(\vb{x}_1)\chi_i(\vb{x}_2) | \mathscr{H} | \chi_i(\vb{x}_1)\chi_j(\vb{x}_2)} + \Braket{\chi_j(\vb{x}_1)\chi_i(\vb{x}_2) | \mathscr{H} | \chi_j(\vb{x}_1)\chi_i(\vb{x}_2)}\right)\\
&= \dfrac{1}{2}\qty(\varepsilon_i + \varepsilon_j - 0 - 0 + \varepsilon_i + \varepsilon_j)\\
&= \varepsilon_i + \varepsilon_j
\end{aligned}
\end{equation}

\ex{2.5}
\begin{equation}\label{key}
\begin{aligned}
\Braket{K | L} &= \dfrac{1}{2}\Braket{\chi_i(\vb{x}_1)\chi_j(\vb{x}_2) - \chi_j(\vb{x}_1)\chi_i(\vb{x}_2) | \chi_k(\vb{x}_1)\chi_l(\vb{x}_2) - \chi_l(\vb{x}_1)\chi_k(\vb{x}_2)}\\
&= \dfrac{1}{2}\qty(\Braket{\chi_i|\chi_k}\Braket{\chi_j|\chi_l} - \Braket{\chi_i|\chi_l}\Braket{\chi_j|\chi_k} - \Braket{\chi_j|\chi_k}\Braket{\chi_i|\chi_l} + \Braket{\chi_j|\chi_l}\Braket{\chi_i|\chi_k})\\
&= \dfrac{1}{2}\qty(\delta_{ik}\delta_{jl} - \delta_{il}\delta_{jk} - \delta_{jk}\delta_{il} + \delta_{jl}\delta_{ik})\\
&= \delta_{ik}\delta_{jl} - \delta_{il}\delta_{jk}
\end{aligned}
\end{equation}

\subsubsection{The Hartree-Fock Approximation}
\subsubsection{The Minimal Basis $ \ce{H_2} $ Model}
\ex{2.6}
\begin{equation}\label{key}
\Braket{\psi_1 | \psi_1} = \dfrac{1}{2(1+S_{12})}\qty(\Braket{\phi_1|\phi_1} + 2\Braket{\phi_1|\phi_2} + \Braket{\phi_2|\phi_2}) = \dfrac{2 + 2S_{12}}{2(1+S_{12})} = 1
\end{equation}
\begin{equation}\label{key}
\Braket{\psi_2 | \psi_2} = \dfrac{1}{2(1-S_{12})}\qty(\Braket{\phi_1|\phi_1} - 2\Braket{\phi_1|\phi_2} + \Braket{\phi_2|\phi_2}) = \dfrac{2 - 2S_{12}}{2(1-S_{12})} = 1
\end{equation}
\begin{equation}\label{key}
\Braket{\psi_1 | \psi_2} = \dfrac{1}{2\sqrt{1+S_{12}}\sqrt{1-S_{12}}}\qty(\Braket{\phi_1|\phi_1} - \Braket{\phi_2|\phi_2}) = 0
\end{equation}
\subsubsection{Excited Determinants}
\subsubsection{Form of the Exact Wfn and CI}
\ex{2.7}
\begin{equation}\label{key}
C_{72}^{42} = 164307576757973059488
\end{equation}




\end{document}
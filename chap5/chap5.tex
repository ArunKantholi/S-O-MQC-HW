%\documentclass[UTF8]{ctexart} % use larger type; default would be 10pt
\documentclass[a4paper]{article}
\usepackage{xeCJK}
%\usepackage[utf8]{inputenc} % set input encoding (not needed with XeLaTeX)

%%% Examples of Article customizations
% These packages are optional, depending whether you want the features they provide.
% See the LaTeX Companion or other references for full information.

%%% PAGE DIMENSIONS
\usepackage{geometry} % to change the page dimensions
\geometry{a4paper} % or letterpaper (US) or a5paper or....
\geometry{margin=1in} % for example, change the margins to 2 inches all round
% \geometry{landscape} % set up the page for landscape
%   read geometry.pdf for detailed page layout information

\usepackage{graphicx} % support the \includegraphics command and options

% \usepackage[parfill]{parskip} % Activate to begin paragraphs with an empty line rather than an indent

%%% PACKAGES
\usepackage{booktabs} % for much better looking tables
\usepackage{array} % for better arrays (eg matrices) in maths
\usepackage{paralist} % very flexible & customisable lists (eg. enumerate/itemize, etc.)
\usepackage{verbatim} % adds environment for commenting out blocks of text & for better verbatim
\usepackage{subfig} % make it possible to include more than one captioned figure/table in a single float
% These packages are all incorporated in the memoir class to one degree or another...

%%% HEADERS & FOOTERS
\usepackage{fancyhdr} % This should be set AFTER setting up the page geometry
\pagestyle{fancy} % options: empty , plain , fancy
\renewcommand{\headrulewidth}{0pt} % customise the layout...
\lhead{}\chead{}\rhead{}
\lfoot{}\cfoot{\thepage}\rfoot{}

%%% SECTION TITLE APPEARANCE
\usepackage{sectsty}
%\allsectionsfont{\sffamily\mdseries\upshape} % (See the fntguide.pdf for font help)
% (This matches ConTeXt defaults)

%%% ToC (table of contents) APPEARANCE
\usepackage[nottoc,notlof,notlot]{tocbibind} % Put the bibliography in the ToC
\usepackage[titles,subfigure]{tocloft} % Alter the style of the Table of Contents
%\renewcommand{\cftsecfont}{\rmfamily\mdseries\upshape}
%\renewcommand{\cftsecpagefont}{\rmfamily\mdseries\upshape} % No bold!

%%% END Article customizations

%%% The "real" document content comes below...

\setlength{\parindent}{0pt}
\usepackage{physics}
\usepackage{amsmath}
%\usepackage{symbols}
\usepackage{AMSFonts}
\usepackage{bm}
%\usepackage{eucal}
\usepackage{mathrsfs}
\usepackage{amssymb}
\usepackage{float}
\usepackage{multicol}
\usepackage{abstract}
\usepackage{empheq}
\usepackage{extarrows}
\usepackage{textcomp}
\usepackage{fontspec}

\setmainfont{CMU Serif}
\setsansfont{CMU Sans Serif}
\setmonofont{CMU Typewriter Text}

\usepackage{braket}
\usepackage{siunitx}
\sisetup{
	separate-uncertainty = true,
	inter-unit-product = \ensuremath{{}\cdot{}}
}
\usepackage{mhchem}
\usepackage{multirow}
\usepackage{booktabs}

\DeclareMathOperator{\p}{\prime}
\DeclareMathOperator{\ti}{\times}
\DeclareMathOperator{\intinf}{\int_0^\infty}
\DeclareMathOperator{\intdinf}{\int_{-\infty}^\infty}
\DeclareMathOperator{\intzpi}{\int_0^\pi}
\DeclareMathOperator{\intztpi}{\int_0^{2\pi}}
\DeclareMathOperator{\sumninf}{\sum_{n=1}^{\infty}}
\DeclareMathOperator{\sumninfz}{\sum_{n=0}^\infty}
\DeclareMathOperator{\sumiinf}{\sum_{i=1}^{\infty}}
\DeclareMathOperator{\sumiinfz}{\sum_{i=0}^\infty}
\DeclareMathOperator{\sumkinf}{\sum_{k=1}^{\infty}}
\DeclareMathOperator{\sumkinfz}{\sum_{k=0}^\infty}
\DeclareMathOperator{\e}{\mathrm{e}}
\DeclareMathOperator{\I}{\mathrm{i}}
\DeclareMathOperator{\Arg}{\mathrm{Arg}}
\DeclareMathOperator{\ra}{\rightarrow}
\DeclareMathOperator{\llra}{\longleftrightarrow}
\DeclareMathOperator{\lra}{\longrightarrow}
\DeclareMathOperator{\dlra}{\Leftrightarrow}
\DeclareMathOperator{\dra}{\Rightarrow}
\newcommand{\bkk}[1]{\Braket{#1|#1}}
\newcommand{\bk}[2]{\Braket{#1|#2}}
\newcommand{\bkev}[2]{\Braket{#2|#1|#2}}



\DeclareMathOperator{\hV}{\hat{\vb{V}}}

\DeclareMathOperator{\hx}{\hat{\vb{x}}}
\DeclareMathOperator{\hy}{\hat{\vb{y}}}
\DeclareMathOperator{\hz}{\hat{\vb{z}}}

\DeclareMathOperator{\hA}{\hat{\vb{A}}}

\DeclareMathOperator{\hQ}{\hat{\vb{Q}}}
\DeclareMathOperator{\hI}{\hat{\vb{I}}}
\DeclareMathOperator{\psis}{\psi^\ast}
\DeclareMathOperator{\Psis}{\Psi^\ast}
\DeclareMathOperator{\hi}{\hat{\vb{i}}}
\DeclareMathOperator{\hj}{\hat{\vb{j}}}
\DeclareMathOperator{\hk}{\hat{\vb{k}}}
\DeclareMathOperator{\hr}{\hat{\vb{r}}}
\DeclareMathOperator{\hT}{\hat{\vb{T}}}
\DeclareMathOperator{\hH}{\hat{H}}
\DeclareMathOperator{\hh}{\hat{h}}               % helicity
\DeclareMathOperator{\hL}{\hat{\vb{L}}}
\DeclareMathOperator{\hp}{\hat{\vb{p}}}

\DeclareMathOperator{\ha}{\hat{\vb{a}}}
\DeclareMathOperator{\hs}{\hat{\vb{s}}}
\DeclareMathOperator{\hS}{\hat{\vb{S}}}
\DeclareMathOperator{\hSigma}{\hat{\bm\Sigma}}
\DeclareMathOperator{\hJ}{\hat{\vb{J}}}
\DeclareMathOperator{\hP}{\hat{\vb{P}}}          % Parity
\DeclareMathOperator{\hC}{\hat{\vb{C}}} 
\DeclareMathOperator{\Tdv}{-\dfrac{\hbar^2}{2m}\dv[2]{x}}
\DeclareMathOperator{\Tna}{-\dfrac{\hbar^2}{2m}\nabla^2}
\DeclareMathOperator{\vna}{\vnabla}
\DeclareMathOperator{\nna}{\nabla^2}
\newcommand{\naCarExpd}[1]{\pdv[2]{#1}{x} + \pdv[2]{#1}{y} + \pdv[2]{#1}{z}}
\newcommand{\naCyl}{\qty[\dfrac{1}{\rho}\pdv{\rho}\qty(\rho\pdv{\rho}) + \dfrac{1}{\rho^2}\pdv[2]{\phi} + \pdv[2]{z}]}

%\DeclareMathOperator{\g#0}{\gamma^0}
%\DeclareMathOperator{\g1}{\gamma^1}
%\DeclareMathOperator{\g2}{\gamma^2}
%\DeclareMathOperator{\g3}{\gamma^3}
%\DeclareMathOperator{\g5}{\gamma^5}
\newcommand{\g}[1]{\gamma^{#1}}
\DeclareMathOperator{\gmuu}{\gamma^\mu}
\DeclareMathOperator{\gmud}{\gamma_\mu}
\newcommand{\G}[2]{g^{#1#2}}


%% MQC
\DeclareMathOperator{\sH}{\mathscr{H}}
\DeclareMathOperator{\sA}{\mathscr{A}}
\newcommand{\iden}{{\large \bm{1}}}
\newcommand{\qed}{$ \Square $}
\newcommand{\tPhi}{\tilde{\Phi} }
\newcommand{\hsP}{\hat{\mathscr{P}}}
\newcommand{\hsS}{\hat{\mathscr{S}}}
\DeclareMathOperator{\core}{\mathrm{core}}
\DeclareMathOperator{\GF}{\mathrm{GF}}
\DeclareMathOperator{\SF}{\mathrm{SF}}
\DeclareMathOperator{\corr}{\mathrm{corr}}
\DeclareMathOperator{\gvb}{\mathrm{GVB}}
\DeclareMathOperator{\eff}{\mathrm{eff}}

\newcommand{\subsbul}{\subsection*{$ \bullet $}}
\newcommand{\ex}[1]{\paragraph{Ex #1}}
\newcommand{\subex}[1]{\subparagraph{#1}}
\newcommand{\dis}{\displaystyle}


\numberwithin{equation}{subsection}
%\setcounter{secnumdepth}{4}
\setcounter{tocdepth}{4}
\allowdisplaybreaks[1]

\usepackage{xcolor}
\definecolor{codegray}{gray}{0.9}
\newfontfamily\Consolas{Consolas}
\newcommand{\code}[1]{\colorbox{codegray}{{\Consolas#1}}}

\title{\textbf{Modern Quantum Chemistry, Szabo \& Ostlund}\\HW}
\author{wsr
\vspace{5pt}\\
}
\date{\today} % Activate to display a given date or no date (if empty),
         % otherwise the current date is printed 

\begin{document}
% \boldmath

\maketitle

\tableofcontents

\newpage

\setcounter{section}{4}
\section{Pair and Coupled-pair Theories}
\subsection{The Independent Electron Pair Approximation}
\ex{5.1}
\subex{a.}
\begin{align}
{^1 E}_{\corr}(\text{FO}) &= \dfrac{|\Braket{1\bar{1}||2\bar{2}}|^2}{\varepsilon_1 + \varepsilon_1 - \varepsilon_2 - \varepsilon_2} \notag\\
&= \dfrac{\abs{\Braket{1\bar{1}|2\bar{2}} - \Braket{1\bar{1}|\bar{2} 2}}^2}{2\varepsilon_1 - 2\varepsilon_2} \notag\\
&= \dfrac{\abs{[12|\bar{1}\bar{2}] - [1\bar{2}|\bar{1} 2]}^2}{2\varepsilon_1 - 2\varepsilon_2} \notag\\
&= \dfrac{K_{12}^2}{2(\varepsilon_1 - \varepsilon_2)} 
\end{align}
\subex{b.}
\begin{align}
{^1 E}_{\corr} &= \Delta - \Delta\sqrt{1 + \dfrac{K_{12}^2}{\Delta^2}} \notag\\
&= \Delta - \Delta\qty(1 + \dfrac{K_{12}^2}{2\Delta^2}) \notag\\
&= - \dfrac{K_{12}^2}{2\Delta}) \notag\\
&\approx \dfrac{K_{12}^2}{2(\varepsilon_1 - \varepsilon_2)} 
\end{align}

\ex{5.2}
From Eq. 5.9a and 5.9b in the textbook, we get
\begin{align}
\sum_{t<u} c_{1_i\bar{1}_i}^{tu} \Braket{\Psi_0 | \sH | \Psi_{1_i\bar{1}_i}^{tu}} &= e_{1_i\bar{1}_i} \\
\Braket{\Psi_{1_i\bar{1}_i}^{rs} | \sH | \Psi_0} + \sum_{t<u} \Braket{\Psi_{1_i\bar{1}_i}^{rs} | \sH - E_0 | \Psi_{1_i\bar{1}_i}^{tu}} c_{1_i\bar{1}_i}^{tu}  &= e_{1_i\bar{1}_i} c_{1_i\bar{1}_i}^{rs} 
\end{align}
$ \therefore $
\begin{align}
c_{1_i\bar{1}_i}^{2_i\bar{2}_i} \Braket{\Psi_0 | \sH | \Psi_{1_i\bar{1}_i}^{2_i\bar{2}_i}} &= e_{1_i\bar{1}_i} \label{5.2a}\\
\Braket{\Psi_{1_i\bar{1}_i}^{2_i\bar{2}_i} | \sH | \Psi_0} + \sum_{t<u} \Braket{\Psi_{1_i\bar{1}_i}^{2_i\bar{2}_i} | \sH - E_0 | \Psi_{1_i\bar{1}_i}^{tu}} c_{1_i\bar{1}_i}^{tu}  &= e_{1_i\bar{1}_i} c_{1_i\bar{1}_i}^{2_i\bar{2}_i}   \label{5.2b}
\end{align}
$ \eqref{5.2a} $ gives
\begin{align}
K_{12} c_{1_i\bar{1}_i}^{2_i\bar{2}_i} = e_{1_i\bar{1}_i}
\end{align}
$ \eqref{5.2b} $ gives
\begin{align}
K_{12} + \sum_{jk} \Braket{\Psi_{1_i\bar{1}_i}^{2_i\bar{2}_i} | \sH - E_0 | \Psi_{1_i\bar{1}_i}^{2_j\bar{2}_k}} c_{1_i\bar{1}_i}^{2_j\bar{2}_k}  &= e_{1_i\bar{1}_i} c_{1_i\bar{1}_i}^{2_i\bar{2}_i} 
\end{align}
Since
\begin{align}
\Braket{\Psi_{1_i\bar{1}_i}^{2_i\bar{2}_i} | \sH - E_0 | \Psi_{1_i\bar{1}_i}^{2_j\bar{2}_k}} c_{1_i\bar{1}_i}^{2_j\bar{2}_k} = \left\{
\mqty{2\Delta & j=k=i\\
0 & j=k\neq i\\
0 & i=j=\neq k}
\right.
\end{align}
we have
\begin{align}
K_{12} + 2\Delta c_{1_i\bar{1}_i}^{2_i\bar{2}_i}  &= e_{1_i\bar{1}_i} c_{1_i\bar{1}_i}^{2_i\bar{2}_i} 
\end{align}

\ex{5.3}
\begin{align}
{^2 E}_{\corr}(\text{FO}) &= \sum_i \dfrac{\abs{\Braket{1_i\bar{1}_i||2_i\bar{2}_i}}^2}{\varepsilon_1 + \varepsilon_1 - \varepsilon_2 - \varepsilon_2} \notag\\
&= 2\times \dfrac{K_{12}^2}{2(\varepsilon_1 - \varepsilon_2)} \notag\\
&= \dfrac{K_{12}^2}{(\varepsilon_1 - \varepsilon_2)}
\end{align}

\subsubsection{Invariance under Unitary Transformations: An Example}
\ex{5.4}
\begin{align}
\ket{a\bar{a} b\bar{b}} &= 2^{-1/2} \qty(\ket{1_1\bar{a} b\bar{b}} + \ket{1_2\bar{a} b\bar{b}}) \notag\\
&= 2^{-1} \qty(\ket{1_1\bar{1}_1 b\bar{b}} + \ket{1_1\bar{1}_2 b\bar{b}}  + \ket{1_2\bar{1}_1 b\bar{b}} + \ket{1_2\bar{1}_2 b\bar{b}}) \notag\\
&= 2^{-2} (
\ket{1_1\bar{1}_1 1_1\bar{1}_1} - \ket{1_1\bar{1}_1 1_1\bar{1}_2} - \ket{1_1\bar{1}_1 1_2\bar{1}_1} + \ket{1_1\bar{1}_1 1_2\bar{1}_2} \notag\\
&{}\quad + \ket{1_1\bar{1}_2 1_1\bar{1}_1} - \ket{1_1\bar{1}_2 1_1\bar{1}_2} - \ket{1_1\bar{1}_2 1_2\bar{1}_1} + \ket{1_1\bar{1}_2 1_2\bar{1}_2} \notag\\
&{}\quad + \ket{1_2\bar{1}_1 1_1\bar{1}_1} - \ket{1_2\bar{1}_1 1_1\bar{1}_2} - \ket{1_2\bar{1}_1 1_2\bar{1}_1} + \ket{1_2\bar{1}_1 1_2\bar{1}_2} \notag\\
&{}\quad + \ket{1_2\bar{1}_2 1_1\bar{1}_1} - \ket{1_2\bar{1}_2 1_1\bar{1}_2} - \ket{1_2\bar{1}_2 1_2\bar{1}_1} + \ket{1_2\bar{1}_2 1_2\bar{1}_2}
) \notag\\
&= 2^{-2} (
2\ket{1_1\bar{1}_1 1_1\bar{1}_1}   + 2\ket{1_1\bar{1}_1 1_2\bar{1}_2} 
- 2\ket{1_1\bar{1}_2 1_1\bar{1}_2} - 2\ket{1_1\bar{1}_2 1_2\bar{1}_1}  
) \notag\\
&= 2^{-2} ( 2\ket{1_1\bar{1}_1 1_2\bar{1}_2} - 2\ket{1_1 \bar{1}_1 1_2\bar{1}_2}  ) \notag\\
&= \ket{1_1\bar{1}_1 1_2\bar{1}_2}
\end{align}

\ex{5.5}
\begin{align}
\Braket{\Psi_0 | \sH | \Psi_{a\bar{a}}^{**}} &= 2^{-1/2}\qty(\Braket{\Psi_0 | \sH | \Psi_{a\bar{a}}^{r\bar{r}}} + \Braket{\Psi_0 | \sH | \Psi_{a\bar{a}}^{s\bar{s}}}) \notag\\
&= 2^{-1/2}\qty(2\times \dfrac{1}{2}K_{12} ) \notag\\
&= 2^{-1/2} K_{12}
\end{align}
\begin{align}
\Braket{\Psi_{a\bar{a}}^{**} | \sH - E_0 | \Psi_{a\bar{a}}^{**}} &= 2^{-1} (\Braket{\Psi_{a\bar{a}}^{r\bar{r}} | \sH - E_0 | \Psi_{a\bar{a}}^{r\bar{r}}} 
+ \Braket{\Psi_{a\bar{a}}^{r\bar{r}} | \sH - E_0 | \Psi_{a\bar{a}}^{s\bar{s}}} \notag\\
&{}\quad
+ \Braket{\Psi_{a\bar{a}}^{s\bar{s}} | \sH - E_0 | \Psi_{a\bar{a}}^{r\bar{r}}} 
+ \Braket{\Psi_{a\bar{a}}^{s\bar{s}} | \sH - E_0 | \Psi_{a\bar{a}}^{s\bar{s}}}) \notag\\
&= 2^{-1} \Bigg[ \qty(2h_{11} + 2h_{22} + \dfrac{1}{2}J_{11} + \dfrac{1}{2}J_{22} + 2J_{12} - K_{12}) - (4h_{11} + 2J_{11}) \notag\\
&{}\qquad + \dfrac{1}{2}J_{22} + \dfrac{1}{2}J_{22} \notag\\
&{}\qquad + \qty(2h_{11} + 2h_{22} + \dfrac{1}{2}J_{11} + \dfrac{1}{2}J_{22} + 2J_{12} - K_{12}) - (4h_{11} + 2J_{11})
 \Bigg] \notag\\
&= 2^{-1} \qty(-2h_{11} + 2h_{22} - \dfrac{3}{2}J_{11} + J_{22} + 2J_{12} - K_{12})\times 2 \notag\\
&= -2h_{11} + 2h_{22} - \dfrac{3}{2}J_{11} + J_{22} + 2J_{12} - K_{12}
\end{align}
Since
\begin{equation}\label{key}
\varepsilon_2 - \varepsilon_1 = h_{22} - h_{11} + 2J_{12} - K_{12} - J_{11}
\end{equation}
we have
\begin{align}
\Braket{\Psi_{a\bar{a}}^{**} | \sH - E_0 | \Psi_{a\bar{a}}^{**}} &= 2(\varepsilon_2 - \varepsilon_1) - 2J_{12} + K_{12} + \dfrac{1}{2}J_{11} + J_{22}
\end{align}

\ex{5.6}
Since
\begin{equation}\label{key}
\ket{\Psi_{a\bar{b}}^{**}} = 2^{-1/2} (\ket{\Psi_{a\bar{b}}^{r\bar{s}}} + \ket{\Psi_{a\bar{b}}^{s\bar{r}}})
\end{equation}
\begin{align}
\Braket{\Psi_0 | \sH | \Psi_{a\bar{b}}^{**}} &= 2^{-1/2}\qty(\Braket{\Psi_0 | \sH | \Psi_{a\bar{b}}^{r\bar{s}}} + \Braket{\Psi_0 | \sH | \Psi_{a\bar{b}}^{s\bar{r}}}) \notag\\
&= 2^{-1/2}\qty(\Braket{a\bar{b}||r\bar{s}} + \Braket{a\bar{b}||s\bar{r}} ) \notag\\
&= 2^{-1/2} \qty((ar|bs) + (as|br)) \notag\\
&= 2^{-1/2} K_{12}
\end{align}
\begin{align}
\Braket{\Psi_{a\bar{b}}^{**} | \sH - E_0 | \Psi_{a\bar{b}}^{**}} &= 2^{-1} (\Braket{\Psi_{a\bar{b}}^{r\bar{s}} | \sH - E_0 | \Psi_{a\bar{b}}^{r\bar{s}}} 
+ \Braket{\Psi_{a\bar{b}}^{r\bar{s}} | \sH - E_0 | \Psi_{a\bar{b}}^{s\bar{r}}} \notag\\
&{}\quad
+ \Braket{\Psi_{a\bar{b}}^{s\bar{r}} | \sH - E_0 | \Psi_{a\bar{b}}^{r\bar{s}}} 
+ \Braket{\Psi_{a\bar{b}}^{s\bar{r}} | \sH - E_0 | \Psi_{a\bar{b}}^{s\bar{r}}}) \notag\\
&= 2^{-1} \Bigg[ \qty(2h_{11} + 2h_{22} + \dfrac{1}{2}J_{11} + \dfrac{1}{2}J_{22} + 2J_{12} - K_{12}) - (4h_{11} + 2J_{11}) \notag\\
&{}\qquad + \dfrac{1}{2}J_{22} + \dfrac{1}{2}J_{22} \notag\\
&{}\qquad + \qty(2h_{11} + 2h_{22} + \dfrac{1}{2}J_{11} + \dfrac{1}{2}J_{22} + 2J_{12} - K_{12}) - (4h_{11} + 2J_{11})
\Bigg] \notag\\
&= ... \notag\\
&= 2(\varepsilon_2 - \varepsilon_1) - 2J_{12} + K_{12} + \dfrac{1}{2}J_{11} + J_{22} \equiv 2\Delta'
\end{align}
Thus the equations determining $ e_{a\bar{b}} $ are identical to that of $ e_{a\bar{a}} $. Similarly, $ e_{\bar{a}b} $ shares the same equations with them.\\
$ \therefore\; e_{a\bar{b}} = e_{\bar{a}b} = e_{a\bar{a}}$.

\ex{5.7}
\subex{a.}
As shown in Ex 5.5, 5.6
\begin{align}
\Braket{\Psi_0 | \sH | \Psi_{a\bar{a}}^{**}} 
= \Braket{\Psi_0 | \sH | \Psi_{a\bar{b}}^{**}} 
= \Braket{\Psi_0 | \sH | \Psi_{\bar{a}b}^{**}} = 2^{-1/2}K_{12}
\end{align}
\begin{align}
\Braket{\Psi_{a\bar{a}}^{**} | \sH - E_0 | \Psi_{a\bar{a}}^{**}} 
= \Braket{\Psi_{a\bar{b}}^{**} | \sH - E_0 | \Psi_{a\bar{b}}^{**}} 
= \Braket{\Psi_{\bar{a}b}^{**} | \sH - E_0 | \Psi_{\bar{a}b}^{**}} = 2\Delta' 
\end{align}
Similarly, we get
\begin{align}
\Braket{\Psi_0 | \sH | \Psi_{b\bar{b}}^{**}} = 2^{-1/2}K_{12}
\end{align}
\begin{align}
\Braket{\Psi_{b\bar{b}}^{**} | \sH - E_0 | \Psi_{b\bar{b}}^{**}} 
 = 2\Delta' 
\end{align}
For the rest, 
\begin{align}
\Braket{\Psi_{a\bar{a}}^{**} | \sH - E_0 | \Psi_{b\bar{b}}^{**}} 
&= 2^{-1} (\Braket{\Psi_{a\bar{a}}^{r\bar{r}} | \sH - E_0 | \Psi_{b\bar{b}}^{r\bar{r}}} 
+ \Braket{\Psi_{a\bar{a}}^{r\bar{r}} | \sH - E_0 | \Psi_{b\bar{b}}^{s\bar{s}}} \notag\\
&{}\quad
+ \Braket{\Psi_{a\bar{a}}^{s\bar{s}} | \sH - E_0 | \Psi_{b\bar{b}}^{r\bar{r}}} 
+ \Braket{\Psi_{a\bar{a}}^{s\bar{s}} | \sH - E_0 | \Psi_{b\bar{b}}^{s\bar{s}}}) \notag\\
&= 2^{-1} \qty[ 
\Braket{b\bar{b} || a\bar{a}} 
+ 0 + 0 + \Braket{b\bar{b} || a\bar{a}}
] \notag\\
&= (ab|ab) \notag\\
&= \dfrac{1}{2}J_{11}
\end{align}
\begin{align}
\Braket{\Psi_{a\bar{a}}^{**} | \sH - E_0 | \Psi_{a\bar{b}}^{**}} 
&= 2^{-1} (\Braket{\Psi_{a\bar{a}}^{r\bar{r}} | \sH - E_0 | \Psi_{a\bar{b}}^{r\bar{s}}} 
+ \Braket{\Psi_{a\bar{a}}^{r\bar{r}} | \sH - E_0 | \Psi_{a\bar{b}}^{s\bar{r}}} \notag\\
&{}\quad
+ \Braket{\Psi_{a\bar{a}}^{s\bar{s}} | \sH - E_0 | \Psi_{a\bar{b}}^{r\bar{s}}} 
+ \Braket{\Psi_{a\bar{a}}^{s\bar{s}} | \sH - E_0 | \Psi_{a\bar{b}}^{s\bar{r}}}) \notag\\
&= 2^{-1} \qty[ 
\Braket{\bar{r}\bar{b} || \bar{a}\bar{s}} 
- \Braket{r\bar{b} || s\bar{a}} + \Braket{s\bar{b} || r\bar{a}} - \Braket{\bar{s}\bar{b} || \bar{a}\bar{r}}
] \notag\\
&= 2^{-1}[(ra|bs) - (rs|ba) - (rs|ba) - (sr|ba) + (sa|br) - (sr|ba) ] \notag\\
&= 2^{-1}[(ra|bs) + (sa|br) - 4(ab|sr) ] \notag\\
&= 2^{-1}\qty[2\times \dfrac{1}{2}K_{12} - 4\times \dfrac{1}{2}J_{12}] \notag\\
&= \dfrac{1}{2} K_{12} - J_{12}
\end{align}
Similarly, we get
\begin{align}
\Braket{\Psi_{a\bar{b}}^{**} | \sH - E_0 | \Psi_{\bar{a}b}^{**}} = \dfrac{1}{2}J_{11}
\end{align}
\begin{align}
\Braket{\Psi_{a\bar{a}}^{**} | \sH - E_0 | \Psi_{\bar{a}b}^{**}} = \Braket{\Psi_{b\bar{b}}^{**} | \sH - E_0 | \Psi_{a\bar{b}}^{**}} = \Braket{\Psi_{b\bar{b}}^{**} | \sH - E_0 | \Psi_{\bar{a}b}^{**}} = \dfrac{1}{2}K_{12} - J_{12}
\end{align}
thus the DCI equation is
\begin{equation}\label{key}
\mqty(0 & 2^{-1/2}K_{12} & 2^{-1/2}K_{12} & 2^{-1/2}K_{12} & 2^{-1/2}K_{12} \\
2^{-1/2}K_{12} & 2\Delta' & \tfrac{1}{2}J_{11} & \tfrac{1}{2}K_{12} - J_{12} & \tfrac{1}{2}K_{12} - J_{12} \\
2^{-1/2}K_{12} & \tfrac{1}{2}J_{11} & 2\Delta' & \tfrac{1}{2}K_{12} - J_{12} & \tfrac{1}{2}K_{12} - J_{12} \\
2^{-1/2}K_{12} & \tfrac{1}{2}K_{12} - J_{12} & \tfrac{1}{2}K_{12} - J_{12} & 2\Delta' &\tfrac{1}{2}J_{11} \\
2^{-1/2}K_{12} & \tfrac{1}{2}K_{12} - J_{12} & \tfrac{1}{2}K_{12} - J_{12} & \tfrac{1}{2}J_{11} & 2\Delta' )
\mqty(1\\ c_1\\ c_2\\ c_3\\ c_4) = {^2 E}_{\corr}(\text{DCI}) 
\mqty(1\\ c_1\\ c_2\\ c_3\\ c_4)
\end{equation}

\subex{b.}
By solving the DCI equation above (see \code{5-7.nb}), we get
\begin{align}
{^2 E}_{\corr}(\text{DCI}) &= \dfrac{2\Delta' + \tfrac{1}{2}J_{11} + 2(\tfrac{1}{2}K_{12} - J_{12}) - \sqrt{16(2^{-1/2}K_{12})^2 + [2\Delta' + \tfrac{1}{2}J_{11} + 2(\tfrac{1}{2}K_{12} - J_{12})]^2}}{2} \notag\\
%&= 
\end{align}
and
\begin{align}
c_1 = c_2 = c_3 = c_4 = \dfrac{2\Delta' + \tfrac{1}{2}J_{11} + 2(\tfrac{1}{2}K_{12} - J_{12}) + \sqrt{16(2^{-1/2}K_{12})^2 + [2\Delta' + \tfrac{1}{2}J_{11} + 2(\tfrac{1}{2}K_{12} - J_{12})]^2}}{8\times 2^{-1/2}K_{12}}
\end{align}
Since
\begin{align}
2\Delta' = 2(\varepsilon_2 - \varepsilon_1) - 2J_{12} + K_{12} + \dfrac{1}{2}J_{11} + J_{22} \\
2\Delta = 2(\varepsilon_2 - \varepsilon_1) + J_{11} + J_{22} - 4J_{12} + 2K_{12}
\end{align}
we have
\begin{equation}\label{key}
2\Delta = 2\Delta' + \dfrac{1}{2}J_{11} - 2J_{12} + K_{12}
\end{equation}
$ \therefore $
\begin{align}
{^2 E}_{\corr}(\text{DCI}) &= \dfrac{2\Delta - \sqrt{8K_{12}^2 + (2\Delta)^2}}{2} \notag\\
&= \Delta - \sqrt{2K_{12}^2 + \Delta^2}
\end{align}
\begin{align}
c_1 = c_2 = c_3 = c_4  &= \dfrac{2\Delta + \sqrt{8K_{12}^2 + (2\Delta)^2}}{4\sqrt{2}K_{12}} \notag\\
&= \dfrac{\Delta + \sqrt{2K_{12}^2 + \Delta^2}}{2\sqrt{2}K_{12}}
\end{align}

\ex{5.8}
\begin{align}
E_{\corr}(\text{FO}) &= \sum_{A<B}\sum_{R<S} \dfrac{\abs{\Braket{AB||RS}}^2}{\varepsilon_A + \varepsilon_B - \varepsilon_R - \varepsilon_S} \notag\\
&= \dfrac{\abs{\Braket{a\bar{a}||r\bar{r}}}^2 + \abs{\Braket{a\bar{a}||r\bar{s}}}^2 + \abs{\Braket{a\bar{a}||s\bar{r}}}^2 + \abs{\Braket{a\bar{a}||s\bar{s}}}^2}{\varepsilon_1 + \varepsilon_1 - \varepsilon_2 - \varepsilon_2} %\notag\\
%&{}\quad 
+ \dfrac{\abs{\Braket{a\bar{b}||r\bar{r}}}^2 + \abs{\Braket{a\bar{b}||r\bar{s}}}^2 + \abs{\Braket{a\bar{b}||s\bar{r}}}^2 + \abs{\Braket{a\bar{b}||s\bar{s}}}^2}{\varepsilon_1 + \varepsilon_1 - \varepsilon_2 - \varepsilon_2} \notag\\
&{}\quad + \dfrac{\abs{\Braket{b\bar{a}||r\bar{r}}}^2 + \abs{\Braket{b\bar{a}||r\bar{s}}}^2 + \abs{\Braket{b\bar{a}||s\bar{r}}}^2 + \abs{\Braket{b\bar{a}||s\bar{s}}}^2}{\varepsilon_1 + \varepsilon_1 - \varepsilon_2 - \varepsilon_2} %\notag\\
%&{}\quad 
+ \dfrac{\abs{\Braket{b\bar{b}||r\bar{r}}}^2 + \abs{\Braket{b\bar{b}||r\bar{s}}}^2 + \abs{\Braket{b\bar{b}||s\bar{r}}}^2 + \abs{\Braket{b\bar{b}||s\bar{s}}}^2}{\varepsilon_1 + \varepsilon_1 - \varepsilon_2 - \varepsilon_2} \notag\\
&= \dfrac{\abs{(ar|ar)}^2 + \abs{(ar|as)}^2 + \abs{(as|ar)}^2 + \abs{(as|as)}^2}{2(\varepsilon_1 - \varepsilon_2)} 
+ \dfrac{\abs{(ar|br)}^2 + \abs{(ar|bs)}^2 + \abs{(as|br)}^2 + \abs{(as|bs)}^2}{2(\varepsilon_1 - \varepsilon_2)} \notag\\
&{}\quad + \dfrac{\abs{(br|ar)}^2 + \abs{(br|as)}^2 + \abs{(bs|ar)}^2 + \abs{(bs|as)}^2}{2(\varepsilon_1 - \varepsilon_2)} 
+ \dfrac{\abs{(br|br)}^2 + \abs{(br|bs)}^2 + \abs{(bs|br)}^2 + \abs{(bs|bs)}^2}{2(\varepsilon_1 - \varepsilon_2)} \notag\\
&= \dfrac{\abs{\tfrac{1}{2}K_{12}}^2 + 0 + 0 + \abs{\tfrac{1}{2}K_{12}}^2}{2(\varepsilon_1 - \varepsilon_2)} 
+ \dfrac{0 + \abs{\tfrac{1}{2}K_{12}}^2 + \abs{\tfrac{1}{2}K_{12}}^2 + 0}{2(\varepsilon_1 - \varepsilon_2)} \notag\\
&{}\quad + \dfrac{0 + 0 + \abs{\tfrac{1}{2}K_{12}}^2 + \abs{\tfrac{1}{2}K_{12}}^2}{2(\varepsilon_1 - \varepsilon_2)} 
+ \dfrac{\abs{\tfrac{1}{2}K_{12}}^2 + 0 + 0 + \abs{\tfrac{1}{2}K_{12}}^2}{2(\varepsilon_1 - \varepsilon_2)} \notag\\
&= \dfrac{2K_{12}^2 }{2(\varepsilon_1 - \varepsilon_2)}
\end{align}

\ex{5.9}
\subex{a.}
\begin{align}
^2 E_{\corr}(\text{EN(L)}) &= -\sum_{a<b} \sum_{r<s} \dfrac{\abs{\Braket{\Psi_0|\sH|\Psi_{ab}^{rs}}}^2}{\Braket{\Psi_{ab}^{rs} | \sH - E_0 | \Psi_{ab}^{rs}}} \notag\\
&= -
\end{align}

\ex{5.10}

\subsubsection{Some Illustrative Calculations}

\subsection{Coupled-pair Theories}
\subsubsection{The Coupled-cluster Approximation}

\subsubsection{The Cluster Expansion of the Wave Function}
\ex{5.11}
Eq. 5.49 gives
\begin{align}
\ket{\Phi_0} &= \ket{1_1 \bar{1}_1 1_2 \bar{1}_2} 
+ c_{1_1 \bar{1}_1}^{2_1 \bar{2}_1} \ket{2_1\bar{2}_1 1_2\bar{1}_2}
+ c_{1_2 \bar{1}_2}^{2_2 \bar{2}_2} \ket{1_1\bar{1}_1 2_2\bar{2}_2}
+ c_{1_1 \bar{1}_1 1_2 \bar{1}_2}^{2_1 \bar{2}_1 2_2 \bar{2}_2} \ket{2_1\bar{2}_1 2_2\bar{2}_2} \notag\\
&= \qty[1
+ c_{1_1 \bar{1}_1}^{2_1 \bar{2}_1} a^\dagger_{2_1}a^\dagger_{\bar{2}_1} a_{\bar{1}_1}a_{1_1} 
+ c_{1_2 \bar{1}_2}^{2_2 \bar{2}_2} a^\dagger_{2_2}a^\dagger_{\bar{2}_2} a_{\bar{1}_2}a_{1_2}  
+ c_{1_1 \bar{1}_1 1_2 \bar{1}_2}^{2_1 \bar{2}_1 2_2 \bar{2}_2} a^\dagger_{2_1}a^\dagger_{\bar{2}_1}a^\dagger_{2_2}a^\dagger_{\bar{2}_2} a_{\bar{1}_2}a_{1_2}a_{\bar{1}_1}a_{1_1} ] \ket{1_1\bar{1}_1 1_2\bar{1}_2} %\notag\\ 
\end{align}
while
\begin{align}
& \quad \exp(c_{1_1 \bar{1}_1}^{2_1 \bar{2}_1} a^\dagger_{2_1}a^\dagger_{\bar{2}_1} a_{\bar{1}_1}a_{1_1} + c_{1_2 \bar{1}_2}^{2_2 \bar{2}_2} a^\dagger_{2_2}a^\dagger_{\bar{2}_2} a_{\bar{1}_2}a_{1_2}
) \ket{1_1 \bar{1}_1 1_2 \bar{1}_2} \notag\\
&= \qty[ 1 
+ \qty(c_{1_1 \bar{1}_1}^{2_1 \bar{2}_1} a^\dagger_{2_1}a^\dagger_{\bar{2}_1} a_{\bar{1}_1}a_{1_1} + c_{1_2 \bar{1}_2}^{2_2 \bar{2}_2} a^\dagger_{2_2}a^\dagger_{\bar{2}_2} a_{\bar{1}_2}a_{1_2}) 
+ \qty(c_{1_1 \bar{1}_1}^{2_1 \bar{2}_1} a^\dagger_{2_1}a^\dagger_{\bar{2}_1} a_{\bar{1}_1}a_{1_1} + c_{1_2 \bar{1}_2}^{2_2 \bar{2}_2} a^\dagger_{2_2}a^\dagger_{\bar{2}_2} a_{\bar{1}_2}a_{1_2})^2 + \cdots
] \ket{1_1 \bar{1}_1 1_2 \bar{1}_2} %\notag\\
%&= 
\end{align}
since we cannot annihilate or create any orbital twice, the terms over 3rd power must be zero, thus
\begin{align}
& \quad \exp(c_{1_1 \bar{1}_1}^{2_1 \bar{2}_1} a^\dagger_{2_1}a^\dagger_{\bar{2}_1} a_{\bar{1}_1}a_{1_1} + c_{1_2 \bar{1}_2}^{2_2 \bar{2}_2} a^\dagger_{2_2}a^\dagger_{\bar{2}_2} a_{\bar{1}_2}a_{1_2}
) \ket{1_1 \bar{1}_1 1_2 \bar{1}_2} \notag\\
&= \qty[ 1 
+ \qty(c_{1_1 \bar{1}_1}^{2_1 \bar{2}_1} a^\dagger_{2_1}a^\dagger_{\bar{2}_1} a_{\bar{1}_1}a_{1_1} + c_{1_2 \bar{1}_2}^{2_2 \bar{2}_2} a^\dagger_{2_2}a^\dagger_{\bar{2}_2} a_{\bar{1}_2}a_{1_2}) 
+ \qty(c_{1_1 \bar{1}_1}^{2_1 \bar{2}_1} a^\dagger_{2_1}a^\dagger_{\bar{2}_1} a_{\bar{1}_1}a_{1_1} + c_{1_2 \bar{1}_2}^{2_2 \bar{2}_2} a^\dagger_{2_2}a^\dagger_{\bar{2}_2} a_{\bar{1}_2}a_{1_2})^2
] \ket{1_1 \bar{1}_1 1_2 \bar{1}_2} \notag\\
&= \left[ 1 
+ \qty(c_{1_1 \bar{1}_1}^{2_1 \bar{2}_1} a^\dagger_{2_1}a^\dagger_{\bar{2}_1} a_{\bar{1}_1}a_{1_1} + c_{1_2 \bar{1}_2}^{2_2 \bar{2}_2} a^\dagger_{2_2}a^\dagger_{\bar{2}_2} a_{\bar{1}_2}a_{1_2}) 
  + \qty(c_{1_1 \bar{1}_1}^{2_1 \bar{2}_1} a^\dagger_{2_1}a^\dagger_{\bar{2}_1} a_{\bar{1}_1}a_{1_1})^2 
  + \qty(c_{1_2 \bar{1}_2}^{2_2 \bar{2}_2} a^\dagger_{2_2}a^\dagger_{\bar{2}_2} a_{\bar{1}_2}a_{1_2})^2 \right. \notag\\
&{} \qquad \left.
  + c_{1_1 \bar{1}_1}^{2_1 \bar{2}_1}c_{1_2 \bar{1}_2}^{2_2 \bar{2}_2} a^\dagger_{2_1}a^\dagger_{\bar{2}_1} a^\dagger_{2_2}a^\dagger_{\bar{2}_2} a_{\bar{1}_1}a_{1_1}  a_{\bar{1}_2}a_{1_2}
\right] \ket{1_1 \bar{1}_1 1_2 \bar{1}_2} \notag\\
&= \qty[1 
+ c_{1_1 \bar{1}_1}^{2_1 \bar{2}_1} a^\dagger_{2_1}a^\dagger_{\bar{2}_1} a_{\bar{1}_1}a_{1_1} 
+ c_{1_2 \bar{1}_2}^{2_2 \bar{2}_2} a^\dagger_{2_2}a^\dagger_{\bar{2}_2} a_{\bar{1}_2}a_{1_2}
+ c_{1_1 \bar{1}_1}^{2_1 \bar{2}_1}c_{1_2 \bar{1}_2}^{2_2 \bar{2}_2} a^\dagger_{2_1}a^\dagger_{\bar{2}_1} a^\dagger_{2_2}a^\dagger_{\bar{2}_2} a_{\bar{1}_1}a_{1_1}  a_{\bar{1}_2}a_{1_2} 
] \ket{1_1 \bar{1}_1 1_2 \bar{1}_2} \notag\\
&= \qty[1 
+ c_{1_1 \bar{1}_1}^{2_1 \bar{2}_1} a^\dagger_{2_1}a^\dagger_{\bar{2}_1} a_{\bar{1}_1}a_{1_1} 
+ c_{1_2 \bar{1}_2}^{2_2 \bar{2}_2} a^\dagger_{2_2}a^\dagger_{\bar{2}_2} a_{\bar{1}_2}a_{1_2}
+ c_{1_1 \bar{1}_1 1_2 \bar{1}_2}^{2_1 \bar{2}_1 2_2 \bar{2}_2}  a^\dagger_{2_1}a^\dagger_{\bar{2}_1} a^\dagger_{2_2}a^\dagger_{\bar{2}_2} a_{\bar{1}_1}a_{1_1}  a_{\bar{1}_2}a_{1_2} 
] \ket{1_1 \bar{1}_1 1_2 \bar{1}_2} 
\end{align}

\subsubsection{Linear CCA and the Coupled-Electron Pair Approximation}
\ex{5.12}
\subex{a.}
The diagonal elements of $ \vb{D} $ is
\begin{align}
\vb{D}_{rasb, rasb} &= \Braket{\Psi_{ab}^{rs} | \sH - E_0 | \Psi_{ab}^{rs}} %\notag\\
%&= \Braket{\Psi_{ab}^{rs} | \sH | \Psi_{ab}^{rs}} - E_0 \notag\\
%&= h_{rr} + h_{ss} - h_{aa} - h_{bb} + 
\end{align}
thus
\begin{align}
E_{\corr} &= -\vb{B}^\dagger \vb{D} \vb{B} \notag\\
&= - \dfrac{\Braket{\Psi_0 | \sH | \Psi_{ab}^{rs}}^\dagger \Braket{\Psi_0 | \sH | \Psi_{ab}^{rs}} }{\Braket{\Psi_{ab}^{rs} | \sH - E_0 | \Psi_{ab}^{rs}}} \notag\\
&= - \dfrac{\abs{\Braket{\Psi_0 | \sH | \Psi_{ab}^{rs}}}^2 }{\Braket{\Psi_{ab}^{rs} | \sH - E_0 | \Psi_{ab}^{rs}}} 
\end{align}
which matches Eq. 5.15 and 5.16.
\subex{b.}









\subsubsection{Some Illustrative Calculations}

\subsection{Many-electron Theories with Single Particle Hamiltonians}
\ex{5.13}
\begin{equation}\label{key}
C = \dfrac{-H_{11} + H_{22} - \sqrt{H_{11}^2 + 4H_{12}H_{21} - 2H_{11}H_{22} + H_{22}^2}}{2H_{12}}
\end{equation}
\begin{align}\label{key}
\varepsilon_1 &= H_{11} + H_{12}C \notag\\
&= H_{11} + \dfrac{-H_{11} + H_{22} - \sqrt{H_{11}^2 + 4H_{12}H_{21} - 2H_{11}H_{22} + H_{22}^2}}{2} \notag\\
&=  \dfrac{H_{11} + H_{22} - \sqrt{H_{11}^2 + 4H_{12}H_{21} - 2H_{11}H_{22} + H_{22}^2}}{2}
\end{align}
while the eigenvalues of the matrix is
\begin{equation}\label{key}
\dfrac{H_{11} + H_{22} \pm \sqrt{H_{11}^2 + 4H_{12}H_{21} - 2H_{11}H_{22} + H_{22}^2}}{2}
\end{equation}


\subsubsection{The Relaxation Energy via CI, IEPA, CEPA and CCA}
%\paragraph{1. Full CI}
\ex{5.14}
\subex{a.}
\begin{align}
\Braket{\Psi_0 | \sH | \Psi_b^s} &= \Braket{\Psi_0 | \sum_i [h_0(i) + v(i)] | \Psi_b^s} \notag\\
&= v_{bs}
\end{align}
\subex{b.}
Similarly
\begin{align}
\Braket{\Psi_a^r | \sH | \Psi_0} = v_{ra}
\end{align}
\subex{c.}
\begin{align}
\Braket{\Psi_a^r | \sH - E_0 | \Psi_b^s} 
&= \Braket{\Psi_a^r | \sH | \Psi_b^s} - E_0\Braket{\Psi_a^r | \Psi_b^s} \notag\\
&= \left\{\mqty{
	0 + 0 & a\neq b, r\neq s\\
	v_{rs} + 0 & a=b, r\neq s\\
	-v_{ba} + 0 & a\neq b, r=s\\
	E_0 + \varepsilon_r^{(0)} + v_{rr} - \varepsilon_a^{(0)} - v_{aa} - E_0 & a=b, r=s
}\right. \notag\\
&= \left\{\mqty{
	0  & a\neq b, r\neq s\\
	v_{rs}  & a=b, r\neq s\\
	-v_{ba}  & a\neq b, r=s\\
	\varepsilon_r^{(0)} + v_{rr} - \varepsilon_a^{(0)} - v_{aa} & a=b, r=s
}\right.
\end{align}
\subex{d.}
Since we cannot create or annihilate an orbital twice,
\begin{align}
\Braket{\Psi_a^r | \sH - E_0 | \Psi_{ab}^{rs}} = 
\left\{\mqty{
	v_{bs} & a\neq b, r\neq s\\
	0 & \text{otherwise}
}\right.
\end{align}

%\paragraph{2. SCI}

%\paragraph{3. IEPA}

%\paragraph{4. CCA}

%\paragraph{5. L-CCA}

%\paragraph{6. CEPA}

\ex{5.15}
\subex{a.}
\begin{align}
\Ket{\Phi_0} &= 
a_1b_1\cdot 0 + a_1 b_2 \ket{\chi_1^{(0)}\chi_2^{(0)}} 
+ a_1 b_3\ket{\chi_1^{(0)}\chi_3^{(0)}} + a_1 b_4\ket{\chi_1^{(0)}\chi_4^{(0)}}
\notag\\
&\quad{} + a_2 b_1\ket{\chi_2^{(0)}\chi_1^{(0)}} + a_2 b_2\cdot 0
+ a_2 b_3\ket{\chi_2^{(0)}\chi_3^{(0)}} + a_2 b_4\ket{\chi_2^{(0)}\chi_4^{(0)}}
\notag\\
&\quad{}
+ a_3 b_1\ket{\chi_3^{(0)}\chi_1^{(0)}} + a_3 b_2\ket{\chi_3^{(0)}\chi_2^{(0)}}
+ a_3 b_3\cdot 0 + a_3 b_4\ket{\chi_3^{(0)}\chi_4^{(0)}}
\notag\\
&\quad{}
+ a_4 b_1\ket{\chi_4^{(0)}\chi_1^{(0)}} + a_4 b_2\ket{\chi_4^{(0)}\chi_2^{(0)}}
+ a_4 b_3\ket{\chi_4^{(0)}\chi_3^{(0)}} + a_4 b_4\cdot 0
\notag\\
&= (a_1 b_2 - a_2 b_1) \ket{\chi_1^{(0)}\chi_2^{(0)}} 
+ (a_1 b_3 - a_3 b_1)\ket{\chi_1^{(0)}\chi_3^{(0)}} 
+ (a_1 b_4 - a_4 b_1)\ket{\chi_1^{(0)}\chi_4^{(0)}} \notag\\
&\quad{}
+ (a_2 b_3 - a_3 b_2)\ket{\chi_2^{(0)}\chi_3^{(0)}} 
+ (a_2 b_4 - a_4 b_2)\ket{\chi_2^{(0)}\chi_4^{(0)}}
+ (a_3 b_4 - a_4 b_3)\ket{\chi_3^{(0)}\chi_4^{(0)}}
\end{align}
thus
\begin{align}
\Ket{\Phi_0} = 
\end{align}
\subex{b.}





\subsubsection{The Resonance Energy of Polyenes in H\"uckel Theory}
\ex{5.16}
\begin{align}
\vb{H} = 
\mqty(\alpha & \beta & 0 & 0 & 0 & \beta\\
      \beta & \alpha & \beta & 0 & 0 & 0 \\
      0 & \beta & \alpha & \beta & 0 & 0 \\
      0 & 0 & \beta & \alpha & \beta & 0 \\
      0 & 0 & 0 & \beta & \alpha & \beta \\
      \beta & 0 & 0 & 0 & \beta & \alpha \\
      )
\end{align}
the eigenvalues are
\begin{equation}\label{key}
\alpha - 2\beta, \alpha - \beta, \alpha - \beta, \alpha + \beta, \alpha + \beta, \alpha + 2\beta
\end{equation}
while from Eq. 5.131, we get 
\begin{align}
\varepsilon_i = \alpha + 2\beta\cos\dfrac{\pi i}{3} \quad (i=0,\pm 1, \pm 2, 3)
\end{align}
i.e.
\begin{equation}\label{key}
\{\varepsilon_i\} = \{\alpha+2\beta, \alpha + \beta, \alpha + \beta, \alpha - \beta, \alpha - \beta, \alpha - 2\beta, \}
\end{equation}
which is identical to those eigenvalues.

The total energy is
\begin{align}
\mathscr{E}_0 &= 2(\alpha+2\beta + \alpha+\beta + \alpha+\beta)\\
&= 6\alpha + 8\beta
\end{align}
which agrees with Eq. 5.132.

\ex{5.17}
For Eq. 5.139
\begin{align}
\Braket{i|j} &= \dfrac{1}{2}\qty(\bra{\phi_{2i-1}} + \bra{\phi_{2i}}) \qty(\ket{\phi_{2j-1}} + \ket{\phi_{2j}}) \notag\\
&= \dfrac{1}{2}\qty(\delta_{2i-1,2j-1} + 0 + 0 + \delta_{2i,2j}) \notag\\
&= \dfrac{1}{2}\qty(\delta_{i,j} + \delta_{i,j}) \notag\\
&= \delta_{i,j}
\end{align}
$ \Braket{i^*|j^*} $ is similar.
\begin{align}
\Braket{i|j^*} &= \dfrac{1}{2}\qty(\bra{\phi_{2i-1}} + \bra{\phi_{2i}}) \qty(\ket{\phi_{2j-1}} - \ket{\phi_{2j}}) \notag\\
&= \dfrac{1}{2}\qty(\delta_{2i-1,2j-1} - 0 + 0 - \delta_{2i,2j}) \notag\\
&= \dfrac{1}{2}\qty(\delta_{i,j} - \delta_{i,j}) \notag\\
&= 0
\end{align}

For Eq. 5.140
\begin{align}
\Braket{i | h_{\eff} | i} &= \dfrac{1}{2}\qty(\bra{\phi_{2i-1}} + \bra{\phi_{2i}}) h_{\eff} \qty(\ket{\phi_{2i-1}} + \ket{\phi_{2i}}) \notag\\
&= \dfrac{1}{2} (\alpha + \beta + \beta + \alpha) \notag\\
&= \alpha + \beta
\end{align}
\begin{align}
\Braket{i^* | h_{\eff} | i^*} &= \dfrac{1}{2}\qty(\bra{\phi_{2i-1}} - \bra{\phi_{2i}}) h_{\eff} \qty(\ket{\phi_{2i-1}} - \ket{\phi_{2i}}) \notag\\
&= \dfrac{1}{2} (\alpha - \beta - \beta + \alpha) \notag\\
&= \alpha - \beta
\end{align}
\begin{align}
\Braket{i | h_{\eff} | i\pm 1} &= \dfrac{1}{2}\qty(\bra{\phi_{2i-1}} + \bra{\phi_{2i}}) h_{\eff} \qty(\ket{\phi_{2i-1\pm 2}} + \ket{\phi_{2i\pm 2}}) \notag\\
&= \left\{ 
\mqty{\dfrac{1}{2} (0 + 0 + \beta + 0) \qquad +  \\ ~\\
	  \dfrac{1}{2} (0 + \beta + 0 + 0) \qquad -
}\right. \notag\\
&= \beta/2
\end{align}
\begin{align}
\Braket{i^* | h_{\eff} | (i\pm 1)^*} &= \dfrac{1}{2} \qty(\bra{\phi_{2i-1}} - \bra{\phi_{2i}}) h_{\eff} \qty(\ket{\phi_{2i-1\pm 2}} - \ket{\phi_{2i\pm 2}}) \notag\\
&= \left\{ 
\mqty{\dfrac{1}{2} (0 - 0 - \beta + 0) \qquad +  \\ ~\\
	\dfrac{1}{2} (0 - \beta - 0 + 0) \qquad -
}\right. \notag\\
&= -\beta/2
\end{align}
\begin{align}
\Braket{i | h_{\eff} | (i\pm 1)^*} &= \dfrac{1}{2}\qty(\bra{\phi_{2i-1}} + \bra{\phi_{2i}}) h_{\eff} \qty(\ket{\phi_{2i-1\pm 2}} - \ket{\phi_{2i\pm 2}}) \notag\\
&= \left\{ 
\mqty{\dfrac{1}{2} (0 - 0 + \beta - 0) \qquad +  \\ ~\\
	\dfrac{1}{2} (0 - \beta + 0 - 0) \qquad -
}\right. \notag\\
&= \pm\beta/2
\end{align}

\ex{5.18}
\begin{align}
\Braket{\Psi_0 | \sH | \overset{*}{1}} &= 2^{-1/2} \Braket{\Psi_0 | \sH | \Psi_1^{2*} - \Psi_1^{3*}} \notag\\
&= 2^{-1/2}[\beta/2 - (-\beta/2)] \notag\\
&= 2^{-1/2}\beta
\end{align}
\begin{align}
\Braket{ \overset{*}{1} | \sH - E_0 | \overset{*}{1}} &= \dfrac{1}{2} \Braket{\Psi_1^{2*} - \Psi_1^{3*} | \sH - E_0 | \Psi_1^{2*} - \Psi_1^{3*}}
\notag\\
&= \dfrac{1}{2} \qty[
\Braket{\Psi_1^{2*} - \Psi_1^{3*} | \sH | \Psi_1^{2*} - \Psi_1^{3*}} 
- \Braket{\Psi_1^{2*} - \Psi_1^{3*} | E_0 | \Psi_1^{2*} - \Psi_1^{3*}}
] \notag\\
&= \dfrac{1}{2} \qty[ 2(\alpha-\beta) - 2(-\beta/2) - 2E_0] \notag\\
&= \alpha - \beta/2 - E_0 \notag\\
&= -\dfrac{3}{2}\beta
\end{align}
thus
\begin{align}
2^{-1/2}\beta c &= e_1 \\
2^{-1/2}\beta -\dfrac{3}{2}\beta c &= e_1 c
\end{align}
the solutions are
\begin{equation}\label{key}
c = \dfrac{-3 \pm \sqrt{17}}{2\sqrt{2}} \qquad 
e_1 = \dfrac{-3 \pm\sqrt{17}}{4}\beta
\end{equation}
and we take
\begin{equation}\label{key}
e_1 = \dfrac{-3 + \sqrt{17}}{4}\beta
\end{equation}

\ex{5.19}
\subex{a)}
\begin{align}
\ket{\Psi_1} = \ket{\Psi_0} + c_1\ket{\Psi_1^{1*}} + c_2\ket{\Psi_1^{2*}} + \cdots + c_n\ket{\Psi_1^{n*}}
\end{align}
Since
\begin{align}
\Braket{\Psi_0 | \sH | \Psi_1^{1*} } &= 0\\
\Braket{\Psi_0 | \sH | \Psi_1^{2*} } &= \beta/2 \\
\Braket{\Psi_0 | \sH | \Psi_1^{j*} } &= 0 \qquad (1<j<n)\\
\Braket{\Psi_0 | \sH | \Psi_1^{n*} } &= -\beta/2 
\end{align}
thus, 
\begin{align}
\ket{\Psi_1} = \ket{\Psi_0} + c\Ket{ \overset{*}{1}} 
\end{align}
\begin{equation}\label{key}
\Ket{ \overset{*}{1}} = 2^{-1/2} \qty(\Ket{\Psi_1^{2*} } - \Ket{\Psi_1^{n*} })
\end{equation}
As before, we get
\begin{align}
\Braket{\Psi_0 | \sH | \overset{*}{1}} %&= 2^{-1/2} \Braket{\Psi_0 | \sH | \Psi_1^{2*} - \Psi_1^{3*}} \notag\\
%&= 2^{-1/2}[\beta/2 - (-\beta/2)] \notag\\
&= 2^{-1/2}\beta
\end{align}
but
\begin{align}
\Braket{ \overset{*}{1} | \sH - E_0 | \overset{*}{1}} 
%&= \dfrac{1}{2} \Braket{\Psi_1^{2*} - \Psi_1^{n*} | \sH - E_0 | \Psi_1^{2*} - \Psi_1^{n*}}
%\notag\\
&= \dfrac{1}{2} \qty[
\Braket{\Psi_1^{2*} - \Psi_1^{3*} | \sH | \Psi_1^{2*} - \Psi_1^{3*}} 
- \Braket{\Psi_1^{2*} - \Psi_1^{3*} | E_0 | \Psi_1^{2*} - \Psi_1^{3*}}
] \notag\\
&= \dfrac{1}{2} \qty[ 2(\alpha-\beta) - 2\times 0 - 2E_0] \notag\\
&= \alpha - \beta - E_0 \notag\\
&= -2\beta
\end{align}
thus
\begin{equation}\label{key}
e_1 = \qty(-1 + \dfrac{\sqrt{6}}{2})\beta
\end{equation}
\begin{align}\label{key}
E_R(\text{IEPA}) &= Ne_1 \notag\\
&= \qty(-1 + \dfrac{\sqrt{6}}{2})N\beta \notag\\
&= 0.2247 N\beta
\end{align}

\subex{b)}
As $ N=10 $,
\begin{align}
\ket{\Psi_1} = \ket{\Psi_0} + c_1\ket{\Psi_1^{1*}} + c_2\ket{\Psi_1^{2*}} + c_3\ket{\Psi_1^{3*}} + + c_4\ket{\Psi_1^{4*}} + c_5\ket{\Psi_1^{5*}}
\end{align}
As before, let
\begin{equation}\label{key}
\Ket{\overset{*}{1}} = 2^{-1/2}\qty(\ket{\Psi_1^{1*}} - \ket{\Psi_1^{5*}})
\end{equation}
\begin{equation}\label{key}
\ket{\Psi_1} = \ket{\Psi_0} + c_1\Ket{\overset{*}{1}} + c_3\ket{\Psi_1^{3*}} + + c_4\ket{\Psi_1^{4*}}
\end{equation}
then the "particle" equations will be
\begin{align}
\Braket{\Psi_0 | \sH | \overset{*}{1}} c_1 + \Braket{\Psi_0 | \sH | \Psi_1^{3*}}c_3 + \Braket{\Psi_0 | \sH | \Psi_1^{4*}}c_4 &= e_1 \\
\Braket{\overset{*}{1} | \sH | \Psi_0} 
+ \Braket{\overset{*}{1} | \sH | \Psi_1^{3*}}c_3
+ \Braket{\overset{*}{1} | \sH | \Psi_1^{4*}}c_4
+ \Braket{\overset{*}{1} | \sH-E_0 | \overset{*}{1}}c_1 &= e_1 c_1 \\
\Braket{\Psi_1^{3*} | \sH | \Psi_0} 
+ \Braket{\Psi_1^{3*} | \sH | \overset{*}{1}}c_1
+ \Braket{\Psi_1^{3*} | \sH | \Psi_1^{4*}}c_4
+ \Braket{\Psi_1^{3*} | \sH-E_0 | \Psi_1^{3*}}c_3 &= e_1 c_3 \\
\Braket{\Psi_1^{4*} | \sH | \Psi_0} 
+ \Braket{\Psi_1^{4*} | \sH | \overset{*}{1}}c_1
+ \Braket{\Psi_1^{4*} | \sH | \Psi_1^{3*}}c_3
+ \Braket{\Psi_1^{4*} | \sH-E_0 | \Psi_1^{4*}}c_4 &= e_1 c_4 
\end{align}
where
\begin{align}
\Braket{\Psi_0 | \sH | \overset{*}{1}} = 2^{-1/2}\beta \\
\Braket{\Psi_0 | \sH | \Psi_1^{3*}} = 0\\
\Braket{\Psi_0 | \sH | \Psi_1^{4*}} = 0
\end{align}
\begin{align}
\Braket{\overset{*}{1} | \sH-E_0 | \overset{*}{1}} = -2\beta
\end{align}
\begin{align}
\Braket{\Psi_1^{3*} | \sH-E_0 | \Psi_1^{3*}} = \Braket{\Psi_1^{4*} | \sH-E_0 | \Psi_1^{4*}} = \alpha - \beta - E_0 = -2\beta
\end{align}
\begin{align}
\Braket{\overset{*}{1} | \sH | \Psi_1^{3*}} &= 2^{-1/2}\qty[\Braket{\Psi_1^{2*} | \sH | \Psi_1^{3*}} - \Braket{\Psi_1^{5*} | \sH | \Psi_1^{3*}}] \notag\\
&= 2^{-1/2}(-\beta/2)
\end{align}
\begin{align}
\Braket{\overset{*}{1} | \sH | \Psi_1^{4*}} &= 2^{-1/2}\qty[\Braket{\Psi_1^{2*} | \sH | \Psi_1^{4*}} - \Braket{\Psi_1^{5*} | \sH | \Psi_1^{4*}}] \notag\\
&= 2^{-1/2} (\beta/2)
\end{align}
\begin{align}
\Braket{\Psi_1^{3*} | \sH | \Psi_1^{4*}} 
&= -\beta/2
\end{align}
thus
\begin{align}
2^{-1/2}\beta c_1 &= e_1 \\
2^{-1/2}\beta + 2^{-1/2}(-\beta/2) c_3 + 2^{-1/2}(\beta/2) c_4 + (-2\beta) c_1 &= e_1 c_1\\
2^{-1/2}(-\beta/2) c_1 + (-\beta/2) c_4 + (-2\beta) c_3 = e_1 c_3\\
2^{-1/2} (\beta/2) c_1 + (-\beta/2) c_3 + (-2\beta) c_4 = e_1 c_4
\end{align}
or
\begin{align}
\mqty(0 & 2^{-1/2}\beta & 0 & 0\\
      2^{-1/2}\beta & -2\beta & 2^{-1/2}(-\beta/2) & 2^{-1/2}(\beta/2)\\
      0 & 2^{-1/2}(-\beta/2) & -2\beta & -\beta/2\\
      0 & 2^{-1/2}(\beta/2) & -\beta/2 & -2\beta)
\mqty(1 \\ c_1 \\ c_3 \\ c_4)
= e_1 \mqty(1 \\ c_1 \\ c_3 \\ c_4)
\end{align}
the eigenvalues are
\begin{equation}\label{key}
-\dfrac{5}{2}\beta \;\text{ or roots of }\; 
(2e_1/\beta)^3 + 7(2e_1/\beta)^2 + 9(2e_1/\beta) - 6 = 0
\end{equation}
rearrange the cubic equation, we get
\begin{align}
4e_1^3 + 14\beta e_1^2 + 9\beta^2 e_1 - 3\beta^3 = 0
\end{align}
\begin{equation}\label{key}
e_1 = -2.4627\beta, -1.2760\beta, 0.2387\beta
\end{equation}
so we take
\begin{equation}\label{key}
e_1 = 0.2387\beta
\end{equation}

\ex{5.20}
\begin{align}
\Braket{\overset{*}{1} | \sH | \overset{*}{2}} 
&= \dfrac{1}{2} \Braket{\Psi_1^{2*} - \Psi_1^{3*} | \sH | \Psi_2^{3*} - \Psi_2^{1*}} \notag\\
&= -\dfrac{1}{2} \Braket{\Psi_1^{3*} | \sH | \Psi_2^{3*} } \notag\\
&= -\dfrac{1}{2} (-1)\Braket{2 | h_{\eff} | 1} \notag\\
&= -\dfrac{1}{2} (-1)\beta/2 \notag\\
&= \beta/4
\end{align}
\begin{align}
\Braket{\overset{*}{1} | \sH | \overset{*}{3}} 
&= \dfrac{1}{2} \Braket{\Psi_1^{2*} - \Psi_1^{3*} | \sH | \Psi_3^{1*} - \Psi_3^{2*}} \notag\\
&= -\dfrac{1}{2} \Braket{\Psi_1^{2*} | \sH | \Psi_3^{2*} } \notag\\
&= -\dfrac{1}{2} (-1)\beta/2 \notag\\
&= \beta/4
\end{align}
\begin{align}
\Braket{\overset{*}{2} | \sH | \overset{*}{3}} 
&= \dfrac{1}{2} \Braket{\Psi_2^{3*} - \Psi_2^{1*} | \sH | \Psi_3^{1*} - \Psi_3^{2*}} \notag\\
&= -\dfrac{1}{2} \Braket{\Psi_2^{1*} | \sH | \Psi_3^{1*} } \notag\\
&= -\dfrac{1}{2} (-1)\beta/2 \notag\\
&= \beta/4
\end{align}
For SCI,
\begin{equation}\label{key}
\sum_{bs} v_{bs} c_b^s = E_R(\text{SCI})
\end{equation}
\begin{equation}\label{key}
v_{ra} + (\varepsilon_r^{(0)} + v_{rr})c_a^r + \sum_s v_{rs}c_a^s - (\varepsilon_a^{(0)} + v_{aa})c_a^r - \sum_b v_{ba}c_b^r = E_R(\text{SCI}) c_a^r
\end{equation}
thus
\begin{align}
6c\Braket{\overset{*}{i} | \sH | \Psi_0} &= E_R(\text{SCI}) \\
\Braket{\overset{*}{i} | \sH | \Psi_0} 
+ c\Braket{\overset{*}{i} | \sH-E_0 | \overset{*}{i}}
+ \sum_{j\neq i} c\Braket{\overset{*}{j} | \sH | \overset{*}{i}} 
%- \sum_{j\neq i} c\Braket{\overset{*}{j} | \sH | \overset{*}{i}} 
&= E_R(\text{SCI}) c
\end{align}
i.e.
\begin{align}
6c\times 2^{-1/2}\beta &= E_R(\text{SCI}) \\
2^{-1/2}\beta + c\qty(-\dfrac{3}{2}\beta + 2\times \beta/4) 
&= E_R(\text{SCI}) c 
\end{align}
$ \therefore $
\begin{align}
6c\times 2^{-1/2}\beta &= E_R(\text{SCI}) \\
2^{-1/2}\beta - c\beta &= E_R(\text{SCI}) c 
\end{align}
the solutions are
\begin{equation}\label{key}
E_R(\text{SCI}) = \dfrac{-1\pm\sqrt{13}}{2}\beta
\end{equation}
we take
\begin{equation}\label{key}
E_R(\text{SCI}) = \dfrac{-1+\sqrt{13}}{2}\beta
\end{equation}

\ex{5.21}
It's clear that
\begin{equation}\label{key}
\Braket{\Psi_0 | \sH |\overset{*}{i}} = 2^{-1/2}\beta
\end{equation}
while
\begin{align}
\Braket{\overset{*}{i} | \sH-E_0 | \overset{*}{j}} 
&= \Braket{\overset{*}{i} | \sH | \overset{*}{j}} - E_0\delta_{ij} \notag\\
&= \Braket{\Psi_i^{(i+1)*} - \Psi_i^{(i-1)*} | \sH | \Psi_j^{(j+1)*} - \Psi_j^{(j-1)*}} - E_0\delta_{ij} %\notag\\
\end{align}
If $ i=j $,
\begin{align}
\Braket{\overset{*}{i} | \sH-E_0 | \overset{*}{j}} 
%&= \Braket{\overset{*}{i} | \sH | \overset{*}{j}} - E_0\delta_{ij} \notag\\
&= \dfrac{1}{2}\Braket{\Psi_i^{(i+1)*} - \Psi_i^{(i-1)*} | \sH | \Psi_i^{(i+1)*} - \Psi_i^{(i-1)*}} - E_0 \notag\\
%%% See 5.19
&= \dfrac{1}{2}\times 2 (\alpha-\beta) - E_0 \notag\\
&= -2\beta
\end{align}
else,
\begin{align}
\Braket{\overset{*}{i} | \sH-E_0 | \overset{*}{j}} 
%&= \Braket{\overset{*}{i} | \sH | \overset{*}{j}} - E_0\delta_{ij} \notag\\
&= \dfrac{1}{2}\Braket{\Psi_i^{(i+1)*} - \Psi_i^{(i-1)*} | \sH | \Psi_j^{(j+1)*} - \Psi_j^{(j-1)*}} \notag\\
&= - \dfrac{1}{2}\Braket{\Psi_i^{(i+1)*} | \sH |  \Psi_j^{(j-1)*}} 
- \dfrac{1}{2}\Braket{\Psi_i^{(i-1)*} | \sH |  \Psi_j^{(j+1)*}} \notag\\
&= 0
\end{align}
thus
\begin{equation}\label{key}
\Braket{\overset{*}{i} | \sH-E_0 | \overset{*}{j}} = -2\beta \delta_{ij}
\end{equation}
Similar to Ex. 5.20, the SCI equations are
\begin{align}
Nc\times 2^{-1/2}\beta &= E_R(\text{SCI}) \\
2^{-1/2}\beta + c(-2\beta + 0) &= E_R(\text{SCI})c
\end{align}
$ \therefore $
\begin{equation}\label{key}
E_R(\text{SCI}) = \dfrac{-2 + \sqrt{2N + 4}}{2}\beta 
= \qty[\sqrt{1+N/2} - 1] \beta
\end{equation}

\end{document}
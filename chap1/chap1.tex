%\documentclass[UTF8]{ctexart} % use larger type; default would be 10pt
\documentclass[a4paper]{article}
\usepackage{xeCJK}
%\usepackage[utf8]{inputenc} % set input encoding (not needed with XeLaTeX)

%%% Examples of Article customizations
% These packages are optional, depending whether you want the features they provide.
% See the LaTeX Companion or other references for full information.

%%% PAGE DIMENSIONS
\usepackage{geometry} % to change the page dimensions
\geometry{a4paper} % or letterpaper (US) or a5paper or....
\geometry{margin=1in} % for example, change the margins to 2 inches all round
% \geometry{landscape} % set up the page for landscape
%   read geometry.pdf for detailed page layout information

\usepackage{graphicx} % support the \includegraphics command and options

% \usepackage[parfill]{parskip} % Activate to begin paragraphs with an empty line rather than an indent

%%% PACKAGES
\usepackage{booktabs} % for much better looking tables
\usepackage{array} % for better arrays (eg matrices) in maths
\usepackage{paralist} % very flexible & customisable lists (eg. enumerate/itemize, etc.)
\usepackage{verbatim} % adds environment for commenting out blocks of text & for better verbatim
\usepackage{subfig} % make it possible to include more than one captioned figure/table in a single float
% These packages are all incorporated in the memoir class to one degree or another...

%%% HEADERS & FOOTERS
\usepackage{fancyhdr} % This should be set AFTER setting up the page geometry
\pagestyle{fancy} % options: empty , plain , fancy
\renewcommand{\headrulewidth}{0pt} % customise the layout...
\lhead{}\chead{}\rhead{}
\lfoot{}\cfoot{\thepage}\rfoot{}

%%% SECTION TITLE APPEARANCE
\usepackage{sectsty}
\allsectionsfont{\sffamily\mdseries\upshape} % (See the fntguide.pdf for font help)
% (This matches ConTeXt defaults)

%%% ToC (table of contents) APPEARANCE
\usepackage[nottoc,notlof,notlot]{tocbibind} % Put the bibliography in the ToC
\usepackage[titles,subfigure]{tocloft} % Alter the style of the Table of Contents
\renewcommand{\cftsecfont}{\rmfamily\mdseries\upshape}
\renewcommand{\cftsecpagefont}{\rmfamily\mdseries\upshape} % No bold!

%%% END Article customizations

%%% The "real" document content comes below...

\setlength{\parindent}{0pt}
\usepackage{physics}
\usepackage{amsmath}
%\usepackage{symbols}
\usepackage{AMSFonts}
\usepackage{bm}
%\usepackage{eucal}
\usepackage{mathrsfs}
\usepackage{amssymb}
\usepackage{float}
\usepackage{multicol}
\usepackage{abstract}
\usepackage{empheq}
\usepackage{extarrows}
\usepackage{textcomp}
\usepackage{fontspec}
\usepackage{braket}
\usepackage{siunitx}
\sisetup{
	separate-uncertainty = true,
	inter-unit-product = \ensuremath{{}\cdot{}}
}

\DeclareMathOperator{\p}{\prime}
\DeclareMathOperator{\ti}{\times}
\DeclareMathOperator{\intinf}{\int_0^\infty}
\DeclareMathOperator{\intdinf}{\int_{-\infty}^\infty}
\DeclareMathOperator{\intzpi}{\int_0^\pi}
\DeclareMathOperator{\intztpi}{\int_0^{2\pi}}
\DeclareMathOperator{\sumninf}{\sum_{n=1}^{\infty}}
\DeclareMathOperator{\sumninfz}{\sum_{n=0}^\infty}
\DeclareMathOperator{\sumiinf}{\sum_{i=1}^{\infty}}
\DeclareMathOperator{\sumiinfz}{\sum_{i=0}^\infty}
\DeclareMathOperator{\sumkinf}{\sum_{k=1}^{\infty}}
\DeclareMathOperator{\sumkinfz}{\sum_{k=0}^\infty}
\DeclareMathOperator{\e}{\mathrm{e}}
\DeclareMathOperator{\I}{\mathrm{i}}
\DeclareMathOperator{\Arg}{\mathrm{Arg}}
\DeclareMathOperator{\ra}{\rightarrow}
\DeclareMathOperator{\llra}{\longleftrightarrow}
\DeclareMathOperator{\lra}{\longrightarrow}
\DeclareMathOperator{\dlra}{\Leftrightarrow}
\DeclareMathOperator{\dra}{\Rightarrow}
\newcommand{\bkk}[1]{\Braket{#1|#1}}
\newcommand{\bk}[2]{\Braket{#1|#2}}
\newcommand{\bkev}[2]{\Braket{#2|#1|#2}}



\DeclareMathOperator{\hV}{\hat{\vb{V}}}

\DeclareMathOperator{\hx}{\hat{\vb{x}}}
\DeclareMathOperator{\hy}{\hat{\vb{y}}}
\DeclareMathOperator{\hz}{\hat{\vb{z}}}

\DeclareMathOperator{\hA}{\hat{\vb{A}}}

\DeclareMathOperator{\hQ}{\hat{\vb{Q}}}
\DeclareMathOperator{\hI}{\hat{\vb{I}}}
\DeclareMathOperator{\psis}{\psi^\ast}
\DeclareMathOperator{\Psis}{\Psi^\ast}
\DeclareMathOperator{\hi}{\hat{\vb{i}}}
\DeclareMathOperator{\hj}{\hat{\vb{j}}}
\DeclareMathOperator{\hk}{\hat{\vb{k}}}
\DeclareMathOperator{\hr}{\hat{\vb{r}}}
\DeclareMathOperator{\hT}{\hat{\vb{T}}}
\DeclareMathOperator{\hH}{\hat{H}}
\DeclareMathOperator{\hh}{\hat{h}}               % helicity
\DeclareMathOperator{\hL}{\hat{\vb{L}}}
\DeclareMathOperator{\hp}{\hat{\vb{p}}}

\DeclareMathOperator{\ha}{\hat{\vb{a}}}
\DeclareMathOperator{\hS}{\hat{\vb{S}}}
\DeclareMathOperator{\hSigma}{\hat{\bm\Sigma}}
\DeclareMathOperator{\hJ}{\hat{\vb{J}}}
\DeclareMathOperator{\hP}{\hat{\vb{P}}}          % Parity
\DeclareMathOperator{\hC}{\hat{\vb{C}}} 
\DeclareMathOperator{\Tdv}{-\dfrac{\hbar^2}{2m}\dv[2]{x}}
\DeclareMathOperator{\Tna}{-\dfrac{\hbar^2}{2m}\nabla^2}
\DeclareMathOperator{\vna}{\vnabla}
\DeclareMathOperator{\nna}{\nabla^2}
\newcommand{\naCarExpd}[1]{\pdv[2]{#1}{x} + \pdv[2]{#1}{y} + \pdv[2]{#1}{z}}
\newcommand{\naCyl}{\qty[\dfrac{1}{\rho}\pdv{\rho}\qty(\rho\pdv{\rho}) + \dfrac{1}{\rho^2}\pdv[2]{\phi} + \pdv[2]{z}]}

%\DeclareMathOperator{\g#0}{\gamma^0}
%\DeclareMathOperator{\g1}{\gamma^1}
%\DeclareMathOperator{\g2}{\gamma^2}
%\DeclareMathOperator{\g3}{\gamma^3}
%\DeclareMathOperator{\g5}{\gamma^5}
\newcommand{\g}[1]{\gamma^{#1}}
\DeclareMathOperator{\gmuu}{\gamma^\mu}
\DeclareMathOperator{\gmud}{\gamma_\mu}
\newcommand{\G}[2]{g^{#1#2}}

\newcommand{\subsbul}{\subsection*{$ \bullet $}}
\newcommand{\ex}[1]{\paragraph{Ex #1}}
\newcommand{\subex}[1]{\subparagraph{#1}}
\newcommand{\dis}{\displaystyle}
\newcommand{\iden}{{\large \bm{1}}}
\newcommand{\qed}{$ \Square $}
\newcommand{\tPhi}{\tilde{\Phi} }

\numberwithin{equation}{subsection}
%\setcounter{secnumdepth}{4}
\setcounter{tocdepth}{4}
%\allowdisplaybreaks[4]

\usepackage{xcolor}
\definecolor{codegray}{gray}{0.9}
\newfontfamily\Consolas{Consolas}
\newcommand{\code}[1]{\colorbox{codegray}{{\Consolas#1}}}

\title{\textbf{Modern Quantum Chemistry, Szabo \& Ostlund}\\HW}
\author{王石嵘
\vspace{5pt}\\
%161240065\\
%Email: shirong\_wang@berkeley.edu
}
\date{\today} % Activate to display a given date or no date (if empty),
         % otherwise the current date is printed 

\begin{document}
% \boldmath

\maketitle

\tableofcontents

\newpage

%\setcounter{section}{1}
\section{Mathematical Review}
\subsection{Linear Algebra}
\subsubsection{3-D Vector Algebra}
\ex{1.1}
\subex{a)}
\begin{equation}\label{key}
\mathcal{O}\vb{e}_j = \sum_{i=1}^3 \vb{e}_i O_{ij}
\end{equation}
\begin{equation}\label{key}
\vb{e}_i \cdot \mathcal{O}\vb{e}_j = \vb{e}_i \cdot \sum_{i=1}^3 \vb{e}_i O_{ij} = O_{ij}
\end{equation}
\subex{b)}
\begin{equation}\label{key}
\begin{aligned}
\vb{b} &= \mathcal{O}\vb{a} = \sum_{i=1}^3 a_i\sum_{j=1}^3 \vb{e}_jO_{ji}\\
&=\sum_{j=1}^3 a_j\sum_{i=1}^3 \vb{e}_iO_{ij} = \sum_{i=1}^3 \vb{e}_i\sum_{j=1}^3 a_jO_{ij}
\end{aligned}
\end{equation}
thus
\begin{equation}\label{key}
\vb{b}_i = \sum_{j=1}^3 a_jO_{ij}
\end{equation}

%.
\ex{1.2} 
\begin{equation}\label{key}
[\vb{A},\vb{B}] = \mqty[0& -2& 4\\2& 0& 3\\-4& -3& 0]
\end{equation}
\begin{equation}\label{key}
\{\vb{A},\vb{B}\} = \mqty[0& 0& -2\\0& -2& 3\\-2& 3& -2]
\end{equation}

\subsubsection{Matrices}
\ex{1.3}
\begin{equation}\label{key}
(AB)_{nk} = \sum_m^M A_{nm}B_{mk}
\end{equation}
\begin{equation}\label{key}
(AB)^\dagger_{kn} = (AB)_{nk}^* = \sum_m^M A_{nm}^*B_{mk}^* = \sum_m^M B^\dagger_{km} A^\dagger_{mn} = (B^\dagger A^\dagger)_{kn}
\end{equation}
thus
\begin{equation}\label{key}
(\vb{A}\vb{B})^\dagger = \vb{B}^\dagger \vb{A}^\dagger
\end{equation}

\ex{1.4}
\subex{a.}
suppose $ \vb{A} $ is $ N\cross M $ and $ \vb{B} $ is $ M\cross N $
\begin{equation}\label{key}
\tr \vb{A}\vb{B} = \sum_n^N (AB)_{nn} = \sum_n^N \sum_m^M A_{nm}B_{mn} = \sum_m^M\sum_n^N  B_{mn}A_{nm} = \sum_m^M (BA)_{mm} = \tr \vb{B}\vb{A}
\end{equation}
\subex{b.}
\begin{equation}\label{key}
\vb{AB}(\vb{AB})^{-1} = \iden
\end{equation}
\begin{equation}\label{key}
\vb{B}^{-1}\vb{A}^{-1}\vb{A}\vb{B}(\vb{AB})^{-1} = \vb{B}^{-1}\vb{A}^{-1}\iden
\end{equation}
\begin{equation}\label{key}
\vb{B}^{-1}(\vb{A}^{-1}\vb{A})\vb{B}(\vb{AB})^{-1} = \vb{B}^{-1}\vb{A}^{-1}
\end{equation}
\begin{equation}\label{key}
\vb{B}^{-1} \iden \vb{B}(\vb{AB})^{-1} = \vb{B}^{-1}\vb{A}^{-1}
\end{equation}
thus
\begin{equation}\label{key}
(\vb{AB})^{-1} = \vb{B}^{-1}\vb{A}^{-1}
\end{equation}
\subex{c.}
\begin{equation}\label{key}
\vb{B} = \vb{U}^\dagger \vb{A} \vb{U}
\end{equation}huhhj
\begin{equation}\label{key}
\vb{U} \vb{B} \vb{U}^\dagger = \vb{U} \vb{U}^\dagger \vb{A} \vb{U} \vb{U}^\dagger = \iden \vb{A} \iden = \vb{A}
\end{equation}
\subex{d.}
$ \because \vb{C}$ is Hermitian, $ \therefore $
\begin{equation}\label{key}
\vb{C} = \vb{C}^\dagger
\end{equation}
\begin{equation}\label{key}
\vb{A}\vb{B} = (\vb{A}\vb{B})^\dagger = \vb{B}^\dagger \vb{A}^\dagger
\end{equation}
Since $ \vb{A} $, $ \vb{B} $ are Hermitian,
\begin{equation}\label{key}
\vb{A}\vb{B} = \vb{B}^\dagger \vb{A}^\dagger = \vb{B}\vb{A}
\end{equation}
$ \therefore $
\begin{equation}\label{key}
[\vb{A}, \vb{B}] = \vb{A}\vb{B} - \vb{B}\vb{A} = 0
\end{equation}
i.e. $ \vb{A} $, $ \vb{B} $ commute
\subex{e.}
Since $ \vb{A} $ is Hermitian,
\begin{equation}\label{key}
\vb{A} = \vb{A}^\dagger
\end{equation}
thus
\begin{equation}\label{key}
(\vb{A}^{1-})^\dagger \vb{A} = (\vb{A}^{1-})^\dagger \vb{A}^\dagger = (\vb{A}\vb{A}^{-1})^\dagger = \iden^\dagger = \iden
\end{equation}
thus
\begin{equation}\label{key}
(\vb{A}^{1-})^\dagger \vb{A} \vb{A}^{-1} = \vb{A}^{-1}
\end{equation}
\begin{equation}\label{key}
(\vb{A}^{1-})^\dagger = \vb{A}^{-1}
\end{equation}
i.e. $ \vb{A}^{-1} $, if it exists, is Hermitian.
\subex{f.}
Suppose
\begin{equation}\label{key}
\vb{A}^{-1} = \mqty(x & y\\z & w)
\end{equation}
thus
\begin{equation}\label{key}
\mqty(A_{11} & A_{12}\\A_{21} & A_{22})\mqty(x & y\\z & w) = \mqty(1 & 0\\0 & 1)
\end{equation}
the solution is
\begin{equation}\label{key}
\mqty{
	&x = \dfrac{A_{22}}{A_{11}A_{22} - A_{12}A_{21}}\\
	&y = \dfrac{-A_{12}}{A_{11}A_{22} - A_{12}A_{21}}\\
	&z = \dfrac{-A_{21}}{A_{11}A_{22} - A_{12}A_{21}}\\
	&w = \dfrac{A_{11}}{A_{11}A_{22} - A_{12}A_{21}}\\ 
}
\end{equation}
thus
\begin{equation}\label{key}
\vb{A}^{-1} = \dfrac{1}{\det(\vb{A})} \mqty(A_{22} & -A_{12}\\ -A_{21} & A_{11})
\end{equation}

\subsubsection{Determinants}
%.
\ex{1.5}
Suppose 
\begin{equation}\label{key}
\vb{A} = \mqty(A_{11} & A_{12}\\A_{21} & A_{22})
\end{equation}
\subex{1.}
\begin{equation}\label{key}
\mqty|0 & 0\\A_{21} & A_{22}| = 0\cdot A_{22} - 0\cdot A_{21} = 0
\end{equation}
\begin{equation}\label{key}
\mqty|0 & A_{12}\\0 & A_{22}| = 0\cdot A_{22} - 0\cdot A_{12} = 0
\end{equation}
\subex{2.}
\begin{equation}\label{key}
\det(\vb{A}) = A_{11}A_{22} -  0\cdot 0 = A_{11}A_{22}
\end{equation}
\subex{3.}
\begin{equation}\label{key}
\det(\vb{A}) = A_{11}A_{22} -  A_{12}A_{21}
\end{equation}
\begin{equation}\label{key}
\mqty| A_{21} & A_{22}\\A_{11} & A_{12} | = A_{21}A_{12} - A_{22}A_{11} = -\det(\vb{A})
\end{equation}
\subex{4.}
\begin{equation}\label{key}
\det(\vb{A}^\dagger)^* = \mqty|A_{11}^* & A_{21}^*\\A_{12}^* & A_{22}^*|^* = (A_{11}^*A_{22}^* - A_{21}^*A_{12}^*)^* = A_{11}A_{22} -  A_{12}A_{21} = \det(\vb{A})
\end{equation}
\subex{5.}
Suppose $ \vb{B} = \mqty(B_{11} & B_{12}\\B_{21} & B_{22}) $
\begin{equation}\label{key}
\begin{aligned}
\det(\vb{A}\vb{B}) &= 
\mqty|A_{11}B_{11}+A_{12}B_{21} & A_{11}B_{12}+A_{12}B_{22}\\
A_{21}B_{11}+A_{22}B_{21} & A_{21}B_{12}+A_{22}B_{22}|\\
&= (A_{11}B_{11}+A_{12}B_{21})(A_{21}B_{12}+A_{22}B_{22}) - (A_{11}B_{12}+A_{12}B_{22})(A_{21}B_{11}+A_{22}B_{21})\\
&= A_{11}B_{11}A_{21}B_{12} + A_{11}B_{11}A_{22}B_{22} + A_{12}B_{21}A_{21}B_{12} + A_{12}B_{21}A_{22}B_{22}\\
&\quad - (A_{11}B_{12}A_{21}B_{11} +  A_{11}B_{12}A_{22}B_{21} + A_{12}B_{22}A_{21}B_{11} + A_{12}B_{22}A_{22}B_{21})\\
&= A_{11}B_{11}A_{22}B_{22} + A_{12}B_{21}A_{21}B_{12} - A_{11}B_{12}A_{22}B_{21} - A_{12}B_{22}A_{21}B_{11}
\end{aligned}
\end{equation}
\begin{equation}\label{key}
\begin{aligned}
\det(\vb{A})\det(\vb{B}) &= (A_{11}A_{22} -  A_{12}A_{21})(B_{11}B_{22} -  B_{12}B_{21})\\
&= A_{11}A_{22}B_{11}B_{22} - A_{11}A_{22}B_{12}B_{21} - A_{12}A_{21}B_{11}B_{22} + A_{12}A_{21}B_{12}B_{21}\\
&= A_{11}B_{11}A_{22}B_{22} + A_{12}B_{21}A_{21}B_{12}- A_{11}B_{12}A_{22}B_{21} - A_{12}B_{22}A_{21}B_{11} \\
\end{aligned}
\end{equation}
$ \therefore $
\begin{equation}\label{key}
\det(\vb{A})\det(\vb{B}) = \det(\vb{A}\vb{B})
\end{equation}

\ex{1.6}
%.
\subex{6.}
If two rows (e.g. $ i $th and $ j $th) are equal
\begin{equation}\label{key}
\det(\vb{A}) = \mqty|  & ... & ... & \\
      A_{i1} & A_{i2} & ... & A_{in}\\
        & ... & ... & \\
      A_{j1} & A_{j2} & ... & A_{jn}\\
        & ... & ... & \\|
\xlongequal{1.5.3} -\mqty|  & ... & ... & \\
A_{j1} & A_{j2} & ... & A_{jn}\\
& ... & ... & \\
A_{i1} & A_{i2} & ... & A_{in}\\
& ... & ... & \\|
= - \mqty|  & ... & ... & \\
A_{i1} & A_{i2} & ... & A_{in}\\
& ... & ... & \\
A_{j1} & A_{j2} & ... & A_{jn}\\
& ... & ... & \\|
\end{equation}
i.e.
\begin{equation}\label{key}
\det(\vb{A}) = -\det(\vb{A})
\end{equation}
thus
\begin{equation}\label{key}
\det(\vb{A}) = 0
\end{equation}
\subex{7.}
From Ex 1.5.5, we have
\begin{equation}\label{key}
\det(\vb{A})\det(\vb{A}^{-1}) = \det(\iden) = 1
\end{equation}
thus
\begin{equation}\label{key}
\det(\vb{A}^{-1}) = \det(\vb{A})^{-1}
\end{equation}
\subex{8.}
\begin{equation}\label{key}
\vb{A}\vb{A}^\dagger = \iden \;\dra\; \det(\vb{A})\det(\vb{A}^\dagger) = \det(\iden) = 1
\end{equation}
From Ex 1.5.4, we have
\begin{equation}\label{key}
\det(\vb{A})\det(\vb{A})^* = 1
\end{equation}
\subex{9.}
From Ex 1.5.5, we get
\begin{equation}\label{key}
\det(\vb{U}^\dagger)\det(\vb{O})\det(\vb{U}) = \det(\bm\Omega)
\end{equation}
and
\begin{equation}\label{key}
\det(\vb{U}^\dagger)\det(\vb{U}) = \det(\iden) = 1
\end{equation}
$ \therefore $
\begin{equation}\label{key}
\det(\vb{O}) = \det(\bm\Omega)
\end{equation}

%.
\ex{1.7}
If $ \det(\vb{A}) \neq 0 $, thus $ \vb{A}^{-1} $ exists, we have
\begin{equation}\label{key}
\vb{A}^{-1}\vb{A}\vb{c} = \bm{0} \;\dra\; \vb{c} = \bm{0}
\end{equation}
$  \square $

\subsubsection{N-D Complex Vector Spaces}

\subsubsection{Change of Basis}
\ex{1.8}
\begin{equation}\label{key}
\Omega_{\alpha\beta} = \sum_{ij} U_{\alpha i}^\dagger O_{ij} U_{j\beta}
\end{equation}
gives
\begin{equation}\label{key}
\begin{aligned}
\tr\bm\Omega &= \sum_\alpha \Omega_{\alpha\alpha} = \sum_\alpha \sum_{ij} U_{\alpha i}^\dagger O_{ij} U_{j\alpha}\\
&= \sum_{ij} O_{ij} \sum_\alpha U_{j\alpha} U_{\alpha i}^\dagger = \sum_{ij} O_{ij} \delta_{ji} = \tr\vb{O}
\end{aligned}
\end{equation}
%while
%\begin{equation}\label{key}
%O_{ij} = \sum_{\alpha\beta} U_{i\alpha} \Omega_{\alpha\beta} U_{\beta j}^\dagger
%\end{equation}
%gives
%\begin{equation}\label{key}
%\tr \vb{O} = \sum_{i} O_{ii} = \sum_{\alpha\beta} U_{i\alpha} \Omega_{\alpha\beta} U_{i\beta}^*
%\end{equation}

\subsubsection{The Eigenvalue Problem}
%.
\ex{1.9}
\begin{equation}\label{key}
\vb{O}\vb{U} = \vb{U}\bm\omega \;\dra\; \vb{O}\mqty(\vb{c}^1 & \vb{c}^2 & \cdots & \vb{c}^N) = \mqty(\omega_1\vb{c}_1 & \omega_2\vb{c}_2 & \cdots & \omega_N\vb{c}_N)
\end{equation}
thus
\begin{equation}\label{key}
\vb{O}\vb{c}^\alpha = \omega_\alpha \vb{c}^\alpha
\end{equation}
\ex{1.10}
\begin{equation}\label{key}
\left\{
\begin{aligned}
&O_{11}-\omega + O_{12}c = 0\\
&O_{21} + (O_{22}-\omega)c = 0
\end{aligned}\right.
\end{equation}
\begin{equation}\label{key}
(O_{11}-\omega)(O_{22}-\omega) - O_{21}O_{12} = 0
\end{equation}
\begin{equation}\label{key}
\omega^2 - (O_{11}+O_{22})\omega + O_{11}O_{22} - O_{21}O_{12} = 0
\end{equation}
\begin{equation}\label{key}
\left\{\begin{aligned}
&\omega_1 = \dfrac{1}{2}\qty(O_{11}+O_{22} - \sqrt{(O_{11}-O_{22})^2 + 4O_{21}O_{12}})\\
&\omega_2 = \dfrac{1}{2}\qty(O_{11}+O_{22} + \sqrt{(O_{11}-O_{22})^2 + 4O_{21}O_{12}})
\end{aligned}\right.
\end{equation}

%.
\ex{1.11}
\subex{a)}
\begin{equation}\label{key}
\mqty| 3-\omega & 1\\1 & 3-\omega| = 0 \;\dra\; (3-\omega)^2 - 1 = 0
\end{equation}
Eigenvalues
\begin{equation}\label{key}
\omega_1 = 2 \quad \omega_2 = 4
\end{equation}
Eigenvectors
\begin{equation}\label{key}
\vb{c}^1 = \mqty(1\\-1) \quad \vb{c}^2 = \mqty(1\\1)
\end{equation}
\begin{equation}\label{key}
\mqty|3-\omega & 1\\1 & 2-\omega| = 0 \;\dra\; (3-\omega)(2-\omega) - 1 = 0
\end{equation}
Eigenvalues
\begin{equation}\label{key}
\omega_1 = \dfrac{5+\sqrt{5}}{2} \quad \omega_2 = \dfrac{5-\sqrt{5}}{2}
\end{equation}
Eigenvectors
\begin{equation}\label{key}
\vb{c}^1 = \mqty(\dfrac{1}{2}(1+\sqrt{5})\\1) \quad \vb{c}^2 = \mqty(\dfrac{1}{2}(1-\sqrt{5})\\1)
\end{equation}
\subex{b)}
\begin{equation}\label{key}
\theta_0 = \dfrac{1}{2}\tan^{-1}\dfrac{2O_{12}}{O_{11}-O_{12}}
\end{equation}
for $ \vb{A} $
\begin{equation}\label{key}
\theta_0 = \dfrac{1}{2}\tan^{-1}\dfrac{2\cross 1}{3-3} = \dfrac{\pi}{4} 
\end{equation}
Eigenvalues
\begin{equation}\label{key}
\omega_1 = 2 \quad \omega_2 = 4
\end{equation}
Eigenvectors
\begin{equation}\label{key}
\vb{c}^1 = \mqty(\sqrt{2}/2\\-\sqrt{2}/2) \quad \vb{c}^2 = \mqty(\sqrt{2}/2\\ \sqrt{2}/2)
\end{equation}
for $ \vb{B} $
\begin{equation}\label{key}
\theta_0 = \dfrac{1}{2}\tan^{-1}\dfrac{2\cross 1}{3-2} = \dfrac{1}{2}\tan^{-1}2 
\end{equation}
Eigenvalues
\begin{equation}\label{key}
\omega_1 = \dfrac{10}{5+\sqrt{5}} = \dfrac{5-\sqrt{5}}{2} \quad \omega_2 = \dfrac{10}{5-\sqrt{5}} = \dfrac{5+\sqrt{5}}{2}
\end{equation}
Eigenvectors
\begin{equation}\label{key}
\vb{c}^1 
= \mqty(\sqrt{\dfrac{\sqrt{5}+5}{10}}\vspace{5pt}\\ \sqrt{\dfrac{2}{\sqrt{5}+5}}) 
= \sqrt{\dfrac{2}{\sqrt{5}+5}} 
\mqty(\dfrac{1+\sqrt{5}}{2} \vspace{5pt}\\1) 
\end{equation}
\begin{equation}\label{key}
\vb{c}^2
= \mqty(\sqrt{\dfrac{2}{\sqrt{5}+5}}\vspace{5pt}\\ -\sqrt{\dfrac{\sqrt{5}+5}{10}}) 
= -\sqrt{\dfrac{\sqrt{5}+5}{10}}
\mqty(\dfrac{1-\sqrt{5}}{2} \vspace{5pt}\\1) 
\end{equation}
Details are in \code{chap1-1.nb}

\subsubsection{Functions of Matrices}
\ex{1.12}
\subex{a.}
\begin{equation}\label{key}
\vb{A}^n = \vb{U}\vb{a}^n\vb{U}^\dagger
\end{equation}
\begin{equation}\label{key}
\det(\vb{A}^n) = \det(\vb{U})\det(\vb{a}^n)\det(\vb{U}^\dagger) =  \det(\vb{U})\det(\vb{U}^\dagger)
\mqty|a_1^n & & & \\
      & a_2^n & & \\
      & & \ddots & \\
      & & & a_N^n | = a_1^n a_2^n \cdots a_N^n
\end{equation}
\subex{b.}
From 1.4.a, we have
\begin{equation}\label{key}
\tr \vb{A}^n = \tr(\vb{U}\vb{a}^n\vb{U}^\dagger) = \tr(\vb{U}\vb{U}^\dagger\vb{a}^n) = \tr(\vb{a}^n) = \sum_{\alpha=1}^N a_\alpha^n
\end{equation}
\subex{c.}
\begin{equation}\label{key}
\vb{U}^\dagger(\omega\iden - \vb{A})\vb{U} = \omega\iden - \vb{a}
\end{equation}
\begin{equation}\label{key}
(\omega\iden - \vb{A})^{-1} = [(\vb{U}(\omega\iden - \vb{a})\vb{U}^\dagger]^{-1} = \vb{U}(\omega\iden - \vb{a})^{-1}\vb{U}^\dagger
\end{equation}
while
\begin{equation}\label{key}
(\omega\iden - \vb{a})^{-1} = 
\mqty(\omega-a_{1} & & & \\
      & \omega-a_{2} & & \\
      & & \ddots & \\
      & & & \omega-a_{N} )^{-1}
= \mqty(\dfrac{1}{\omega-a_{1}} & & & \\
        & \dfrac{1}{\omega-a_{2}} & & \\
        & & \ddots & \\
        & & & \dfrac{1}{\omega-a_{N}} )
\end{equation}
thus
\begin{equation}\label{key}
\vb{G}(\omega) = (\omega\iden - \vb{A})^{-1} = \vb{U}
\mqty(\dfrac{1}{\omega-a_{1}} & & & \\
& \dfrac{1}{\omega-a_{2}} & & \\
& & \ddots & \\
& & & \dfrac{1}{\omega-a_{N}} )
\vb{U}^\dagger
\end{equation}
\begin{equation}\label{key}
\vb{G}(\omega)_{ij} = \sum_\alpha U_{i\alpha}\dfrac{1}{\omega-a_\alpha}U^\dagger_{\alpha j} = \sum_\alpha \dfrac{U_{i\alpha}U_{j\alpha}^*}{\omega-a_\alpha}
\end{equation}
Since $U_{i\alpha} = \Braket{i | \alpha}  $, $ U^\dagger_{\alpha j} = U_{j\alpha}^* = \Braket{\alpha | j} $
\begin{equation}\label{key}
\vb{G}(\omega)_{ij} = \sum_\alpha \dfrac{\Braket{i | \alpha} \Braket{\alpha | j}}{\omega-a_\alpha}
\end{equation}
\ex{1.13}
The eigenvalues and eigenvectors of $ \vb{A} $ are
\begin{equation}\label{key}
\omega_1 = a-b \quad \omega_2 = a+b
\end{equation}
\begin{equation}\label{key}
\vb{c}^1 = \mqty(1\\ -1) \quad \vb{c}^2 = \mqty(1\\ 1)
\end{equation}
\begin{equation}\label{key}
\vb{A} = \vb{U}\vb{a}\vb{U}^\dagger 
=\dfrac{\sqrt{2}}{2}\mqty(1 & 1\\1 & -1)\mqty(a+b & 0\\0 & a-b) \dfrac{\sqrt{2}}{2}\mqty(1 & 1\\1 & -1)
\end{equation}
\begin{equation}\label{key}
\begin{aligned}
f(\vb{A}) &= \vb{U}f(\vb{a})\vb{U}^\dagger = \dfrac{1}{2}\mqty(1 & 1\\1 & -1) \mqty(f(a+b) & 0\\0 & f(a-b)) \mqty(1 & 1\\1 & -1)\\
&= \dfrac{1}{2}\mqty(f(a+b) & f(a-b)\\ f(a+b) & -f(a-b)) \mqty(1 & 1\\1 & -1)\\
&= \dfrac{1}{2}\mqty(f(a+b)+f(a-b) & f(a+b)-f(a-b) \\ f(a+b)-f(a-b) & f(a+b)+f(a-b) )
\end{aligned}
\end{equation}

\subsection{Orthogonal Functions, Eigenfunctions, and Operators}
\ex{1.14}
\begin{equation}\label{key}
\intdinf \dd x a(x)\delta(x) = \lim_{\varepsilon\ra 0} \int_{-\varepsilon}^\varepsilon \dd x a(x)\dfrac{1}{2\varepsilon} = \lim_{\varepsilon\ra 0} \dfrac{1}{2\varepsilon} \int_{-\varepsilon}^\varepsilon \dd x a(x) 
\xlongequal{\text{L'H\^opital}} \lim_{\varepsilon\ra 0} \dfrac{a(\varepsilon) - [-a(-\varepsilon)]}{2} = a(0)
\end{equation}
\ex{1.15}
\begin{equation}\label{key}
\begin{aligned}
\int \dd x \psi_j^*(x)\mathcal{O}\psi_i(x) &= \int\dd x \psi_j^*(x)\sum_k\psi_k(x)O_{ki} = \sum_k O_{ki}\int\dd x \psi_j^*(x)\psi_k(x)\\
&= \sum_k O_{ki}\delta_{jk} =  O_{ji}
\end{aligned}
\end{equation}
In bra-ket notation, (1) becomes
\begin{equation}\label{key}
\mathcal{O}\ket{i} = \sum_j \ket{j} \Braket{j | \mathcal{O} | i}
\end{equation}
which is identical to Eq.(1.55) in the textbook.
\ex{1.16}
With bra-ket notation,
\begin{equation}\label{key}
\mathcal{O}\sumiinf c_i\ket{i} = \omega\sumiinf c_i\ket{i}
\end{equation}
Multiply by $ \bra{j} $
\begin{equation}\label{key}
\sumiinf c_i \Braket{j | \mathcal{O} | i} = \omega\sumiinf c_i \Braket{j | i} = \omega c_j
\end{equation}
i.e.
\begin{equation}\label{key}
\sumiinf O_{ji} c_i = \omega c_j
\end{equation}
\begin{equation}\label{key}
\vb{O}\vb{c} = \omega\vb{c}
\end{equation}
It's similar to prove that without bra-ket notation.

\ex{1.17}
\subex{a.}
\begin{equation}\label{key}
\int\dd x \Bra{i|x}\Braket{x|j} = \Braket{i|j} = \delta_{ij}
\end{equation}
i.e.
\begin{equation}\label{key}
\int\dd x \psi_i^*(x)\Psi_j(x) = \delta_{ij}
\end{equation}
\subex{b.}
\begin{equation}\label{key}
\sumiinf \Braket{x|i}\Braket{i|x'} = \Braket{x|x'} = \delta(x-x')
\end{equation}
thus
\begin{equation}\label{key}
 \sumiinf \psi_i^*(x)\psi_i(x') = \sumiinf \Braket{x|i}\Braket{i|x'} = \delta(x-x')
\end{equation}
\subex{c.}
\begin{equation}\label{key}
\int\dd x \Braket{x'|x}\Braket{x|a} = \Braket{x'|a}
\end{equation}
thus
\begin{equation}\label{key}
\int\dd x \delta(x' - x) a(x) = a(x')
\end{equation}
i.e.
\begin{equation}\label{key}
\int\dd x' \delta(x-x')a(x') = a(x)
\end{equation}
\subex{d.}
\begin{equation}\label{key}
\Braket{x' | \mathcal{O} | a} = \int\dd x \Braket{x' | \mathcal{O} | x} \Braket{x|a} = \Braket{x'|b}
\end{equation}
$ \therefore $
\begin{equation}\label{key}
\mathcal{O}a(x') = \int\dd x O(x',x)a(x) = b(x')
\end{equation}
i.e.
\begin{equation}\label{key}
b(x) = \mathcal{O}a(x) = \int\dd x' O(x,x')a(x')
\end{equation}
\subex{e.}
\begin{equation}\label{key}
\begin{aligned}
O(x,x') &= \Braket{x | \mathcal{O} | x'} = \bra{x} \qty(\sum_i \ket{i}\bra{i}) \mathcal{O} \qty(\sum_j\ket{j}\bra{j}) \ket{x'}\\
&= \sum_{ij} \Braket{x|i}\Braket{i | \mathcal{O} | j} \Braket{j|x'}\\
&= \sum_{ij} \psi_i(x) O_{ij} \psi_j^*(x')
\end{aligned}
\end{equation}

\subsection{The Variation Method}
\subsubsection{The Variation Principle}
\ex{1.18}
\begin{equation}\label{key}
\begin{aligned}
\mathscr{E} &= \dfrac{\Braket{\tilde{\Phi} | -\tfrac{1}{2}\dv[2]{x}-\delta(x) | \tilde{\Phi}}}{\Braket{\tilde{\Phi} | \tilde{\Phi}}} = \dfrac{N^2 \intdinf\dd x \e^{-\alpha x^2}\qty[-\dfrac{1}{2}(-2\alpha+4\alpha^2 x^2) - \delta(x)]\e^{-\alpha x^2}}{N^2\intdinf\dd x \e^{-2\alpha x^2}}\\
&= \dfrac{\alpha\dfrac{\pi^{1/2}}{(2\alpha)^{1/2}} - 2\alpha^2\dfrac{2\pi^{1/2}}{4(2\alpha)^{3/2}} - 1}{\dfrac{\pi^{1/2}}{(2\alpha)^{1/2}}}\\
&= \dfrac{\alpha \pi^{1/2} - \alpha^2\dfrac{\pi^{1/2}}{(2\alpha)} - (2\alpha)^{1/2}}{\pi^{1/2}}\\
&= \alpha - \dfrac{1}{2}\alpha - \dfrac{(2\alpha)^{1/2}}{\pi^{1/2}}\\
&= \dfrac{1}{2}\alpha - \dfrac{(2\alpha)^{1/2}}{\pi^{1/2}}
\end{aligned}
\end{equation}
Let $ \dv{\mathscr{E}}{\alpha} = 0 $, we have
\begin{equation}\label{key}
\dfrac{1}{2} - \dfrac{1}{(2\pi\alpha)^{1/2}} = 0 \;\dra\; \alpha = \dfrac{2}{\pi}
\end{equation}
thus
\begin{equation}\label{key}
\mathscr{E}_{min} = -\dfrac{1}{\pi}
\end{equation}

\ex{1.19}
\begin{equation}\label{key}
\begin{aligned}
\mathscr{E} &= \dfrac{\Braket{\tilde{\Phi} | -\dfrac{1}{2}\nabla^2 - \dfrac{1}{r} | \tilde{\Phi}}}{\Braket{\tilde{\Phi} | \tilde{\Phi}}} = \dfrac{N^2 \cdot 4\pi \intdinf r^2\dd r \e^{-\alpha r^2}\qty[-\dfrac{1}{2}(4 \alpha^2 r^2 - 6\alpha ) - \dfrac{1}{r}]\e^{-\alpha r^2}}{N^2\cdot 4\pi \intdinf r^2\dd r \e^{-2\alpha r^2}}\\
&= \dfrac{-2\alpha^2\dfrac{24\pi^{1/2}}{64(2\alpha)^{5/2}} + 3\alpha\dfrac{2\pi^{1/2}}{8(2\alpha)^{3/2}} - \dfrac{1}{2(2\alpha)} }{\dfrac{2\pi^{1/2}}{8(2\alpha)^{3/2}}}\\
&= -2\alpha^2\dfrac{12}{8(2\alpha)} + 3\alpha - \dfrac{2(2\alpha)^{1/2}}{\pi^{1/2} } \\
&= \dfrac{3}{2}\alpha - \dfrac{2(2\alpha)^{1/2}}{\pi^{1/2} }
\end{aligned}
\end{equation}
Let $ \dv{\mathscr{E}}{r} = 0 $,
\begin{equation}\label{key}
\dfrac{3}{2} - \dfrac{2}{\sqrt{2\pi\alpha}} = 0 \;\dra\; \alpha = \dfrac{8}{9\pi}
\end{equation}
\begin{equation}\label{key}
\mathscr{E}_{min} = \dfrac{4}{3\pi} - \dfrac{8}{3\pi} = -\dfrac{4}{3\pi}
\end{equation}

\ex{1.20}
\begin{equation}\label{key}
\begin{aligned}
\omega(\theta) &= \vb{c}^\dagger \vb{O} \vb{c} = \mqty(\cos\theta & \sin\theta) \mqty(O_{11}\cos\theta + O_{12}\sin\theta\\ O_{12}\cos\theta + O_{22}\sin\theta)\\
&= O_{11}\cos^2\theta + 2O_{12}\cos\theta\sin\theta + O_{22}\sin^2\theta\\
\end{aligned}
\end{equation}
Let $ \dv{\omega}{\theta} = 0 $, thus
\begin{equation}\label{key}
O_{11} (-2\cos\theta\sin\theta) + O_{12}\cdot 2\cos 2\theta + O_{22}\cdot 2\sin\theta\cos\theta = 0
\end{equation}
\begin{equation}\label{key}
(O_{22}-O_{11})\sin 2\theta + 2O_{12}\cos 2\theta = 0
\end{equation}
\begin{equation}\label{key}
\theta = \dfrac{1}{2}\arctan\dfrac{2O_{12}}{O_{11}-O_{22}}
\end{equation}
%\begin{equation}\label{key}
%\omega_{min} = \dfrac{1}{2}\qty(O_{11}+O_{22} \pm \sqrt{(O_{11}-O_{22})^2 + 4O_{12}^2})
%\end{equation}
\begin{equation}\label{key}
\omega = O_{11}\cos^2\theta + O_{12}\sin 2\theta + O_{22}\sin^2\theta
\end{equation}
which are exactly the results in Eq. (1.105) and Eq. (1.106a) in the textbook. We get the result because the trial vector $ \vb{c} $ is the exact eigenvector of $ \vb{O} $.

\subsubsection{The Linear Variational Problem}
\ex{1.21}
\subex{a.}
\begin{equation}\label{key}
\Braket{\tPhi' | \tPhi'} = 1 = \sum_{\alpha\beta} \Braket{\tPhi' | \Phi_\alpha} \Braket{\Phi_\alpha | \Phi_\beta} \Braket{\Phi_\beta | \tPhi'}
\end{equation}
Since $ \Braket{\tPhi' | \Phi_0} = 0 $, we have
\begin{equation}\label{key}
\sum_{\alpha=1}^\infty \sum_{\beta=1}^\infty \Braket{\tPhi' | \Phi_\alpha} \Braket{\Phi_\alpha | \Phi_\beta} \Braket{\Phi_\beta | \tPhi'} = 1
\end{equation}
thus
\begin{equation}\label{key}
\sum_{\alpha=1}^\infty \sum_{\beta=1}^\infty \Braket{\tPhi' | \Phi_\alpha} \delta_{\alpha\beta} \Braket{\Phi_\beta | \tPhi'} = 1
\end{equation}
\begin{equation}\label{key}
\sum_{\alpha=1}^\infty \Braket{\tPhi' | \Phi_\alpha} \Braket{\Phi_\alpha | \tPhi'} = 1
\end{equation}
\begin{equation}\label{key}
\sum_{\alpha=1}^\infty \abs{\Braket{\Phi_\alpha | \tPhi'}}^2 = 1
\end{equation}
Similarly,
\begin{equation}\label{key}
\Braket{\tPhi' | \mathscr{H} | \tPhi'} = \sum_{\alpha\beta} \Braket{\tPhi' | \Phi_\alpha} \Braket{\Phi_\alpha | \mathscr{H} | \Phi_\beta} \Braket{\Phi_\beta | \tPhi'} = \sum_{\alpha=1}^\infty \sum_{\beta=1}^\infty \Braket{\tPhi' | \Phi_\alpha} \Braket{\Phi_\alpha | \mathscr{H} | \Phi_\beta} \Braket{\Phi_\beta | \tPhi'}
\end{equation}
From Eq. (1.170) from the textbook, we get
\begin{equation}\label{key}
\Braket{\Phi_\alpha | \mathscr{H} | \Phi_\beta} = \mathscr{E}_\alpha \delta_{\alpha\beta}
\end{equation}
thus
\begin{equation}\label{key}
\Braket{\tPhi' | \mathscr{H} | \tPhi'} =\sum_{\alpha=1}^\infty \abs{\Braket{\Phi_\alpha | \tPhi'}}^2 \mathscr{E}_\alpha \geq \sum_{\alpha=1}^\infty \abs{\Braket{\Phi_\alpha | \tPhi'}}^2 \mathscr{E}_1 = \mathscr{E}_1
\end{equation}
\subex{b.}
\begin{equation}\label{key}
\Braket{\tPhi' | \tPhi'} = 1 = \qty(x^* \bra{\tPhi_0} + y^*\bra{\tPhi_1}) \qty(x\ket{\tPhi_0} + y\ket{\tPhi_1}) = \abs{x}^2 + \abs{y}^2
\end{equation}
\subex{c.}
\begin{equation}\label{key}
\begin{aligned}
\Braket{\tPhi' | \mathscr{H} | \tPhi'} &= \abs{x}^2 \Braket{\tPhi_0 | \mathscr{H} | \tPhi_0} + \abs{y}^2 \Braket{\tPhi_1 | \mathscr{H} | \tPhi_1} + x^* y \Braket{\tPhi_0 | \mathscr{H} | \tPhi_1} + x y^* \Braket{\tPhi_1 | \mathscr{H} | \tPhi_0}\\
&= E_1 - \abs{x}^2(E_1 - E_0)
\end{aligned}
\end{equation}
thus
\begin{equation}\label{key}
\mathscr{E}_1 \leq \Braket{\tPhi' | \mathscr{H} | \tPhi'} \leq E_1 - \abs{x}^2(E_1-E_1) = E_1
\end{equation}

\ex{1.22}
\begin{equation}\label{key}
\begin{aligned}
H_{11} &= \Braket{1s | \mathscr{H} | 1s} = -\dfrac{1}{2} + F\Braket{1s| r\cos\theta | 1s} = -\dfrac{1}{2}\\
H_{12} &= H_{21} = \Braket{1s | \mathscr{H} | 2p_z} = 0 + F\Braket{1s| r\cos\theta | 2p_z} = \dfrac{512 \sqrt{2} \pi }{243}F\\
H_{22} &= \Braket{2p_z | \mathscr{H} | 2p_z} = -\dfrac{1}{8} + F\Braket{2p_z| r\cos\theta | 2p_z} = -\dfrac{1}{8}
\end{aligned}
\end{equation}
Suppose $ \vb{c} = \mqty(\cos p\\ \sin p) $,
\begin{equation}\label{key}
p = \dfrac{1}{2}\arctan\dfrac{2H_{12}}{H_{11} - H_{22}} = -\dfrac{1}{2} \tan ^{-1}\qty(\dfrac{2048 \sqrt{2} F}{729})
\end{equation}
thus
\begin{equation}\label{key}
\mathscr{E}(F) = H_{11}\cos^2 p + H_{12}\sin 2p + H_{22}\sin^2 p = -\dfrac{1}{2}-\dfrac{262144}{177147}F^2+\mathcal{O}(F^3)
\end{equation}
$ \therefore $
\begin{equation}\label{key}
\alpha = 2\cross\dfrac{262144}{177147} = 2.96
\end{equation}


\end{document}
%\documentclass[UTF8]{ctexart} % use larger type; default would be 10pt
\documentclass[a4paper]{article}
\usepackage{xeCJK}
%\usepackage[utf8]{inputenc} % set input encoding (not needed with XeLaTeX)

%%% Examples of Article customizations
% These packages are optional, depending whether you want the features they provide.
% See the LaTeX Companion or other references for full information.

%%% PAGE DIMENSIONS
\usepackage{geometry} % to change the page dimensions
\geometry{a4paper} % or letterpaper (US) or a5paper or....
\geometry{margin=1in} % for example, change the margins to 2 inches all round
% \geometry{landscape} % set up the page for landscape
%   read geometry.pdf for detailed page layout information

\usepackage{graphicx} % support the \includegraphics command and options

% \usepackage[parfill]{parskip} % Activate to begin paragraphs with an empty line rather than an indent

%%% PACKAGES
\usepackage{booktabs} % for much better looking tables
\usepackage{array} % for better arrays (eg matrices) in maths
\usepackage{paralist} % very flexible & customisable lists (eg. enumerate/itemize, etc.)
\usepackage{verbatim} % adds environment for commenting out blocks of text & for better verbatim
\usepackage{subfig} % make it possible to include more than one captioned figure/table in a single float
% These packages are all incorporated in the memoir class to one degree or another...

%%% HEADERS & FOOTERS
\usepackage{fancyhdr} % This should be set AFTER setting up the page geometry
\pagestyle{fancy} % options: empty , plain , fancy
\renewcommand{\headrulewidth}{0pt} % customise the layout...
\lhead{}\chead{}\rhead{}
\lfoot{}\cfoot{\thepage}\rfoot{}

%%% SECTION TITLE APPEARANCE
\usepackage{sectsty}
%\allsectionsfont{\sffamily\mdseries\upshape} % (See the fntguide.pdf for font help)
% (This matches ConTeXt defaults)

%%% ToC (table of contents) APPEARANCE
\usepackage[nottoc,notlof,notlot]{tocbibind} % Put the bibliography in the ToC
\usepackage[titles,subfigure]{tocloft} % Alter the style of the Table of Contents
%\renewcommand{\cftsecfont}{\rmfamily\mdseries\upshape}
%\renewcommand{\cftsecpagefont}{\rmfamily\mdseries\upshape} % No bold!

%%% END Article customizations

%%% The "real" document content comes below...

\setlength{\parindent}{0pt}
\usepackage{physics}
\usepackage{amsmath}
%\usepackage{symbols}
\usepackage{AMSFonts}
\usepackage{bm}
%\usepackage{eucal}
\usepackage{mathrsfs}
\usepackage{amssymb}
\usepackage{float}
\usepackage{multicol}
\usepackage{abstract}
\usepackage{empheq}
\usepackage{extarrows}
\usepackage{textcomp}
\usepackage{fontspec}

\setmainfont{CMU Serif}
\setsansfont{CMU Sans Serif}
\setmonofont{CMU Typewriter Text}

\usepackage{braket}
\usepackage{siunitx}
\sisetup{
	separate-uncertainty = true,
	inter-unit-product = \ensuremath{{}\cdot{}}
}
\usepackage{mhchem}

\DeclareMathOperator{\p}{\prime}
\DeclareMathOperator{\ti}{\times}
\DeclareMathOperator{\intinf}{\int_0^\infty}
\DeclareMathOperator{\intdinf}{\int_{-\infty}^\infty}
\DeclareMathOperator{\intzpi}{\int_0^\pi}
\DeclareMathOperator{\intztpi}{\int_0^{2\pi}}
\DeclareMathOperator{\sumninf}{\sum_{n=1}^{\infty}}
\DeclareMathOperator{\sumninfz}{\sum_{n=0}^\infty}
\DeclareMathOperator{\sumiinf}{\sum_{i=1}^{\infty}}
\DeclareMathOperator{\sumiinfz}{\sum_{i=0}^\infty}
\DeclareMathOperator{\sumkinf}{\sum_{k=1}^{\infty}}
\DeclareMathOperator{\sumkinfz}{\sum_{k=0}^\infty}
\DeclareMathOperator{\e}{\mathrm{e}}
\DeclareMathOperator{\I}{\mathrm{i}}
\DeclareMathOperator{\Arg}{\mathrm{Arg}}
\DeclareMathOperator{\ra}{\rightarrow}
\DeclareMathOperator{\llra}{\longleftrightarrow}
\DeclareMathOperator{\lra}{\longrightarrow}
\DeclareMathOperator{\dlra}{\Leftrightarrow}
\DeclareMathOperator{\dra}{\Rightarrow}
\newcommand{\bkk}[1]{\Braket{#1|#1}}
\newcommand{\bk}[2]{\Braket{#1|#2}}
\newcommand{\bkev}[2]{\Braket{#2|#1|#2}}



\DeclareMathOperator{\hV}{\hat{\vb{V}}}

\DeclareMathOperator{\hx}{\hat{\vb{x}}}
\DeclareMathOperator{\hy}{\hat{\vb{y}}}
\DeclareMathOperator{\hz}{\hat{\vb{z}}}

\DeclareMathOperator{\hA}{\hat{\vb{A}}}

\DeclareMathOperator{\hQ}{\hat{\vb{Q}}}
\DeclareMathOperator{\hI}{\hat{\vb{I}}}
\DeclareMathOperator{\psis}{\psi^\ast}
\DeclareMathOperator{\Psis}{\Psi^\ast}
\DeclareMathOperator{\hi}{\hat{\vb{i}}}
\DeclareMathOperator{\hj}{\hat{\vb{j}}}
\DeclareMathOperator{\hk}{\hat{\vb{k}}}
\DeclareMathOperator{\hr}{\hat{\vb{r}}}
\DeclareMathOperator{\hT}{\hat{\vb{T}}}
\DeclareMathOperator{\hH}{\hat{H}}
\DeclareMathOperator{\hh}{\hat{h}}               % helicity
\DeclareMathOperator{\hL}{\hat{\vb{L}}}
\DeclareMathOperator{\hp}{\hat{\vb{p}}}

\DeclareMathOperator{\ha}{\hat{\vb{a}}}
\DeclareMathOperator{\hs}{\hat{\vb{s}}}
\DeclareMathOperator{\hS}{\hat{\vb{S}}}
\DeclareMathOperator{\hSigma}{\hat{\bm\Sigma}}
\DeclareMathOperator{\hJ}{\hat{\vb{J}}}
\DeclareMathOperator{\hP}{\hat{\vb{P}}}          % Parity
\DeclareMathOperator{\hC}{\hat{\vb{C}}} 
\DeclareMathOperator{\Tdv}{-\dfrac{\hbar^2}{2m}\dv[2]{x}}
\DeclareMathOperator{\Tna}{-\dfrac{\hbar^2}{2m}\nabla^2}
\DeclareMathOperator{\vna}{\vnabla}
\DeclareMathOperator{\nna}{\nabla^2}
\newcommand{\naCarExpd}[1]{\pdv[2]{#1}{x} + \pdv[2]{#1}{y} + \pdv[2]{#1}{z}}
\newcommand{\naCyl}{\qty[\dfrac{1}{\rho}\pdv{\rho}\qty(\rho\pdv{\rho}) + \dfrac{1}{\rho^2}\pdv[2]{\phi} + \pdv[2]{z}]}

%\DeclareMathOperator{\g#0}{\gamma^0}
%\DeclareMathOperator{\g1}{\gamma^1}
%\DeclareMathOperator{\g2}{\gamma^2}
%\DeclareMathOperator{\g3}{\gamma^3}
%\DeclareMathOperator{\g5}{\gamma^5}
\newcommand{\g}[1]{\gamma^{#1}}
\DeclareMathOperator{\gmuu}{\gamma^\mu}
\DeclareMathOperator{\gmud}{\gamma_\mu}
\newcommand{\G}[2]{g^{#1#2}}


%% MQC
\DeclareMathOperator{\sH}{\mathscr{H}}
\DeclareMathOperator{\sA}{\mathscr{A}}
\newcommand{\iden}{{\large \bm{1}}}
\newcommand{\qed}{$ \Square $}
\newcommand{\tPhi}{\tilde{\Phi} }
\newcommand{\hsP}{\hat{\mathscr{P}}}
\newcommand{\hsS}{\hat{\mathscr{S}}}
\DeclareMathOperator{\core}{\mathrm{core}}
\DeclareMathOperator{\GF}{\mathrm{GF}}
\DeclareMathOperator{\SF}{\mathrm{SF}}



\newcommand{\subsbul}{\subsection*{$ \bullet $}}
\newcommand{\ex}[1]{\paragraph{Ex #1}}
\newcommand{\subex}[1]{\subparagraph{#1}}
\newcommand{\dis}{\displaystyle}


\numberwithin{equation}{subsection}
%\setcounter{secnumdepth}{4}
\setcounter{tocdepth}{4}
\allowdisplaybreaks[1]

\usepackage{xcolor}
\definecolor{codegray}{gray}{0.9}
\newfontfamily\Consolas{Consolas}
\newcommand{\code}[1]{\colorbox{codegray}{{\Consolas#1}}}

\title{\textbf{Modern Quantum Chemistry, Szabo \& Ostlund}\\HW}
\author{王石嵘
\vspace{5pt}\\
%161240065\\
%Email: shirong\_wang@berkeley.edu
}
\date{\today} % Activate to display a given date or no date (if empty),
         % otherwise the current date is printed 

\begin{document}
% \boldmath

\maketitle

\tableofcontents

\newpage

\setcounter{section}{2}
\section{The Hartree-Fock Approximation}
\subsection{The HF Equations}
\subsubsection{The Coulomb and Exchange Operators}
\subsubsection{The Fock Operator}
\ex{3.1}
\begin{align}
\Braket{\chi_i | \hat{f} | \chi_j} &= \Braket{ \chi_i(1) | h(1) + \sum_{b} [\mathscr{J}_b(1) - \mathscr{K}_b(1)] | \chi_j(1)} \notag\\
&= [i|h|j] + \sum_{b\neq j}\qty[\Braket{\chi_i(1) \chi_b(2) | \dfrac{1}{r_{12}} | \chi_b(2)\chi_j(1)} - \Braket{\chi_i(1)\chi_b(2) | \dfrac{1}{r_{12}} | \chi_b(1)\chi_j(2)}] \notag\\
&= [i|h|j] + \sum_{b\neq j}\qty([ij|bb] - [ib|bj]) 
\end{align}
Since
\begin{equation}\label{key}
[ij|jj] - [ij|jj] = 0
\end{equation}
we have
\begin{align}
\Braket{\chi_i | \hat{f} | \chi_j}
&= \Braket{i|h|j} + \sum_b\qty(\Braket{ib|jb} - \Braket{ib|bj}) \notag\\
&= \Braket{i|h|j} + \sum_b\Braket{ib||jb}
\end{align}

\subsection{Derivation of the HF Equations}
\subsubsection{Functional Variation}
\subsubsection{Minimization of the Energy of a Single Determinant}
\ex{3.2}
Take the complex conjugate of
\begin{equation}\label{key}
\mathscr{L}[\{\chi_\alpha\}] = E_0[\{\chi_\alpha\}] - \sum_a^N\sum_b^N \varepsilon_{ba}([a|b] - \delta_{ab})
\end{equation}
we have
\begin{equation}\label{key}
\mathscr{L}[\{\chi_\alpha\}]^* = E_0[\{\chi_\alpha\}]^* - \sum_a^N\sum_b^N \varepsilon_{ba}^*([a|b]^* - \delta_{ab}^*)
\end{equation}
i.e.
\begin{equation}\label{key}
\mathscr{L}[\{\chi_\alpha\}] = E_0[\{\chi_\alpha\}] - \sum_a^N\sum_b^N \varepsilon_{ba}^*([b|a] - \delta_{ab})
\end{equation}
thus
\begin{equation}\label{key}
\sum_a^N\sum_b^N \varepsilon_{ba}([a|b] - \delta_{ab}) = \sum_a^N\sum_b^N \varepsilon_{ba}^*([b|a] - \delta_{ab}) = \sum_b^N\sum_a^N \varepsilon_{ab}^*([a|b] - \delta_{ba})
\end{equation}
$ \therefore $
\begin{equation}\label{key}
\varepsilon_{ba} = \varepsilon_{ab}^*
\end{equation}

\ex{3.3}
$ \because $
\begin{align}
[\delta\chi_a | h | \chi_a] &= [\chi_a | h | \delta\chi_a]^*\\
[\chi_a\delta\chi_a | \chi_b\chi_b] &= [\delta\chi_a\chi_a | \chi_b\chi_b]^*\\
[\chi_a\chi_a | \chi_b\delta\chi_b] &= [\chi_a\chi_a | \delta\chi_b\chi_b]^*\\
[\chi_a\chi_b | \chi_b\delta\chi_a] &= [\chi_b\delta\chi_a | \chi_a\chi_b] = [\delta\chi_a\chi_b | \chi_b\chi_a]^*\\
[\chi_a\chi_b | \delta\chi_b\chi_a] &= [\delta\chi_b\chi_a | \chi_a\chi_b] = [\chi_a\delta\chi_b | \chi_b\chi_a]^*
\end{align}
$ \therefore $
\begin{align}
\delta E_0 &= \sum_a^N[\delta\chi_a | h | \chi_a] 
+ \dfrac{1}{2}\sum_a^N\sum_b^N \qty([\delta\chi_a\chi_a | \chi_b\chi_b] + [\chi_a\chi_a | \delta\chi_b\chi_b]) \notag\\
&\quad{} - \dfrac{1}{2}\sum_a^N\sum_b^N \qty([\delta\chi_a\chi_b | \chi_b\chi_a] + [\chi_a\chi_b | \delta\chi_b\chi_a]) + \text{complex conjugates}
\end{align}
while
\begin{align}
\sum_a^N\sum_b^N [\chi_a\chi_a | \delta\chi_b\chi_b] &= \sum_b^N\sum_a^N [\chi_b\chi_b | \delta\chi_a\chi_a] = \sum_a^N\sum_b^N [\delta\chi_a\chi_a | \chi_b\chi_b]\\
\sum_a^N\sum_b^N [\chi_a\chi_b | \delta\chi_b\chi_a] &= \sum_b^N\sum_a^N [\chi_b\chi_a | \delta\chi_a\chi_b] = \sum_a^N\sum_b^N [\delta\chi_a\chi_b | \chi_b\chi_a]
\end{align}
thus
\begin{equation}\label{key}
\delta E_0 = \sum_a^N[\delta\chi_a | h | \chi_a] 
+ \sum_a^N\sum_b^N \qty([\delta\chi_a\chi_a | \chi_b\chi_b] - [\delta\chi_a\chi_b | \chi_b\chi_a]) + \text{complex conjugates}
\end{equation}

\subsubsection{The Canonical HF Equations}

\subsection{Interpretation of Solutions to the HF Equations}
\subsubsection{Orbital Energies and Koopmans' Theorem}
\ex{3.4}
\begin{align}
f_{ij} = \Braket{\chi_i | f | \chi_j} = \Braket{i|h|j} + \sum_b\Braket{ib||jb}
\end{align}
\begin{align}
f_{ji}^* &= \Braket{\chi_j | f | \chi_i}^* = \Braket{j|h|i}^* + \sum_b\Braket{jb||ib}^*\notag\\
&= \Braket{i|h|j} + \sum_b\Braket{ib||jb}\notag\\
&= f_{ij}
\end{align}
thus the Fock operator is Hermitian.

\ex{3.5}
\begin{align}
\text{IP} &= ^{N-2}E - E_0 \notag\\
&= \sum_{a\neq c,d}\Braket{a|h|a} + \dfrac{1}{2}\sum_{a\neq c,d}\sum_{b\neq c,d} \Braket{ab||ab}  - \qty[\sum_a\Braket{a|h|a} + \dfrac{1}{2}\sum_a\sum_b \Braket{ab||ab} ] \notag\\
&= -\Braket{c|h|c} - \Braket{d|h|d} -  \dfrac{1}{2}\sum_{a\neq c,d}\Braket{ac||ac}  -  \dfrac{1}{2}\sum_{a\neq c,d}\Braket{ad||ad} - \dfrac{1}{2}\sum_{b\neq c,d} \Braket{cb||cb} - \dfrac{1}{2}\sum_{b\neq c,d} \Braket{db||db}  - \Braket{cd||cd} \notag\\
&= -\Braket{c|h|c} - \Braket{d|h|d} -  \sum_{a\neq c,d}\Braket{ac||ac}  -  \sum_{a\neq c,d}\Braket{ad||ad}   - \Braket{cd||cd} \notag\\
&= -\Braket{c|h|c} - \Braket{d|h|d} - \qty( \sum_{a\neq c}\Braket{ac||ac} - \Braket{dc||dc}) -  \qty(\sum_{a\neq d}\Braket{ad||ad} - \Braket{cd||cd})  - \Braket{cd||cd} \notag\\
&= -\varepsilon_c - \varepsilon_d + \Braket{cd|cd} - \Braket{cd|dc}
\end{align}

\ex{3.6}
\begin{align}
^N E_0 - ^{N+1}E^r &= \sum_a\Braket{a|h|a} + \dfrac{1}{2}\sum_a\sum_b \Braket{ab||ab} \notag\\
&{}\quad - \qty[\sum_a\Braket{a|h|a} + \Braket{r|h|r} + \dfrac{1}{2}\sum_a\sum_b \Braket{ab||ab} + \dfrac{1}{2}\sum_b \Braket{rb||rb} + \dfrac{1}{2}\sum_a \Braket{ar||ar}] \notag\\
&= - \Braket{r|h|r} - \dfrac{1}{2}\sum_b \Braket{rb||rb} - \dfrac{1}{2}\sum_b \Braket{br||br} \notag\\
&= - \Braket{r|h|r} - \sum_b \Braket{rb||rb}
\end{align}

\subsubsection{Brillouin's Theorem}
\subsubsection{The HF Hamiltonian}
\ex{3.7}
Suppose $ \mathscr{H}_0 $ commutes with $ \mathscr{P}_n $,
\begin{align}\label{key}
\mathscr{H}_0\ket{\Psi_0} &= \mathscr{H}_0 \dfrac{1}{\sqrt{N!}} \sum_n^{N!}(-1)^{p_n} \mathscr{P}_n \qty{\sum_i^N  f(i)\chi_j(1)\cdots\chi_k(N)} \notag\\
&= \dfrac{1}{\sqrt{N!}} \sum_n^{N!}(-1)^{p_n} \mathscr{P}_n \qty{ (\varepsilon_j + \cdots + \varepsilon_k) \chi_j(1)\cdots\chi_k(N)} \notag\\
&= \sum_a \varepsilon_a
\end{align}
Now we show $ \mathscr{H}_0 $ commutes with $ \mathscr{P}_n $, for example, $ \mathscr{P}_{ab} $
\begin{equation}\label{key}
\mathscr{P}_{ab}\mathscr{H}_0 = \mathscr{P}_{ab}(\cdots + f(a) + \cdots + f(b) + \cdots) = (\cdots + f(b) + \cdots + f(a) + \cdots)\mathscr{P}_{ab} = \mathscr{H}_0 \mathscr{P}_{ab}
\end{equation}

\ex{3.8}
\begin{equation}\label{key}
\mathscr{V} = \sum_i^N \sum_{j>i}^N \mathscr{O}_2 - \sum_i^N\sum_b^N [\mathscr{G}_b(i) - \mathscr{K}_b(i)]
\end{equation}
thus
\begin{align}
\Braket{\Psi_0 | \mathscr{V} | \Psi_0} &= \sum_i^N \sum_{j>i}^N \Braket{\Psi_0 | \mathscr{O}_2 | \Psi_0} - \sum_i^N\sum_b^N [\Braket{\Psi_0 | \mathscr{G}_b(i) - \mathscr{K}_b(i)| \Psi_0}] \notag\\
&= \dfrac{1}{2}\sum_a^N \sum_b^N \Braket{ab||ab} - \sum_i^N\sum_b^N [\Braket{ib|ib} - \Braket{ib|bi}] \notag\\
&= -\dfrac{1}{2}\sum_a^N \sum_b^N \Braket{ab||ab}
\end{align}

\subsection{Restricted Closed-shell HF: The Roothaan Equations}
\subsubsection{Closed-shell HF: Restricted Spin Orbitals}
\ex{3.9}
\begin{align}
\varepsilon_i %&= \Braket{\chi_i | h | \chi_i} + \sum_b^N \Braket{\chi_i\chi_b || \chi_i\chi_b} \notag\\
&= (i|h|i) + \sum_b^N(\Braket{ib|ib} - \Braket{ib|bi}) \notag\\
&= (i|h|i) + \sum_c^{N/2}(\Braket{ic|ic} - \Braket{ic|ci}) + \sum_{\bar{c}}^{N/2}(\Braket{i\bar{c}|i\bar{c}} - \Braket{i\bar{c}|\bar{c}i}) 
\end{align}
Assume $ \chi_j $ has $ \alpha $ spin, since assuming $ \alpha $ or $ \beta $ is identical
\begin{align}
\varepsilon_i &= (i|h|i) + \sum_c^{N/2}\qty[ (ic|ic)\Braket{\alpha|\alpha}\Braket{\alpha|\alpha} 
- (ic|ci)\Braket{\alpha|\alpha}\Braket{\alpha|\alpha}]
+ \sum_c^{N/2}\qty[(ic|ic)\Braket{\alpha|\alpha}\Braket{\beta|\beta} 
- (ic|ci)\Braket{\alpha|\beta}\Braket{\beta|\alpha}] \notag\\
&= (i|h|i) + \sum_c^{N/2}\qty[ 2(ic|ic) - (ic|ci)] \notag\\
&= (i|h|i) + \sum_n^{N/2} (2J_{ib} - K_{ib})
\end{align}

\subsubsection{Introduction of a Basis: The Roothaan Equations}
\ex{3.10}
\begin{align}
(\vb{C}^\dagger \vb{S} \vb{C})_{\mu\nu} &= \sum_i\sum_j C^\dagger_{\mu i} S_{ij} C_{j\nu} \notag\\
&= \sum_i\sum_j C^*_{i\mu} \Braket{\phi_i| \phi_j} C_{j\nu} \notag\\
&= \Braket{\phi_\mu | \phi_\nu} \notag\\
&= \delta_{\mu\nu}
\end{align}
thus
\begin{equation}\label{key}
\vb{C}^\dagger \vb{S} \vb{C} = \iden
\end{equation}

\subsubsection{The Charge Density}
\ex{3.11}
\begin{align}
\rho(\vb{r}) &= \Braket{\Psi_0 | \hat{\rho}(\vb{r}) | \Psi_0} \notag\\
&= \sum_i^N \dfrac{1}{N!} \sum_I^{N!}\sum_J^{N!} (-1)^{p_I} (-1)^{p_J} \int \dd \vb{x}_1\cdots \dd \vb{x}_N \hsP_I\{\chi_1(1)\cdots\chi_N(N)\}^* \delta(\vb{r}_i - \vb{r}) \hsP_J\{\chi_1(1)\cdots\chi_N(N)\}
\end{align}
Since $ \{\chi_m\} $ are orthogonal,
\begin{align}
\rho(\vb{r}) 
&= \sum_i^N \dfrac{1}{N!} \sum_I^{N!} \int \dd \vb{x}_1\cdots \dd \vb{x}_N \hsP_I\{\chi_1(1)\cdots\chi_N(N)\}^* \delta(\vb{r}_i - \vb{r}) \hsP_I\{\chi_1(1)\cdots\chi_N(N)\} \notag\\
&= \sum_i^N \dfrac{1}{N!} (N-1)!\sum_s^N\int \dd \vb{x}_i \chi_s^*(\vb{x}_i) \delta(\vb{r}_i - \vb{r}) \chi_s(\vb{x}_i) \notag\\
&= \sum_i^N \dfrac{1}{N}\cdot 2\sum_s^{N/2}\int \dd \vb{r}_i \phi_s(\vb{r}_i) \delta(\vb{r}_i - \vb{r}) \phi_s(\vb{r}_i) \notag\\
&= \sum_i^N \dfrac{2}{N} \sum_s^{N/2} \phi_s(\vb{r}) \phi_s(\vb{r}) \notag\\
&= N \dfrac{2}{N} \sum_s^{N/2} \phi_s(\vb{r}) \phi_s(\vb{r}) \notag\\
&= 2 \sum_s^{N/2} \phi_s(\vb{r}) \phi_s(\vb{r}) 
\end{align}

\ex{3.12}
From Ex 3.10, we have
\begin{equation}\label{key}
\vb{C}^\dagger \vb{S} \vb{C} = \iden
\end{equation}
i.e.
\begin{equation}\label{key}
\sum_i^K\sum_j^K C^*_{i\mu} S_{ij} C_{j\nu} = \delta_{\mu\nu}
\end{equation}
thus
\begin{align}
(\vb{P}\vb{S}\vb{P})_{\mu\sigma} &= \sum_\nu^K\sum_\lambda^K P_{\mu\nu} S_{\nu\lambda} P_{\lambda\sigma} \notag\\
&= 4\sum_\nu^K\sum_\lambda^K \sum_a^{N/2}C_{\mu a}C_{\nu a}^* S_{\nu\lambda} \sum_b^{N/2}C_{\lambda b}C_{\sigma b}^* \notag\\
&= 4\sum_a^{N/2}\sum_b^{N/2} C_{\mu a}\qty(\sum_\nu^K\sum_\lambda^K C_{\nu a}^* S_{\nu\lambda} C_{\lambda b}) C_{\sigma b}^* \notag\\
&= 4\sum_a^{N/2}\sum_b^{N/2}  C_{\mu a} \delta_{ab} C_{\sigma b}^* \notag\\
&= 4\sum_a^{N/2} C_{\mu a} C_{\sigma a}^* \notag\\
&= 2 P_{\mu\sigma}
\end{align}
thus
\begin{equation}\label{key}
\vb{P}\vb{S}\vb{P} = 2\vb{P}
\end{equation}

\ex{3.13}
Eq. 3.122 shows
\begin{equation}\label{key}
f(\vb{r}_1) = h(\vb{r}_1) + \sum_a^{N/2}\int \dd\vb{r}_2 \psi_a^*(\vb{r}_2) (2-\hsP_{12}) r_{12}^{-1} \psi_a(\vb{r}_2)
\end{equation}
thus
\begin{align}
f(\vb{r}_1) &= h(\vb{r}_1) 
+ \sum_a^{N/2}\int \dd\vb{r}_2 
\sum_\sigma C_{\sigma a}^* \phi_\sigma^*(\vb{r}_2) (2-\hsP_{12}) r_{12}^{-1} \sum_\lambda C_{\lambda a} \phi_\lambda(\vb{r}_2) \notag\\
&= h(\vb{r}_1) 
+ \sum_\sigma\sum_\lambda \qty(\sum_a^{N/2} C_{\sigma a}^*C_{\lambda a} )
\int \dd\vb{r}_2 \phi_\sigma^*(\vb{r}_2) (2-\hsP_{12}) r_{12}^{-1}  \phi_\lambda(\vb{r}_2) \notag\\
&= h(\vb{r}_1) 
+ \dfrac{1}{2} \sum_{\sigma,\lambda} P_{\lambda\sigma}
\int \dd\vb{r}_2 \phi_\sigma^*(\vb{r}_2) (2-\hsP_{12}) r_{12}^{-1}  \phi_\lambda(\vb{r}_2) 
\end{align}

\subsubsection{Expression for the Fock Matrix}
\ex{3.14}
In expression $ (\mu\nu|\lambda\sigma) $, there are three interchangeable pairs, i.e. $\mu\leftrightarrow\nu $, $ \lambda\leftrightarrow\sigma $, and $ \mu\nu\leftrightarrow\lambda\sigma $. Thus $ (\mu\nu|\lambda\sigma) $ has an 8-fold symmetry. Similarly, %$ ,  $ have 4-fold symmetry, 
$ (\mu\mu|\lambda\sigma), (\mu\nu|\mu\lambda), (\mu\nu|\mu\nu), (\mu\mu|\sigma\sigma) $ has 2-fold symmetry, and $ (\mu\mu|\mu\nu), (\mu\mu|\mu\mu) $ has 1-fold symmetry.\\
Therefore, the number of unique 2e integrals is
\begin{table}[H]
	\centering
	\begin{tabular}{ccc}
		\hline
		expression & number & $ K=100 $\\ \hline
		 $ (\mu\nu|\lambda\sigma) $ & $ K(K-1)(K-2)(K-3)/8 $ & 11763675 \\
		 $ (\mu\mu|\lambda\sigma)$ & $ K(K-1)(K-2)/2 $& 485100\\
		 $ (\mu\nu|\mu\lambda)$ & $ K(K-1)(K-2)/2 $& 485100\\
		 $ (\mu\nu|\mu\nu) $ & $ K(K-1)/2 $ & 4950\\
		 $ (\mu\mu|\sigma\sigma) $ & $ K(K-1)/2 $ & 4950\\
		 $ (\mu\mu|\mu\nu) $ & $ K(K-1) $ & 9900\\
		 $ (\mu\mu|\mu\mu) $ & $ K $ & 100\\
		 \hline
	\end{tabular}
\end{table}
thus the total number is $ \num{12753775} $.

\subsubsection{Orthogonalization of the Basis}
\ex{3.15}
$ \because $
\begin{equation}\label{key}
\vb{U}^\dagger \vb{S} \vb{U} = \vb{s}
\end{equation}
$ \therefore $
\begin{equation}\label{key}
\vb{S} \vb{U} = \vb{U} \vb{s}
\end{equation}
i.e.
\begin{equation}\label{key}
\sum_\nu S_{\mu\nu} U_{\nu i} = U_{\mu i} s_i
\end{equation}
thus
\begin{equation}\label{key}
\sum_\mu U_{\mu i}^* \sum_\nu S_{\mu\nu} U_{\nu i} = \sum_\mu U_{\mu i}^* U_{\mu i} s_i
\end{equation}
\begin{equation}\label{key}
\sum_\mu \sum_\nu U_{\mu i}^* \Braket{\phi_\mu| \phi_\nu} U_{\nu i} = s_i \sum_\mu \abs{ U_{\mu i}}^2 
\end{equation}
Suppose
\begin{equation}\label{key}
\phi'_i = \sum_\nu U_{\nu i}\phi_\nu
\end{equation}
thus
\begin{equation}\label{key}
\Braket{\phi'_i | \phi'_i} = s_i \sum_\mu \abs{ U_{\mu i}}^2 
\end{equation}
$ \because $
\begin{equation}\label{key}
\Braket{\phi'_i | \phi'_i} > 0 \qquad \abs{ U_{\mu i}}^2 > 0
\end{equation}
$ \therefore $
\begin{equation}\label{key}
s_i > 0
\end{equation}

\ex{3.16}
\subex{$ \bullet $} (3.174)\\
Since ($ \phi,\phi',\psi $ are row vectors)
%\begin{equation}\label{key}
%\bm\phi' = \bm\phi \vb{X}
%\end{equation}
\begin{equation}\label{key}
\bm\psi = \bm\phi \vb{C} 
\end{equation}
\begin{equation}\label{key}
\bm\psi = \bm\phi' \vb{C}' = \bm\phi \vb{X} \vb{C}'
\end{equation}
we have
\begin{equation}\label{key}
\vb{C} = \vb{X} \vb{C}'
\end{equation}
i.e.
\begin{equation}\label{key}
\vb{C}' = \vb{X}^{-1} \vb{C}
\end{equation}

\subex{$ \bullet $} (3.177)
\begin{align}
F'_{\mu\nu} &= \Braket{\phi'_\mu | f | \phi'_\nu} \notag\\
&= \Braket{\sum_i \phi_i X_{i\mu} | f | \sum_j \phi_j X_{j\nu}} \notag\\
&= \sum_i\sum_j X^*_{i\mu} X_{j\nu} \Braket{\phi_i | f | \phi_j} \notag\\
&= \sum_i\sum_j X^*_{i\mu} F_{ij} X_{j\nu} 
\end{align}
i.e.
\begin{equation}\label{key}
\vb{F}' = \vb{X}^\dagger \vb{F} \vb{X}
\end{equation}

\subsubsection{The SCF Procedure}

\subsubsection{Expectation Values and Population Analysis}
\ex{3.17}
From (3.148) in the textbook, we get
\begin{equation}\label{key}
F_{\mu\nu} = H_{\mu\nu}^{\core} + G_{\mu\nu} = H_{\mu\nu}^{\core} + \sum_a^{N/2}[ 2(\mu\nu|aa) - (\mu a| a\nu)]
\end{equation}
thus
\begin{align}
E_0 &= \sum_a^{N/2} [2 h_{aa} + \sum_b^{N/2} (2 J_{ab} - K_{ab})] \notag\\
&= 2 \sum_a^{N/2} (a|h|a) + \sum_a^{N/2}\sum_b^{N/2} [2 (aa|bb) - (ab|ba)] \notag\\
&= 2 \sum_a^{N/2} \sum_\mu\sum_\nu C_{\mu a}^* C_{\nu a} (\mu| h |\nu) 
+ \sum_a^{N/2}\sum_b^{N/2} \qty[2\sum_\mu\sum_\nu C_{\mu a}^* C_{\nu a} (\mu\nu|bb) - \sum_\mu\sum_\nu C_{\mu a}^* C_{\nu a} (\mu b| b\nu)] \notag\\
&= \sum_\mu\sum_\nu P_{\nu\mu} H_{\mu\nu}^{\core} + \dfrac{1}{2}\sum_b^{N/2}\sum_\mu\sum_\nu[2 P_{\nu\mu}(\mu\nu|bb) - P_{\nu\mu}(\mu b|b\nu)] \notag\\
&= \sum_\mu\sum_\nu P_{\nu\mu} [H_{\mu\nu}^{\core} + \dfrac{1}{2}G_{\mu\nu}] \notag\\
&= \dfrac{1}{2}\sum_\mu\sum_\nu P_{\nu\mu} [H_{\mu\nu}^{\core} + F_{\mu\nu}] \notag\\
\end{align}

\ex{3.18}
For symmetrically orthogonalized basis,
\begin{equation}\label{key}
\vb{C}' = \vb{S}^{1/2}\vb{C}
\end{equation}
thus
\begin{align}
P'_{\mu\nu} &=  2\sum_a^{N/2} C'_{\mu a}C'^*_{\nu a} \notag\\
&=  2\sum_a^{N/2}  \sum_i S^{1/2}_{\mu i} C_{i a} \sum_j S^{1/2*}_{\nu j} C^*_{ja}  \notag\\
&=  \sum_i\sum_j  S^{1/2}_{\mu i} \qty( 2\sum_a^{N/2} C_{i a} C^*_{ja} ) S^{1/2*}_{\nu j}  \notag\\
&= \sum_i\sum_j  S^{1/2}_{\mu i} P_{ij} S^{1/2*}_{\nu j}  \notag\\
&= \sum_i\sum_j  S^{1/2}_{\mu i} P_{ij} S^{1/2}_{j\nu}
\end{align}
i.e.
\begin{equation}\label{key}
\vb{P}' = \vb{S}^{1/2} \vb{P} \vb{S}^{1/2}
\end{equation}
thus
\begin{equation}\label{key}
\sum_\mu (\vb{S}^{1/2} \vb{P} \vb{S}^{1/2})_{\mu\mu} = \sum_\mu \vb{P}'_{\mu\mu}
\end{equation}

\subsection{Model Calculations on \ce{H_2} and \ce{HeH^+}}
\subsubsection{The $ 1s $ Minimal STO-3G Basis Set}
\ex{3.19}
\begin{align}
\phi_{1s}^{\GF}(\alpha,\vb{r}-\vb{R}_A)\phi_{1s}^{\GF}(\alpha,\vb{r}-\vb{R}_B) &= \qty(\dfrac{2\alpha}{\pi})^{3/4} \e^{-\alpha\abs{\vb{r}-\vb{R}_A}^2} \qty(\dfrac{2\beta}{\pi})^{3/4} \e^{-\beta\abs{\vb{r}-\vb{R}_B}^2} \notag\\
&= \qty(\dfrac{2\alpha}{\pi})^{3/4} \qty(\dfrac{2\beta}{\pi})^{3/4} \e^{-\alpha\abs{\vb{r}-\vb{R}_A}^2 - \beta\abs{\vb{r}-\vb{R}_B}^2} \notag\\
&= \qty(\dfrac{2\alpha}{\pi})^{3/4} \qty(\dfrac{2\beta}{\pi})^{3/4} \exp(-\qty[(\alpha+\beta)\abs{\vb{r}}^2 
	- 2\vb{r}\cdot(\alpha\vb{R}_A + \beta\vb{R}_B) 
	+ \alpha\abs{\vb{R}_A}^2 +\beta\abs{\vb{R}_B}^2])
\end{align}
Let
\begin{equation}\label{key}
p = \alpha + \beta \qquad \vb{R}_P = \dfrac{\alpha\vb{R}_A + \beta\vb{R}_B} {\alpha+\beta}
\end{equation}
we have
\begin{align}
\phi_{1s}^{\GF}(\alpha,\vb{r}-\vb{R}_A)\phi_{1s}^{\GF}(\alpha,\vb{r}-\vb{R}_B)
&= \qty(\dfrac{2\alpha}{\pi}\dfrac{\beta}{\pi})^{3/4} 
\exp(-\qty[p\abs{\vb{r}}^2 
	- 2\vb{r}\cdot(p\vb{R}_P) 
	+ \alpha\abs{\vb{R}_A}^2 +\beta\abs{\vb{R}_B}^2]) \notag\\
&= \qty(\dfrac{2\alpha}{\pi}\dfrac{2\beta}{\pi})^{3/4} 
\exp(-\qty[p\abs{\vb{r} - \vb{R}_P}^2 
	- p\abs{\vb{R}_P}^2
	+ \alpha\abs{\vb{R}_A}^2 +\beta\abs{\vb{R}_B}^2]) \notag\\
&= \qty(\dfrac{2\alpha\beta/p}{\pi})^{3/4} \qty(\dfrac{2p}{\pi})^{3/4} 
\e^{-p\abs{\vb{r} - \vb{R}_P}^2 }
\exp(p\abs{\vb{R}_P}^2 - \alpha\abs{\vb{R}_A}^2 - \beta\abs{\vb{R}_B}^2) 
\end{align}
Let
\begin{equation}\label{key}
\phi_{1s}^{\GF}(\alpha,\vb{r}-\vb{R}_A)\phi_{1s}^{\GF}(\alpha,\vb{r}-\vb{R}_B) = K_{AB} \qty(\dfrac{2p}{\pi})^{3/4} 
\e^{-p\abs{\vb{r} - \vb{R}_P}^2 }
\end{equation}
thus
\begin{align}
K_{AB} &= \qty(\dfrac{2\alpha\beta/p}{\pi})^{3/4} 
\exp(p\abs{\vb{R}_P}^2 - \alpha\abs{\vb{R}_A}^2 -\beta\abs{\vb{R}_B}^2) \notag\\
&= \qty(\dfrac{2\alpha\beta/p}{\pi})^{3/4} 
\exp(\dfrac{1}{p}(
\alpha^2\abs{\vb{R}_A}^2 +\beta^2\abs{\vb{R}_B}^2 + 2\alpha\beta\vb{R}_A\cdot\vb{R}_B )
- \alpha\abs{\vb{R}_A}^2 -\beta\abs{\vb{R}_B}^2) \notag\\
&= \qty(\dfrac{2\alpha\beta/p}{\pi})^{3/4} 
\exp(\dfrac{1}{p} \qty(\alpha^2\abs{\vb{R}_A}^2 +\beta^2\abs{\vb{R}_B}^2 + 2\alpha\beta\vb{R}_A\cdot\vb{R}_B 
- p\alpha\abs{\vb{R}_A}^2 -p\beta\abs{\vb{R}_B}^2)) \notag\\
&= \qty(\dfrac{2\alpha\beta/p}{\pi})^{3/4} 
\exp(\dfrac{1}{p} \qty(-\alpha\beta\abs{\vb{R}_A}^2 -\alpha\beta\abs{\vb{R}_B}^2 + 2\alpha\beta\vb{R}_A\cdot\vb{R}_B )) \notag\\
&= \qty(\dfrac{2\alpha\beta}{p\pi})^{3/4} 
\exp(-\dfrac{\alpha\beta}{p} \abs{\vb{R}_A - \vb{R}_B}^2 ) 
\end{align}

\ex{3.20}
At $ r=0 $,
\begin{align}
\phi_{1s}^{\text{CGF}}(\zeta=1.0, \text{STO-1G}) &= \num{0.267656} \\
\phi_{1s}^{\text{CGF}}(\zeta=1.0, \text{STO-2G}) &= \num{0.389383} \\
\phi_{1s}^{\text{CGF}}(\zeta=1.0, \text{STO-3G}) &= \num{0.454986} 
\end{align}
while
\begin{equation}\label{key}
\phi_{1s}^{\text{SF}}(\zeta=1.0) = \dfrac{1}{\sqrt{\pi}} = \num{0.56419}
\end{equation}

\subsubsection{STO-3G \ce{H_2}}
\ex{3.21}
\begin{equation}\label{key}
\phi_{1s}^{\text{CGF}}(\zeta=1.0, \text{STO-1G}) = \phi_{1s}^{\text{GF}}(\num{0.270950})
\end{equation}
Since $ \alpha = \alpha_{(\zeta=1.0)}\cross \zeta^2 $,
\begin{equation}\label{key}
\phi_{1s}^{\text{CGF}}(\zeta=1.24, \text{STO-1G}) = \phi_{1s}^{\text{GF}}(\num{0.416613})
\end{equation}
thus
\begin{align}
S_{12} &= K_{AB} \qty(\dfrac{2\cdot 2\alpha}{\pi})^{3/4} \int \dd\vb{r} \e^{-2\alpha\abs{\vb{r}-\vb{R}_P}^2} \notag\\
&= \qty(\dfrac{2\alpha}{2\pi})^{3/4} \e^{-\tfrac{\alpha}{2}R^2} \qty(\dfrac{2\cdot 2\alpha}{\pi})^{3/4} \int \dd\vb{r} \e^{-2\alpha\abs{\vb{r}-\vb{R}_A}^2} \notag\\
&= \qty(\dfrac{2\alpha}{\pi})^{3/2} \e^{-\tfrac{\alpha}{2}R^2} 4\pi \int \dd r r^2 \e^{-2\alpha r^2} \notag\\
&= \qty(\dfrac{2\alpha}{\pi})^{3/2} \e^{-\tfrac{\alpha}{2}R^2} 4\pi \dfrac{\sqrt{\pi}}{8\sqrt{2} \alpha^{3/2}} \notag\\
&= \e^{-\tfrac{\alpha}{2}R^2}  
\end{align}
At $ R = 1.4, \alpha = \num{0.416613} $,
\begin{equation}\label{key}
S_{12} = \num{0.6648}
\end{equation}

\ex{3.22}
Let 
\begin{equation}\label{key}
\psi_1 = c_1 (\phi_1 + \phi_2) \qquad \psi_2 = c_2 (\phi_1 - \phi_2)
\end{equation}
\begin{align}
1 &= \Braket{\phi_1|\psi_1} = c_1^2 (S_{11} + S_{12} + S_{21} + S_{22}) \notag\\
&= c_1^2 (2 + 2S_{12})
\end{align}
$ \therefore $
\begin{equation}\label{key}
c_1 = [2(1 + S_{12})]^{-1/2}
\end{equation}
\begin{align}
1 &= \Braket{\phi_2|\psi_2} = c_2^2 (S_{11} - S_{12} - S_{21} + S_{22}) \notag\\
&= c_2^2 (2 - 2S_{12})
\end{align}
$ \therefore $
\begin{equation}\label{key}
c_2 = [2(1 - S_{12})]^{-1/2}
\end{equation}

\ex{3.23}
Suppose
\begin{equation}\label{key}
\psi_1 = c_1 (\phi_1 + \phi_2) \qquad \psi_2 = c_2 (\phi_1 - \phi_2)
\end{equation}
thus
\begin{equation}\label{key}
\vb{H}^{\core} \vb{C} = \vb{S} \vb{C} \bm\varepsilon
\end{equation}
\begin{equation}\label{key}
\mqty(H_{11}^{\core} & H_{12}^{\core}\\ H_{21}^{\core} & H_{22}^{\core}) \mqty(c_1 & c_2\\ c_1 & -c_2) = \mqty(S_{11} & S_{12}\\ S_{21} & S_{22}) \mqty(c_1 & c_2\\ c_1 & -c_2) \mqty(\varepsilon_1 & 0\\ 0 &\varepsilon_2 ) 
\end{equation}
\begin{equation}\label{key}
\mqty((H_{11}^{\core} + H_{12}^{\core}) c_1 & (H_{11}^{\core} - H_{12}^{\core}) c_2 \\ 
(H_{21}^{\core} + H_{22}^{\core}) c_1 & (H_{21}^{\core} - H_{22}^{\core})c_2)
= \mqty((S_{11} + S_{12}) c_1\varepsilon_1 & (S_{11} - S_{12}) c_2\varepsilon_2\\
(S_{21} + S_{22}) c_1\varepsilon_1 & (S_{21} - S_{22}) c_2\varepsilon_2)
\end{equation}
$ \therefore $
\begin{equation}\label{key}
\left\{ \mqty{\varepsilon_1 = (H_{11}^{\core} + H_{12}^{\core})/(1 + S_{12}) \\
\varepsilon_2 = (H_{11}^{\core} - H_{12}^{\core})/(1 - S_{12})}
\right.
\end{equation}
\begin{align}
\varepsilon_1 = (-1.1204-0.9584)/(1+0.6593) = -1.2528\\
\varepsilon_2 = (-1.1204+0.9584)/(1-0.6593) = -0.4755
\end{align}

\ex{3.24}


\ex{3.25}


\ex{3.26}


\ex{3.27}


\subsubsection{An SCF Calculation on STO-3G \ce{HeH^+}}
\ex{3.28}
\begin{align}
\vb{X}^\dagger_{\text{Schmidt}} \vb{S} \vb{X}_{\text{Schmidt}} &= \mqty(1 & 0 \\ -S_{12}/\sqrt{1-S_{12}^2} & 1/\sqrt{1-S_{12}^2}) 
\mqty(1 & S_{12} \\ S_{12} & 1) 
\mqty(1 & -S_{12}/\sqrt{1-S_{12}^2}\\ 0 & 1/\sqrt{1-S_{12}^2}) \notag\\
&= \mqty(1 & S_{12} \\ 0 & \sqrt{1-S_{12}^2}) 
\mqty(1 & -S_{12}/\sqrt{1-S_{12}^2}\\ 0 & 1/\sqrt{1-S_{12}^2}) \notag\\
&= \mqty(1 & 0\\ 0 & 1)
\end{align}
thus the Schmidt transformation produces orthonormal basis.

\ex{3.29}
\begin{align}
E_0(R\ra \infty) &= \dfrac{1}{2}\sum_\mu\sum_\nu P_{\nu\mu}(R\ra \infty) [2H_{\mu\nu}^{\core} + G_{\mu\nu}] 
\end{align}
where
\begin{equation}\label{key}
P_{\nu\mu}(R\ra \infty) = \mqty( 2 & 0\\ 0 & 0)
\end{equation}
\begin{align}\label{key}
G_{\mu\nu} &= \sum_\lambda\sum_\sigma P_{\lambda\sigma}(R\ra\infty) \qty[(\mu\nu|\sigma\lambda) - \dfrac{1}{2}(\mu\lambda|\sigma\nu)] \notag\\
&= 2\qty[(\mu\nu|\phi_1\phi_1) - \dfrac{1}{2}(\mu\phi_1|\phi_1\nu)]
\end{align}
thus
\begin{align}
E_0(R\ra \infty) &= \dfrac{1}{2}\sum_\mu\sum_\nu P_{\nu\mu}(R\ra \infty) [2H_{\mu\nu}^{\core} + G_{\mu\nu}] \notag\\
&= \dfrac{1}{2}\times 2[2H_{11}^{\core} + G_{11}] \notag\\
&= 2(T_{11} + V^1_{11}) + 2\qty[(\phi_1\phi_1|\phi_1\phi_1) - \dfrac{1}{2}(\phi_1\phi_1|\phi_1\phi_1)] \notag\\
&= 2T_{11} + 2V^1_{11} + (\phi_1\phi_1|\phi_1\phi_1) 
\end{align}

\subsection{Polyatomic Basis Sets}
\subsubsection{Contracted Gaussian Functions}
\subsubsection{Minimal Basis Sets: STO-3G}
\subsubsection{Double Zeta Basis Sets: 4-31G}
\ex{3.30}
The outer basis function
\begin{equation}\label{key}
\phi''_{1s}(\vb{r}) = g_{1s}(0.298073, \vb{r})
\end{equation}
The inner basis function
\begin{equation}\label{key}
\phi'_{1s}(\vb{r}) = N[0.46954 g_{1s}(1.242567, \vb{r}) 
+ 0.15457 g_{1s}(5.782948, \vb{r}) + 0.02373 g_{1s}(38.47497, \vb{r})]
\end{equation}
Renormalize it, we get
\begin{equation}\label{key}
N = 1.689
\end{equation}
thus
\begin{equation}\label{key}
\phi'_{1s}(\vb{r}) = 0.79330 g_{1s}(1.242567, \vb{r}) 
+ 0.26115 g_{1s}(5.782948, \vb{r}) + 0.04009 g_{1s}(38.47497, \vb{r})
\end{equation}

\subsubsection{Polarized Basis Sets: 6-31G* and 6-31G**}


\end{document}
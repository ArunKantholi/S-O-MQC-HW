%\documentclass[UTF8]{ctexart} % use larger type; default would be 10pt
\documentclass[a4paper]{article}
\usepackage{xeCJK}
%\usepackage[utf8]{inputenc} % set input encoding (not needed with XeLaTeX)

%%% Examples of Article customizations
% These packages are optional, depending whether you want the features they provide.
% See the LaTeX Companion or other references for full information.

%%% PAGE DIMENSIONS
\usepackage{geometry} % to change the page dimensions
\geometry{a4paper} % or letterpaper (US) or a5paper or....
\geometry{margin=1in} % for example, change the margins to 2 inches all round
% \geometry{landscape} % set up the page for landscape
%   read geometry.pdf for detailed page layout information

\usepackage{graphicx} % support the \includegraphics command and options

% \usepackage[parfill]{parskip} % Activate to begin paragraphs with an empty line rather than an indent

%%% PACKAGES
\usepackage{booktabs} % for much better looking tables
\usepackage{array} % for better arrays (eg matrices) in maths
\usepackage{paralist} % very flexible & customisable lists (eg. enumerate/itemize, etc.)
\usepackage{verbatim} % adds environment for commenting out blocks of text & for better verbatim
\usepackage{subfig} % make it possible to include more than one captioned figure/table in a single float
% These packages are all incorporated in the memoir class to one degree or another...

%%% HEADERS & FOOTERS
\usepackage{fancyhdr} % This should be set AFTER setting up the page geometry
\pagestyle{fancy} % options: empty , plain , fancy
\renewcommand{\headrulewidth}{0pt} % customise the layout...
\lhead{}\chead{}\rhead{}
\lfoot{}\cfoot{\thepage}\rfoot{}

%%% SECTION TITLE APPEARANCE
\usepackage{sectsty}
%\allsectionsfont{\sffamily\mdseries\upshape} % (See the fntguide.pdf for font help)
% (This matches ConTeXt defaults)

%%% ToC (table of contents) APPEARANCE
\usepackage[nottoc,notlof,notlot]{tocbibind} % Put the bibliography in the ToC
\usepackage[titles,subfigure]{tocloft} % Alter the style of the Table of Contents
%\renewcommand{\cftsecfont}{\rmfamily\mdseries\upshape}
%\renewcommand{\cftsecpagefont}{\rmfamily\mdseries\upshape} % No bold!

%%% END Article customizations

%%% The "real" document content comes below...

\setlength{\parindent}{0pt}
\usepackage{physics}
\usepackage{amsmath}
%\usepackage{symbols}
\usepackage{AMSFonts}
\usepackage{bm}
%\usepackage{eucal}
\usepackage{mathrsfs}
\usepackage{amssymb}
\usepackage{float}
\usepackage{multicol}
\usepackage{abstract}
\usepackage{empheq}
\usepackage{extarrows}
\usepackage{textcomp}
\usepackage{fontspec}

\setmainfont{CMU Serif}
\setsansfont{CMU Sans Serif}
\setmonofont{CMU Typewriter Text}

\usepackage{braket}
\usepackage{siunitx}
\sisetup{
	separate-uncertainty = true,
	inter-unit-product = \ensuremath{{}\cdot{}}
}
\usepackage{mhchem}

\DeclareMathOperator{\p}{\prime}
\DeclareMathOperator{\ti}{\times}
\DeclareMathOperator{\intinf}{\int_0^\infty}
\DeclareMathOperator{\intdinf}{\int_{-\infty}^\infty}
\DeclareMathOperator{\intzpi}{\int_0^\pi}
\DeclareMathOperator{\intztpi}{\int_0^{2\pi}}
\DeclareMathOperator{\sumninf}{\sum_{n=1}^{\infty}}
\DeclareMathOperator{\sumninfz}{\sum_{n=0}^\infty}
\DeclareMathOperator{\sumiinf}{\sum_{i=1}^{\infty}}
\DeclareMathOperator{\sumiinfz}{\sum_{i=0}^\infty}
\DeclareMathOperator{\sumkinf}{\sum_{k=1}^{\infty}}
\DeclareMathOperator{\sumkinfz}{\sum_{k=0}^\infty}
\DeclareMathOperator{\e}{\mathrm{e}}
\DeclareMathOperator{\I}{\mathrm{i}}
\DeclareMathOperator{\Arg}{\mathrm{Arg}}
\DeclareMathOperator{\ra}{\rightarrow}
\DeclareMathOperator{\llra}{\longleftrightarrow}
\DeclareMathOperator{\lra}{\longrightarrow}
\DeclareMathOperator{\dlra}{\Leftrightarrow}
\DeclareMathOperator{\dra}{\Rightarrow}
\newcommand{\bkk}[1]{\Braket{#1|#1}}
\newcommand{\bk}[2]{\Braket{#1|#2}}
\newcommand{\bkev}[2]{\Braket{#2|#1|#2}}



\DeclareMathOperator{\hV}{\hat{\vb{V}}}

\DeclareMathOperator{\hx}{\hat{\vb{x}}}
\DeclareMathOperator{\hy}{\hat{\vb{y}}}
\DeclareMathOperator{\hz}{\hat{\vb{z}}}

\DeclareMathOperator{\hA}{\hat{\vb{A}}}

\DeclareMathOperator{\hQ}{\hat{\vb{Q}}}
\DeclareMathOperator{\hI}{\hat{\vb{I}}}
\DeclareMathOperator{\psis}{\psi^\ast}
\DeclareMathOperator{\Psis}{\Psi^\ast}
\DeclareMathOperator{\hi}{\hat{\vb{i}}}
\DeclareMathOperator{\hj}{\hat{\vb{j}}}
\DeclareMathOperator{\hk}{\hat{\vb{k}}}
\DeclareMathOperator{\hr}{\hat{\vb{r}}}
\DeclareMathOperator{\hT}{\hat{\vb{T}}}
\DeclareMathOperator{\hH}{\hat{H}}
\DeclareMathOperator{\hh}{\hat{h}}               % helicity
\DeclareMathOperator{\hL}{\hat{\vb{L}}}
\DeclareMathOperator{\hp}{\hat{\vb{p}}}

\DeclareMathOperator{\ha}{\hat{\vb{a}}}
\DeclareMathOperator{\hs}{\hat{\vb{s}}}
\DeclareMathOperator{\hS}{\hat{\vb{S}}}
\DeclareMathOperator{\hSigma}{\hat{\bm\Sigma}}
\DeclareMathOperator{\hJ}{\hat{\vb{J}}}
\DeclareMathOperator{\hP}{\hat{\vb{P}}}          % Parity
\DeclareMathOperator{\hC}{\hat{\vb{C}}} 
\DeclareMathOperator{\Tdv}{-\dfrac{\hbar^2}{2m}\dv[2]{x}}
\DeclareMathOperator{\Tna}{-\dfrac{\hbar^2}{2m}\nabla^2}
\DeclareMathOperator{\vna}{\vnabla}
\DeclareMathOperator{\nna}{\nabla^2}
\newcommand{\naCarExpd}[1]{\pdv[2]{#1}{x} + \pdv[2]{#1}{y} + \pdv[2]{#1}{z}}
\newcommand{\naCyl}{\qty[\dfrac{1}{\rho}\pdv{\rho}\qty(\rho\pdv{\rho}) + \dfrac{1}{\rho^2}\pdv[2]{\phi} + \pdv[2]{z}]}

%% MQC
\DeclareMathOperator{\sH}{\mathscr{H}}
\DeclareMathOperator{\sA}{\mathscr{A}}
\newcommand{\iden}{{\large \bm{1}}}
\newcommand{\qed}{$ \Square $}
\newcommand{\tPhi}{\tilde{\Phi} }
\newcommand{\hsP}{\hat{\mathscr{P}}}
\newcommand{\hsS}{\hat{\mathscr{S}}}

%\DeclareMathOperator{\g#0}{\gamma^0}
%\DeclareMathOperator{\g1}{\gamma^1}
%\DeclareMathOperator{\g2}{\gamma^2}
%\DeclareMathOperator{\g3}{\gamma^3}
%\DeclareMathOperator{\g5}{\gamma^5}
\newcommand{\g}[1]{\gamma^{#1}}
\DeclareMathOperator{\gmuu}{\gamma^\mu}
\DeclareMathOperator{\gmud}{\gamma_\mu}
\newcommand{\G}[2]{g^{#1#2}}

\newcommand{\subsbul}{\subsection*{$ \bullet $}}
\newcommand{\ex}[1]{\paragraph{Ex #1}}
\newcommand{\subex}[1]{\subparagraph{#1}}
\newcommand{\dis}{\displaystyle}


\numberwithin{equation}{subsection}
%\setcounter{secnumdepth}{4}
\setcounter{tocdepth}{4}
\allowdisplaybreaks[1]

\usepackage{xcolor}
\definecolor{codegray}{gray}{0.9}
\newfontfamily\Consolas{Consolas}
\newcommand{\code}[1]{\colorbox{codegray}{{\Consolas#1}}}

\title{\textbf{Modern Quantum Chemistry, Szabo \& Ostlund}\\HW}
\author{王石嵘
\vspace{5pt}\\
%161240065\\
%Email: shirong\_wang@berkeley.edu
}
\date{\today} % Activate to display a given date or no date (if empty),
         % otherwise the current date is printed 

\begin{document}
% \boldmath

\maketitle

\tableofcontents

\newpage

\setcounter{section}{2}
\section{The Hartree-Fock Approximation}
\subsection{The HF Equations}
\subsubsection{The Coulomb and Exchange Operators}
\subsubsection{The Fock Operator}
\ex{3.1}
\begin{align}
\Braket{\chi_i | \hat{f} | \chi_j} &= \Braket{ \chi_i(1) | h(1) + \sum_{b} [\mathscr{J}_b(1) - \mathscr{K}_b(1)] | \chi_j(1)} \notag\\
&= [i|h|j] + \sum_{b\neq j}\qty[\Braket{\chi_i(1) \chi_b(2) | \dfrac{1}{r_{12}} | \chi_b(2)\chi_j(1)} - \Braket{\chi_i(1)\chi_b(2) | \dfrac{1}{r_{12}} | \chi_b(1)\chi_j(2)}] \notag\\
&= [i|h|j] + \sum_{b\neq j}\qty([ij|bb] - [ib|bj]) 
\end{align}
Since
\begin{equation}\label{key}
[ij|jj] - [ij|jj] = 0
\end{equation}
we have
\begin{align}
\Braket{\chi_i | \hat{f} | \chi_j}
&= \Braket{i|h|j} + \sum_b\qty(\Braket{ib|jb} - \Braket{ib|bj}) \notag\\
&= \Braket{i|h|j} + \sum_b\Braket{ib||jb}
\end{align}

\subsection{Derivation of the HF Equations}
\subsubsection{Functional Variation}
\subsubsection{Minimization of the Energy of a Single Determinant}
\ex{3.2}
Take the complex conjugate of
\begin{equation}\label{key}
\mathscr{L}[\{\chi_\alpha\}] = E_0[\{\chi_\alpha\}] - \sum_a^N\sum_b^N \varepsilon_{ba}([a|b] - \delta_{ab})
\end{equation}
we have
\begin{equation}\label{key}
\mathscr{L}[\{\chi_\alpha\}]^* = E_0[\{\chi_\alpha\}]^* - \sum_a^N\sum_b^N \varepsilon_{ba}^*([a|b]^* - \delta_{ab}^*)
\end{equation}
i.e.
\begin{equation}\label{key}
\mathscr{L}[\{\chi_\alpha\}] = E_0[\{\chi_\alpha\}] - \sum_a^N\sum_b^N \varepsilon_{ba}^*([b|a] - \delta_{ab})
\end{equation}
thus
\begin{equation}\label{key}
\sum_a^N\sum_b^N \varepsilon_{ba}([a|b] - \delta_{ab}) = \sum_a^N\sum_b^N \varepsilon_{ba}^*([b|a] - \delta_{ab}) = \sum_b^N\sum_a^N \varepsilon_{ab}^*([a|b] - \delta_{ba})
\end{equation}
$ \therefore $
\begin{equation}\label{key}
\varepsilon_{ba} = \varepsilon_{ab}^*
\end{equation}

\ex{3.3}
$ \because $
\begin{align}
[\delta\chi_a | h | \chi_a] &= [\chi_a | h | \delta\chi_a]^*\\
[\chi_a\delta\chi_a | \chi_b\chi_b] &= [\delta\chi_a\chi_a | \chi_b\chi_b]^*\\
[\chi_a\chi_a | \chi_b\delta\chi_b] &= [\chi_a\chi_a | \delta\chi_b\chi_b]^*\\
[\chi_a\chi_b | \chi_b\delta\chi_a] &= [\chi_b\delta\chi_a | \chi_a\chi_b] = [\delta\chi_a\chi_b | \chi_b\chi_a]^*\\
[\chi_a\chi_b | \delta\chi_b\chi_a] &= [\delta\chi_b\chi_a | \chi_a\chi_b] = [\chi_a\delta\chi_b | \chi_b\chi_a]^*
\end{align}
$ \therefore $
\begin{align}
\delta E_0 &= \sum_a^N[\delta\chi_a | h | \chi_a] 
+ \dfrac{1}{2}\sum_a^N\sum_b^N \qty([\delta\chi_a\chi_a | \chi_b\chi_b] + [\chi_a\chi_a | \delta\chi_b\chi_b]) \notag\\
&\quad{} - \dfrac{1}{2}\sum_a^N\sum_b^N \qty([\delta\chi_a\chi_b | \chi_b\chi_a] + [\chi_a\chi_b | \delta\chi_b\chi_a]) + \text{complex conjugates}
\end{align}
while
\begin{align}
\sum_a^N\sum_b^N [\chi_a\chi_a | \delta\chi_b\chi_b] &= \sum_b^N\sum_a^N [\chi_b\chi_b | \delta\chi_a\chi_a] = \sum_a^N\sum_b^N [\delta\chi_a\chi_a | \chi_b\chi_b]\\
\sum_a^N\sum_b^N [\chi_a\chi_b | \delta\chi_b\chi_a] &= \sum_b^N\sum_a^N [\chi_b\chi_a | \delta\chi_a\chi_b] = \sum_a^N\sum_b^N [\delta\chi_a\chi_b | \chi_b\chi_a]
\end{align}
thus
\begin{equation}\label{key}
\delta E_0 = \sum_a^N[\delta\chi_a | h | \chi_a] 
+ \sum_a^N\sum_b^N \qty([\delta\chi_a\chi_a | \chi_b\chi_b] - [\delta\chi_a\chi_b | \chi_b\chi_a]) + \text{complex conjugates}
\end{equation}

\subsubsection{The Canonical HF Equations}

\subsection{Interpretation of Solutions to the HF Equations}
\subsubsection{Orbital Energies and Koopmans' Theorem}
\ex{3.4}
\begin{align}
f_{ij} = \Braket{\chi_i | f | \chi_j} = \Braket{i|h|j} + \sum_b\Braket{ib||jb}
\end{align}
\begin{align}
f_{ji}^* &= \Braket{\chi_j | f | \chi_i}^* = \Braket{j|h|i}^* + \sum_b\Braket{jb||ib}^*\notag\\
&= \Braket{i|h|j} + \sum_b\Braket{ib||jb}\notag\\
&= f_{ij}
\end{align}
thus the Fock operator is Hermitian.

\ex{3.5}
\begin{align}
\text{IP} &= ^{N-2}E - E_0 \notag\\
&= \sum_{a\neq c,d}\Braket{a|h|a} + \dfrac{1}{2}\sum_{a\neq c,d}\sum_{b\neq c,d} \Braket{ab||ab}  - \qty[\sum_a\Braket{a|h|a} + \dfrac{1}{2}\sum_a\sum_b \Braket{ab||ab} ] \notag\\
&= -\Braket{c|h|c} - \Braket{d|h|d} -  \dfrac{1}{2}\sum_{a\neq c,d}\Braket{ac||ac}  -  \dfrac{1}{2}\sum_{a\neq c,d}\Braket{ad||ad} - \dfrac{1}{2}\sum_{b\neq c,d} \Braket{cb||cb} - \dfrac{1}{2}\sum_{b\neq c,d} \Braket{db||db}  - \Braket{cd||cd} \notag\\
&= -\Braket{c|h|c} - \Braket{d|h|d} -  \sum_{a\neq c,d}\Braket{ac||ac}  -  \sum_{a\neq c,d}\Braket{ad||ad}   - \Braket{cd||cd} \notag\\
&= -\Braket{c|h|c} - \Braket{d|h|d} - \qty( \sum_{a\neq c}\Braket{ac||ac} - \Braket{dc||dc}) -  \qty(\sum_{a\neq d}\Braket{ad||ad} - \Braket{cd||cd})  - \Braket{cd||cd} \notag\\
&= -\varepsilon_c - \varepsilon_d + \Braket{cd|cd} - \Braket{cd|dc}
\end{align}

\ex{3.6}
\begin{align}
^N E_0 - ^{N+1}E^r &= \sum_a\Braket{a|h|a} + \dfrac{1}{2}\sum_a\sum_b \Braket{ab||ab} \notag\\
&{}\quad - \qty[\sum_a\Braket{a|h|a} + \Braket{r|h|r} + \dfrac{1}{2}\sum_a\sum_b \Braket{ab||ab} + \dfrac{1}{2}\sum_b \Braket{rb||rb} + \dfrac{1}{2}\sum_a \Braket{ar||ar}] \notag\\
&= - \Braket{r|h|r} - \dfrac{1}{2}\sum_b \Braket{rb||rb} - \dfrac{1}{2}\sum_b \Braket{br||br} \notag\\
&= - \Braket{r|h|r} - \sum_b \Braket{rb||rb}
\end{align}

\subsubsection{Brillouin's Theorem}
\subsubsection{The HF Hamiltonian}
\ex{3.7}
Suppose $ \mathscr{H}_0 $ commutes with $ \mathscr{P}_n $,
\begin{align}\label{key}
\mathscr{H}_0\ket{\Psi_0} &= \mathscr{H}_0 \dfrac{1}{\sqrt{N!}} \sum_n^{N!}(-1)^{p_n} \mathscr{P}_n \qty{\sum_i^N  f(i)\chi_j(1)\cdots\chi_k(N)} \notag\\
&= \dfrac{1}{\sqrt{N!}} \sum_n^{N!}(-1)^{p_n} \mathscr{P}_n \qty{ (\varepsilon_j + \cdots + \varepsilon_k) \chi_j(1)\cdots\chi_k(N)} \notag\\
&= \sum_a \varepsilon_a
\end{align}
Now we show $ \mathscr{H}_0 $ commutes with $ \mathscr{P}_n $, for example, $ \mathscr{P}_{ab} $
\begin{equation}\label{key}
\mathscr{P}_{ab}\mathscr{H}_0 = \mathscr{P}_{ab}(\cdots + f(a) + \cdots + f(b) + \cdots) = (\cdots + f(b) + \cdots + f(a) + \cdots)\mathscr{P}_{ab} = \mathscr{H}_0 \mathscr{P}_{ab}
\end{equation}

\ex{3.8}
\begin{equation}\label{key}
\mathscr{V} = \sum_i^N \sum_{j>i}^N \mathscr{O}_2 - \sum_i^N\sum_b^N [\mathscr{G}_b(i) - \mathscr{K}_b(i)]
\end{equation}
thus
\begin{align}
\Braket{\Psi_0 | \mathscr{V} | \Psi_0} &= \sum_i^N \sum_{j>i}^N \Braket{\Psi_0 | \mathscr{O}_2 | \Psi_0} - \sum_i^N\sum_b^N [\Braket{\Psi_0 | \mathscr{G}_b(i) - \mathscr{K}_b(i)| \Psi_0}] \notag\\
&= \dfrac{1}{2}\sum_a^N \sum_b^N \Braket{ab||ab} - \sum_i^N\sum_b^N [\Braket{ib|ib} - \Braket{ib|bi}] \notag\\
&= -\dfrac{1}{2}\sum_a^N \sum_b^N \Braket{ab||ab}
\end{align}

\subsection{Restricted Closed-shell HF: The Roothaan Equations}
\subsubsection{Closed-shell HF: Restricted Spin Orbitals}
\ex{3.9}
\begin{align}
\varepsilon_i %&= \Braket{\chi_i | h | \chi_i} + \sum_b^N \Braket{\chi_i\chi_b || \chi_i\chi_b} \notag\\
&= (i|h|i) + \sum_b^N(\Braket{ib|ib} - \Braket{ib|bi}) \notag\\
&= (i|h|i) + \sum_c^{N/2}(\Braket{ic|ic} - \Braket{ic|ci}) + \sum_{\bar{c}}^{N/2}(\Braket{i\bar{c}|i\bar{c}} - \Braket{i\bar{c}|\bar{c}i}) 
\end{align}
Assume $ \chi_j $ has $ \alpha $ spin, since assuming $ \alpha $ or $ \beta $ is identical
\begin{align}
\varepsilon_i &= (i|h|i) + \sum_c^{N/2}\qty[ (ic|ic)\Braket{\alpha|\alpha}\Braket{\alpha|\alpha} 
- (ic|ci)\Braket{\alpha|\alpha}\Braket{\alpha|\alpha}]
+ \sum_c^{N/2}\qty[(ic|ic)\Braket{\alpha|\alpha}\Braket{\beta|\beta} 
- (ic|ci)\Braket{\alpha|\beta}\Braket{\beta|\alpha}] \notag\\
&= (i|h|i) + \sum_c^{N/2}\qty[ 2(ic|ic) - (ic|ci)] \notag\\
&= (i|h|i) + \sum_n^{N/2} (2J_{ib} - K_{ib})
\end{align}

\subsubsection{Introduction of a Basis: The Roothaan Equations}
\ex{3.10}
\begin{align}
(\vb{C}^\dagger \vb{S} \vb{C})_{\mu\nu} &= \sum_i\sum_j C^\dagger_{\mu i} S_{ij} C_{j\nu} \notag\\
&= \sum_i\sum_j C^\dagger_{\mu i} \Braket{\phi_i| \phi_j} C_{j\nu} \notag\\
&= \Braket{\phi_\mu | \phi_\nu} \notag\\
&= \delta_{\mu\nu}
\end{align}
thus
\begin{equation}\label{key}
\vb{C}^\dagger \vb{S} \vb{C} = \iden
\end{equation}

\subsubsection{The Charge Density}
\ex{3.11}




\end{document}
%\documentclass[UTF8]{ctexart} % use larger type; default would be 10pt
\documentclass[a4paper]{article}
\usepackage{xeCJK}
%\usepackage[utf8]{inputenc} % set input encoding (not needed with XeLaTeX)

%%% Examples of Article customizations
% These packages are optional, depending whether you want the features they provide.
% See the LaTeX Companion or other references for full information.

%%% PAGE DIMENSIONS
\usepackage{geometry} % to change the page dimensions
\geometry{a4paper} % or letterpaper (US) or a5paper or....
\geometry{margin=1in} % for example, change the margins to 2 inches all round
% \geometry{landscape} % set up the page for landscape
%   read geometry.pdf for detailed page layout information

\usepackage{graphicx} % support the \includegraphics command and options

% \usepackage[parfill]{parskip} % Activate to begin paragraphs with an empty line rather than an indent

%%% PACKAGES
\usepackage{booktabs} % for much better looking tables
\usepackage{array} % for better arrays (eg matrices) in maths
\usepackage{paralist} % very flexible & customisable lists (eg. enumerate/itemize, etc.)
\usepackage{verbatim} % adds environment for commenting out blocks of text & for better verbatim
\usepackage{subfig} % make it possible to include more than one captioned figure/table in a single float
% These packages are all incorporated in the memoir class to one degree or another...

%%% HEADERS & FOOTERS
\usepackage{fancyhdr} % This should be set AFTER setting up the page geometry
\pagestyle{fancy} % options: empty , plain , fancy
\renewcommand{\headrulewidth}{0pt} % customise the layout...
\lhead{}\chead{}\rhead{}
\lfoot{}\cfoot{\thepage}\rfoot{}

%%% SECTION TITLE APPEARANCE
\usepackage{sectsty}
%\allsectionsfont{\sffamily\mdseries\upshape} % (See the fntguide.pdf for font help)
% (This matches ConTeXt defaults)

%%% ToC (table of contents) APPEARANCE
\usepackage[nottoc,notlof,notlot]{tocbibind} % Put the bibliography in the ToC
\usepackage[titles,subfigure]{tocloft} % Alter the style of the Table of Contents
%\renewcommand{\cftsecfont}{\rmfamily\mdseries\upshape}
%\renewcommand{\cftsecpagefont}{\rmfamily\mdseries\upshape} % No bold!

%%% END Article customizations

%%% The "real" document content comes below...

\setlength{\parindent}{0pt}
\usepackage{physics}
\usepackage{amsmath}
%\usepackage{symbols}
\usepackage{AMSFonts}
\usepackage{bm}
%\usepackage{eucal}
\usepackage{mathrsfs}
\usepackage{amssymb}
\usepackage{float}
\usepackage{multicol}
\usepackage{abstract}
\usepackage{empheq}
\usepackage{extarrows}
\usepackage{textcomp}
\usepackage{fontspec}

\setmainfont{CMU Serif}
\setsansfont{CMU Sans Serif}
\setmonofont{CMU Typewriter Text}

\usepackage{braket}
\usepackage{siunitx}
\sisetup{
	separate-uncertainty = true,
	inter-unit-product = \ensuremath{{}\cdot{}}
}
\usepackage{mhchem}
\usepackage{multirow}

\DeclareMathOperator{\p}{\prime}
\DeclareMathOperator{\ti}{\times}
\DeclareMathOperator{\intinf}{\int_0^\infty}
\DeclareMathOperator{\intdinf}{\int_{-\infty}^\infty}
\DeclareMathOperator{\intzpi}{\int_0^\pi}
\DeclareMathOperator{\intztpi}{\int_0^{2\pi}}
\DeclareMathOperator{\sumninf}{\sum_{n=1}^{\infty}}
\DeclareMathOperator{\sumninfz}{\sum_{n=0}^\infty}
\DeclareMathOperator{\sumiinf}{\sum_{i=1}^{\infty}}
\DeclareMathOperator{\sumiinfz}{\sum_{i=0}^\infty}
\DeclareMathOperator{\sumkinf}{\sum_{k=1}^{\infty}}
\DeclareMathOperator{\sumkinfz}{\sum_{k=0}^\infty}
\DeclareMathOperator{\e}{\mathrm{e}}
\DeclareMathOperator{\I}{\mathrm{i}}
\DeclareMathOperator{\Arg}{\mathrm{Arg}}
\DeclareMathOperator{\ra}{\rightarrow}
\DeclareMathOperator{\llra}{\longleftrightarrow}
\DeclareMathOperator{\lra}{\longrightarrow}
\DeclareMathOperator{\dlra}{\Leftrightarrow}
\DeclareMathOperator{\dra}{\Rightarrow}
\newcommand{\bkk}[1]{\Braket{#1|#1}}
\newcommand{\bk}[2]{\Braket{#1|#2}}
\newcommand{\bkev}[2]{\Braket{#2|#1|#2}}



\DeclareMathOperator{\hV}{\hat{\vb{V}}}

\DeclareMathOperator{\hx}{\hat{\vb{x}}}
\DeclareMathOperator{\hy}{\hat{\vb{y}}}
\DeclareMathOperator{\hz}{\hat{\vb{z}}}

\DeclareMathOperator{\hA}{\hat{\vb{A}}}

\DeclareMathOperator{\hQ}{\hat{\vb{Q}}}
\DeclareMathOperator{\hI}{\hat{\vb{I}}}
\DeclareMathOperator{\psis}{\psi^\ast}
\DeclareMathOperator{\Psis}{\Psi^\ast}
\DeclareMathOperator{\hi}{\hat{\vb{i}}}
\DeclareMathOperator{\hj}{\hat{\vb{j}}}
\DeclareMathOperator{\hk}{\hat{\vb{k}}}
\DeclareMathOperator{\hr}{\hat{\vb{r}}}
\DeclareMathOperator{\hT}{\hat{\vb{T}}}
\DeclareMathOperator{\hH}{\hat{H}}
\DeclareMathOperator{\hh}{\hat{h}}               % helicity
\DeclareMathOperator{\hL}{\hat{\vb{L}}}
\DeclareMathOperator{\hp}{\hat{\vb{p}}}

\DeclareMathOperator{\ha}{\hat{\vb{a}}}
\DeclareMathOperator{\hs}{\hat{\vb{s}}}
\DeclareMathOperator{\hS}{\hat{\vb{S}}}
\DeclareMathOperator{\hSigma}{\hat{\bm\Sigma}}
\DeclareMathOperator{\hJ}{\hat{\vb{J}}}
\DeclareMathOperator{\hP}{\hat{\vb{P}}}          % Parity
\DeclareMathOperator{\hC}{\hat{\vb{C}}} 
\DeclareMathOperator{\Tdv}{-\dfrac{\hbar^2}{2m}\dv[2]{x}}
\DeclareMathOperator{\Tna}{-\dfrac{\hbar^2}{2m}\nabla^2}
\DeclareMathOperator{\vna}{\vnabla}
\DeclareMathOperator{\nna}{\nabla^2}
\newcommand{\naCarExpd}[1]{\pdv[2]{#1}{x} + \pdv[2]{#1}{y} + \pdv[2]{#1}{z}}
\newcommand{\naCyl}{\qty[\dfrac{1}{\rho}\pdv{\rho}\qty(\rho\pdv{\rho}) + \dfrac{1}{\rho^2}\pdv[2]{\phi} + \pdv[2]{z}]}

%\DeclareMathOperator{\g#0}{\gamma^0}
%\DeclareMathOperator{\g1}{\gamma^1}
%\DeclareMathOperator{\g2}{\gamma^2}
%\DeclareMathOperator{\g3}{\gamma^3}
%\DeclareMathOperator{\g5}{\gamma^5}
\newcommand{\g}[1]{\gamma^{#1}}
\DeclareMathOperator{\gmuu}{\gamma^\mu}
\DeclareMathOperator{\gmud}{\gamma_\mu}
\newcommand{\G}[2]{g^{#1#2}}


%% MQC
\DeclareMathOperator{\sH}{\mathscr{H}}
\DeclareMathOperator{\sA}{\mathscr{A}}
\newcommand{\iden}{{\large \bm{1}}}
\newcommand{\qed}{$ \Square $}
\newcommand{\tPhi}{\tilde{\Phi} }
\newcommand{\hsP}{\hat{\mathscr{P}}}
\newcommand{\hsS}{\hat{\mathscr{S}}}
\DeclareMathOperator{\core}{\mathrm{core}}
\DeclareMathOperator{\GF}{\mathrm{GF}}
\DeclareMathOperator{\SF}{\mathrm{SF}}
\DeclareMathOperator{\corr}{\mathrm{corr}}


\newcommand{\subsbul}{\subsection*{$ \bullet $}}
\newcommand{\ex}[1]{\paragraph{Ex #1}}
\newcommand{\subex}[1]{\subparagraph{#1}}
\newcommand{\dis}{\displaystyle}


\numberwithin{equation}{subsection}
%\setcounter{secnumdepth}{4}
\setcounter{tocdepth}{4}
\allowdisplaybreaks[1]

\usepackage{xcolor}
\definecolor{codegray}{gray}{0.9}
\newfontfamily\Consolas{Consolas}
\newcommand{\code}[1]{\colorbox{codegray}{{\Consolas#1}}}

\title{\textbf{Modern Quantum Chemistry, Szabo \& Ostlund}\\HW}
\author{王石嵘
\vspace{5pt}\\
%161240065\\
%Email: shirong\_wang@berkeley.edu
}
\date{\today} % Activate to display a given date or no date (if empty),
         % otherwise the current date is printed 

\begin{document}
% \boldmath

\maketitle

\tableofcontents

\newpage

\setcounter{section}{3}
\section{Configuration Interaction}
\subsection{Multiconfigurational Wave Functions and the Structure of Full CI Matrix}

\subsubsection{Intermediate Normalization and an Expression for the Correlation Energy}
\ex{4.1}
If $ a\notin \{c,d,e\} $ and $ r\notin\{t,u,v\} $,
\begin{equation}\label{key}
\Braket{\Psi_a^r | \mathscr{H} | \Psi_{cde}^{tuv}} = 0
\end{equation}
Let's suppose $ a = e $, thus
\begin{equation}\label{key}
\Braket{\Psi_a^r | \mathscr{H} | \Psi_{cde}^{tuv}} = \Braket{\Psi_a^r | \mathscr{H} | \Psi_{acd}^{vtu}} 
\end{equation}
if $ r\neq v $, this term will still be zero, thus
\begin{equation}\label{key}
\sum_{c<d<e,t<u<v} c_{cde}^{tuv} \Braket{\Psi_a^r | \mathscr{H} | \Psi_{cde}^{tuv}} = \sum_{c<d,t<u} c_{acd}^{rtu} \Braket{\Psi_a^r | \mathscr{H} | \Psi_{acd}^{rtu}} 
\end{equation}

\ex{4.2}
\begin{equation}\label{key}
\mqty|-E_{\corr} & K_{12}\\ K_{12} & 2\Delta-E_{\corr}| = 0
\end{equation}
\begin{equation}\label{key}
-E_{\corr}(2\Delta-E_{\corr}) - K_{12}^2 = 0
\end{equation}
\begin{equation}\label{key}
E_{\corr} = \dfrac{2\Delta \pm \sqrt{4\Delta^2 + 4K_{12}^2}}{2} = \Delta \pm \sqrt{\Delta^2 + K_{12}^2}
\end{equation}
choosing the lowest eigenvalue,
\begin{equation}\label{key}
E_{\corr} = \Delta - \sqrt{\Delta^2 + K_{12}^2}
\end{equation}

\ex{4.3}
At $ R = 1.4 $,
\begin{align}
\Delta &= \varepsilon_2 - \varepsilon_1 + \dfrac{1}{2}(J_{11} + J_{22}) - 2J_{12} + K_{12} \notag\\
&= 0.6703 + 0.5782 + \dfrac{1}{2}(0.6746 + 0.6975) - 2\times 0.6636 + 0.1813 \notag\\
&= 0.78865
\end{align}
\begin{align}
E_{\corr} = \Delta - \sqrt{\Delta^2 + K_{12}^2} = 0.78865 - \sqrt{0.78865^2 + 0.1813^2} = -0.020571
\end{align}
\begin{equation}\label{key}
c = \dfrac{E_{\corr}}{K_{12}} = \dfrac{-0.020571}{0.1813} = -0.1135
\end{equation}
As $ R\ra\infty $, $ \varepsilon_2 - \varepsilon_1 \ra 0 $, all 2e integrals $ \ra \dfrac{1}{2}(\phi_1\phi_1|\phi_1\phi_1) $, thus
\begin{align}
\lim\limits_{R\ra\infty}\Delta = 0 + \lim\lim\limits_{R\ra\infty} \qty[\dfrac{1}{2}(J_{11} + J_{22}) - 2J_{12} + K_{12}] = 0
\end{align}
\begin{equation}\label{key}
\lim\limits_{R\ra\infty} E_{\corr} =  -\lim\limits_{R\ra\infty} K_{12}
\end{equation}
\begin{equation}\label{key}
\lim\limits_{R\ra\infty} c = \lim\limits_{R\ra\infty} \dfrac{E_{\corr}}{K_{12}} = -1
\end{equation}
As $ R\ra\infty $, the full CI wave function will be
\begin{equation}\label{key}
\ket{\Phi_0} = \ket{\Psi_0} - \ket{\Psi_{1\bar{1}}^{2\bar{2}}} = \ket{\psi_1\bar\psi_1} - \ket{\psi_2\bar\psi_2}
\end{equation}
Since
\begin{align}
\psi_1 &= \dfrac{1}{\sqrt{2(1 + S_{12})}}(\phi_1 + \phi_2) \\
\psi_2 &= \dfrac{1}{\sqrt{2(1 - S_{12})}}(\phi_1 - \phi_2)
\end{align}
we get
\begin{align}
\ket{\psi_1\bar\psi_1} &= \dfrac{1}{2(1 + S_{12})} \qty(\ket{\phi_1\bar\phi_1} + \ket{\phi_1\bar\phi_2} + \ket{\phi_2\bar\phi_1} + \ket{\phi_2\bar\phi_2}) \\
\ket{\psi_2\bar\psi_2} &= \dfrac{1}{2(1 - S_{12})} \qty(\ket{\phi_1\bar\phi_1} - \ket{\phi_1\bar\phi_2} - \ket{\phi_2\bar\phi_1} + \ket{\phi_2\bar\phi_2})
\end{align}
As $ R\ra\infty $, $ S_{12}\ra 0 $, thus
\begin{align}
\ket{\Phi_0} = \ket{\psi_1\bar\psi_1} - \ket{\psi_2\bar\psi_2} = \dfrac{1}{2}\qty(\ket{\phi_1\bar\phi_2} + \ket{\phi_2\bar\phi_1})
\end{align}
Renormalize it, we get
\begin{equation}\label{key}
\ket{\Phi_0} = \dfrac{1}{\sqrt{2}}\qty(\ket{\phi_1\bar\phi_2} + \ket{\phi_2\bar\phi_1})
\end{equation}

\subsection{Doubly Exited CI}

\subsection{Some Illustrative Calculations}

\subsection{Natural Orbitals and the 1-Particle Reduced DM}
\ex{4.4}
\begin{equation}\label{key}
\gamma_{ij} = \int\dd\vb{x}_1\dd\vb{x}'_1 \chi_i^*(\vb{x}_1) \gamma(\vb{x}_1,\vb{x}'_1) \chi_j(\vb{x}'_1)
\end{equation}
\begin{align}
\gamma_{ji}^* &= \int\dd\vb{x}_1\dd\vb{x}'_1 \chi_j(\vb{x}_1) \gamma^*(\vb{x}_1,\vb{x}'_1) \chi_i^*(\vb{x}'_1) \notag\\
&= \int\dd\vb{x}'_1\dd\vb{x}_1 \chi_j(\vb{x}'_1) \gamma^*(\vb{x}'_1,\vb{x}_1) \chi_i^*(\vb{x}_1) \notag\\
&= \int\dd\vb{x}'_1\dd\vb{x}_1 \chi_j(\vb{x}'_1) \gamma(\vb{x}'_1,\vb{x}_1) \chi_i^*(\vb{x}_1) \notag\\
&= \gamma_{ij}
\end{align}
$ \therefore \bm\gamma$ is Hermitian.

\ex{4.5}
\begin{align}
\Braket{\Phi | \Phi} 
&= \dfrac{1}{N} \int\dd\vb{x}_1  \gamma(\vb{x}_1,\vb{x}_1) \notag\\
&= \int\dd\vb{x}_1  \sum_{ij} \chi_i(\vb{x}_1) \gamma_{ij} \chi_j^*(\vb{x}_1) \notag\\
&= \dfrac{1}{N} \sum_{ij} \qty[\int\dd\vb{x}_1 \chi_j^*(\vb{x}_1) \chi_i(\vb{x}_1)] \gamma_{ij}  \notag\\
&= \dfrac{1}{N} \sum_{ij} \delta_{ji} \gamma_{ij} \notag\\
&= \dfrac{1}{N} \tr \bm\gamma
\end{align}
thus
\begin{equation}\label{key}
\tr \bm\gamma = N
\end{equation}

\ex{4.6}
\subex{a.}
\begin{align}
\Braket{\Phi | \mathscr{O}_1 | \Phi} &= \sum_i \Braket{\Phi | h(\vb{x}_1) | \Phi} \notag\\
&= N \int\dd\vb{x}_1 \int\dd\vb{x}_2\cdots\dd\vb{x}_N \Phi^*(\vb{x}_1,\cdots,\vb{x}_N) h(\vb{x}_1) \Phi(\vb{x}_1,\cdots,\vb{x}_N) \notag\\
&= N \dfrac{1}{N}\int\dd\vb{x}_1 [h(\vb{x}_1) \gamma(\vb{x}_1,\vb{x}'_1)]_{\vb{x}'_1=\vb{x}_1} \notag\\
&= \int\dd\vb{x}_1 [h(\vb{x}_1) \gamma(\vb{x}_1,\vb{x}'_1)]_{\vb{x}'_1=\vb{x}_1}
\end{align}
\subex{b.}
\begin{align}
\Braket{\Phi | \mathscr{O}_1 | \Phi} 
&= \int\dd\vb{x}_1 [h(\vb{x}_1) \gamma(\vb{x}_1,\vb{x}'_1)]_{\vb{x}'_1=\vb{x}_1} \notag\\
&= \int\dd\vb{x}_1 [h(\vb{x}_1) \sum_{ij} \chi_i(\vb{x}_1) \gamma_{ij} \chi_j^*(\vb{x}'_1)]_{\vb{x}'_1=\vb{x}_1} \notag\\
&= \sum_{ij} \qty[\int\dd\vb{x}_1 \chi_j^*(\vb{x}_1) h(\vb{x}_1)\chi_i(\vb{x}_1)] \gamma_{ij}  \notag\\
&= \sum_{ij} h_{ji} \gamma_{ij} \notag\\
&= \sum_j (\vb{h}\bm\gamma)_{jj} \notag\\
&= \tr (\vb{h}\bm\gamma)
\end{align}

\ex{4.7}
\subex{a.}
\begin{align}
\Braket{\Phi | \mathscr{O}_1 | \Phi} 
&= \sum_{ij} \Braket{i | h | j} \Braket{\Phi | a_i^+ a_j | \Phi}
\end{align}
while
\begin{align}
\Braket{\Phi | \mathscr{O}_1 | \Phi} 
&= \sum_{ij} h_{ij} \gamma_{ji} 
\end{align}
$ \therefore $
\begin{equation}\label{key}
\gamma_{ji} = \Braket{\Phi | a_i^+ a_j | \Phi}
\end{equation}
i.e.
\begin{equation}\label{key}
\gamma_{ij} = \Braket{\Phi | a_j^+ a_i | \Phi}
\end{equation}
\subex{b.}
\begin{equation}\label{key}
\gamma_{ij}^{\text{HF}} = \Braket{\Psi_0 | a_j^+ a_i | \Psi_0} %= \delta_{ij} - \Braket{\Psi_0 | a_i a_j^+ | \Psi_0}
\end{equation}
If $ i $ is unoccupied, thus $ \gamma_{ij}^{\text{HF}} = 0 $ as we cannot annihilate electrons from it. If $ j $ is unoccupied, \\
$ \gamma_{ij}^{\text{HF}} = \delta_{ij} - \Braket{\Psi_0 | a_i a_j^+ | \Psi_0} = \delta_{ij} - \delta_{ij} = 0 $.\\
Otherwise, when $ i,j $ are occupied, it's clear that $ \gamma_{ij}^{\text{HF}} = \delta_{ij} $.\\
Thus,
\begin{equation}\label{key}
\gamma_{ij}^{\text{HF}} = \left\{
\mqty{\delta_{ij} & i,j\text{ are occupied} \vspace{5pt}\\
	  0 & \text{otherwise} }\right.
\end{equation}

\ex{4.8}
\subex{a.}
Since 
%$ \ket{^1\Psi_1^r}, \ket{^1\Psi_{11}^{rs}} $ can be expressed as linear combinations of $ \ket{\psi_i\bar\psi_j} $
\begin{equation}\label{key}
\ket{^1\Phi_0} = c_0\ket{\psi_1\bar\psi_1} + \sum_{r=2}^K c_1^r \dfrac{1}{\sqrt{2}}\qty(\ket{\psi_1\bar\psi_r} + \ket{\psi_r\bar\psi_1}) + \dfrac{1}{2}\sum_{r=2}^K\sum_{s=2}^K c_{11}^{rs}\dfrac{1}{\sqrt{2}} \qty(\ket{\psi_r\bar\psi_s} + \ket{\psi_s\bar\psi_r})
\end{equation}
we can write
\begin{equation}\label{key}
\ket{^1\Phi_0} = \sum_i^K \sum_j^K C_{ij}\ket{\psi_i\bar\psi_j}
\end{equation}
When one or two of $ i,j $ equals $ 1 $, it is clear that $ C_{ij} = C_{ji} $. Otherwise, $ c_{11}^{rs} = c_{11}^{sr} $. \\
Thus, $ \vb{C} $ is symmetric.
\subex{b.}
\begin{align}
\gamma(\vb{x}_1,\vb{x}'_1) &= 2\int\dd\vb{x}_2 \sum_{ij} C_{ij} \dfrac{1}{\sqrt{2}} \qty(\psi_i(\vb{x}_1)\bar\psi_j(\vb{x}_2) - \psi_i(\vb{x}_2)\bar\psi_j(\vb{x}_1)) 
\sum_{kl} C^*_{kl} \dfrac{1}{\sqrt{2}} \qty(\psi_k^*(\vb{x}'_1)\bar\psi_l^*(\vb{x}_2) - \psi_k^*(\vb{x}_2)\bar\psi_l^*(\vb{x}'_1)) \notag\\
&= \sum_{ij}\sum_{kl} C_{ij} C^*_{kl} \int\dd\vb{x}_2 \qty(\psi_i(\vb{x}_1)\bar\psi_j(\vb{x}_2) - \psi_i(\vb{x}_2)\bar\psi_j(\vb{x}_1)) 
\qty(\psi_k^*(\vb{x}'_1)\bar\psi_l^*(\vb{x}_2) - \psi_k^*(\vb{x}_2)\bar\psi_l^*(\vb{x}'_1)) \notag\\
&= \sum_{ij}\sum_{kl} C_{ij} C^*_{kl} 
\qty[\psi_i(\vb{x}_1)\psi_k^*(\vb{x}'_1)\delta_{jl}
 + \bar\psi_j(\vb{x}_1)\bar\psi_l^*(\vb{x}'_1)\delta_{ik}
] \notag\\
&= \sum_{ij}\sum_k C_{ij} C^*_{kj} \psi_i(\vb{x}_1)\psi_k^*(\vb{x}'_1)
+ \sum_{ij}\sum_l C_{ij} C^*_{il} \bar\psi_j(\vb{x}_1)\bar\psi_l^*(\vb{x}'_1)
 \notag\\
&= \sum_{ik} (\vb{C}\vb{C}^\dagger)_{ik} \psi_i(\vb{x}_1)\psi_k^*(\vb{x}'_1)
+ \sum_{jl} (\vb{C}^\dagger\vb{C})_{lj} \bar\psi_j(\vb{x}_1)\bar\psi_l^*(\vb{x}'_1)
\notag\\
&= \sum_{ij} (\vb{C}\vb{C}^\dagger)_{ij} \psi_i(\vb{x}_1)\psi_j^*(\vb{x}'_1)
+ \sum_{ij} (\vb{C}\vb{C}^\dagger)_{ji} \bar\psi_i(\vb{x}_1)\bar\psi_j^*(\vb{x}'_1)
\notag\\
&= \sum_{ij} (\vb{C}\vb{C}^\dagger)_{ij} \qty[\psi_i(1)\psi_j^*(1')
+ \bar\psi_i(1)\bar\psi_j^*(1')]
\end{align}
\subex{c.}
%Since $ \vb{U} $ is unitary and $ \vb{C} $ is symmetric,
\begin{equation}\label{key}
\vb{d} = \vb{U}^\dagger\vb{C}\vb{U} 
\end{equation}
\begin{equation}\label{key}
\vb{d}^\dagger = (\vb{U}^\dagger\vb{C}\vb{U} )^\dagger = \vb{U}^\dagger\vb{C}^\dagger\vb{U} 
\end{equation}
Since $ \vb{U} $ is unitary
\begin{equation}\label{key}
\vb{d}^2 = \vb{d}\vb{d}^\dagger = \vb{U}^\dagger\vb{C}\vb{U} \vb{U}^\dagger\vb{C}^\dagger\vb{U} = \vb{U}^\dagger\vb{C}\vb{C}^\dagger\vb{U}
\end{equation}
\subex{d.}
Since
\begin{equation}\label{key}
\psi_k = \sum_i U^\dagger_{ik} \zeta_i
\end{equation}
\begin{align}
\gamma(\vb{x}_1,\vb{x}'_1) &= \sum_{ij} (\vb{C}\vb{C}^\dagger)_{ij} \qty[\psi_i(1)\psi_j^*(1') 
+ \bar\psi_i(1)\bar\psi_j^*(1')] \notag\\
&=  \sum_{ij} (\vb{C}\vb{C}^\dagger)_{ij} \qty[\sum_k U^\dagger_{ki}\zeta_k(1)\sum_l U^{\dagger*}_{lj}\zeta_l^*(1')
+ \sum_k U^\dagger_{ki}\bar\zeta_k(1)\sum_l U^{\dagger*}_{lj}\bar\zeta_l^*(1')] \notag\\
&=  \sum_k\sum_l \sum_{ij} U^\dagger_{ki} (\vb{C}\vb{C}^\dagger)_{ij} U_{jl} \qty[\zeta_k(1) \zeta_l^*(1')
+ \bar\zeta_k(1)\bar\zeta_l^*(1')] \notag\\
&=  \sum_k\sum_l (\vb{U}^\dagger \vb{C}\vb{C}^\dagger \vb{U})_{kl} \qty[\zeta_k(1) \zeta_l^*(1')
+ \bar\zeta_k(1)\bar\zeta_l^*(1')] \notag\\
&=  \sum_k\sum_l d^2_k \delta_{kl} \qty[\zeta_k(1) \zeta_l^*(1')
+ \bar\zeta_k(1)\bar\zeta_l^*(1')] \notag\\
&= \sum_k d^2_k \qty[\zeta_k(1) \zeta_k^*(1')
+ \bar\zeta_k(1) \bar\zeta_k^*(1')] 
\end{align}
\subex{e.}
\begin{align}
\ket{^1\Phi_0} &= \sum_i^K \sum_j^K C_{ij} \ket{\psi_i\bar\psi_j} \notag\\
&= \sum_i^K \sum_j^K C_{ij} \Ket{ \qty(\sum_k U^\dagger_{ki} \zeta_k)\qty(\sum_l U^\dagger_{lj} \bar\zeta_l)} \notag\\
&= \sum_i^K \sum_j^K \sum_k\sum_l U^\dagger_{ki} C_{ij} U_{jl} \ket{ \zeta_k  \bar\zeta_l} \notag\\
&= \sum_k\sum_l d_{k}\delta_{kl} \ket{ \zeta_k  \bar\zeta_l} \notag\\
&= \sum_k d_{k} \ket{ \zeta_k  \bar\zeta_k}
\end{align}

\subsection{The MCSCF and the GVB Methods}





\end{document}
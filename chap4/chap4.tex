%\documentclass[UTF8]{ctexart} % use larger type; default would be 10pt
\documentclass[a4paper]{article}
\usepackage{xeCJK}
%\usepackage[utf8]{inputenc} % set input encoding (not needed with XeLaTeX)

%%% Examples of Article customizations
% These packages are optional, depending whether you want the features they provide.
% See the LaTeX Companion or other references for full information.

%%% PAGE DIMENSIONS
\usepackage{geometry} % to change the page dimensions
\geometry{a4paper} % or letterpaper (US) or a5paper or....
\geometry{margin=1in} % for example, change the margins to 2 inches all round
% \geometry{landscape} % set up the page for landscape
%   read geometry.pdf for detailed page layout information

\usepackage{graphicx} % support the \includegraphics command and options

% \usepackage[parfill]{parskip} % Activate to begin paragraphs with an empty line rather than an indent

%%% PACKAGES
\usepackage{booktabs} % for much better looking tables
\usepackage{array} % for better arrays (eg matrices) in maths
\usepackage{paralist} % very flexible & customisable lists (eg. enumerate/itemize, etc.)
\usepackage{verbatim} % adds environment for commenting out blocks of text & for better verbatim
\usepackage{subfig} % make it possible to include more than one captioned figure/table in a single float
% These packages are all incorporated in the memoir class to one degree or another...

%%% HEADERS & FOOTERS
\usepackage{fancyhdr} % This should be set AFTER setting up the page geometry
\pagestyle{fancy} % options: empty , plain , fancy
\renewcommand{\headrulewidth}{0pt} % customise the layout...
\lhead{}\chead{}\rhead{}
\lfoot{}\cfoot{\thepage}\rfoot{}

%%% SECTION TITLE APPEARANCE
\usepackage{sectsty}
%\allsectionsfont{\sffamily\mdseries\upshape} % (See the fntguide.pdf for font help)
% (This matches ConTeXt defaults)

%%% ToC (table of contents) APPEARANCE
\usepackage[nottoc,notlof,notlot]{tocbibind} % Put the bibliography in the ToC
\usepackage[titles,subfigure]{tocloft} % Alter the style of the Table of Contents
%\renewcommand{\cftsecfont}{\rmfamily\mdseries\upshape}
%\renewcommand{\cftsecpagefont}{\rmfamily\mdseries\upshape} % No bold!

%%% END Article customizations

%%% The "real" document content comes below...

\setlength{\parindent}{0pt}
\usepackage{physics}
\usepackage{amsmath}
%\usepackage{symbols}
\usepackage{amsfonts}
\usepackage{bm}
%\usepackage{eucal}
\usepackage{mathrsfs}
\usepackage{amssymb}
\usepackage{float}
\usepackage{multicol}
\usepackage{abstract}
\usepackage{empheq}
\usepackage{extarrows}
\usepackage{textcomp}
\usepackage{fontspec}

%\setmainfont{CMU Serif}
%\setsansfont{CMU Sans Serif}
%\setmonofont{CMU Typewriter Text}

\usepackage{braket}
\usepackage{siunitx}
\sisetup{
	separate-uncertainty = true,
	inter-unit-product = \ensuremath{{}\cdot{}}
}
\usepackage{mhchem}
\usepackage{multirow}
\usepackage{booktabs}

\DeclareMathOperator{\p}{\prime}
\DeclareMathOperator{\ti}{\times}
\DeclareMathOperator{\intinf}{\int_0^\infty}
\DeclareMathOperator{\intdinf}{\int_{-\infty}^\infty}
\DeclareMathOperator{\intzpi}{\int_0^\pi}
\DeclareMathOperator{\intztpi}{\int_0^{2\pi}}
\DeclareMathOperator{\sumninf}{\sum_{n=1}^{\infty}}
\DeclareMathOperator{\sumninfz}{\sum_{n=0}^\infty}
\DeclareMathOperator{\sumiinf}{\sum_{i=1}^{\infty}}
\DeclareMathOperator{\sumiinfz}{\sum_{i=0}^\infty}
\DeclareMathOperator{\sumkinf}{\sum_{k=1}^{\infty}}
\DeclareMathOperator{\sumkinfz}{\sum_{k=0}^\infty}
\DeclareMathOperator{\e}{\mathrm{e}}
\DeclareMathOperator{\I}{\mathrm{i}}
\DeclareMathOperator{\Arg}{\mathrm{Arg}}
\DeclareMathOperator{\ra}{\rightarrow}
\DeclareMathOperator{\llra}{\longleftrightarrow}
\DeclareMathOperator{\lra}{\longrightarrow}
\DeclareMathOperator{\dlra}{\Leftrightarrow}
\DeclareMathOperator{\dra}{\Rightarrow}
\newcommand{\bkk}[1]{\Braket{#1|#1}}
\newcommand{\bk}[2]{\Braket{#1|#2}}
\newcommand{\bkev}[2]{\Braket{#2|#1|#2}}



\DeclareMathOperator{\hV}{\hat{\vb{V}}}

\DeclareMathOperator{\hx}{\hat{\vb{x}}}
\DeclareMathOperator{\hy}{\hat{\vb{y}}}
\DeclareMathOperator{\hz}{\hat{\vb{z}}}

\DeclareMathOperator{\hA}{\hat{\vb{A}}}

\DeclareMathOperator{\hQ}{\hat{\vb{Q}}}
\DeclareMathOperator{\hI}{\hat{\vb{I}}}
\DeclareMathOperator{\psis}{\psi^\ast}
\DeclareMathOperator{\Psis}{\Psi^\ast}
\DeclareMathOperator{\hi}{\hat{\vb{i}}}
\DeclareMathOperator{\hj}{\hat{\vb{j}}}
\DeclareMathOperator{\hk}{\hat{\vb{k}}}
\DeclareMathOperator{\hr}{\hat{\vb{r}}}
\DeclareMathOperator{\hT}{\hat{\vb{T}}}
\DeclareMathOperator{\hH}{\hat{H}}
\DeclareMathOperator{\hh}{\hat{h}}               % helicity
\DeclareMathOperator{\hL}{\hat{\vb{L}}}
\DeclareMathOperator{\hp}{\hat{\vb{p}}}

\DeclareMathOperator{\ha}{\hat{\vb{a}}}
\DeclareMathOperator{\hs}{\hat{\vb{s}}}
\DeclareMathOperator{\hS}{\hat{\vb{S}}}
\DeclareMathOperator{\hSigma}{\hat{\bm\Sigma}}
\DeclareMathOperator{\hJ}{\hat{\vb{J}}}
\DeclareMathOperator{\hP}{\hat{\vb{P}}}          % Parity
\DeclareMathOperator{\hC}{\hat{\vb{C}}} 
\DeclareMathOperator{\Tdv}{-\dfrac{\hbar^2}{2m}\dv[2]{x}}
\DeclareMathOperator{\Tna}{-\dfrac{\hbar^2}{2m}\nabla^2}
\DeclareMathOperator{\vna}{\vnabla}
\DeclareMathOperator{\nna}{\nabla^2}
\newcommand{\naCarExpd}[1]{\pdv[2]{#1}{x} + \pdv[2]{#1}{y} + \pdv[2]{#1}{z}}
\newcommand{\naCyl}{\qty[\dfrac{1}{\rho}\pdv{\rho}\qty(\rho\pdv{\rho}) + \dfrac{1}{\rho^2}\pdv[2]{\phi} + \pdv[2]{z}]}

%\DeclareMathOperator{\g#0}{\gamma^0}
%\DeclareMathOperator{\g1}{\gamma^1}
%\DeclareMathOperator{\g2}{\gamma^2}
%\DeclareMathOperator{\g3}{\gamma^3}
%\DeclareMathOperator{\g5}{\gamma^5}
\newcommand{\g}[1]{\gamma^{#1}}
\DeclareMathOperator{\gmuu}{\gamma^\mu}
\DeclareMathOperator{\gmud}{\gamma_\mu}
\newcommand{\G}[2]{g^{#1#2}}


%% MQC
\DeclareMathOperator{\sH}{\mathscr{H}}
\DeclareMathOperator{\sA}{\mathscr{A}}
\newcommand{\iden}{{\large \bm{1}}}
\newcommand{\qed}{$ \Square $}
\newcommand{\tPhi}{\tilde{\Phi} }
\newcommand{\hsP}{\hat{\mathscr{P}}}
\newcommand{\hsS}{\hat{\mathscr{S}}}
\DeclareMathOperator{\core}{\mathrm{core}}
\DeclareMathOperator{\GF}{\mathrm{GF}}
\DeclareMathOperator{\SF}{\mathrm{SF}}
\DeclareMathOperator{\corr}{\mathrm{corr}}
\DeclareMathOperator{\gvb}{\mathrm{GVB}}


\newcommand{\subsbul}{\subsection*{$ \bullet $}}
\newcommand{\ex}[1]{\paragraph{Ex #1}}
\newcommand{\subex}[1]{\subparagraph{#1}}
\newcommand{\dis}{\displaystyle}


\numberwithin{equation}{subsection}
%\setcounter{secnumdepth}{4}
\setcounter{tocdepth}{4}
\allowdisplaybreaks[1]

\usepackage{xcolor}
\definecolor{codegray}{gray}{0.9}
\newfontfamily\Consolas{Consolas}
\newcommand{\code}[1]{\colorbox{codegray}{{\Consolas#1}}}

\title{\textbf{Modern Quantum Chemistry, Szabo \& Ostlund}\\HW}
\author{wsr
\vspace{5pt}\\
}
\date{\today} % Activate to display a given date or no date (if empty),
         % otherwise the current date is printed 

\begin{document}
% \boldmath

\maketitle

\tableofcontents

\newpage

\setcounter{section}{3}
\section{Configuration Interaction}
\subsection{Multiconfigurational Wave Functions and the Structure of Full CI Matrix}

\subsubsection{Intermediate Normalization and an Expression for the Correlation Energy}
\ex{4.1}
If $ a\notin \{c,d,e\} $ and $ r\notin\{t,u,v\} $,
\begin{equation}\label{key}
\Braket{\Psi_a^r | \mathscr{H} | \Psi_{cde}^{tuv}} = 0
\end{equation}
Let's suppose $ a = e $, thus
\begin{equation}\label{key}
\Braket{\Psi_a^r | \mathscr{H} | \Psi_{cde}^{tuv}} = \Braket{\Psi_a^r | \mathscr{H} | \Psi_{acd}^{vtu}} 
\end{equation}
if $ r\neq v $, this term will still be zero, thus
\begin{equation}\label{key}
\sum_{c<d<e,t<u<v} c_{cde}^{tuv} \Braket{\Psi_a^r | \mathscr{H} | \Psi_{cde}^{tuv}} = \sum_{c<d,t<u} c_{acd}^{rtu} \Braket{\Psi_a^r | \mathscr{H} | \Psi_{acd}^{rtu}} 
\end{equation}

\ex{4.2}
\begin{equation}\label{key}
\mqty|-E_{\corr} & K_{12}\\ K_{12} & 2\Delta-E_{\corr}| = 0
\end{equation}
\begin{equation}\label{key}
-E_{\corr}(2\Delta-E_{\corr}) - K_{12}^2 = 0
\end{equation}
\begin{equation}\label{key}
E_{\corr} = \dfrac{2\Delta \pm \sqrt{4\Delta^2 + 4K_{12}^2}}{2} = \Delta \pm \sqrt{\Delta^2 + K_{12}^2}
\end{equation}
choosing the lowest eigenvalue,
\begin{equation}\label{key}
E_{\corr} = \Delta - \sqrt{\Delta^2 + K_{12}^2}
\end{equation}

\ex{4.3}
At $ R = 1.4 $,
\begin{align}
\Delta &= \varepsilon_2 - \varepsilon_1 + \dfrac{1}{2}(J_{11} + J_{22}) - 2J_{12} + K_{12} \notag\\
&= 0.6703 + 0.5782 + \dfrac{1}{2}(0.6746 + 0.6975) - 2\times 0.6636 + 0.1813 \notag\\
&= 0.78865
\end{align}
\begin{align}
E_{\corr} = \Delta - \sqrt{\Delta^2 + K_{12}^2} = 0.78865 - \sqrt{0.78865^2 + 0.1813^2} = -0.020571
\end{align}
\begin{equation}\label{key}
c = \dfrac{E_{\corr}}{K_{12}} = \dfrac{-0.020571}{0.1813} = -0.1135
\end{equation}
As $ R\ra\infty $, $ \varepsilon_2 - \varepsilon_1 \ra 0 $, all 2e integrals $ \ra \dfrac{1}{2}(\phi_1\phi_1|\phi_1\phi_1) $, thus
\begin{align}
\lim\limits_{R\ra\infty}\Delta = 0 + \lim\limits_{R\ra\infty} \qty[\dfrac{1}{2}(J_{11} + J_{22}) - 2J_{12} + K_{12}] = 0
\end{align}
\begin{equation}\label{key}
\lim\limits_{R\ra\infty} E_{\corr} =  -\lim\limits_{R\ra\infty} K_{12}
\end{equation}
\begin{equation}\label{key}
\lim\limits_{R\ra\infty} c = \lim\limits_{R\ra\infty} \dfrac{E_{\corr}}{K_{12}} = -1
\end{equation}
As $ R\ra\infty $, the full CI wave function will be
\begin{equation}\label{key}
\ket{\Phi_0} = \ket{\Psi_0} - \ket{\Psi_{1\bar{1}}^{2\bar{2}}} = \ket{\psi_1\bar\psi_1} - \ket{\psi_2\bar\psi_2}
\end{equation}
Since
\begin{align}
\psi_1 &= \dfrac{1}{\sqrt{2(1 + S_{12})}}(\phi_1 + \phi_2) \\
\psi_2 &= \dfrac{1}{\sqrt{2(1 - S_{12})}}(\phi_1 - \phi_2)
\end{align}
we get
\begin{align}
\ket{\psi_1\bar\psi_1} &= \dfrac{1}{2(1 + S_{12})} \qty(\ket{\phi_1\bar\phi_1} + \ket{\phi_1\bar\phi_2} + \ket{\phi_2\bar\phi_1} + \ket{\phi_2\bar\phi_2}) \\
\ket{\psi_2\bar\psi_2} &= \dfrac{1}{2(1 - S_{12})} \qty(\ket{\phi_1\bar\phi_1} - \ket{\phi_1\bar\phi_2} - \ket{\phi_2\bar\phi_1} + \ket{\phi_2\bar\phi_2})
\end{align}
As $ R\ra\infty $, $ S_{12}\ra 0 $, thus
\begin{align}
\ket{\Phi_0} = \ket{\psi_1\bar\psi_1} - \ket{\psi_2\bar\psi_2} = \ket{\phi_1\bar\phi_2} + \ket{\phi_2\bar\phi_1}
\end{align}
Renormalize it, we get
\begin{equation}\label{key}
\ket{\Phi_0} = \dfrac{1}{\sqrt{2}}\qty(\ket{\phi_1\bar\phi_2} + \ket{\phi_2\bar\phi_1})
\end{equation}

\subsection{Doubly Exited CI}

\subsection{Some Illustrative Calculations}

\subsection{Natural Orbitals and the 1-Particle Reduced DM}
\ex{4.4}
\begin{equation}\label{key}
\gamma_{ij} = \int\dd\vb{x}_1\dd\vb{x}'_1 \chi_i^*(\vb{x}_1) \gamma(\vb{x}_1,\vb{x}'_1) \chi_j(\vb{x}'_1)
\end{equation}
\begin{align}
\gamma_{ji}^* &= \int\dd\vb{x}_1\dd\vb{x}'_1 \chi_j(\vb{x}_1) \gamma^*(\vb{x}_1,\vb{x}'_1) \chi_i^*(\vb{x}'_1) \notag\\
&= \int\dd\vb{x}'_1\dd\vb{x}_1 \chi_j(\vb{x}'_1) \gamma^*(\vb{x}'_1,\vb{x}_1) \chi_i^*(\vb{x}_1) \notag\\
&= \int\dd\vb{x}'_1\dd\vb{x}_1 \chi_j(\vb{x}'_1) \gamma(\vb{x}_1,\vb{x}'_1) \chi_i^*(\vb{x}_1) \notag\\
&= \gamma_{ij}
\end{align}
$ \therefore \bm\gamma$ is Hermitian.

\ex{4.5}
\begin{align}
\Braket{\Phi | \Phi} 
&= \dfrac{1}{N} \int\dd\vb{x}_1  \gamma(\vb{x}_1,\vb{x}_1) \notag\\
&= \dfrac{1}{N} \int\dd\vb{x}_1  \sum_{ij} \chi_i(\vb{x}_1) \gamma_{ij} \chi_j^*(\vb{x}_1) \notag\\
&= \dfrac{1}{N} \sum_{ij} \qty[\int\dd\vb{x}_1 \chi_j^*(\vb{x}_1) \chi_i(\vb{x}_1)] \gamma_{ij}  \notag\\
&= \dfrac{1}{N} \sum_{ij} \delta_{ji} \gamma_{ij} \notag\\
&= \dfrac{1}{N} \tr \bm\gamma
\end{align}
thus
\begin{equation}\label{key}
\tr \bm\gamma = N
\end{equation}

\ex{4.6}
\subex{a.}
\begin{align}
\Braket{\Phi | \mathscr{O}_1 | \Phi} &= \sum_i \Braket{\Phi | h(\vb{x}_1) | \Phi} \notag\\
&= N \int\dd\vb{x}_1 \int\dd\vb{x}_2\cdots\dd\vb{x}_N \Phi^*(\vb{x}_1,\cdots,\vb{x}_N) h(\vb{x}_1) \Phi(\vb{x}_1,\cdots,\vb{x}_N) \notag\\
&= N \dfrac{1}{N}\int\dd\vb{x}_1 [h(\vb{x}_1) \gamma(\vb{x}_1,\vb{x}'_1)]_{\vb{x}'_1=\vb{x}_1} \notag\\
&= \int\dd\vb{x}_1 [h(\vb{x}_1) \gamma(\vb{x}_1,\vb{x}'_1)]_{\vb{x}'_1=\vb{x}_1}
\end{align}
\subex{b.}
\begin{align}
\Braket{\Phi | \mathscr{O}_1 | \Phi} 
&= \int\dd\vb{x}_1 [h(\vb{x}_1) \gamma(\vb{x}_1,\vb{x}'_1)]_{\vb{x}'_1=\vb{x}_1} \notag\\
&= \int\dd\vb{x}_1 [h(\vb{x}_1) \sum_{ij} \chi_i(\vb{x}_1) \gamma_{ij} \chi_j^*(\vb{x}'_1)]_{\vb{x}'_1=\vb{x}_1} \notag\\
&= \sum_{ij} \qty[\int\dd\vb{x}_1 \chi_j^*(\vb{x}_1) h(\vb{x}_1)\chi_i(\vb{x}_1)] \gamma_{ij}  \notag\\
&= \sum_{ij} h_{ji} \gamma_{ij} \notag\\
&= \sum_j (\vb{h}\bm\gamma)_{jj} \notag\\
&= \tr (\vb{h}\bm\gamma)
\end{align}

\ex{4.7}
\subex{a.}
\begin{align}
\Braket{\Phi | \mathscr{O}_1 | \Phi} 
&= \sum_{ij} \Braket{i | h | j} \Braket{\Phi | a_i^+ a_j | \Phi}
\end{align}
while
\begin{align}
\Braket{\Phi | \mathscr{O}_1 | \Phi} 
&= \sum_{ij} h_{ij} \gamma_{ji} 
\end{align}
$ \therefore $
\begin{equation}\label{key}
\gamma_{ji} = \Braket{\Phi | a_i^+ a_j | \Phi}
\end{equation}
i.e.
\begin{equation}\label{key}
\gamma_{ij} = \Braket{\Phi | a_j^+ a_i | \Phi}
\end{equation}
\subex{b.}
\begin{equation}\label{key}
\gamma_{ij}^{\text{HF}} = \Braket{\Psi_0 | a_j^+ a_i | \Psi_0} %= \delta_{ij} - \Braket{\Psi_0 | a_i a_j^+ | \Psi_0}
\end{equation}
If $ i $ is unoccupied, thus $ \gamma_{ij}^{\text{HF}} = 0 $ as we cannot annihilate electrons from it. If $ j $ is unoccupied, \\
$ \gamma_{ij}^{\text{HF}} = \delta_{ij} - \Braket{\Psi_0 | a_i a_j^+ | \Psi_0} = \delta_{ij} - \delta_{ij} = 0 $.\\
Otherwise, when $ i,j $ are occupied, it's clear that $ \gamma_{ij}^{\text{HF}} = \delta_{ij} $.\\
Thus,
\begin{equation}\label{key}
\gamma_{ij}^{\text{HF}} = \left\{
\mqty{\delta_{ij} & i,j\text{ are occupied} \vspace{5pt}\\
	  0 & \text{otherwise} }\right.
\end{equation}

\ex{4.8}
\subex{a.}
Since 
%$ \ket{^1\Psi_1^r}, \ket{^1\Psi_{11}^{rs}} $ can be expressed as linear combinations of $ \ket{\psi_i\bar\psi_j} $
\begin{equation}\label{key}
\ket{^1\Phi_0} = c_0\ket{\psi_1\bar\psi_1} + \sum_{r=2}^K c_1^r \dfrac{1}{\sqrt{2}}\qty(\ket{\psi_1\bar\psi_r} + \ket{\psi_r\bar\psi_1}) + \dfrac{1}{2}\sum_{r=2}^K\sum_{s=2}^K c_{11}^{rs}\dfrac{1}{\sqrt{2}} \qty(\ket{\psi_r\bar\psi_s} + \ket{\psi_s\bar\psi_r})
\end{equation}
we can write
\begin{equation}\label{key}
\ket{^1\Phi_0} = \sum_i^K \sum_j^K C_{ij}\ket{\psi_i\bar\psi_j}
\end{equation}
When one or two of $ i,j $ equals $ 1 $, it is clear that $ C_{ij} = C_{ji} $. Otherwise, $ c_{11}^{rs} = c_{11}^{sr} $. \\
Thus, $ \vb{C} $ is symmetric.
\subex{b.}
\begin{align}
\gamma(\vb{x}_1,\vb{x}'_1) &= 2\int\dd\vb{x}_2 \sum_{ij} C_{ij} \dfrac{1}{\sqrt{2}} \qty(\psi_i(\vb{x}_1)\bar\psi_j(\vb{x}_2) - \psi_i(\vb{x}_2)\bar\psi_j(\vb{x}_1)) 
\sum_{kl} C^*_{kl} \dfrac{1}{\sqrt{2}} \qty(\psi_k^*(\vb{x}'_1)\bar\psi_l^*(\vb{x}_2) - \psi_k^*(\vb{x}_2)\bar\psi_l^*(\vb{x}'_1)) \notag\\
&= \sum_{ij}\sum_{kl} C_{ij} C^*_{kl} \int\dd\vb{x}_2 \qty(\psi_i(\vb{x}_1)\bar\psi_j(\vb{x}_2) - \psi_i(\vb{x}_2)\bar\psi_j(\vb{x}_1)) 
\qty(\psi_k^*(\vb{x}'_1)\bar\psi_l^*(\vb{x}_2) - \psi_k^*(\vb{x}_2)\bar\psi_l^*(\vb{x}'_1)) \notag\\
&= \sum_{ij}\sum_{kl} C_{ij} C^*_{kl} 
\qty[\psi_i(\vb{x}_1)\psi_k^*(\vb{x}'_1)\delta_{jl}
 + \bar\psi_j(\vb{x}_1)\bar\psi_l^*(\vb{x}'_1)\delta_{ik}
] \notag\\
&= \sum_{ij}\sum_k C_{ij} C^*_{kj} \psi_i(\vb{x}_1)\psi_k^*(\vb{x}'_1)
+ \sum_{ij}\sum_l C_{ij} C^*_{il} \bar\psi_j(\vb{x}_1)\bar\psi_l^*(\vb{x}'_1)
 \notag\\
&= \sum_{ik} (\vb{C}\vb{C}^\dagger)_{ik} \psi_i(\vb{x}_1)\psi_k^*(\vb{x}'_1)
+ \sum_{jl} (\vb{C}^\dagger\vb{C})_{lj} \bar\psi_j(\vb{x}_1)\bar\psi_l^*(\vb{x}'_1)
\notag\\
&= \sum_{ij} (\vb{C}\vb{C}^\dagger)_{ij} \psi_i(\vb{x}_1)\psi_j^*(\vb{x}'_1)
+ \sum_{ij} (\vb{C}\vb{C}^\dagger)_{ji} \bar\psi_i(\vb{x}_1)\bar\psi_j^*(\vb{x}'_1)
\notag\\
&= \sum_{ij} (\vb{C}\vb{C}^\dagger)_{ij} \qty[\psi_i(1)\psi_j^*(1')
+ \bar\psi_i(1)\bar\psi_j^*(1')]
\end{align}
\subex{c.}
%Since $ \vb{U} $ is unitary and $ \vb{C} $ is symmetric,
\begin{equation}\label{key}
\vb{d} = \vb{U}^\dagger\vb{C}\vb{U} 
\end{equation}
\begin{equation}\label{key}
\vb{d}^\dagger = (\vb{U}^\dagger\vb{C}\vb{U} )^\dagger = \vb{U}^\dagger\vb{C}^\dagger\vb{U} 
\end{equation}
Since $ \vb{U} $ is unitary
\begin{equation}\label{key}
\vb{d}^2 = \vb{d}\vb{d}^\dagger = \vb{U}^\dagger\vb{C}\vb{U} \vb{U}^\dagger\vb{C}^\dagger\vb{U} = \vb{U}^\dagger\vb{C}\vb{C}^\dagger\vb{U}
\end{equation}
\subex{d.}
Since
\begin{equation}\label{key}
\psi_k = \sum_i U^\dagger_{ik} \zeta_i
\end{equation}
\begin{align}
\gamma(\vb{x}_1,\vb{x}'_1) &= \sum_{ij} (\vb{C}\vb{C}^\dagger)_{ij} \qty[\psi_i(1)\psi_j^*(1') 
+ \bar\psi_i(1)\bar\psi_j^*(1')] \notag\\
&=  \sum_{ij} (\vb{C}\vb{C}^\dagger)_{ij} \qty[\sum_k U^\dagger_{ki}\zeta_k(1)\sum_l U^{\dagger*}_{lj}\zeta_l^*(1')
+ \sum_k U^\dagger_{ki}\bar\zeta_k(1)\sum_l U^{\dagger*}_{lj}\bar\zeta_l^*(1')] \notag\\
&=  \sum_k\sum_l \sum_{ij} U^\dagger_{ki} (\vb{C}\vb{C}^\dagger)_{ij} U_{jl} \qty[\zeta_k(1) \zeta_l^*(1')
+ \bar\zeta_k(1)\bar\zeta_l^*(1')] \notag\\
&=  \sum_k\sum_l (\vb{U}^\dagger \vb{C}\vb{C}^\dagger \vb{U})_{kl} \qty[\zeta_k(1) \zeta_l^*(1')
+ \bar\zeta_k(1)\bar\zeta_l^*(1')] \notag\\
&=  \sum_k\sum_l d^2_k \delta_{kl} \qty[\zeta_k(1) \zeta_l^*(1')
+ \bar\zeta_k(1)\bar\zeta_l^*(1')] \notag\\
&= \sum_k d^2_k \qty[\zeta_k(1) \zeta_k^*(1')
+ \bar\zeta_k(1) \bar\zeta_k^*(1')] 
\end{align}
\subex{e.}
\begin{align}
\ket{^1\Phi_0} &= \sum_i^K \sum_j^K C_{ij} \ket{\psi_i\bar\psi_j} \notag\\
&= \sum_i^K \sum_j^K C_{ij} \Ket{ \qty(\sum_k U^\dagger_{ki} \zeta_k)\qty(\sum_l U^\dagger_{lj} \bar\zeta_l)} \notag\\
&= \sum_i^K \sum_j^K \sum_k\sum_l U^\dagger_{ki} C_{ij} U_{jl} \ket{ \zeta_k  \bar\zeta_l} \notag\\
&= \sum_k\sum_l d_{k}\delta_{kl} \ket{ \zeta_k  \bar\zeta_l} \notag\\
&= \sum_k d_{k} \ket{ \zeta_k  \bar\zeta_k}
\end{align}

\subsection{The MCSCF and the GVB Methods}
\ex{4.9}
\subex{a.}
\begin{align}
\Braket{u|u} &= \dfrac{1}{a^2 + b^2} \Braket{a\psi_A + b\psi_B | a\psi_A + b\psi_B} \notag\\
&= \dfrac{1}{a^2 + b^2} (a^2 + b^2) \notag\\
&= 1
\end{align}
\begin{align}
\Braket{v|v} &= \dfrac{1}{a^2 + b^2} \Braket{a\psi_A - b\psi_B | a\psi_A - b\psi_B} \notag\\
&= \dfrac{1}{a^2 + b^2} (a^2 + b^2) \notag\\
&= 1
\end{align}
\begin{align}
\Braket{u|v} &= \dfrac{1}{a^2 + b^2} \Braket{a\psi_A + b\psi_B | a\psi_A - b\psi_B} \notag\\
&= \dfrac{a^2 - b^2}{a^2 + b^2} 
\end{align}
\subex{b.}
\begin{align}
\ket{\Psi_{\gvb}} &= [2(1 + S^2)]^{-1/2} [u(1)v(2) + u(2)v(1)] 2^{-1/2} [\alpha(1)\beta(2) - \alpha(2)\beta(1)] \notag\\
&= \qty[2 + 2\qty(\dfrac{a^2-b^2}{a^2+b^2})^2]^{-1/2} (a^2 + b^2)^{-1} \notag\\
&{}\quad \cross [(a\psi_A(1) + b\psi_B(1))(a\psi_A(2) - b\psi_B(2)) + (a\psi_A(2) + b\psi_B(2))(a\psi_A(1) - b\psi_B(1))] \notag\\
&{}\quad \cross 2^{-1/2} [\alpha(1)\beta(2) - \alpha(2)\beta(1)] \notag\\
&= \qty[2(a^2 + b^2)^2 + 2\qty(a^2-b^2)^2]^{-1/2} [2a^2\psi_A(1)\psi_A(2) - 2b^2\psi_B(1)\psi_B(2)] \cross 2^{-1/2} [\alpha(1)\beta(2) - \alpha(2)\beta(1)] \notag\\
&= \qty[4(a^4 + b^4)]^{-1/2} [2a^2\psi_A(1)\psi_A(2) - 2b^2\psi_B(1)\psi_B(2)] \cross 2^{-1/2} [\alpha(1)\beta(2) - \alpha(2)\beta(1)] \notag\\
&= \qty(a^4 + b^4)^{-1/2} [a^2\psi_A(1)\psi_A(2) - b^2\psi_B(1)\psi_B(2)] \cross 2^{-1/2} [\alpha(1)\beta(2) - \alpha(2)\beta(1)]
\end{align}
i.e.
\begin{align}
\ket{\Psi_{\gvb}} &= \qty(a^4 + b^4)^{-1/2}a^2 \cross 2^{-1/2} \psi_A(1)\psi_A(2)[\alpha(1)\beta(2) - \alpha(2)\beta(1)] \notag\\
&{}\quad - \qty(a^4 + b^4)^{-1/2}b^2 \cross 2^{-1/2} \psi_B(1)\psi_B(2)[\alpha(1)\beta(2) - \alpha(2)\beta(1)]  \notag\\
&= \qty(a^4 + b^4)^{-1/2}a^2 \ket{\psi_A\bar\psi_A} - \qty(a^4 + b^4)^{-1/2}b^2 \ket{\psi_B\bar\psi_B}
\end{align}
thus $ \ket{\Psi_{\gvb}} $ is identical to $ \ket{\Psi^{\text{MCSCF}}} $.

\subsection{Truncated CI and the Size-consistency Problem}
\ex{4.10}
\begin{align}
\Braket{\Psi_0 | \sH | 1_1\bar{1}_1 2_1 \bar{2}_1} &= \Braket{1_2\bar{1}_2 || 2_1 \bar{2}_1} \notag\\
&= \Braket{1_2\bar{1}_2 | 2_1 \bar{2}_1} - \Braket{1_2\bar{1}_2 | \bar{2}_1 2_1 } \notag\\
&= [1_2 2_1 | \bar{1}_2\bar{2}_1] - [1_2 \bar{2}_1 | \bar{1}_2 2_1] \notag\\
&= (1_2 2_1 | 1_2 2_1) \notag\\
&= 0
\end{align}
\begin{align}
\Braket{2_1\bar{2}_1 1_2\bar{1}_2 | \sH | 1_1\bar{1}_1 2_1 \bar{2}_1} &= \Braket{2_1\bar{2}_1 1_2\bar{1}_2 | \sH | 2_1 \bar{2}_1 1_1\bar{1}_1} \notag\\
&= \Braket{1_2\bar{1}_2 || 1_1\bar{1}_1} \notag\\
&= \Braket{1_2\bar{1}_2 | 1_1\bar{1}_1} - \Braket{1_2\bar{1}_2 | \bar{1}_1 1_1} \notag\\
&= [1_2 1_1 | \bar{1}_2\bar{1}_1] - [1_2 \bar{1}_1 | \bar{1}_2 1_1] \notag\\
&= (1_2 1_1 | 1_2 1_1) \notag\\
&= 0
\end{align}
\begin{align}
\Braket{1_1\bar{1}_1 2_2\bar{2}_2 | \sH | 1_1\bar{1}_1 2_1 \bar{2}_1} 
%&= \Braket{2_1\bar{2}_1 1_2\bar{1}_2 | \sH | 2_1 \bar{2}_1 1_1\bar{1}_1} \notag\\
&= \Braket{2_2\bar{2}_2 || 2_1 \bar{2}_1} \notag\\
&= \Braket{2_2\bar{2}_2 | 2_1 \bar{2}_1} - \Braket{2_2\bar{2}_2 | \bar{2}_1 2_1} \notag\\
&= [2_2 2_1 | \bar{2}_2\bar{2}_1] - [2_2 \bar{2}_1 | \bar{2}_2 2_1] \notag\\
&= (2_2 2_1 | 2_2 2_1) \notag\\
&= 0
\end{align}

\ex{4.11}
\begin{align}
\dfrac{^N E_{\corr}(\text{DCI})}{N} = \dfrac{\Delta - (\Delta^2 + N K_{12}^2)^{1/2}}{N}
\end{align}
From Ex 4.3, we get $ \Delta = 0.78865 $, $ K_{12} = 0.1813 $, thus
\begin{table}[H]
	\centering
	\begin{tabular}{cc}
		\hline 
		\specialrule{0em}{1pt}{2pt}
		$ N $ & $ ^N E_{\corr}(\text{DCI})/N $ 	\\ \hline
		1   & -0.02057\\
		10  & -0.01864\\
		100 & -0.01188\\ \hline
	\end{tabular}
\end{table}

\ex{4.12}
\subex{a.}
In addition to the matrix elements obtained in Eq. 4.56 in the textbook, we need to calculate the rest, i.e. those involving $ \ket{2_1\bar{2}_1 2_2\bar{2}_2} $.
\begin{align}
\Braket{\Psi_0 | \sH | 2_1\bar{2}_1 2_2\bar{2}_2} &= 0 
\end{align}
\begin{align}
\Braket{2_1 \bar{2}_1 1_2\bar{1}_2 | \sH | 2_1\bar{2}_1 2_2\bar{2}_2} &= \Braket{1_2\bar{1}_2 || 2_2\bar{2}_2} \notag\\
&= \Braket{1_2\bar{1}_2 | 2_2\bar{2}_2} - \Braket{1_2\bar{1}_2 | \bar{2}_2 2_2} \notag\\
&= [1_2 2_2|\bar{1}_2 \bar{2}_2] - [1_2 \bar{2}_2 |\bar{1}_2 2_2] \notag\\
&= (12|12) \notag\\
&= K_{12} \\
\Braket{1_1 \bar{1}_1 2_2\bar{2}_2 | \sH | 2_1\bar{2}_1 2_2\bar{2}_2} &= \Braket{1_1 \bar{1}_1 || 2_1\bar{2}_1} \notag\\
&= \Braket{1_1 \bar{1}_1 | 2_1\bar{2}_1} - \Braket{1_1 \bar{1}_1 |\bar{2}_1 2_1} \notag\\
&= [1_1 2_1|\bar{1}_1 \bar{2}_1] - [1_1 \bar{2}_1 |\bar{1}_1 2_1] \notag\\
&= (12|12) \notag\\
&= K_{12}
\end{align}
\begin{align}
\Braket{2_1\bar{2}_1 2_2\bar{2}_2 | \sH - E_0 | 2_1\bar{2}_1 2_2\bar{2}_2} &= 4h_{22} + 2J_{22} - 4h_{11} - 2J_{11} \notag\\
&= 4\Delta
\end{align}
thus the full CI equation is
\begin{equation}\label{key}
\mqty(0 & K_{12} & K_{12} & 0\\
      K_{12} & 2\Delta & 0 & K_{12}\\
      K_{12} & 0 & 2\Delta & K_{12}\\
      0 & K_{12} & K_{12} & 4\Delta)
\mqty(1\\ c_1 \\ c_2 \\ c_3) = {^2 E}_{\corr} \mqty(1\\ c_1 \\ c_2 \\ c_3)
\end{equation}
\iffalse
\subex{b.}
\begin{align}
K_{12} (c_1 + c_2) &= {^2 E}_{\corr} \\
K_{12} + 2\Delta c_1 + K_{12}c_3 &= {^2 E}_{\corr}c_1 \label{b2}\\
K_{12} + 2\Delta c_2 + K_{12}c_3 &= {^2 E}_{\corr}c_2 \label{b3}\\
K_{12}(c_1 + c_2) + 4\Delta c_3 &= {^2 E}_{\corr}c_3 
\end{align}
From $ \eqref{b2}\eqref{b3} $, we get
\begin{equation}\label{key}
(2\Delta - {^2 E}_{\corr})(c_1 - c_2) = 0
\end{equation}
thus
\begin{equation}\label{key}
c_1 = c_2
\end{equation}
$ \therefore $
\begin{equation}\label{key}
{^2 E}_{\corr} = 2 K_{12} c_1
\end{equation}
\fi
\subex{e.}
Directly solve the full CI equation (see \code{4-11,12.nb}), we get the lowest eigenvalue
\begin{equation}\label{key}
{^2 E}_{\corr} = 2[\Delta - \sqrt{\Delta^2 + K_{12}^2}]
\end{equation}

\ex{4.13}
\begin{align}
{^1 E}_{\corr}(\text{exact}) &= \Delta - \sqrt{\Delta^2 + K_{12}^2} \notag\\
&= \Delta - \Delta\sqrt{1 + \dfrac{K_{12}^2}{\Delta^2}} \notag\\
&\approx \Delta - \Delta\qty(1 + \dfrac{1}{2}\dfrac{K_{12}^2}{\Delta^2}) \notag\\
&\approx -\dfrac{1}{2}\dfrac{K_{12}^2}{\Delta}
\end{align}
\begin{align}
{^N E}_{\corr}(\text{DCI}) &= \Delta - \sqrt{\Delta^2 + N K_{12}^2} \notag\\
&= \Delta - \Delta\sqrt{1 + \dfrac{N K_{12}^2}{\Delta^2}} \notag\\
&\approx \Delta - \Delta\qty(1 + \dfrac{1}{2}\dfrac{N K_{12}^2}{\Delta^2}) \notag\\
&\approx -\dfrac{1}{2}\dfrac{N K_{12}^2}{\Delta}
\end{align}

\ex{4.14}
\subex{a.}
\begin{align}
{^N E}_{\corr}(\text{DCI}) &= \Delta - \sqrt{\Delta^2 + N K_{12}^2} \notag\\
&= \Delta - \Delta\sqrt{1 + \dfrac{N K_{12}^2}{\Delta^2}} \notag\\
&= \Delta - \Delta\qty(1 + \dfrac{1}{2}\dfrac{N K_{12}^2}{\Delta^2} - \dfrac{1}{8}\dfrac{N^2 K_{12}^4}{\Delta^4} + \cdots) \notag\\
&= -\dfrac{1}{2}\dfrac{N K_{12}^2}{\Delta} + \dfrac{1}{8}\dfrac{N^2 K_{12}^4}{\Delta^3} + \cdots
\end{align}
\subex{b.}
\begin{align}
c_0^2 = \dfrac{1}{1 + N c_1^2}
\end{align}
thus
\begin{equation}\label{key}
1 - c_0^2 = \dfrac{N c_1^2}{1 + N c_1^2}
\end{equation}
\subex{c.}
\begin{align}
c_1 &= \dfrac{K_{12}}{{^N E}_{\corr}(\text{DCI}) - 2\Delta} \notag\\
&= \dfrac{K_{12}}{-\dfrac{1}{2}\dfrac{N K_{12}^2}{\Delta} + \dfrac{1}{8}\dfrac{N^2 K_{12}^4}{\Delta^3} - 2\Delta + \cdots} \notag\\
&= \dfrac{1}{-\dfrac{1}{2}\dfrac{N K_{12}}{\Delta} + \dfrac{1}{8}\dfrac{N^2 K_{12}^3}{\Delta^3} - 2\dfrac{\Delta}{K_{12}} + \cdots} \notag\\
&= -\dfrac{1}{2}\dfrac{K_{12}}{\Delta} + \cdots
\end{align}
\subex{d.}
\begin{align}
\Delta E_{\text{Davidson}} &= (1 - c_0^2) \; {^N E}_{\corr}(\text{DCI}) \\
&= \dfrac{N (-K_{12}/2\Delta)^2}{1 + N (-K_{12}/2\Delta)^2} \qty(-\dfrac{1}{2}\dfrac{N K_{12}^2}{\Delta} + \dfrac{1}{8}\dfrac{N^2 K_{12}^4}{\Delta^3} + \cdots) \notag\\
&= N \dfrac{K_{12}^2}{4\Delta^2} \qty(-\dfrac{1}{2}\dfrac{N K_{12}^2}{\Delta} + \dfrac{1}{8}\dfrac{N^2 K_{12}^4}{\Delta^3} + \cdots) \notag\\
&= -\dfrac{N^2 K_{12}^4}{8\Delta^3} +\cdots
\end{align}
\subex{e.}
\begin{align}
\Delta E_{\text{Davidson}} &= (1 - c_0^2) \; {^N E}_{\corr}(\text{DCI}) \notag\\
&= \dfrac{N c_1^2}{1 + N c_1^2} NK_{12}c_1 \notag\\
&= \dfrac{N^2 K_{12} c_1^3}{1 + N c_1^2}
\end{align}
while
\begin{equation}\label{key}
c_1 = {^N E}_{\corr}(\text{DCI}) / NK_{12}
\end{equation}
thus
\begin{align}
\Delta E_{\text{Davidson}} &= \dfrac{N^2 K_{12} c_1^3}{1 + N c_1^2} \notag\\
&= \dfrac{[{^N E}_{\corr}(\text{DCI})]^3 / NK_{12}^2}{1 + [{^N E}_{\corr}(\text{DCI})]^2 / NK_{12}^2} \notag\\
&= \dfrac{[{^N E}_{\corr}(\text{DCI})]^3}{NK_{12}^2 + [{^N E}_{\corr}(\text{DCI})]^2 }
\end{align}
Since
\begin{align}
{^N E}_{\corr}(\text{DCI}) &= \Delta - \sqrt{\Delta^2 + N K_{12}^2} \\
{^N E}_{\corr}(\text{exact}) &= N\qty[\Delta - \sqrt{\Delta^2 + K_{12}^2}]
\end{align}
\newpage
The values of $ {^N E}_{\corr}(\text{DCI}) $, $ {^N E}_{\corr}(\text{exact}) $, $ \Delta E_{\text{Davidson}} $ for $ N=1,\cdots, 20, 100 $ are as follows.
\begin{table}[H]
	%\centering
	\begin{tabular}{cccc}
	\hline
	\specialrule{0em}{1pt}{2pt}
	$ N $ & $ {^N E}_{\corr}(\text{DCI}) $ & $ {^N E}_{\corr}(\text{exact}) $ & $ \Delta E_{\text{Davidson}} $  \\ \hline
    1 & -0.020571	& -0.020571 &  -0.0002615 \\
    2 & -0.040632	& -0.041142 &  -0.0009954 \\
    3 & -0.060219	& -0.061713 &  -0.0021360	 \\
    4 & -0.079364	& -0.082284 &  -0.0036282	 \\
    5 & -0.098095	& -0.102855	 & -0.0054259	 \\
    6 & -0.116439	& -0.123426	 & -0.0074900	 \\
    7 & -0.134419	& -0.143997	 & -0.0097872	 \\
    8 & -0.152055	& -0.164567	 & -0.0122891	 \\
    9 & -0.169367	& -0.185138	 & -0.0149711	 \\
    10 & -0.186371	& -0.205709	 & -0.0178120	 \\
    11 & -0.203084	& -0.22628	 & -0.0207933	 \\
    12 & -0.219519	& -0.246851	 & -0.0238991	 \\
    13 & -0.235691	& -0.267422	 & -0.0271151	 \\
    14 & -0.251612	& -0.287993	 & -0.0304291	 \\
    15 & -0.267292	& -0.308564	 & -0.0338301	 \\
    16 & -0.282743	& -0.329135	 & -0.0373084	 \\
    17 & -0.297975	& -0.349706	 & -0.0408554	 \\
    18 & -0.312996	& -0.370277	 & -0.0444636	 \\
    19 & -0.327814	& -0.390848	 & -0.0481262	 \\
    20 & -0.342439	& -0.411419	 & -0.0518370	 \\
    100 & -1.188450 & -2.057090  & -0.3571950 \\ 
    \hline
	\end{tabular}
\end{table}
The values and errors of DCI energies and DCI energies with Davidson correction are as follows.
\begin{table}[H]
	%\centering
	\begin{tabular}{ccccc}
		\hline
		\specialrule{0em}{1pt}{2pt}
		 $ N $ & $ {^N E}_{\corr}(\text{DCI}) / {^N E}_{\corr}(\text{exact}) $      & Error/\% & $ [{^N E}_{\corr}(\text{DCI}) + \Delta E_{\text{Davidson}}] / {^N E}_{\corr}(\text{exact}) $ & Error/\% \\ \hline
	    1 & 1.0000	 & 0.00	& 1.0127	& -1.27  \\
		2 & 0.9876     & 1.24	& 1.0118	& -1.18  \\
		3 & 0.9758	 & 2.42	& 1.0104	& -1.04  \\
		4 & 0.9645	 & 3.55	& 1.0086	& -0.86 \\
		5 & 0.9537	 & 4.63	& 1.0065	& -0.65 \\
		6 & 0.9434	 & 5.66	& 1.0041	& -0.41 \\
		7 & 0.9335	 & 6.65	& 1.0015	& -0.15 \\
		8 & 0.9240	 & 7.60	& 0.9986	& 0.14  \\
		9 & 0.9148	 & 8.52	& 0.9957	& 0.43  \\
		10 & 0.9060	 & 9.40	& 0.9926	& 0.74  \\
		11 & 0.8975	 & 10.25	& 0.9894	& 1.06   \\
		12 & 0.8893	 & 11.07	& 0.9861	& 1.39   \\
		13 & 0.8813	 & 11.87	& 0.9827	& 1.73   \\
		14 & 0.8737	 & 12.63	& 0.9793	& 2.07   \\
		15 & 0.8662	 & 13.38	& 0.9759	& 2.41   \\
		16 & 0.8591	 & 14.10	& 0.9724	& 2.76   \\
		17 & 0.8521	 & 14.79	& 0.9689	& 3.11   \\
		18 & 0.8453	 & 15.47	& 0.9654	& 3.46   \\
		19 & 0.8387	 & 16.13	& 0.9619	& 3.81   \\
		20 & 0.8323	 & 16.77	& 0.9583	& 4.17   \\
		100 & 0.5777 & 42.23    & 0.7514    & 24.86  \\
		\hline
	\end{tabular}
\end{table}

\newpage
\subex{f.}
From data of Saxe et al., we get
\begin{equation}\label{key}
E_{\corr}(\text{DCI}) = -0.139340 \qquad c_0 = 0.97938
\end{equation}
thus
\begin{align}
\Delta E_{\text{Davidson}} &= (1 - c_0^2)E_{\corr}(\text{DCI}) \notag\\
&= (1 - 0.97938^2)\times(-76.129178) \notag\\
&= -0.005687
\end{align}
thus
\begin{table}[H]
	\begin{tabular}{ccc}
		\hline
		& correlation energy & error wrt full CI \\ \hline
		DCI + Davidson & -0.145027 & 0.003181\\
		DQCI           & -0.145859 & 0.002349\\
		Full CI        & -0.148208 & 0\\ \hline
	\end{tabular}
\end{table}

\ex{4.15}
%For $ N $ independent $ \ce{H_2} $ molecules, the normalized 
\begin{align}
\Braket{\Psi_0 | \Phi_0} &= \prod_{i=1}^N \qty[(1+c^2)^{-1/2}\Braket{1_i\bar{1}_i|1_i\bar{1}_i} + c(1+c^2)^{-1/2}\Braket{1_i\bar{1}_i|2_i\bar{2}_i}] \notag\\
&= (1+c^2)^{-N/2}
\end{align}
Since 
\begin{equation}\label{key}
c = \dfrac{{^1 E}_{\corr}}{K_{12}} = \dfrac{-0.020571}{0.1813} = -0.1135
\end{equation}
we get
\begin{table}[H]
	\begin{tabular}{cc}
		\hline
		$ N $ & $ \Braket{\Psi_0 | \Phi_0} $ \\ \hline
		1 & 0.9936 \\
		10 & 0.9380 \\
		100 & 0.5273 \\ \hline
	\end{tabular}
\end{table}


\end{document}